\documentclass[twoside]{MATH77}
\usepackage{multicol}
\usepackage[fleqn,reqno,centertags]{amsmath}
\begin{document}
\begtoc

\silentfootnote{$^\copyright$1997 Calif. Inst. of Technology, \thisyear \ Math \`a la Carte, Inc.}

\centerline{\LARGE \bf MATH77 and {\em mathc90}}\vspace{15pt}

\centerline{\Large \bf Mathematical Subprogram Libraries}
\vspace{5pt}

\centerline{\Large \bf for Fortran~77 and ANSI C}
\vspace{20pt}

\centerline{\Large \bf Table of Contents}\vspace{10pt}

\begin{description}
\item[\large \bf 1.]  {\large {\bf Introduction}}

\vspace{-20pt}
\begin{tabbing}
\hspace{.45in}\=\\
\phantom{1}1.0 \> Release~6.0 of MATH77\\
\phantom{1}1.1 \> Purpose and Scope\\
\phantom{1}1.2 \> Access to the MATH77 and {\em mathc90} Libraries\\
\phantom{1}1.3 \> Conventions Followed in the Code and Documentation\\
\end{tabbing}\vspace{-15pt}

\item[\large \bf 2.]  {\large {\bf Mathematical Functions}}

\vspace{-20pt}
\begin{tabbing}
\hspace{.45in}\=\\
\phantom{1}2.1 \> Inverse Hyperbolic Functions\\
\phantom{1}2.2 \> Error Function\\
\phantom{1}2.3 \> Gamma and Log-Gamma Functions\\
\phantom{1}2.4 \> Bessel Functions $J_0$, $J_1$, $Y_0$ and $Y_1$\\
\phantom{1}2.5 \> Bessel Functions of General Orders $J_\nu $ and $Y_\nu $\\
\phantom{1}2.6 \> Bessel Functions $I_0$, $I_1$, $K_0$ and $K_1$\\
\phantom{1}2.7 \> $\{$ Empty $\}$\\
\phantom{1}2.8 \> Complete Elliptic Integrals $K$ and $E$\\
\phantom{1}2.9 \> Incomplete Elliptic Integrals\\
\phantom{1}2.10 \> Exponential Integrals Ei and $E_1$\\
\phantom{1}2.11 \> Finite Legendre Series\\
\phantom{1}2.12 \> Finite Laguerre Series\\
\phantom{1}2.13 \> Inverse Error Function and Inverse Complementary Error Function\\
\phantom{1}2.14 \> Sine and Cosine Integrals\\
\phantom{1}2.15 \> Procedures to Avoid Loss of Precision: $\ln (1+x)$, etc\\
\phantom{1}2.16 \> Complex Error Function $w(z)$\\
\phantom{1}2.17 \> Fresnel Integrals\\
\phantom{1}2.18 \> Digamma or $\psi $ Function\\
\phantom{1}2.19 \> Incomplete Gamma Function Ratio\\
\phantom{1}2.20 \> Binomial Coefficients\\
\end{tabbing}\vspace{-15pt}

\item[\large \bf 3.]  {\large {\bf Pseudorandom Number Generation}}

\vspace{-20pt}
\begin{tabbing}
\hspace{.45in}\=\\
\phantom{1}3.1 \> Uniform Random Numbers\\
\phantom{1}3.2 \> Gaussian (Normal) Random Numbers and Vectors\\
\phantom{1}3.3 \> Random Numbers: Exponential, Rayleigh, and Poisson\\
\end{tabbing}\vspace{-15pt}

\item[\large \bf 4.]  {\large {\bf Linear Systems of Equations and Linear
Least-Squares}}

\vspace{-20pt}
\begin{tabbing}
\hspace{.45in}\=\\
\phantom{1}4.1 \> Square Nonsingular Systems of Linear Equations\\
\phantom{1}4.2 \> Linear Least-Squares and Covariance Matrix\\
\phantom{1}4.3 \> Singular Value Decomposition and Analysis\\
\phantom{1}4.4 \> Sequential Preprocessing of Linear Least-Squares Data\\
\phantom{1}4.5 \> Sequential Solution of a Banded Least-Squares Problem\\
\phantom{1}4.6 \> Solution of a Positive-Definite System with Cholesky
Factorization\\
\end{tabbing}\vspace{-15pt}

\newpage
\item[\large \bf 5.]  {\large {\bf Matrix Eigenvalues and Eigenvectors}}

\vspace{-20pt}
\begin{tabbing}
\hspace{.45in}\=\\
\phantom{1}5.1 \> Eigenvalues and Eigenvectors of a Symmetric Matrix\\
\phantom{1}5.2 \> Eigenvalues and Eigenvectors of a Hermitian Complex Matrix\\
\phantom{1}5.3 \> Eigenvalues of an Unsymmetric Matrix\\
\phantom{1}5.4 \> Eigenvalues and Eigenvectors of an Unsymmetric Matrix\\
\end{tabbing}\vspace{-15pt}

\item[\large \bf 6.]  {\large {\bf Matrix-Vector Utility Subprograms}}

\vspace{-20pt}
\begin{tabbing}
\hspace{.45in}\=\\
\phantom{1}6.1 \> Vector and Matrix Output\\
\phantom{1}6.2 \> Extended Vector and Matrix Output\\
\phantom{1}6.3 \> Basic Linear Algebra Subprograms (BLAS1)\\
\phantom{1}6.4 \> One Householder Transformation\\
\end{tabbing}\vspace{-15pt}

\item[\large \bf 7.]  {\large {\bf Polynomial Root Finding}}

\vspace{-20pt}
\begin{tabbing}
\hspace{.45in}\=\\
\phantom{1}7.1 \> Roots of a Polynomial\\
\phantom{1}7.2 \> Roots of a Quadratic Polynomial\\
\phantom{1}7.3 \> Compute Polynomial Coefficients from Roots\\
\end{tabbing}\vspace{-15pt}

\item[\large \bf 8.]  {\large {\bf Nonlinear Equation Solving}}

\vspace{-20pt}
\begin{tabbing}
\hspace{.45in}\=\\
\phantom{1}8.1 \> Zero of a Univariate Function\\
\phantom{1}8.2 \> Solve System of Nonlinear Equations\\
\phantom{1}8.3 \> Check Code for Computing Derivatives\\
\end{tabbing}\vspace{-15pt}

\item[\large \bf 9.]  {\large {\bf Minimization}}

\vspace{-20pt}
\begin{tabbing}
\hspace{.45in}\=\\
\phantom{1}9.1 \> Local Minimum of a Univariate Function\\
\phantom{1}9.2 \> Local Minimum of a Multivariate Function, with Linear Constraints\\
\phantom{1}9.3 \> Nonlinear Least-Squares\\
\end{tabbing}\vspace{-15pt}

\item[\large \bf 10.]  {\large {\bf Finite Fourier Transforms}}

\vspace{-20pt}
\begin{tabbing}
\hspace{.45in}\=\\
10.0 \> Overview of Fourier Transforms and Spectral Analysis\\
10.1 \> One-Dimensional Real Fourier Transforms\\
10.2 \> Trigonometric, Cosine, and Sine Fourier Transforms\\
10.3 \> Complex Fourier Transform\\
10.4 \> Multi-dimensional Real Fourier Transform\\
10.5 \> Primitive Fast Fourier Transform\\
\end{tabbing}\vspace{-15pt}

\item[\large \bf 11.]  {\large {\bf Curve Fitting}}

\vspace{-20pt}
\begin{tabbing}
\hspace{.45in}\=\\
11.1 \> Polynomial Least-Squares Curve Fit\\
11.2 \> Evaluation, Integration, and Differentiation of Polynomials\\
11.3 \> Conversion between Chebyshev and Monomial Representations of a
Polynomial\\
11.4 \> Least-Squares Cubic Spline Fit\\
11.5 \> Least-Squares Data Fitting Using $K^{th}$ Order Splines with
Constraints\\
11.6 \> Low-level Subprograms for Operations on Splines\\
\end{tabbing}\vspace{-15pt}

\item[\large \bf 12.] {\large {\bf Table Look-Up and Interpolation}}

\vspace{-20pt}
\begin{tabbing}
\hspace{.45in}\=\\
12.1 \> One-Dimensional Table Look Up, Interpolation, and
Differentiation\\
12.2 \> Multi-Dimensional Table Look Up, Interpolation, and
Differentiation\\
12.3 \> Table Look-up With Hermite Cubic Interpolation\\
12.4 \> C$^0$ and C$^1$ Surface Interpolation to Scattered Data\\
\end{tabbing}\vspace{-15pt}

\item[\large \bf 13.] {\large {\bf Definite Integrals (Quadrature)}}

\vspace{-20pt}
\begin{tabbing}
\hspace{.45in}\=\\
13.0 \> Effective Use of the Quadrature Software\\
13.1 \> Numerical Evaluation of Integrals Over One Dimension\\
13.2 \> Numerical Evaluation of Integrals Over More Than One Dimension\\
\end{tabbing}\vspace{-15pt}

\newpage
\item[\large \bf 14.]  {\large {\bf Ordinary Differential Equations}}

\vspace{-20pt}
\begin{tabbing}
\hspace{.45in}\=\\
14.1 \> Variable Order Adams Method for Ordinary~Differential~Equations\\
14.2 \> Explicit Runge-Kutta Method for Ordinary~Differential~Equations\\
\end{tabbing}\vspace{-15pt}

\item[\large \bf 15.]  {\large {\bf Statistics}}

\vspace{-20pt}
\begin{tabbing}
\hspace{.45in}\=\\
15.1 \> Basic Statistics and Histogram\\
15.2 \> Cumulative Distribution Function and Percentage Points for
Normal Probability Distribution\\
15.3 \> Cumulative Distribution Function for Chi-Square Probability
Distribution\\
15.4 \> Cumulative Distribution Function for Poisson Probability
Distribution\\
\end{tabbing}\vspace{-15pt}

\item[\large \bf 16.]  {\large {\bf Graphics}}

\vspace{-20pt}
\begin{tabbing}
\hspace{.45in}\=\\
16.1 \> Character-based Graphics --- One or More XY Graphs\\
16.2 \> Character-based Graphics --- Single Print Line\\
16.3 \> Plotting Using \TeX \\
\end{tabbing}\vspace{-15pt}

\item[\large \bf 17.]  {\large {\bf Special Arithmetic}}

\vspace{-20pt}
\begin{tabbing}
\hspace{.45in}\=\\
17.1 \> Computation Using Derivative Arrays or Univariate Taylor Series\\
17.2 \> Computation Using Partial Derivative Arrays or
Multivariate~Taylor~Series\\
17.3 \> Double Precision Complex Computation\\
\end{tabbing}\vspace{-15pt}

\item[\large \bf 18.]  {\large {\bf Sorting}}

\vspace{-20pt}
\begin{tabbing}
\hspace{.45in}\=\\
18.1 \> Sorting One-dimensional Arrays in Memory\\
18.2 \> Sorting Data of Arbitrary Structure in Memory\\
18.3 \> Sorting Partially Ordered Data of Arbitrary Structure in Memory\\
18.4 \> Sorting Data Sets Too Large to Fit in Memory\\
\end{tabbing}\vspace{-15pt}

\item[\large \bf 19.] {\large {\bf Library Utilities}}

\vspace{-20pt}
\begin{tabbing}
\hspace{.45in}\=\\
19.1 \> System Parameters\\
19.2 \> Error Message Processor\\
19.3 \> Extended Error Message Processor\\
19.4 \> Converting Codes to Different Versions\\
19.5 \> Checking the Installed Library\\
%19.6 \> Testing Functions of One and Two Variables\\
19.7 \> Checking and Output of Program Unit Interfaces\\
\end{tabbing}\vspace{-15pt}

%\item[\large \bf 20.]  {\large {\bf Graphs, Sets, Queues}}
%
%\vspace{-20pt}
%\begin{tabbing}
%\hspace{.45in}\=\\
%20.1 \> Construct A Graph Representation\\
%20.2 \> Strongly Connected Components of A Directed Graph\\
%20.3 \> Union and Find on Sets of Sets\\
%20.4 \> Minimum-Cost Spanning Tree of an Undirected Graph\\
%20.5 \> Priority Queues\\
%\end{tabbing}\vspace{-15pt}

\item[\large \bf Appendix A.]  {\large {\bf Files Required by Each Entry}}

\item[\large \bf Appendix B.]  {\large {\bf Entry Names and Common Block
Names}}\rule{0pt}{15pt}

\item[\large \bf Appendix C.]  {\large {\bf Usage of the {\em mathc90}
Library}}\rule{0pt}{15pt}

\item[\large \bf Appendix D.]  {\large {\bf Function Prototypes for the
{\em mathc90} Library}}\rule{0pt}{15pt}

\item[\large \bf Index]\rule{0pt}{15pt}
\end{description}

\end{document}
