\documentclass[twoside]{MATH77}
\usepackage{multicol}
\usepackage[fleqn,reqno,centertags]{amsmath}
\begin{document}
\begmath 7.1 Roots of a Polynomial

\silentfootnote{$^\copyright$1997 Calif. Inst. of Technology, \thisyear \ Math \`a la Carte, Inc.}

\subsection{Purpose}

Given the coefficients $a_i$ of a polynomial of degree $n=$ {NDEG }$>0,$%
\begin{equation*}
a_1z^n+a_2z^{n-1}+...+a_nz+a_{n+1}
\end{equation*}
with $a_1\neq 0$, this subroutine computes the NDEG roots of the polynomial.
Since the roots may be complex, they will always be returned as complex
numbers, even when they are real.

Subroutines are provided for four cases:

\begin{tabular}{ll}
{\bf SPOLZ} & for single precision real coefficients.\\
{\bf DPOLZ} & for double precision real coefficients.\\
{\bf CPOLZ} & for single precision complex coefficients.\\
{\bf ZPOLZ} & for double precision complex coefficients.\\
\end{tabular}

See Section E for remarks on the method of storing double precision complex
numbers.

\subsection{Usage}

\subsubsection{Program Prototype, Single Precision Real Coefficients}

\begin{description}
\item[INTEGER]  \ {\bf NDEG, IERR}

\item[REAL]  \ {\bf A}$(\geq $ NDEG + 1){\bf , H}$(\geq $ NDEG**2)

\item[~~~~COMPLEX]  \ {\bf Z}$(\geq $ NDEG)
\end{description}

Assign values to A(), and NDEG.%
$$
\fbox{{\bf CALL SPOLZ(A, NDEG, Z, H, IERR)}}
$$
Computed quantities are returned in Z() and IERR.

\subsubsection{Program Prototype, Double Precision Real Coefficients}

\begin{description}
\item[INTEGER]  \ {\bf NDEG, IERR}

\item[DOUBLE PRECISION]  \ {\bf A}$(\geq $ NDEG + 1){\bf ,\newline
H}$(\geq $NDEG**2)

\item[DOUBLE PRECISION]  \ {\bf Z}(2, $\geq $ NDEG) or

\item[~~~~COMPLEX*16]  {\bf Z}$(\geq $ NDEG)
\end{description}

Assign values to A(), and NDEG.%
$$
\fbox{{\bf CALL DPOLZ(A, NDEG, Z, H, IERR)}}
$$
Computed quantities are returned in Z() and IERR.

\subsubsection{Program Prototype, Single Precision Complex Coefficients}

\begin{description}
\item[INTEGER]  \ {\bf NDEG, IERR}

\item[COMPLEX]  \ {\bf A}$(\geq $NDEG + 1)

\item[REAL]  \ {\bf H}$(\geq $ 2 * NDEG**2)

\item[COMPLEX]  \ {\bf Z}$(\geq $ NDEG)
\end{description}

Assign values to A(), and NDEG.%
$$
\fbox{{\bf CALL CPOLZ(A, NDEG, Z, H, IERR)}}
$$
Computed quantities are returned in Z() and IERR.

\subsubsection{Program Prototype, Double Precision Complex Coefficients}

\begin{description}
\item[INTEGER]  \ {\bf NDEG, IERR}

\item[DOUBLE PRECISION]  \ {\bf A}(2, $\geq $ NDEG + 1) or

\item[~~~~COMPLEX*16]  \ {\bf A}$(\geq $ NDEG + 1)

\item[DOUBLE PRECISION]  \ {\bf H} $(\geq $2 * NDEG**2)

\item[DOUBLE PRECISION]  \ {\bf Z}(2, $\geq $ NDEG) or

\item[~~~~COMPLEX*16]  \ {\bf Z}$(\geq $ NDEG)
\end{description}

Assign values to A(), and NDEG.%
$$
\fbox{{\bf CALL ZPOLZ(A, NDEG, Z, H, IERR)}}
$$
Computed quantities are returned in Z() and IERR.

\subsubsection{Argument Definitions}

\begin{description}
\item[A()]  [in] Contains the coefficients of a polynomial, high order
coefficient first, with A($1)\neq 0$. The contents of this array will not be
modified by the subroutine.

In the case of ZPOLZ, if the declaration DOUBLE PRECISION A(2, NDEG$+1)$ is
used, the user must store the real part of $a_i$ in A(1, $i$) and the imaginary
part in A(2, $i$).

\item[NDEG]  [in] Degree of the polynomial.

\item[H()]  [scratch] Work space for the subroutine.

\item[Z()]  [out] Contains the polynomial roots stored as complex numbers.
In the cases of DPOLZ and ZPOLZ, if the declaration DOUBLE PRECISION Z($2,$
NDEG) is used, the real part of $z_i$ will be returned in Z($1, i)$ and the
imaginary part in Z($2, i).$

\item[IERR]  [out] Error flag. Set by the subroutine to~0 on normal
termination. Set to $-$1 if A($1)=0$. Set to $-$2 if NDEG $<1$. Set to J $>0$
if the iteration count limit has been exceeded and roots~1 through J have
not been determined.
\end{description}

\subsection{Examples and Remarks}

The program, DRSPOLZ, uses SPOLZ to compute the roots of the two polynomials%
\begin{gather*}
x^3-4x^2+x-4,\quad \text{and}\\
x^5-15x^4+85x^3-225x^2+274x-120
\end{gather*}
The output is shown in ODSPOLZ.

The Fortran~77 standard does not support a double precision complex data
type, although such a type is supported in many compilers using the
declaration, COMPLEX*16. The subroutines described here conform to the
standard and thus do not use COMPLEX*16. For cases in which double precision
complex quantities are communicated using the arrays A() or Z(), these
subroutines store a complex number as an adjacent pair of double precision
numbers, representing respectively the real and imaginary parts of the
complex number.  This is compatible with the Fortran~90 storage
convention for double precision complex.  If the user has the COMPLEX*16
declaration available and wishes to use it, it will generally be compatible
with this storage
convention.

\subsection{Functional Description}

\subparagraph{Method}

The degree, NDEG, and coefficients, $a_1$, $..., a_{NDEG+1}$, are given. An
error condition is reported if NDEG $< 1$ or if $a_1 = 0.$

Let $k+1$ be the index of the last nonzero coefficient. If $k<\text{NDEG}$, the
polynomial has a root at zero repeated $\text{NDEG}-k$ times and these roots
are recorded. The remaining roots will be roots of the polynomial of degree $k$
whose coefficients are the first $k+1$ given coefficients. If $k=1$, the
root $-a_2/a_1$ is recorded.

If  $k>1$, the subroutine forms the $k\times k$ companion matrix $H$ for
this $k^{th}$ degree polynomial. The matrix $H$ is zero except for the first
row,%
\begin{equation*}
h_{1,j}=-a_{j+1}/a_1,\quad j=1, ..., k
\end{equation*}
and the elements just below the diagonal,%
\begin{equation*}
h_{i,i-1}=1,\quad i=2, ..., k
\end{equation*}
A scaling algorithm is applied to $H$ to balance the sizes of the nonzero
elements.  Testing showed that accuracy could be very unsatisfactory if no
scaling were done.  The scaling method used is a modification of the
subroutine BALANC from the EISPACK collection of matrix eigensystem
subroutines \cite{Smith:1974:MER}, \cite{Garbow:1977:MER}.  The
modification avoids treating the elements that are known to be zero due to
the form of the companion matrix.  The search for useful row or column
permutations done in BALANC is also deleted since, with $a_{k+1}$ known to
be nonzero, none of these permutations are possible.

The eigenvalues of the balanced $H$ matrix are then computed using appropriate
code for the QR algorithm from EISPACK. For SPOLZ and DPOLZ the code for the
EISPACK subroutine, HQR, has been incorporated in-line. For CPOLZ or ZPOLZ,
a call is made to SCOMQR or DCOMQR, respectively. These latter two
subroutines are, respectively, single precision and double precision versions
of the EISPACK subroutine, COMQR.

Reference \cite{Goedecker:1994:RAF} reports that compared with other
widely used methods, the conversion of the root finding problem to an
eigenvalue problem is superior in reliability, generally superior in
accuracy, and comparable in speed.

\subparagraph{Accuracy tests}

A root of a polynomial is called ill-conditioned if a small relative change
in the coefficients makes a significantly larger relative change in the
root. Roots that are bunched closely together relative to their magnitude,
or relative to their distance from other roots, tend to be ill-conditioned.
An extreme case of ill-conditioning occurs with multiple roots. If the
coefficients of a polynomial are only known to some relative precision, say $%
p$, then a double root will typically have a relative uncertainty of about $%
p^{1/2}$, and a triple root a relative uncertainty of about $p^{1/3}.$

The accuracy of a computed root is limited by the inherent conditioning of
the root. Rather than directly testing the accuracy of the computed roots,
we have chosen to test the accuracy with which polynomial coefficients could
be reconstituted from the computed roots.

To test these subroutines we selected eight sets of roots to test SPOLZ and
DPOLZ and a different group of eight sets of roots to test CPOLZ and ZPOLZ.
These sets included double and triple roots, roots that formed a small
cluster relative to other roots, roots that differed by a factor of $10^8$,
and both real and complex roots. The number of roots in each set varied
from~1 to~7.

From each set of roots we created 8~test sets by multiplying all roots in
the set by one of the factors, $10^{-3}$, $10^{-2}$, ..., $10^3$, or $10^4$.
This produced a total of 64~test sets for each subroutine.

To execute the test for one subroutine, say xPOLZ, and one set of roots, say
$r_1$, ..., $r_n$, we did the following:

\begin{itemize}
\item[1.]  Compute, in double precision, the coefficients, say $a_i$, of the
monic polynomial having the roots, $r_i$.  Subroutine ZCOEF of
Chapter 15.3 is used for this.

\item[2.]  Use subroutine xPOLZ to compute the roots, say $s_i$, of this
polynomial.

\item[3.]  Compute, in double precision, the coefficients, say $b_i$, of the
monic polynomial having the roots, $s_i.$

\item[4.]  Compute $\delta =$ the maximum of the relative differences
between corresponding coefficients, $a_i$ and $b_i.$

\item[5.]  Compute $\varepsilon =\delta \ /\rho $, where $\rho $ is the
relative precision of the arithmetic being used by xPOLZ. The value for $%
\rho $ was obtained as R1MACH(4) for single precision and D1MACH(4) for
double precision.  (See Chapter~19.1.)
\end{itemize}

Ideally, Step~3 should be done using significantly higher precision than is
used by the subroutine xPOLZ being tested, so the error measures, $\delta $
and $\varepsilon $, will be attributable only to xPOLZ with no inflation
from Step~3. Since we used double precision in Step~3 for all cases, we
conclude that the reported values of $\varepsilon $ are solely attributable
to SPOLZ and CPOLZ in the tests of those subroutines, while the values of $%
\varepsilon $ reported for DPOLZ and ZPOLZ are somewhat inflated due to
errors from Step~3.

These tests were run on an IBM PC/{AT} equipped with the Intel 80287 math
coprocessor. The precision was $\rho \approx 1.2 \times 10^{-7}$ for single
precision and $2.2\times 10^{-16}$ for double precision. The results may be
summarized as follows:

\begin{tabular}{lrr}
    & \multicolumn{1}{c}{\bf
Max. $\varepsilon $ over}  &  \multicolumn{1}{c}{\bf Number of Cases}\\
  \multicolumn{1}{c}{\bf Subroutine} & \multicolumn{1}{c}{\bf 64 test cases} &
  \multicolumn{1}{c}{\bf with $\epsilon >10$}  \\
  \hspace{.15in} SPOLZ   &  35 \hspace{.3in}   &  19 \hspace{.5in}  \\
  \hspace{.15in} DPOLZ   &  57 \hspace{.3in}  &  21 \hspace{.5in} \\
  \hspace{.15in} CPOLZ   &    8 \hspace{.3in}   &    0 \hspace{.5in} \\
  \hspace{.15in} ZPOLZ   &  43  \hspace{.3in}  &    8 \hspace{.5in}
\end{tabular}

\bibliography{math77}
\bibliographystyle{math77}

\subsection{Error Procedures and Restrictions}

The error indicator, IERR, will be set as follows:

\begin{description}
\item[\rm = 0]  \ when no errors are detected.

\item[\rm = $-$1]  \ if A(1) is zero, or A(1,~1) and A(2,~1) are both zero.

\item[\rm = $-$2]  \ if NDEG $<1.$

\item[\rm = J $>$ 0]  \ if the iteration count limit has been exceeded and
roots 1 through J have not been determined. The matrix eigenvalue codes
allow a maximum of 30~iterations for each root. This limit is ample.
\end{description}

When IERR is set nonzero, the subroutine will also issue an error message
using ERMSG, of Chapter~19.2, with an error level of 0.

\subsection{Supporting Information}

\begin{tabular}{@{\bf}l@{\hspace{5pt}}l}
\bf Entry & \hspace{.35in} {\bf Required Files}\vspace{2pt}\\
CPOLZ & \parbox[t]{2.7in}{\hyphenpenalty10000 \raggedright
 AMACH, CPOLZ, ERFIN, ERMSG, SCOMQR\rule[-5pt]{0pt}{8pt}}\\
DPOLZ & \parbox[t]{2.7in}{\hyphenpenalty10000 \raggedright
 AMACH, DPOLZ, ERFIN, ERMSG\rule[-5pt]{0pt}{8pt}}\\
SPOLZ & \parbox[t]{2.7in}{\hyphenpenalty10000 \raggedright
 AMACH, ERFIN, ERMSG, SPOLZ\rule[-5pt]{0pt}{8pt}}\\
ZPOLZ & \parbox[t]{2.7in}{\hyphenpenalty10000 \raggedright
 AMACH, DCOMQR, DZABS, ERFIN, ERMSG, ZPOLZ, ZQUO, ZSQRT}\\
\end{tabular}

Designed by C. L. Lawson, JPL, May~1986. Programmed by C. L. Lawson and S.
Y. Chiu, JPL, May~1986, Feb.~1987.


\begcodenp
\lstset{language=[77]Fortran,showstringspaces=false}
\lstset{xleftmargin=.8in}

\centerline{\bf \large DRSPOLZ}\vspace{10pt}
\lstinputlisting{\codeloc{spolz}}

\vspace{30pt}\centerline{\bf \large ODSPOLZ}\vspace{10pt}
\lstset{language={}}
\lstinputlisting{\outputloc{spolz}}
\end{document}
