\documentclass[twoside]{MATH77}
\usepackage{multicol}
\usepackage[fleqn,reqno,centertags]{amsmath}
\begin{document}
\begmath 5.3 Eigenvalues of an Unsymmetric Matrix

\silentfootnote{$^\copyright$1997 Calif. Inst. of Technology, \thisyear \ Math \`a la Carte, Inc.}

\subsection{Purpose}

Compute all eigenvalues of a real N $\times $ N unsymmetric matrix $A$. Some or all
of the eigenvalues may be complex.

\subsection{Usage}

\subsubsection{Program Prototype, Single Precision}

\begin{description}
\item[REAL]  \ {\bf A}(LDA, $\geq $ N) \ [LDA$\geq $N]{\bf , VR}($\geq $ N)%
{\bf , VI}($\geq $N)

\item[INTEGER]  \ {\bf LDA, N, IFLAG}($\geq $ N)
\end{description}

Assign values to A(,), LDA, and N.
$$
\fbox{{\bf CALL SEVUN(A, LDA, N, VR, VI, IFLAG)}}
$$
Results are returned in VR(), VI(), IFLAG(1). The contents of A(,) will be
modified.

\subsubsection{Argument Definitions}

\begin{description}
\item[A(,), LDA, N]  \ A(,) is [inout], LDA and N are [in]. On entry A(,) must
contain the N $\times $ N matrix $A$ whose eigenvalues are to be computed. The
integer LDA is the dimension of the first subscript of the array A(,).
Require LDA $\geq $ N. On return the contents of A(,) will be modified.

\item[VR(), VI()]  \ [out] The subroutine will store the $J^{th}$ eigenvalue
in VR($J$) and VI($J$), $J =1$, ..., N. The real part is stored in
VR($J$), and
the imaginary part in VI($J$). If the $J^{th}$ eigenvalue is real VI($J$) will
be zero. The eigenvalues will be sorted so that VR(1) $\leq $ VR(2) $\leq $
... $\leq $ VR(N), and if VR($J$) = VR($J$+1) for some $J$ then $|$VI$(J)|
\leq |$VI$(J+1)|.$

Complex eigenvalues will occur in conjugate pairs. Such pairs will be stored
in adjacent locations with the eigenvalue having positive imaginary part
preceding its conjugate partner.

\item[IFLAG()]  \ [out, scratch] The N-array IFLAG() will be used as INTEGER
working space. In addition, the first location, IFLAG(1), will be used to
pass information back to the user as follows:

\begin{itemize}
\item[= 1]  If successful and all eigenvalues are real.

\item[= 2]  If successful and some eigenvalues are complex.
\end{itemize}

See Section E for use of IFLAG(1) in error conditions.
\end{description}

\subsubsection{Modifications for Double Precision}

Change SEVUN to DEVUN, and the REAL type statement to DOUBLE PRECISION.

\subsection{Examples and Remarks}

{\bf Example~1:} Define
\begin{equation*}
A=\left[
\begin{array}{rrr}
-954 & -464 & -2088 \\
792 & 387 & 1728 \\
264 & 128 & 579
\end{array}
\right] .
\end{equation*}
The eigenvalues of $A$ are $\lambda _1 = 3$, $\lambda _2 = 3$, $\lambda _3
= 6$.  This example was constructed by using the matrix called $A^{-1}$ in
Example~3.1, page~29, of \cite{Gregory:1969:ACM} as the matrix of
eigenvectors.  It illustrates the case of a double eigenvalue having a
full set of eigenvectors.

{\bf Example~2:} Define
\begin{equation*}
B=\left[
\begin{array}{ccc}
4 & 1 & 1 \\
2 & 4 & 1 \\
0 & 1 & 4
\end{array}
\right] .
\end{equation*}
This is Example~5.2 of page~82 of~\cite{Gregory:1969:ACM}.  The
eigenvalues are $\lambda _1 = 3 $, $\lambda _2 = 3$, and $\lambda _3 = 6$.
The matrix is defective in that there is only a one-dimensional space of
eigenvectors associated with the eigenvalue~3.

{\bf Example 3:} Define
\begin{equation*}
C=\left[
\begin{array}{rrr}
8 & -1 & -5 \\
-4 & 4 & -2 \\
18 & -5 & -7
\end{array}
\right] .
\end{equation*}
This is Example~5.4 of page~84 of \cite{Gregory:1969:ACM}.  The
eigenvalues are $\lambda _1 = 1 $, $\lambda _2 = 2 + 4i$, and $\lambda _3
= 2 - 4i.$

The demonstration program DRSEVUN below applies SEVUN to compute eigenvalues
for the above three matrices. Results are in the file ODSEVUN.

\subsection{Functional Description}

Given an N $\times $ N real unsymmetric matrix $A$ there exists an N $\times $ N
nonsingular matrix $C$ such that the matrix
\begin{equation*}
U=C^{-1}AC
\end{equation*}
is N $\times $ N upper triangular. The matrices $C$ and $U$ may be complex. The
diagonal elements of $U$ are called the eigenvalues of $A$. This set of N
numbers is uniquely determined by $A$ although $C$ and $U$ are not unique.
Note that $\lambda $ is an eigenvalue of $A$ if and only if $A - \lambda I$
is singular.

This subroutine SEVUN was developed using the subroutines BALANC, ELMHES,
and HQR from the EISPACK package of eigenvalue-eigenvector subroutines,
\cite{Smith:1974:MER}. The Fortran subroutines in EISPACK
are based directly on the earlier set of Algol procedures described in
\cite{Wilkinson:1971:HAC}.

Subroutine SEVUN first calls SEVBH which consists of the two EISPACK
subroutines, BALANC and ELMHES.

BALANC applies similarity permutations to isolate eigenvalues available by
inspection, if any. It then applies diagonal similarity scaling to balance
the size of the matrix elements%
\begin{equation*}
B=D^{-1}P^TAPD.
\end{equation*}
ELMHES reduces $B$ to upper Hessenberg form using stabilized elementary
transformations%
\begin{equation*}
H=G^{-1}BG.
\end{equation*}
The remainder of SEVUN consists of a minor modification of HQR
which applies the QR algorithm to $H$. This is an iterative process that
reduces $H$ to a real nearly-upper-triangular matrix $R$%
\begin{equation*}
R=Q^THQ.
\end{equation*}
The matrix $R$ has a mixture of single elements and $2 \times 2$ blocks on its
diagonal and is otherwise upper triangular.

The eigenvalues of $A$ are the single diagonal elements of $R$ along with the
eigenvalues of the $2 \times 2$ blocks on the diagonal of $R$. These latter
eigenvalues are computed by direct formulas.

SEVUN reorders the eigenvalues to achieve the ordering described
previously in Section~B.

\bibliography{math77}
\bibliographystyle{math77}

\subsection{Error Procedures and Restrictions}

If N $\leq 0$ or if there is convergence failure in the QR algorithm the
error processing subroutine ERMSG of Chapter 19.2 will be called with an
error level 0 to print an error message. Upon return, IFLAG(1) = 3~or~4 to
indicate N $\leq  0$ or convergence failure, respectively.

In these error conditions all computed eigenvalues should be regarded as
invalid.

\subsection{Supporting Information}

The source language is ANSI Fortran~77.

\begin{tabular}{@{\bf}l@{\hspace{5pt}}l}
\bf Entry & \hspace{.35in} {\bf Required Files}\vspace{2pt} \\
DEVUN & \parbox[t]{2.7in}{\hyphenpenalty10000 \raggedright
AMACH, DEVBH, DEVUN, ERFIN, ERMSG\rule[-5pt]{0pt}{8pt}}\\
SEVUN & \parbox[t]{2.7in}{\hyphenpenalty10000 \raggedright
AMACH, ERFIN, ERMSG, SEVBH, SEVUN}\\
\end{tabular}

The EISPACK package of Fortran subroutines was acquired at JPL from Argonne
National Laboratories where it was developed with financial support from the
AEC and the NSF. The subroutine SEVUN was written by F. T. Krogh, JPL,
October~1991.


\begcode

\medskip\
\lstset{language=[77]Fortran,showstringspaces=false}
\lstset{xleftmargin=.8in}

\centerline{\bf \large DRSEVUN}\vspace{10pt}
\lstinputlisting{\codeloc{sevun}}

\vspace{30pt}\centerline{\bf \large ODSEVUN}\vspace{10pt}
\lstset{language={}}
\lstinputlisting{\outputloc{sevun}}
\end{document}
