\documentclass[twoside]{MATH77}
\usepackage{multicol}
\usepackage[fleqn,reqno,centertags]{amsmath}
\begin{document}
\hyphenation{KORDER}
\begmath 11.6 Low-level Subprograms for Operations on Splines

\silentfootnote{$^\copyright$1997 Calif. Inst. of Technology, \thisyear \ Math \`a la Carte, Inc.}

\subsection{Purpose}

This chapter describes five subprograms for spline operations that are used
by the subprograms of the preceding chapter. It is expected that one would
only use these subprograms directly if one has needs more specialized than
are covered by the higher level subprograms of the preceding chapter.

Subroutine DSVALA evaluates at an argument X the values of the derivatives,
of orders~0 through NDERIV, of a spline function represented using the
B-spline basis. DSVALA must be given a difference table of the coefficients
of the spline function. Subroutine DSDIF is provided to compute this
difference table. Once the difference table has been computed and saved, use
of DSVALA is more economical than making NDERIV$+1$ calls to subprogram
DSVAL of the preceding chapter if NDERIV $> 0.$

Subroutine DSFIND does a lookup in a knot array to find a knot subinterval
of nonzero length containing a specified argument X, or the nearest such
subinterval if extrapolation is needed.

Using a knot sequence regarded as defining a B-spline basis function of
order KORDER, subroutine DSBASD computes the values at X of the KORDER
B-spline basis functions (or a derivative of these functions as specified by
IDERIV) that could be nonzero at X. Subprogram DSBASI computes the integral
from X1 to X2 of each of the NCOEF basis functions. The output of these
subprograms is needed in setting up the matrix for curve fitting or
interpolation involving values, derivatives, or integrals of the fitted
spline function.

\subsection{Usage}

\subsubsection{Usage of DSBASD for evaluation of basis functions or their
derivatives}

\paragraph{Program Prototype, Double Precision}
\begin{description}
\item[INTEGER]  \ {\bf KORDER, LEFT, IDERIV}

\item[DOUBLE PRECISION]  \ {\bf TKNOTS}($\geq ncoef+$\\ KORDER){\bf , X,
BDERIV}($\geq $ KORDER)
\end{description}
(See TKNOTS below for the definition of {\em ncoef}.)

Assign values to KORDER, LEFT, TKNOTS, X, and IDERIV.
\begin{center}\vspace{-10pt}
\fbox{\begin{tabular}{@{\bf }c}
CALL DSBASD(KORDER, LEFT, TKNOTS,\\
X, IDERIV, BDERIV)\\
\end{tabular}}
\end{center}
Computed quantities are returned in BDERIV().

\paragraph{Argument Definitions}
\begin{description}
\item[KORDER]  \ [in] KORDER is both the order of the spline basis functions
and the number of basis functions whose derivatives are to be evaluated.

\item[LEFT]  \ [in] Identifies an interval of nonzero length [TKNOTS(LEFT),
TKNOTS(LEFT$+$1)] which is the reference interval for the function
evaluation. DSBASD will evaluate the IDERIV$^{th}$ derivative of the KORDER
basis functions that could be nonzero on this interval. Require KORDER $\leq
$ LEFT $\leq ncoef.$ Except when extrapolation is needed, LEFT should
satisfy TKNOTS(LEFT) $\leq $ X $<$ TKNOTS(LEFT+1). We recommend that the
subroutine DSFIND be used to determine LEFT.

\item[TKNOTS()]  \ [in] The knot sequence [$t_i$: $i=1$, ..., {\em ncoef} +
KORDER], where {\em ncoef} denotes the total number of B-spline basis functions
associated with this knot sequence. The proper interpolation interval, $[a,b]
$, associated with this knot sequence is given by $a=$ TKNOTS(KORDER) and $b=
$ TKNOTS({\em ncoef}+1). Require $t_i\leq t_{i+1}$ for $i=1$, ..., {\em ncoef} +
KORDER $-$ 1; $t_i<t_{i+KORDER}$ for $i=1$, ..., {\em ncoef}; $%
t_{KORDER+1}>t_{KORDER}$; $t_{ncoef}<t_{ncoef+1}$. The knots strictly
between a and $b$ are internal knots. They specify abscissae at which one
polynomial piece ends and the next begins. Successive internal knots may
have the same value. An abscissa appearing with multiplicity $\mu $ means the
order of continuity of the spline at this abscissa will be at least $\text{%
KORDER}-\mu -1$. The knots indexed ahead of $t_{KORDER}$ can all be equal to
$a$, and those indexed after $t_{ncoef+1}$ can all be equal to $b.$

\item[X]  \ [in] Argument at which the IDERIV$^{th}$ derivative of basis
functions are to be evaluated.

\item[IDERIV]  \ [in] Order of derivative to be computed. IDERIV $=0$
specifies function values. Require IDERIV $\geq 0$. Values of derivatives of
order\ $\geq $ KORDER will be zero.

\item[BDERIV()]  \ [out] On return the values at X of the IDERIV$^{th}$
derivative of the basis functions indexed from $\text{LEFT}+1-\text{KORDER}$
through LEFT will be stored in BDERIV($i$), $i=1$, ..., KORDER.
\end{description}
\subsubsection{Usage of DSBASI for evaluation of an integral of basis
functions}

\paragraph{Program Prototype, Double Precision}
\begin{description}
\item[INTEGER]  \ {\bf KORDER, NCOEF, J1, J2}

\item[DOUBLE PRECISION]  \ {\bf TKNOTS}($\geq $ NCOEF +\\ KORDER){\bf , X1,
X2, BASI}($\geq $ NCOEF)
\end{description}
Assign values to KORDER, NCOEF, TKNOTS(), X1, X2, J1, and J2.
\begin{center}
\fbox{\begin{tabular}{@{\bf }c}
CALL DSBASI ( KORDER, NCOEF,\\
TKNOTS, X1, X2, J1, J2, BASI)\\
\end{tabular}}
\end{center}
Computed results are returned in J1, J2, and BASI().

\paragraph{Argument Definitions}
\begin{description}
\item[KORDER]  \ [in] The order of the spline basis functions.

\item[NCOEF]  \ [in] The total number of B-spline basis functions associated
with this knot sequence. Also the number of values to be returned in BASI().

\item[TKNOTS()]  \ [in] As for DSBASD above, with {\em ncoef} replaced by
NCOEF.

\item[X1, X2]  \ [in] Integration is to be done from X1 to X2. Permit X1 $<$
X2 or X1 $\geq $ X2. Generally X1 and X2 should each lie in $[a,b]$, however
extrapolation will be used to return values when this is not the case.

\item[J1, J2]  \ [inout] On entry J1 and J2 must contain integer values. If
J1 is in [1, N] it will be used to start the lookup for X1. Otherwise the
search will start with~1. Similarly for J2.

On return J1 and J2 indicate the portion of the array BASI() that might be
nonzero on return. BASI($i$) might be nonzero if J1 $\leq i\leq $ J2, and BASI$%
(i)=0$ if $i<$ J1 or $i>$ J2.

\item[BASI()]  \ [out] On return, BASI($i$) will contain the value of the
integral of the $i^{th}$ basis function over the range from X1 to X2, for $i=1$,
..., NCOEF. J1 and J2 above indicate which elements might be nonzero.
\end{description}
\subsubsection{Usage of DSDIF to compute the difference table needed by
DSVALA}

\paragraph{Program Prototype, Double Precision}
\begin{description}
\item[INTEGER]  \ {\bf KORDER, NCOEF, NDERIV}

\item[DOUBLE PRECISION]  \ {\bf TKNOTS}($\geq $ NCOEF + \\ KORDER){\bf ,
BCOEF}($\geq $ NCOEF){\bf , \\ BDIF}($\geq $ NCOEF $\times $ (NDERIV+1))
\end{description}
Assign values to KORDER, NCOEF, TKNOTS(), BCOEF(), and NDERIV.
\begin{center}
\fbox{\begin{tabular}{@{\bf }c}
CALL DSDIF (KORDER, NCOEF,\\
TKNOTS, BCOEF, NDERIV, BDIF)\\
\end{tabular}}
\end{center}
Computed results are returned in BDIF().

\paragraph{Argument Definitions}
\begin{description}
\item[KORDER]  \ [in] The order of the spline basis functions.

\item[NCOEF]  \ [in] The total number of B-spline basis functions associated
with this knot sequence.

\item[TKNOTS()]  \ [in] Same specifications as for DSBASI above.

\item[BCOEF()]  \ [in] Array of NCOEF coefficients representing a spline
function relative to a B-spline basis.

\item[NDERIV]  \ [in] Highest order difference to be computed. Since the
difference table BDIF() is intended for use by DSVALA this should correspond
to the largest order derivative one intends to compute using DSVALA.

\item[BDIF()]  \ [out] Will contain a copy of BCOEF() plus differences
through order NDERIV of this array of coefficients. Intended for use by
DSVALA.
\end{description}
\subsubsection{Usage of DSFIND for lookup in a knot sequence}

\paragraph{Program Prototype, Double Precision}
\begin{description}
\item[INTEGER]  \ {\bf IX1, IX2, LEFT, MODE}

\item[DOUBLE PRECISION]  \ {\bf XT}(IX2+1){\bf , X}
\end{description}
Assign values to XT(), IX1, IX2, LEFT, and X.
\begin{center}
\fbox{\begin{tabular}{@{\bf }c}
CALL DSFIND(XT, IX1, IX2, X,\\
LEFT, MODE)\\
\end{tabular}}
\end{center}
Results are returned in LEFT and MODE.

\paragraph{Argument Definitions}
\begin{description}
\item[XT(), IX1, IX2]  \ [in] XT() is the array in which the lookup will be
done. DSFIND will only look at elements from XT(IX1) through XT(IX2).
Require IX1 $<$ IX2, XT(IX1) $<$ XT(IX1+1), XT(IX2$-$1) $<$ XT(IX2), and
XT($i$) $\leq $ XT($i+$1) for $i=$ IX1+1, ..., IX2 $-$~2.

If the lookup is in a knot array of length $korder + ncoef$ associated with a
B-spline basis, one would generally set IX1 $=korder$ and IX2 $=ncoef+1.$ If
the lookup is in a knot array of length $npc+1$ associated with a power
basis, one would generally set IX1 $=1$ and IX2 $=npc+1.$

\item[X]  \ [in] Value to be looked up in XT().

\item[LEFT]  \ [inout] On entry LEFT must contain an integer value. If this
value is in [IX1,~IX2~$-$~1] the lookup will start with this value,
otherwise the lookup starts with IX1 or IX2~$-$~1.

On return LEFT is the index of the left end of the reference interval $%
\langle $ XT(LEFT), XT(LEFT+1) $\rangle $ for X. This will always be an
interval of nonzero length. If X satisfies XT(IX1) $\leq $ X $<$ XT(IX2) then
LEFT will satisfy XT(LEFT) $\leq $ X $<$ XT(LEFT+1). Otherwise, if X $<$
XT(IX1), LEFT $=$ IX1; or if X $\geq $ XT(IX2), LEFT$=$ IX2 $-$ 1. The
polynomial segment defined over this reference interval is intended to be
used for function evaluation at X.

\item[MODE]  \ [out] Indicator of the position of X relative to [XT(IX1),
XT(IX2)]. Set to $-$1 if X is to the left of this interval, to~0 if X is in
this closed interval, and to +1 if X is to the right of this interval.
\end{description}
\subsubsection{Usage of DSVALA for evaluating a sequence of derivatives}

\paragraph{Program Prototype, Double Precision}
\begin{description}
\item[INTEGER]  \ {\bf KORDER, NCOEF, NDERIV}

\item[DOUBLE PRECISION]  \ {\bf TKNOTS}($\geq $ NCOEF +\\ KORDER){\bf , BDIF}%
($\geq $ NCOEF $\times $ (NDERIV+1)){\bf ,\\ X, SVALUE}($\geq $ NDERIV + 1)
\end{description}
Assign values to KORDER, NCOEF, TKNOTS(), NDERIV, BDIF(), and X.
\begin{center}
\fbox{\begin{tabular}{@{\bf }c}
CALL DSVALA(KORDER, NCOEF,\\
TKNOTS, NDERIV, BDIF, X, SVALUE)\\
\end{tabular}}
\end{center}
Computed results are returned in SVALUE().

\paragraph{Argument Definitions}
\begin{description}
\item[KORDER]  \ [in] The order of the spline basis functions.

\item[NCOEF]  \ [in] The total number of B-spline basis functions associated
with this knot sequence.

\item[TKNOTS()]  \ [in] Same specifications as for DSBASI above.

\item[NDERIV]  \ [in] Highest order derivative to be evaluated. Values of
derivatives of order $\geq $ KORDER will be zero.

\item[BDIF()]  \ [in] A difference table of B-spline coefficients computed
by DSDIF.

\item[X]  \ [in] Argument at which values returned in SVALUE() are to be
computed.

\item[SVALUE()]  \ [out] On return, SVALUE($i$+1) contains the value at X of
the $i^{th}$ derivative of the spline function $f$ for $i=0$, ..., NDERIV.
The spline function $f$ is defined by the parameters KORDER, NCOEF,
TKNOTS(), and the coefficients whose difference table is in BDIF().
\end{description}
\subsubsection{Modifications for Single Precision}

For single precision usage change the DOUBLE PRECISION statements to REAL
and change the initial ``D'' in the subprogram names to ``S''.

\subsection{Examples and Remarks}

The program DRDSBASD and output listing ODDSBASD demonstrate the use of the
subprograms of this chapter.

\subsection{Functional Description}

The subprograms of this chapter are described in Section~D of Chapter~11.5.

\bibliography{math77}
\bibliographystyle{math77}

\subsection{Error Procedures and Restrictions}

DSBASD, DSBASI, and DSVALA each contain an internal dimensioning parameter
$kmax = 20$. It is an error if KORDER $> kmax$ in DSBASD, DSBASI, or DSVALA.
The condition IDERIV\ $< 0$ is an error in DSBASD.

In DSFIND, if the search reaches either of the intervals [XT(IX1),
XT(IX1+1)] or [XT(IX2$-$1), XT(IX2)] and the interval is found to have
nonpositive length, the error is reported.

Errors are reported to the library error message processing subroutines
of Chapter~19.2 with a severity level of~2 that will, by default, cause
execution of the program to stop.

Abscissae and weights for 2-point, 6-point, and 10-point Gaussian quadrature
are stored to 40~decimal digits in DSBASI. With infinite precision abscissae
and weights, these formulae would be exact for splines of KORDER up to~20.

\subsection{Supporting Information}

The source language is ANSI Fortran~77.

These subprograms, except DSBASI, are modifications by Lawson of codes
developed by C.\ de Boor, \cite{deBoor:1978:APG}.  Subprogram DSBASI is
based on code due to D.\ E.\ Amos, \cite{Amos:1979:xxx}.

\newpage

\begin{tabular}{@{\bf}l@{\hspace{5pt}}l}
\bf Entry & \hspace{.35in} {\bf Required Files}\vspace{2pt} \\
DSBASD & \parbox[t]{2.7in}{\hyphenpenalty10000 \raggedright
DSBASD, ERFIN, ERMSG, IERM1, IERV1\rule[-5pt]{0pt}{8pt}}\\
DSBASI & \parbox[t]{2.7in}{\hyphenpenalty10000 \raggedright
DERV1, DSBASD, DSBASI, DSFIND, ERFIN, ERMSG, IERM1, IERV1\rule[-5pt]{0pt}{8pt}}\\
DSDIF & \parbox[t]{2.7in}{\hyphenpenalty10000 \raggedright DSDIF\rule[-5pt]{0pt}{8pt}}\\
DSFIND & \parbox[t]{2.7in}{\hyphenpenalty10000 \raggedright
DERV1, DSFIND, ERFIN, ERMSG, IERM1, IERF1\rule[-5pt]{0pt}{8pt}}\\
DSVALA & \parbox[t]{2.7in}{\hyphenpenalty10000 \raggedright
DERV1, DSFIND, ERFIN, ERMSG, IERM1, IERF1}\\
\end{tabular}

\begin{tabular}{@{\bf}l@{\hspace{5pt}}l}
\bf Entry & \hspace{.35in} {\bf Required Files}\vspace{2pt} \\
SSBASD & \parbox[t]{2.7in}{\hyphenpenalty10000 \raggedright
ERFIN, ERMSG, IERM1, IERV1, SSBASD\rule[-5pt]{0pt}{8pt}}\\
SSBASI & \parbox[t]{2.7in}{\hyphenpenalty10000 \raggedright
ERFIN, ERMSG, IERM1, IERV1, SERV1, SSBASD, SSBASI, SSFIND\rule[-5pt]{0pt}{8pt}}\\
SSDIF & \parbox[t]{2.7in}{\hyphenpenalty10000 \raggedright SSDIF\rule[-5pt]{0pt}{8pt}}\\
SSFIND & \parbox[t]{2.7in}{\hyphenpenalty10000 \raggedright
ERFIN, ERMSG, IERM1, IERF1, SERV1, SSFIND\rule[-5pt]{0pt}{8pt}}\\
SSVALA & \parbox[t]{2.7in}{\hyphenpenalty10000 \raggedright
ERFIN, ERMSG, IERM1, IERF1, SERV1, SSFIND}\\
\end{tabular}

\begcode
\bigskip
\lstset{language=[77]Fortran,showstringspaces=false}
\lstset{xleftmargin=.8in}

\centerline{\bf \large DRDSBASD}\vspace{10pt}
\lstinputlisting{\codeloc{dsbasd}}

\vspace{30pt}\centerline{\bf \large ODDSBASD}\vspace{10pt}
\lstset{language={}}
\lstinputlisting{\outputloc{dsbasd}}
\end{document}
