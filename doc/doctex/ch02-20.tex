\documentclass[twoside]{MATH77}
\usepackage{multicol}
\usepackage[fleqn,reqno,centertags]{amsmath}
\begin{document}
\begmath 2.20  Binomial Coefficients

\silentfootnote{$^\copyright$1997 Calif. Inst. of Technology,
  \thisyear \ Math \`a la Carte, Inc.}

\subsection{Purpose}

Compute the binomial coefficient $\displaystyle \binom{n}{k} =
\frac{n!}{(n-k)!\, k!}$ which gives the number of ways one can select
$k$ items from a set containing $n$ items.  Binomial coefficients also
appear in a number of other contexts.

\subsection{Usage}

\subsubsection{Program Prototype, Single Precision}
\begin{description}
\item[REAL] \ {\bf SBINOM, ANS}
\item[INTEGER] \ {\bf N, K}
\end{description}
Assign values to N and K.
$
\fbox{\bf ANS = SBINOM(N, K)}
$

\subsubsection{Argument Definitions}

\begin{description}
\item[N, K] \ [in] The input integers $n$ and $k$ as described in
"Purpose" above.  One must have $0 \leq \text{K} \leq \text{N}.$
\end{description}

\subsubsection{Modifications for Double Precision}

For double-precision usage change the name SBINOM to DBINOM, and
change the REAL declaration to DOUBLE PRECISION.

\subsection{Examples and Remarks}

The program DRSBINOM compares the true values with the results of
computing the binomial coefficients for two large values of N where SBINOM
uses a different algorithm internally.  It then finds the first case where
results using the Pascal triangle give different results from SBINOM (the
true value for the case printed is 67863915, which is slightly closer to
the value given by SBINOM).  If one is computing binomial coefficients for
strictly increasing values of N, one may want to use the Pascal triangle
code in DRSBINOM instead of SBINOM for computing the binomial
coefficients.

\subsection{Functional Description}

For N $\leq$ 150, the routine maintains and uses a table of saved
factors of factorials, $\phi_j$.  If the input N is larger than any
seen so far, this table is extended.  When there is no problem with
overflow, $\phi_j = j \phi_{j-1}$.  To deal with overflow a second
table $i_\nu$ is maintained.  Initially $\nu=0$.  When adding entry
$j$ to the table would cause an overflow, $\nu$ is incremented,
$i_\nu$ is set to $j-1$, and $\phi_j$ to $j$.

To compute the binomial coefficients, the code uses
\begin{equation*}
\binom{n}{k} = \frac{n!}{(n-k)!\,k!}, \quad
j!  = \Big (\prod_{\ell = 1}^{\nu-1} \phi(i_{\ell}) \Big ) \phi (j),
\end{equation*}
with $\nu $ such that $ i_{\nu-1} < j \leq i_\nu$.  When a factorial is so
large it is represented by more than one piece, the pieces are combined in
such a way as to avoid overflow.  Finally, for large values, the result is
refined by setting the result equal to $p \times \lfloor r / p + .5
\rfloor $, where $r$ is the originally computed result, $p$ is either one
prime or the product of two primes in the interval $\big (\max(n-k,\ k),\
n \big ]$ and $ \lfloor \cdot \rfloor$ denotes the integer part.

In the case of IEEE single precision arithmetic, only one piece of the
factorial is needed for $\text{N} \leq 33$, and for IEEE double precision
$150!$ does not overflow.  When only one piece is needed for the factorial
and it is already computed, the binomial coefficient is computed with one
multiply, one divide, and a few tests.


If N $>$ 150, the log gamma function from Chapter~2.3 is used.  In this
case
\begin{equation*}
\binom{n}{k} = e^{\log \Gamma (n+1) - \log \Gamma (n-k+1) -
\log \Gamma (k+1)}
\end{equation*}
It is very easy to shorten the code by replacing the call to the log
gamma function for large N with an error message reporting too large an N.
It is only slightly more difficult to replace the code for keeping pieces
of the factorial with code that calls log gamma whenever N!\ would
overflow.  This would lose some precision as illustrated by the difference
in the errors for N = 150 and N = 151 in the results.


\subsection{Error Procedures and Restrictions}

An error exit is taken if the condition $0 \leq \text{K} \leq \text{N}$
is not satisfied, or if the result would overflow.  Errors are
processed using the routines in Chapter 19.2.  If one references one
of these routines to change the error action, then instead of stopping on
an error, the routine will return with a result of $-1$.

\subsection{Supporting Information}

The source language is ANSI Fortran~77.

\begin{tabular}{@{\bf}l@{\hspace{5pt}}l}
\bf Entry & \hspace{.35in} {\bf Required Files}\vspace{2pt} \\
DBINOM & \parbox[t]{2.7in}{\hyphenpenalty10000 \raggedright
AMACH, DBINOM, DERM1, DERV1, DLGAMA, DGAMMA, ERFIN, ERMSG, IERM1,
IERV1\rule[-5pt]{0pt}{8pt}}\\
SBINOM & \parbox[t]{2.7in}{\hyphenpenalty10000 \raggedright
AMACH, ERFIN, ERMSG, IERM1, IERV1, SBINOM, SERM1, SERV1, SLGAMA,
SGAMMA\rule[-5pt]{0pt}{8pt}}\\ \end{tabular}

Algorithm and code due to F. T. Krogh, JPL, December~1995.


\begcodenp

\lstset{language=[77]Fortran,showstringspaces=false}
\lstset{xleftmargin=.8in}

\centerline{\bf \large DRSBINOM}\vspace{10pt}
\lstinputlisting{\codeloc{sbinom}}

\vspace{20pt}\centerline{\bf \large ODSBINOM}\vspace{0pt}
\lstset{language={}}
\lstinputlisting{\outputloc{sbinom}}
\end{document}
