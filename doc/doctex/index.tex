\documentclass[twoside]{MATH77}
\usepackage{supertabular, ulem}
\begin{document}
\newcommand{\s}{\hspace{1.2pt}}
\normalem
\begindex
Subscripts are used to indicate the type of arguments as follows:
logical$_L$, integer$_i$, real$_r$, double precision$_d$, complex$_z$,
character$_c$, and subroutine or unknown external$_x$.  The text of
arguments indicates how it is used: {\bf defined} (value may also be
referenced), {\em referenced} (no value assigned), possibly defined or
referenced uses the normal font, {\scriptsize not used}, and
\uline{external name}.  Arrays are given in upper case.  The decorations
for the arguments were obtained automatically using the software described
in Chapter~19.7, while the text for the arguments was obtained using a
program that examines the \LaTeX \ files.

\tablehead{\multicolumn{2}{l}{\bf CHAPTER} &
\multicolumn{2}{l}{\bf \quad CALL Statement}\rule[-8pt]{0pt}{10pt}\\}
\begin{supertabular}{@{\qquad \quad}r@{\quad }l@{\ \ }l@{\ }p{4.9in}}
\shrinkheight{.175in}
\input{gen_index}
\end{supertabular}
\end{document}
