\documentclass[twoside]{MATH77}
\usepackage{multicol}
\usepackage[fleqn,reqno,centertags]{amsmath}
\usepackage{euscript}
\begin{document}
\hyphenation{DZVALS}
\begmath 12.4 C$^0$ and C$^1$ Surface Interpolation to Scattered Data

\silentfootnote{$^\copyright$1997 Calif. Inst. of Technology, \thisyear \ Math \`a la Carte, Inc.}

\subsection{Purpose}
Given a set of triples $(x_i,\ y_i,\ z_i)$,
$i$ = 1, \ldots, NP, these routines implicitly construct a conveniently
computable function $f(x,y)$ satisfying the conditions
\begin{equation*}
z_i = f(x_i, y_i), \ \ i=1, \ldots , NP,
\end{equation*}
and support evaluation of $f(x,y)$ and its first partial derivatives for
user-provided values of $(x,y)$.  The data $(x_i,\ y_i)$ are not assumed
to lie in any special pattern, but it is assumed that $(x_i,\ y_i) =
(x_j,\ y_j)$ only if $i = j$.

\subsection{Usage}

To apply this method the user must first call STGGRD to build a Delaunay
triangular grid having the given $(x_i,\ y_i)$ points as nodal points.
For convenience from here on we shall let {$\EuScript S$} denote the given
set of $(x_i,\ y_i)$ points.  The user must specify that $f$ is to have
either C$^0$ or C$^1$ continuity.  For C$^1$ continuity the interpolation
method needs values of first partial derivatives of the desired surface at
the nodal points.  Most commonly the user will not have these values, in
which case the subroutine STGPD can be called to compute estimated partial
derivatives.

Once the grid and partial derivatives (for the C$^1$ method) have been
established, one can evaluate the interpolation function $f$ at any
$(x,y)$ point by calling STGFI.  This evaluation operation can be repeated
for any number of points.  Subroutine STGFI uses STGFND to determine which
triangle of the grid contains the point $(x,y)$, and STGC0 or STGC1 to
evaluate $f(x,y)$ in that triangle.  If the point is outside the
triangular grid STGFI calls STGEXT to compute extrapolated values.

We do not describe the separate usage of STGFND, STGEXT, STGC0, and STGC1,
but a user could learn this from the comments in these subroutines and the
code in STGFI, if desired for specialized applications.

In some applications one wishes to produce a table of values of the
interpolated surface at points of a regular rectangular grid.  The
subroutine STGREC can be called to produce such a table.  STGREC makes
a sequence of calls to STGFI.  The user must call STGGRD and optionally
STGPD before calling STGREC.

The subroutine STGPGR is provided to print the data structures that
define the triangular grid.  This routine may also be useful as a
model if one wishes to interface to a graphics system to draw the
triangular grid.

The following sections address usage for the following:

\begin{tabular}{lp{2.2in}}
B.1. STGGRD & Construct a Delaunay triangular grid\\
B.2. STGPD & Estimate partial derivatives at grid nodes\\
B.3. STGFI & Lookup and interpolate in a triangular grid\\
B.4. STGREC & Construct a rectangular grid of function values by
interpolation in a triangular grid\\
B.5. STGPGR & Print the data structures defining the triangular grid
\end{tabular}

\subsubsection{Construct a Delaunay Triangular Grid}

\paragraph{Program Prototype, Single Precision}

\begin{description}
\item[INTEGER] NP, IP($\geq$NP), MT, TRIANG(MT), MB, B(4,~MB), NT, INFO(3)
\item[REAL] X($\geq$NP), Y($\geq$NP), W($\geq$NP)
\end{description}
Assign values to NP, MT, MB, X(1:NP), and Y(1:NP).
\begin{center}
\fbox{\begin{tabular}{@{\bf }c}
CALL STGGRD (X, Y, NP, IP, W,\\
\rule{.25in}{0pt}TRIANG, MT, B, MB, NT, INFO)
\end{tabular}}
\end{center}
Results are returned in TRIANG(), B(,), NT, and INFO(1:3)
\paragraph{Argument Definitions}

\begin{description}
\item[X(1:NP), Y(1:NP)] [in] The $(x,\ y)$
coordinates of the NP data points constituting the set {$\EuScript S$}.
\item[NP] [in] Number of data points.  Require NP $\geq $ 3.  The
upper limit on NP will depend on the storage available for the array
TRIANG().
\item[IP()] [scratch] Working space.
\item[W()] [scratch] Working space.
\item[TRIANG()] [out] Defines the triangular grid.
Each triangle is defined by six integer indices
identifying up to three neighboring triangles and three vertex
points.  The number of triangles will be
NT = $2\times (\text{NP}-1)-\text{NB}$
where NB is the number of points on the boundary of the convex hull
of {$\EuScript S$}.
\item[MT] [in] The declared dimension of TRIANG.  Must be $\geq
6\times[2\times(\text{NP}-1)-\text{NB}]$.  If NB is not known, $12
\times (\text{NP}-1)$ provides an upper bound for the required value of MT.
\item[B(,)] [out] On return this contains a two--way linked list
identifying the boundary points of the grid.
\item[MB] [in] The declared second dimension of B(,).  Must be larger
by one than the largest number of boundary points in any of the
subsets of {$\EuScript S$} that STGGRD considers.  Setting MB = NP + 1
would always
be large enough, however, a much smaller value for MB will usually
suffice.\newline
The expected number of boundary points in {$\EuScript S$} when
{$\EuScript S$} is a random set of
points uniformly distributed in a disk is $3.4 \times \text{NP}^{1/3}$
for NP $>$ 100.\newline
We suggest setting MB = NP + 1 for NP $\leq $ 15, and MB = an integer
exceeding $6 \times \text{NP}^{1/3}$ for NP $\geq $ 16.
\item[NT] [out] The number of triangles in the
triangular grid.  NT, NB, and NP are related by the equation
$\text{NT} = 2 \times (\text{NP} - 1) - \text{NB}.$
\item[INFO(1:3)] [out] Termination information.\vspace{-5pt}
\begin{description}
\item[INFO(1)] Status on termination.  The grid is complete if INFO(1) = 0
and incomplete or not optimal otherwise.
\begin{itemize}
\item[=0] Normal termination, the triangular grid is complete and optimized.
\item[=1] A complete triangular grid has been constructed, but the process
of optimizing the shapes of the triangular cells did not complete.
This should not happen.
\item[=2] Either all given points are colinear or else some
pair of the points are duplicates.
\item[=3] Some pair of the points are duplicates.
\item[=4] The dimension MB is not large enough.  Setting MB =
NP + 1 will suffice.
\item[=5] The dimension MT is not large enough.
Setting $\text{MT} = 12 \times (\text{NP}-1)$ will suffice.
\end{itemize}
\item[INFO(2)] = NB, the number of boundary points in the convex
hull of {$\EuScript S$}.
\item[INFO(3)] = The minimum value that would have sufficed for
the dimension MB on this execution of STGGRD.
\end{description}
\end{description}

\subsubsection{Estimate Partial Derivatives at Grid Nodes}

\paragraph{Program Prototype, Single Precision}

\begin{description}
\item[INTEGER]  NP, TRIANG($\geq 6 \times$NP), NT,\newline
IWORK($\geq $NP)
\item[REAL] X($\geq$NP), Y($\geq$NP), Z($\geq$NP), DZ(2, $\geq$NP)
\end{description}
If one does not have partial derivative values available, this
subroutine can be used to compute estimates of the partial derivatives
after using subroutine STGGRD to
construct a triangular grid.  The values of X(), Y(), and NP
should be the same as were input to STGGRD.  The values of TRIANG()
and NT should be those produced by STGGRD.  Values must be
assigned to Z(1:NP).
\begin{center}
\fbox{\begin{tabular}{@{\bf }c}
CALL STGPD (X, Y, Z, DZ, NP,\\
\rule{.25in}{0pt}TRIANG, NT, IWORK)
\end{tabular}}
\end{center}
Results are returned in DZ(1:2, 1:NP).
\paragraph{Argument Definitions}

\begin{description}
\item[X(1:NP), Y(1:NP), Z(1:NP)] [in] The user supplied data; Z(i) is
the function value associated with the point (X(i), Y(i)).
\item[DZ(1:2, 1:NP)] [out] Estimates of the first partial derivatives,
stored as:\newline
\hspace{.25in} DZ(1,i) = $\partial f / \partial x$ at (X(i), Y(i))
\newline
\hspace{.25in} DZ(2,i) = $\partial f / \partial y$ at (X(i), Y(i))
\item[NP] [in] Number of data points.
\item[TRIANG(), NT] [in] Defined as for STGGRD.
\item[IWORK(1:NP)] [scratch] Working space.
\end{description}

\subsubsection{Lookup and Interpolate in a Triangular Grid}

\paragraph{Program Prototype, Single Precision}

\begin{description}
\item[INTEGER]  TRIANG($\geq 6 \times$NT), NT, NCONT,\newline
MODE, MB, B(4,~MB)
\item[REAL] X($\geq$NP), Y($\geq$NP), Z($\geq$NP), DZ(2,$\geq$NP),\newline
Q(2), ZOUT, DZOUT(2), SAVWRK(28)
\item[LOGICAL] WANTDZ
\end{description}

This subroutine is intended to be used after a triangular grid has been
constructed using subroutine STGGRD.  The values of X(), Y(), and MB
should be the same as were input to STGGRD.  The values of TRIANG(), NT,
and B(), should be those produced by STGGRD.  The value of NP (used only
in the text here) should be the same as was used in the call to STGGRD.
Values must be assigned to Z(1:NP), NCONT, Q(1:2), WANTDZ, and SAVWRK(1).
If C$^1$ interpolation is being requested, (NCONT = 1), values must be
assigned to DZ(1:2, 1:NP), possibly by use of subroutine STGPD.

\begin{center}
\fbox{\begin{tabular}{@{\bf }c}
CALL STGFI (X, Y, Z, DZ, TRIANG, NT,\\
\rule{.25in}{0pt}
B, MB, NCONT, Q, ZOUT, WANTDZ,\\
\rule{.25in}{0pt}DZOUT, MODE, SAVWRK)
\end{tabular}}
\end{center}

Results are returned in ZOUT, DZOUT(1:2), and MODE, and the contents of
SAVWRK(1:28) will generally be changed.

\paragraph{Argument Definitions}

\begin{description}
\item[X(1:NP), Y(1:NP)] [in] The $(x,\ y)$ coordinates of the data points
constituting the set {$\EuScript S$}, and thus the nodes of the triangular
grid.
\item[Z(1:NP)] [in] Z(i) is the value at (X(i), Y(i)) of the data to
be interpolated.
\item[DZ(1:2, 1:NP)] [in] Values, or estimates of values, of first
partial derivatives at the nodal points.  These are needed only when
NCONT = 1.  Otherwise this array will not be referenced.
\newline
\hspace{.25in} DZ(1,i) = $\partial f / \partial x$ at (X(i), Y(i))
\newline
\hspace{.25in} DZ(2,i) = $\partial f / \partial y$ at (X(i), Y(i))
\item[TRIANG(), NT] [in] Defined as for STGGRD.
\item[B(), MB] [in] Defined as for STGGRD.
\item[NCONT] [in] Set NCONT = 0 for C$^0$ interpolation,
and = 1 for C$^1$ interpolation.
\item[Q(1:2)] [in] Lookup and interpolation will be done for the point
$q = (x, y) =$ (Q(1), Q(2)).
\item[ZOUT] [out] Interpolated value computed by this subroutine.
\item[WANTDZ] [in] Turns on or off the computation of DZOUT(1:2).
\item[DZOUT(1:2)] [out] If WANTDZ = .true., interpolated values of the
partial derivatives at the point $q$ will be computed and returned in
DZOUT(1:2).  No partial derivative interpolation is done if WANTDZ = .false.
\item[MODE] [out] Indicates status on return.
\begin{itemize}
\item[=0] Point $q$ is within the convex hull of the set {$\EuScript S$}, or possibly
outside this triangle by a small tolerance.  Interpolated results are
returned.
\item[=1] Point $q$ is outside the convex hull of {$\EuScript S$} by more
than the built-in small tolerance.  Extrapolated results are returned.
Accuracy degrades rapidly as the evaluation points move away from the
convex hull.
\item[=2] Error condition.  STGFND has tested the point $q$
against
NT triangles without resolving its status.  This would indicate an error
in the definition of the triangular grid or in the design or coding
of the subroutine.  No results are returned.
\end{itemize}
\item[SAVWRK(1:28)] [inout] Array used as work space and to save values
that, in some circumstances, can be reused to reduce the amount of
computation on subsequent calls.  The user must set SAVWRK(1) = 0.0 on the
first call to this subroutine to signal that SAVWRK() does not contain any
saved information at that time.  Generally SAVWRK(1) will be nonzero on
returns from STGFI.  If subsequent calls do not involve changes to X, Y,
Z, DZ, TRIANG, NT, B, MB, or NCONT, the user should not alter SAVWRK() on
the subsequent calls.  If the user changes any of these quantities, the
user should reset SAVWRK(1) = 0.0 to indicate that saved information
should be ignored.  See Section C for further discussion of SAVWRK().
\end{description}

\subsubsection{Construct a Rectangular Grid of Function Values by
Interpolation in a Triangular Grid}

\paragraph{Program Prototype, Single Precision}

\begin{description}
\item[LOGICAL] WANTPD
\item[INTEGER]  NP, TRIANG($\geq 6 \times$NT), NT, NX, NY,\newline
MX, MY, NCONT, MB, B(4,~MB)
\item[REAL] X($\geq$NP), Y($\geq$NP), Z($\geq$NP),
DZ(2, $\geq$NP),\newline
XYLIM(4), ZFILL, ZVALS(MX, MY),\newline
DZVALS(MX, MY, 2)
\end{description}
Assign values to all of the subroutine arguments except ZVALS() and
DZVALS().  STGGRD should be used to set values for TRIANG(), NT and B().
Subroutine STGPD can be used to set values for DZ(,).
\begin{center}
\fbox{\begin{tabular}{@{\bf }c}
CALL STGREC (X, Y, Z, DZ, NP,\\
\rule{.15in}{0pt}TRIANG, NT, B, NB, XYLIM,\\
\rule{.15in}{0pt}NX, NY, ZFILL, ZVALS, MX, MY,\\
\rule{.15in}{0pt}NCONT, WANTPD, DZVALS)
\end{tabular}}
\end{center}
Results are returned in ZVALS(,) and DZVALS(,,1:2)
\paragraph{Argument Definitions}

\begin{description}
\item[X(1:NP), Y(1:NP), Z(1:NP), DZ(1:2, 1:NP)] [in] Values of $x$, $y$,
$z$, $\partial z / \partial x$, and $\partial z / \partial y$ at the NP
points of the triangular grid.  Values are needed in DZ() only if NCONT =
1.  See subroutine STGPD for computation of DZ().  \item[NP] [in] Number
of data points.
\item[TRIANG(), NT] [in] Defined as for STGGRD.
\item[B(), MB] [in] Defined as for STGGRD.
\item[XYLIM(), NX, NY] [in] Specifies the desired rectangular
grid.  The $x$ grid values will run from XYLIM(1) to XYLIM(2) with $\text{NX}
- 2$ equally spaced intermediate values.  The $y$ grid values will run
from XYLIM(3) to XYLIM(4) with $\text{NY} - 2$ equally spaced intermediate
values.
\item[ZFILL] [in] Determines the action to be taken when a point of the
rectangular grid is outside the convex region covered by the triangular
grid.  If ZFILL is zero, extrapolated values will be returned.  Otherwise
the value, ZFILL, will be returned as a function value in ZVALS() and
optionally as a partial derivative value in DZVALS().
\item[ZVALS(,)] [out] An MX by MY array in which STGREC stores the NX
by NY set of values of the function $z = f(x, y)$ defined by
interpolation in the triangular grid.  Thus for $i$ = 1, \ldots, NX, $j$
= 1, \ldots, NY, STGREC computes
\newline
$x_i = \text{XYLIM}(1) + (i-1) \times $\newline
\rule{.75in}{0pt}$[\text{XYLIM(2)} - \text{XYLIM(1)}] /
(\text{NX} - 1)$\newline
$y_j = \text{XYLIM}(3) + (j-1) \times $\newline
\rule{.75in}{0pt}$[\text{XYLIM(4)} - \text{XYLIM(3)}] /
(\text{NY} - 1)$\newline
IF ($(x_i, y_j)$ is in the region covered by the triangular grid)
THEN \newline
\rule{.2in}{0pt}ZVAL($i,j$) = $f(x_i, y_j)$ (using
interpolation)\newline
ELSE\newline
\rule{.2in}{0pt}IF (ZFILL .eq. 0.0E0) THEN\newline
\rule{.4in}{0pt}$\text{ZVAL}(i,j) = f(x_i, y_j)$ (using
extrapolation).\newline
\rule{.2in}{0pt}ELSE\newline
\rule{.4in}{0pt}$\text{ZVAL}(i,j) =$ ZFILL\newline
\rule{.2in}{0pt}END IF\newline
END IF
\item[MX, MY] [in] Dimensions for the array ZVALS(,) and first two
dimensions for DZVALS(,,1:2).  Require MX $\geq $ NX and MY $\geq $ NY.
\item[NCONT] [in] Set NCONT = 0 for C$^0$ interpolation,
and = 1 for C$^1$ interpolation.
\item[WANTPD] [in] Set to .TRUE. if computation of DZVALS(,,1:2) is
desired in addition to ZVALS(,) and to .FALSE. if only ZVALS(,) is
desired.
\item[DZVALS(,,1:2)] [out] An MX by MY by 2 array in which the NX by NY
by 2 set of values of $\partial z / \partial x$ and
$\partial z / \partial y$
will be stored if WANTPD = .TRUE.  As in the computation of ZVALS(,),
the value ZFILL determines the action when an $(x_i, y_j)$ point is
outside the triangular grid.\newline
If WANTPD = .FALSE. the array DZVALS() will not be referenced and the
user can reduce all dimensions of DZVALS() to 1.

\end{description}

\subsubsection{Print the Data Structures defining a Triangular Grid}

\paragraph{Program Prototype, Single Precision}

\begin{description}
\item[INTEGER] NP, TRIANG($\geq 6 \times$NT), B(4,~MB), NB, NT
\item[REAL] X($\geq$NP), Y($\geq$NP)
\end{description}
\begin{center}
\fbox{\begin{tabular}{@{\bf }c}
CALL STGPRG (X, Y, NP, TRIANG,\\
\rule{.25in}{0pt}B, NB, NT)
\end{tabular}}
\end{center}

\paragraph{Argument Definitions}

All arguments are defined as for STGGRD, and should have the values
they had on a return from subroutine STGGRD (NB = INFO(2)).

\subsubsection{Modifications for Double Precision}

Change the first letter of every subroutine name from ``S'' to ``D'', and
change all REAL type statements to DOUBLE PRECISION.

\subsection{Examples and Remarks}
{\bf Example.}
The program DRSTGFI1 contains a set of data:
$(x_i,\ y_i), i = 1, \ldots, 28$.  The program calls STGGRD
to generate a triangular grid having these points as
nodal points.  The program runs two cases of C$^1$ interpolation to data
over this grid.  The interpolation is done at a set of points along
a line cutting across the region covered by the grid.

For the first case, data are generated at the nodal points using a
quadratic function.  Mathematically, the partial derivative
estimation method in STGPD and the C$^1$ interpolation method
in STGC1 are exact when the given $z_i$ data are a quadratic
function of the $(x_i, y_i)$ data.
Thus we observe that the errors in
the interpolated values are of the order of magnitude of the
precision of the machine arithmetic, which was IEEE single
precision arithmetic for the output listed with this writeup.

For the second case, data are generated using an
exponential function.  The partial derivative
estimation method in STGPD and the C$^1$ interpolation method
in STGC1 will both introduce errors in processing this data,
with the result that the interpolated values
will deviate from
the exponential function at points away from the nodal points.

The first and last points at which interpolation is requested in each case
are outside the convex hull of the given $(x_i, y_i)$ data.  The program
used extrapolation for these points.

Note that this program can be altered to run ten cases rather than
just two by changing the comment status of the loop control lines
involving the loop indices ``ncont" and ``KF".  A version of this
program called DRSTGFI has this change made.

{\bf Choosing between C$^0$ and C$^1$ interpolation.}
Since these are interpolation, and not smoothing, methods, they are
most appropriate for data that has very little noise.
The C$^1$ method requires first partial derivative data, whereas the C$^0$
method does not.  If the first partial derivative data have been estimated
from the $z_i$ data by using STGPD, they will have errors that are
larger as the data deviate more from having a locally quadratic shape.
Estimates of first partial derivative values at points $(x_i, y_i)$
on the boundary of the convex hull of {$\EuScript S$} may
be particularly bad.  The interpolation surface defined by the
C$^1$ method can easily have unwanted sharp peaks and valleys due to
errors in the first partial derivative data, or simply due to the
possibility that the shape of the data cannot be well represented
by the type of piecewise cubic surface produced by this method.

The C$^0$ method is more stable than the C$^1$ method,
in the sense that it will not produce
maxima higher than the data or minima lower than the data, however,
being piecewise linear it will have a choppy appearance due to
corners along all the edges of the triangular grid.

{\bf Using multiple copies of SAVWRK().}

Let us say the settings of X(), Y(), Z(), DZ(,), NP, TRIANG(), NT, B(,),
and MB define a {\em grid and data}.  Applications may arise in which the
user wishes to alternate between interpolations using STGFI with different
{\em grid and data} sets.  In such cases one can simply reset SAVWRK(1) =
0.0 each time the {\em grid and data} is changed.  For more computational
efficiency, however, one could use a distinct SAVWRK() array for each
distinct {\em grid and data}, say SAVW1(), SAVW2(), $\ldots$ One could set
SAVW1(1) = 0, SAVW2(1) = 0, $\ldots$, only once initially.  Then on each
call to STGFI, use the SAVW$n$() array associated with the current {\em
grid and data}.  This permits some saving of computational time in lookups
and C$^1$ interpolation.

\subsection{Functional Description}

{\bf Delaunay Triangulation.}
Given a finite set, {$\EuScript S$}, of points in the plane there are
generally many ways one could connect pairs of these points to form
a triangular grid, i.e., a grid having triangular cells.  The
Delaunay triangulation, constructed by subroutine STGGRD, has a number
of favorable mathematical properties, specifically the max-min angle
property, the empty circumcircle property, and duality with
a proximity region tessellation.

The max-min angle property has both a local and a global implication.
Locally, in any pair of triangles that share a side and whose union
is a strictly convex quadrilateral, the size of the smallest of the
six interior angles of the two triangles would not be increased by
replacing these two triangles by two triangles formed by
partitioning the quadrilateral using its other diagonal.
Globally, the smallest interior angle occurring in the whole triangulation
could not be made larger by any other triangulation.

The empty circumcircle property means that for each triangle of the grid,
its circumcircle, i.e., the unique circle passing through its three
vertices, contains no points of the set {$\EuScript S$} in its interior.

The Delaunay triangulation of a set {$\EuScript S$} is unique, with the
exception that if any set of four or more points of {$\EuScript S$} lie
on a circle that contains no points of {$\EuScript S$} in its interior,
it is arbitrary how triangles are formed using these vertices.

A Delaunay triangulation is dual to a proximity tessellation
associated with the names Dirichlet, Voronoi, and Thiessen.
A Dirichlet tessellation associated with a set {$\EuScript S$} is
a partitioning of the whole plane into regions with the property that
the interior of region {\em i} contains all the points of the plane that
are closer to point $(x_i, y_i)$ of {$\EuScript S$} than to any other
point of {$\EuScript S$}.  These regions will be convex, polygonal, and
some will be of infinite extent.  Clearly a one-to-one correspondence
exists between regions of a Dirichlet tessellation and nodes of a
Delaunay triangulation.  It also happens that a correspondence exists
between triangles of a Delaunay triangulation and nodes of a
Dirichlet tessellation.  Furthermore  edges in a Dirichlet tessellation
are perpendicular bisectors of edges in a Delaunay triangulation.

{\bf Algorithms.} The triangulation algorithm implemented in STGGRD is
described in \cite{Lawson:SC1:1977}.  From both theoretical analysis and
timing tests it appears that the execution time for STGGRD is proportional
to NP$^{4/3}$.

The method of estimating partial derivatives implemented in STGPD is
described in \cite{Lawson:SC1:1977}.  To estimate the partial derivatives
at a point $(x_i, y_i)$ of {$\EuScript S$}, this method locates up to 16
points of {$\EuScript S$} that are near to $(x_i, y_i)$ in the sense of
the grid connectivity, and does a uniformly weighted least squares fit of
a 6-parameter quadratic polynomial to the grid data at these points.  A
modification of this method is used if NP $<$ 6.  The first partial
derivatives of this polynomial at $(x_i, y_i)$ are returned as the
estimated partial derivatives.

The lookup method used in STGFND is described in \cite{Lawson:SC1:1977}.
This method starts by testing the given point $q$ for containment in the
triangle whose index is INDTRI (or 1 if INDTRI is out of range.) If $q$ is
outside of this triangle relative to side $j$, the search moves to the
triangle adjacent to the current triangle across side $j$.  This process
is repeated until either a triangle is found that contains $q$, or $q$ is
found to be outside some triangle relative to an edge that is a boundary
edge of the convex hull of {$\EuScript S$}.

The tests for containment produce values that are also needed in
both the C$^0$ and C$^1$ interpolation methods.  To avoid recomputation,
these quantities are stored in an array by STGFND for subsequent
use in STGC0 or STGC1.

This method is quite efficient in general, and is particularly efficient
when a sequence of lookups is done at points that are near to each
other relative to the grid connectivity, and with each lookup being started
in the triangle at which the previous lookup terminated.

C$^0$ continuity implies continuity of values but allows the possibility
of discontinuity of first partial derivatives, whereas
C$^1$ continuity implies continuity of values and first partial
derivatives,  but allows the possibility
of discontinuity of second partial derivatives.

The C$^0$ interpolation method implemented in STGC0
defines a planar surface over each triangular
cell of the grid.  Each such planar surface matches the values at
the three vertices of its triangle.  This gives a piecewise linear
surface over the whole grid.  The surface is continuous over the
whole grid, but generally has jumps in its first derivative values
when crossing from one triangle to another.

For C$^1$ interpolation we use the Clough-Tocher method,
\cite{Clough:1965:FES}, used in finite element computations.  The
particular implementation in STGC1 is described in \cite{Lawson:1976:C1C}.
Additional properties of the induced surface are described in
\cite{Lawson:1976:IC1}.

For one triangular cell of a grid this method implicitly partitions
the triangle into three subtriangles, by inserting new edges from
the vertices to the centroid.  Over each of these subtriangles the
method implicitly defines a cubic polynomial having C$^1$ continuity
across the newly inserted internal edges.  Along each of the original
edges of a triangle the values and first partial derivatives depend
only on the values and first partial derivatives
given at the two vertices of that edge.
This assures that two triangles that share an edge will have the
same values and first partial derivatives along the common edge, and
thus the surface over the whole triangular grid will have C$^1$
continuity.

The C$^0$ method is exact for linear data both in interpolation and
extrapolation.  The C$^1$ method is exact for quadratic data in
interpolation and for linear data in extrapolation.

\bibliography{math77}
\bibliographystyle{math77}

\subsection{Error Procedures and Restrictions}
STGGRD returns status information in INFO(1).
STGFI returns status information in MODE.
The user should encode tests of these status flags.
IERV1 is used for the output of some error messages.
\subsection{Supporting Information}

The source language is ANSI Fortran~77.  A common block STGCOM is
used in the program units in the file STGPD.
A common block DTGCOM is
used in the program units in the file DTGPD.

\begin{tabular}{@{\bf}l@{\hspace{5pt}}l}
\bf Entry & \hspace{.35in} {\bf Required Files}\vspace{2pt} \\
STGGRD & \parbox[t]{2.7in}{\hyphenpenalty10000 \raggedright
SSORTP, STGGRD, STGSET\rule[-5pt]{0pt}{8pt}}\\
STGPD & \parbox[t]{2.7in}{\hyphenpenalty10000 \raggedright
SROT, SROTG, STGPD, STGSET\rule[-5pt]{0pt}{8pt}}\\
STGFI & \parbox[t]{2.7in}{\hyphenpenalty10000 \raggedright
ERFIN, ERMSG, IERM1, IERV1, STGC0, STGC1, STGEXT, STGFI,
STGFND, STGSET\rule[-5pt]{0pt}{8pt}}\\
STGREC & \parbox[t]{2.7in}{\hyphenpenalty10000 \raggedright
ERFIN, ERMSG, IERM1, IERV1, STGC0, STGC1, STGEXT, STGFI,
STGFND, STGREC, STGSET\rule[-5pt]{0pt}{8pt}}\\
STGPRG & \parbox[t]{2.7in}{\hyphenpenalty10000 \raggedright
STGPRG, STGSET\rule[-5pt]{0pt}{8pt}}\\
DTGGRD & \parbox[t]{2.7in}{\hyphenpenalty10000 \raggedright
DSORTP, DTGGRD, DTGSET\rule[-5pt]{0pt}{8pt}}\\
DTGPD & \parbox[t]{2.7in}{\hyphenpenalty10000 \raggedright
DROT, DROTG, DTGPD, DTGSET\rule[-5pt]{0pt}{8pt}}\\
DTGFI & \parbox[t]{2.7in}{\hyphenpenalty10000 \raggedright
DTGC0, DTGC1, DTGEXT, DTGFI, DTGFND, DTGSET, ERFIN,
ERMSG, IERM1, IERV1\rule[-5pt]{0pt}{8pt}}\\
DTGREC & \parbox[t]{2.7in}{\hyphenpenalty10000 \raggedright
DTGC0, DTGC1, DTGEXT, DTGFI, DTGFND, DTGREC, DTGSET,
ERFIN, ERMSG, IERM1, IERV1\rule[-5pt]{0pt}{8pt}}\\
DTGPRG & \parbox[t]{2.7in}{\hyphenpenalty10000 \raggedright
DTGPRG, DTGSET}\\
\end{tabular}

Algorithm developed and subroutine designed and written by
Charles L. Lawson, JPL, 1976.
Adapted to Fortran 77 and the MATH77 library, 1991 and January, 1996.

\begcodenp

\lstset{language=[77]Fortran,showstringspaces=false}
\lstset{xleftmargin=.8in}

\centerline{\bf \large DRSTGFI1}\vspace{10pt}
\lstinputlisting{\codeloc{stgfi1}}
\newpage

\vspace{30pt}\centerline{\bf \large ODSTGFI1}\vspace{10pt}
\lstset{language={}}
\lstinputlisting{\outputloc{stgfi1}}

\end{document}
