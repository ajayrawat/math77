\documentclass[twoside]{MATH77}
\usepackage[\graphtype]{mfpic}
\usepackage{multicol}
\usepackage[fleqn,reqno,centertags]{amsmath}
\begin{document}
\opengraphsfile{pl16-03}
\begmath 16.3 Plotting Using \TeX

\silentfootnote{$^\copyright$1997 Calif. Inst. of Technology, \thisyear \ Math \`a la Carte, Inc.}

\subsection{Purpose}

These subprograms produce plots for documents prepared using the \TeX\ or
\LaTeX\ languages.

\subsection{Usage}

To produce a plot using \TeX\ or \LaTeX, one constructs a program that uses
SPLOT.  SPLOT writes a file of \TeX\ or \LaTeX\ commands.  The file is
then incorporated into a \TeX\ or \LaTeX\ document, which is then processed
by a \TeX\ or \LaTeX\ processor.  Detailed instructions are given in
Section \ref{IntoTeX} below.

Sections \ref{PPSP} through \ref{DP} below describe:\\
\begin{tabular*}{3.20in}{@{}l@{~~~}l}
\quad \ref{PPSP}    & Program Prototype, Single Precision\dotfill \pageref{PPSP}\\
\quad \ref{Argdef}  & Argument Definitions\dotfill \pageref{Argdef}\\
\quad \ref{Numopt}  & Numerical Option Descriptions\dotfill \pageref{Numopt}\\
\quad \ref{Textopt} & Textual Option Descriptions\dotfill \pageref{Textopt}\\
\quad \ref{IntoTeX} & Incorporating the Output Into a \TeX\ or\rule{0.50in}{0pt}\\
 & \LaTeX\ Document\dotfill
 \pageref{IntoTeX}\\
\quad \ref{DP}      & Modifications for Double Precision\dotfill \pageref{DP}\\
\end{tabular*}

\subsubsection{Program Prototype, Single Precision\label{PPSP}}
\begin{description}
\item[INTEGER] \ {\bf NX}
\item[REAL]    \ {\bf XSIZE, YSIZE, X}($\geq$ NX), {\bf Y}($\geq$ NX),
                      {\bf OPT}(see below)
\item[CHARACTER] \ {\bf COPT}(see below)
\end{description}
Assign values to NX, XSIZE, YSIZE, X, Y, OPT and COPT, and use the following
subroutine reference.
\begin{center}
\fbox{\begin{tabular}{@{\bf }c}
CALL SPLOT (XSIZE, YSIZE, X, NX, Y,\\
OPT, COPT)\\
\end{tabular}}
\end{center}
At this point, \TeX\ or \LaTeX\ commands to produce the plot are in the file
{\tt splot.tex} (unless another file name is selected by options).

\subsubsection{Argument Definitions\label{Argdef}}
\begin{description}
\item[XSIZE] \ [in] Physical horizontal size of the drawing area for the plot.
  Default units are inches.
\item[YSIZE] \ [in] Physical vertical size of the drawing area for the plot.
  Default units are inches.  The nominal drawing area for the plot is a
  region of size XSIZE $\times$ YSIZE.  The origin of the physical
  coordinate system is at the lower left corner, and all border lines
  are on the boundary of this region. Physical coordinates outside this
  area may be specified.  The actual area of the image is increased by
  outward-pointing ``tick'' marks (see option \ref{tick} in Section
  \ref{Numopt}), ``tick'' labels (see options \ref{border} and \ref{tick}
  in Section \ref{Numopt}) or annotations placed outside the drawing area
  (see option \ref{text} in Section \ref{Numopt}).

  Many of the options refer to user coordinates that are specified in
  the same units as the points that are plotted.  Different user
  coordinates may be used in a plot, see option~\ref{dataset} in
  Section~\ref{Numopt}.
\item[X]     \ [in] Array of NX abscissas.
\item[NX]    \ [in] Number of abscissas in X and ordinates in Y.
\item[Y]     \ [in] Array of NX ordinates.
  See also option \ref{ncurv} in Section \ref{Numopt}.
\item[OPT]   \ [inout] OPT(1) is status output.  Nonzero values indicate
  a problem for which an error message is printed. Possibilities are
    listed in the comments of subroutine SPLOTE in the file {\tt splot.f}
    or {\tt splot.for}.  Unless one has taken special action by calling
    MESS, see Chapter~19.3, all except for 6 warning messages stop
    execution.  Values from 1--6 are for warnings, 10--19 are for problems
    in COPT, 20--32 are for problems in OPT, values $\geq 40$ are caused
    by problems that probably are due to bugs in this plotting software
    that should be fixed.\\
  Beginning in OPT(2), the user provides option specifications as described
  in Section \ref{Numopt}.
\item[COPT]  \ [in] Character options, and data used by options in OPT, as
  described in Section \ref{Textopt}.
\end{description}

For simplest usage, set OPT(2)~= 0 and COPT~= 'Q'.

\subsubsection{Numerical Option Descriptions\label{Numopt}}
Options are specified by providing a sequence of codes starting in OPT(2),
and by sequences of characters in COPT().

For options specified in OPT(), each code is a whole number, followed by
zero or more ``arguments.''

Arguments are numbers, which depending on the option, may be whole
numbers.  They may be data or indices into the array COPT() indicating the
location of a character datum.  Where we write (Argument X=n) below, n is
the default value for X if the option is not selected.

``Whole numbers'' in OPT() are determined by using the Fortran nint()
intrinsic function, that is, the nearest whole number to OPT(j) is used
if OPT(j) is expected to be a whole number.

If an option code is negative, say OPT(j) $<$ 0, then OPT(j) is not
processed, but the number of ``arguments'' used by option $|$OPT(j)$|$
are skipped.
\end{multicols}
\begin{multicols}{2}[
\begin{center}
\begin{tabular}{|c|p{2.55in}|c|p{2.35in}|}
\multicolumn{4}{c}{\large Summary of Options\rule[-10pt]{0pt}{10pt}}\\
\hline
Option \rule{0pt}{10pt}  &  & Scratch & \\
or Field & {\qquad \raisebox{1.3ex}[0pt][0pt]{Brief Description}} &
File &
\multicolumn{1}{c|}{\raisebox{1.3ex}[0pt][0pt]{Remarks, Other affected
options}}\\ \hline
0 \rule{0pt}{10pt} &   End of options. & & \\
A\ref{opt}/$10^0$  & Finish or not, set units
   &   & Subsequent data  \\
A\ref{opt}/$10^1$  & Interpolation
   & Y & Subsequent data \\
A\ref{opt}/$10^2$  & $\log_{10}$ on X, (polar -- not implemented)
   & Y & Subsequent data,
         \ref{text}--\ref{xtick},
         \ref{line}--\ref{ellipse}  \\
A\ref{opt}/$10^3$  & $\log_{10}$ on Y or polar info.
   & Y & Subsequent data,
         \ref{text}--\ref{xtick},
         \ref{line}--\ref{ellipse}  \\
A\ref{opt}/$10^4$  & Lines at X = 0, Y = 0 &   & \\
A\ref{opt}/$10^5$  & \TeX\ or \LaTeX\
   &   & Use final value \\
\ref{ncurv}   &   For multiple curves
   & Y & Required by some cases in option \ref{symbol}\\
\ref{pen}  & Solid, dashed, dotted curves, width, etc.
   & Y & Subsequent data, \ref{line}, \ref{pline} \\
\ref{border}  &   Border characteristics
   &   & Use final value for each border or axis \\
\ref{tick}, \ref{tick1} &   Length and spacing of border tick marks
   &   & Use final value for each border or axis \\
\ref{dataset} &  Change current data set
   & Y & data, \ref{Xminmax}, \ref{Yminmax}, \ref{border}, \ref{onesymb},
    \ref{text}, \ref{number}, and \ref{line} -- \ref{pellipse}\\
\ref{Xminmax}, \ref{Yminmax} &   Adjust user coordinates, clipping
   &   & Digits $10^{3:4}$ of S\ref{border} \\
\ref{symbol}  &   Plot symbols, error bars, vectors
   & Y & Subsequent data  \\
\ref{onesymb} &   Plot a single symbol, etc.
   & Y & Nothing  \\
\ref{arrow}   &   Draw arrow heads on curves
   & Y & Subsequent data, \ref{line}, \ref{pline}  \\
\ref{linwid}  & Set various line widths & & \\
A\ref{linwid}--D\ref{linwid}
   & Line widths for border, tick, etc. &   & Use final value\\
E\ref{linwid} & Line widths for rectangles and ellipses
   & Y & \ref{rect}, \ref{ellipse}, \ref{prect}, and \ref{pellipse} \\
\ref{text}    &   Text annotation
   & Y & Nothing  \\
\ref{number}  &   Put a number on the plot
   & Y & Nothing  \\
\ref{xtick}   &   Extra tick / annotation
   & Y & Nothing  \\
\ref{bad}     &   Bad data value
   &   & Use final value \\
\ref{nouse_1} &  Not used. & & \\
\ref{line}   &   Draw a line
   & Y & Nothing  \\
\ref{rect}    &   Draw a rectangle
   & Y & Nothing  \\
\ref{ellipse} &   Draw an ellipse
   & Y & Nothing  \\
\ref{pline}--\ref{pellipse} & As for \ref{line}--\ref{ellipse}, but in
physical coordinates & Y & Nothing  \\
\ref{fill}    &   Type of fill for closed regions
   & Y & Subsequent data, \ref{rect}, \ref{ellipse}, \ref{prect},
     \ref{pellipse} \\
\ref{nouse_2} &  Not used. & & \\
\ref{debug}   &   Debug output
   &   & Final value per call, then reset \\
\hline
\end{tabular}
\end{center}
]

The third and fourth columns of the table are described in Section
\ref{SOE}.

\vspace{5pt}{\bf Number\hspace{.4in}Description}\vspace{-5pt}
\renewcommand{\labelenumi}{\bf \theenumi}
\begin{enumerate}
\setcounter{enumi}{-1}
\item   No more options.  Option zero must be specified, and must be last
        in the list of options. If no other options are specified, the
        following defaults apply:
  \begin{itemize}
  \item Physical units are inches.
  \item One curve is plotted, using a solid line 0.5 points wide.
          Points are connected by a B\'ezier curve that is not closed.
  \item XMIN, XMAX, YMIN and YMAX are determined automatically.
  \item All four borders are drawn, with automatically determined and
          labeled ``tick'' marks on the bottom and left borders.
  \item The output is \LaTeX\ in file {\tt splot.tex}.
  \end{itemize}
\item\label{opt}  (Argument N\ref{opt} = 0) N\ref{opt} is a whole number
        in which the decimal digits specify options.  Digits of
        N\ref{opt} mean:
  \begin{itemize}
  \item[$10^0$]
   \begin{itemize}
   \item[0 =] Finish the curve, and the plot. Physical units are in
          inches.
   \item[1 =] Finish the curve, and the plot. Physical units are in
          millimeters.  1 in.\ = 25.4 mm.
   \item[2 =] Finish the curve, and the plot. Physical units are in
          points.  1 in.\ = 72.27pt.
   \item[3 =] Finish this {\tt mfpic} group, but allow the plot to be
          continued on a subsequent call.  Use this for multiple curves if
          you have capacity problems with Metafont.  This option requires
          that in your \TeX \ or \LaTeX \ document the file generated by
          SPLOT be included with a statement of the form
          $\backslash$hbox\{$\backslash$input ... \}, where ... is the
          file generated by SPLOT and there is a space preceding the
          closing ``\}''. In \LaTeX \ one can use mbox in place of hbox.
          (Units are needed only when plot finishes.)
   \item[4 =] Finish the curve but allow the plot to be continued on a
          subsequent call.
   \item[5 =] More data for the current curve may be provided on a
          subsequent call.
   \end{itemize}
        See special considerations in Section C.
  \item[$10^1$]
    \begin{itemize}
    \item[0 =] Interpolate between points using a B\'ezier curve; do not
          close the curve.
    \item[1 =] Interpolate between points using a B\'ezier curve; close
          the curve with a B\'ezier curve.
    \item[2 =] Interpolate between points using a B\'ezier curve; close
          the curve with a straight line.
    \item[3 =] Connect points using straight lines; do not close the
          curve.
    \item[4 =] Connect points using straight lines; close the curve with a
          straight line.
    \item[5 =] Do not connect the points.  This is frequently used with
         option \ref{symbol}.
   \end{itemize}
  \item[$10^2$]
    Specifies how X is used to determine the abscissa, $\xi$ of the point
    plotted, and how a border or axis with labels depending on X is
    labeled.  When $10^\xi$ is output for a label, if the
    major ticks are 1 unit apart in $\xi$, then minor ticks
    (if any) will be spaced logarithmically.
    \newline

    \hspace{.25in}\begin{tabular}{ccc}
    Value & Plot & Labels\\
    0 & $\xi = \text{X}$ & $\xi$\\
    1 & $\xi = \text{X}$ & $10^\xi$\\
    2 & $\xi = \log_{10}\text{X}$ & $\xi$ \\
    3 & $\xi = \log_{10}\text{X}$ & $10^\xi = \text{X}$ \\
    4 & \multicolumn{2}{l}{Polar coordinates, $r = \text{X}\geq 0$}\\
    \end{tabular}\newline

     Options for logarithmic or polar conversion apply to curves, and to
     options \ref{text}, \ref{number}, and \ref{xtick}.  Those for
     logarithmic conversions apply to options \ref{line}--\ref{ellipse}.
     {\bf Polar coordinates are not implemented in the current code.}
     \newline
      In the case of polar coordinates, the points plotted are
    $(\text{X} \cos \text{Y},\, \text{X} \sin \text{Y})$, and
    \begin{itemize}
    \item[$\bullet$] XMIN, XMAX, YMIN and YMAX are interpreted as
          $r_{\min}$, $r_{\max}$, $\theta_{\min}$, and $\theta_{\max}$,
          respectively.
    \item[$\bullet$] $r_{\min} \geq$ 0 is required.
    \item[$\bullet$] Physical coordinates are Cartesian.  The origin
          of the physical coordinate system coincides with the origin
          of the user (polar) coordinate system, and is within the
          figure.
    \item[$\bullet$] In a user Cartesian coordinate system derived
          from the polar coordinate system, $x_{\min}$ is $-r_{\max}$ if
          $\theta_{\min} \leq 180^{\circ} + k \times 360^{\circ} \leq
          \theta_{\max}$ for some integer $k$, else it is $\min(0, r_{\max}
          \cos \theta_{\min}, r_{\max} \cos \theta_{\max})$.  $x_{\max}$,
          $y_{\min}$ and $y_{\max}$ are defined similarly.  Given a point
          $(r, \theta)$ in the user's polar coordinate system, in the
          physical Cartesian coordinate system,
          $\text{X}_{\text{physical}} = \text{XSIZE}
          \frac{r \cos \theta}{x_{\max}-x_{\min}}$, and
          $\text{Y}_{\text{physical}} = \text{YSIZE}
          \frac{r \sin \theta}{y_{\max}-y_{\min}}$.
    \item[$\bullet$] The border indices are $1 \Rightarrow \theta = \theta_{\min}$,
          $2 \Rightarrow r = r_{\min}$, $3 \Rightarrow \theta =
          \theta_{\max}$ and $4 \Rightarrow r = r_{\max}$.
    \end{itemize}
  \item[$10^3$] The preceding table applies with X replaced by Y and
    $\xi$ replaced by $\eta$ unless in polar coordinates.  In the case
    of polar coordinates, \newline

    \hspace{.25in}\begin{tabular}{ccc}
     Value & $\theta$ is given & Labels\\
    0 & in radians & degrees\\
    1 & in degrees & degrees\\
    2 & in radians & radians\\
    3 & in degrees & radians\\
    \end{tabular}
  \item[$10^4$] Let $x_b$ denote a vertical line drawn from border to
    border at the 0 point on the bottom border, if for this border
    XMIN$\,<0<\,$XMAX.  Similarly let $x_t$ be a vertical line
    at 0 as defined by the top border, $y_\ell$ a horizontal line
    relative to the 0 on the left border, and $y_r$ a horizontal line
    relative to the 0 on the right border.  In all cases if the 0 line
    isn't strictly inside the borders, the line is not drawn.
    \begin{itemize}
    \item[0 =] Draw $x_b$ and $y_\ell$.
    \item[1 =] Do not draw any extra lines.
    \item[2 =] Draw $x_b$ only.
    \item[3 =] Draw $y_\ell$ only.
    \item[4 =] Draw $x_t$ only.
    \item[5 =] Draw $y_r$ only.
    \item[6 =] Draw $x_b$ and $y_r$.
    \item[7 =] Draw $x_t$ and $y_\ell$.
    \item[8 =] Draw $x_t$ and $y_r$.
    \end{itemize}
  \item[$10^5$] 0 gives \LaTeX\ output, 1 gives \TeX\ output.
\end{itemize}
\item\label{ncurv}  (Arguments K\ref{ncurv} = NX, L\ref{ncurv} = 1)
        L\ref{ncurv} sets of ordinates are provided in Y.  K\ref{ncurv} is a
        whole number that gives NDIMY, the declared first dimension for Y
        (K\ref{ncurv} $\geq$ NX), and L\ref{ncurv} is a whole number that
        gives NY, the number of sets of ordinates, {\em i.e.}\ the effective
        second dimension for Y.  The total dimension for Y must be $\geq$
        NDIMY~$\times$ NY.  If a curve has been started but not yet
        finished, that is, SPLOT was most recently previously called with
        option \ref{opt}, and with digit $10^0$ of N\ref{opt}~= 5, and NY
        is changed by specifying this option, an error message is
        produced.  NY must not be greater than the parameter MAXNY in
        SPLOT, which is currently 50.

\item\label{pen}  (Argument P\ref{pen} = 50) S\ref{pen} specifies
        whether lines drawn are solid, dashed, or dotted, and the pen width
        for solid or dashed lines.  In the case of dashed lines one can
        indicate the length of the lines and the nominal space between
        them.  In the case of dotted lines, one can specify the diameter
        of the dots and the space between them.  Digits of S\ref{pen} are
        used as follows:
        \begin{itemize}
        \item[$10^0$]  Pen type.
        \begin{itemize}
          \item[= 0] Solid lines. (The only other thing to specify for
          this case is line width in $10^{1:2}$ below.)
          \item[= 1] Dashed lines.
          \item[= 2] Dotted lines.
          \item[= 3,4] As for 1,2, except units for the length of dashes
           or the diameter of the dots is in deci-points instead of in
           points.  See $10^{3:4}$ below.
          \item[= 5--8]  As for 1--4, except units for the length of the
           spaces, see $10^5$ below, are in deci-points instead of in
           points.
          \end{itemize}
        \item[$10^{1:2}$] The width for solid or dashed lines in
        deci-points. A value of 0 gives the default value of 5
        deci-points.  Here are lines of widths 3, 5, 7 and 10 deci-points:
        \vrule width 0.3pt\ \vrule width 0.5pt\ \vrule width 0.7pt\ \vrule
        width 1.0pt\ .
        \item[$10^{3:4}$] The length of the dashed lines or the diameter
        of the dots.  Units are points or deci-points, depending on the
        value of the low order digit of P\ref{pen}.
        \item[$10^5$] The nominal space between the dashed lines or the
        dots.  Units are points or deci-points, depending on the
        value of the low order digit of P\ref{pen}.
    \end{itemize}
        The following table gives vertical lines showing various
        possibilities with both the spacing and the length for dashes and
        dots given in points. \vspace{5pt}

{\small
\hspace{-16pt}\begin{tabular}{*{15}{@{\hspace{9pt}}l}}
        Spacing: & 1 & 1 & 2 & 2 & 2 & 3 & 3 & 3 & 3 & 4 & 4 & 4 & 4 & 4
\\
        Dashes:  & 1 & 2 & 1 & 2 & 3 & 1 & 2 & 3 & 4 & 1 & 3 & 4 & 5 & 7
\\
        Dots:    & 1 &$\frac 32$& 1 &$\frac 32$& 2 & 1 & 2 &$\frac 52$&
        3 & 2 &$\frac 52$& 3 & $\frac 72$& 4

        \end{tabular}
}

   \hspace{34pt} \mbox{\input pl16-03a }

\item\label{border}  (Arguments N\ref{border}, S\ref{border}, T\ref{border} =
        0)  N\ref{border} is a whole number in which every decimal digit
        $k$ for which $1 \leq k \leq 6$ stipulates that S\ref{border} and
        T\ref{border} specify characteristics for border $k$.  Border
        indices are: 1~$\Rightarrow$ bottom, 2~$\Rightarrow$ left,
        3~$\Rightarrow$ top, 4~$\Rightarrow$ right, 5~$\Rightarrow$
        X-axis if XMIN~$<$ 0 ~$<$ XMAX, 6~$\Rightarrow$ Y-axis if YMIN~$<$
        0 ~$<$ YMAX.

        S\ref{border} is a whole number in which the decimal digits have the
        following meanings:
  \begin{itemize}
  \item[$10^0$]
    \begin{itemize}
    \item[0 =] Border $k$ is not drawn, else it is.  If border $k$ is not
            drawn, ``tick'' marks are neither drawn nor labeled.
    \item[1 =] Major and minor ``tick'' marks are neither drawn nor
            labelled.
    \item[2 =] No labels printed at major ``tick'' marks.
    \item[3 =] Labels printed at major ``tick'' marks,
            except those at the ends of the border.
    \item[4 =] Labels printed at major ``tick'' marks
            except those at the left (bottom) end of the border.
    \item[5 =] Labels printed at major ``tick'' marks
            except those at the right (top) end of the border.
    \item[6 =] Labels printed at all major ``tick'' marks.
    \end{itemize}\vspace{5pt}

    Labels are never printed at minor ``tick'' marks, and will be skipped
    at some major tick marks if it appears the labels would overlap.

  \item[$10^1$] Length, in points, of arrow head to be drawn at the right
          or top end of the border.  If 0, there is no arrow head.
  \item[$10^{2:3}$] Points plotted are not allowed to get closer to the
          border than this number of points.
  \item[$10^{4:}$] Gives the space in points to set aside for labels +
          outward pointing tick marks + border captions.  If this is 0,
          this space will be estimated.
  \end{itemize}
   The defaults for bottom and left borders are 6, for top and right
   borders are 1, and for axes, are 0.

        T\ref{border} is a whole number in which decimal digits have the
        following meanings:
  \begin{itemize}
  \item[$10^{0:1}$] The number of minor intervals within each major
          interval. If zero the value is determined automatically.
  \item[$10^2$] Defines how range on variable is expanded.
    \begin{itemize}
    \item[0 =] Expand the range so that a major ``tick'' mark appears at
            automatically-determined ends of border $k$ if the major
            ``tick'' mark would be at zero, else expand the range to a
            minor ``tick'' mark.
    \item[1 =] Expand the range so that a major or minor ``tick''
            mark appears at automatically-determined ends of border $k$.
    \item[2 =] Expand the range so that a major ``tick'' mark appears at
            automatically-determined ends of border $k$.
    \item[else] Do not expand the range.
    \end{itemize}
  \item[$10^{3:}$] If nonzero, this is the index in COPT of the same kind
        of text allowed after a textual option character of 1--6 as
        defined in Section~\ref{Textopt}.  If more than one border/axis is
        indicated, the caption pointed to will be applied to all of them.
        Thus it is best to use COPT options 1-6 as described in Section
        \ref{Textopt} below, unless this is being used for an extra data
        set, see Option \ref{dataset}.
    \end{itemize}

\item\label{tick}  (Arguments N\ref{tick} = 123456, A\ref{tick} = 4.0pt,
        B\ref{tick} = 3.0pt) N\ref{tick} has the same meaning as
        N\ref{border}.  $|$A\ref{tick}$|$ is the physical length of major
        ``tick'' marks along border $k$, and $|$B\ref{tick}$|$ is the
        physical length of minor ``tick'' marks, both in points.
        ``Tick'' marks of positive length are directed inward, while
        ``tick'' marks of negative length are directed outward.  If the
        physical length is $>$ than the length of the plot ($10^5$ should
        do this), the line is drawn the full length of the plot.
 \item\label{tick1} (Arguments N\ref{tick1}, X\ref{tick1}, D\ref{tick1})
        N\ref{tick1} has the same meaning as N\ref{border}.  If
        D\ref{tick1} $>$ 0 then major ``tick'' marks will be drawn at all
        positions for which [XY]MIN~$\leq$ X\ref{tick1}~+ $k$~$\times$
        D\ref{tick1}~$\leq$ [XY]MAX ($k$ integer).  If D\ref{tick1} $>$ 0
        this option supersedes digits $10^{0:1}$ of T\ref{border}.
\item\label{dataset} (Argument B\ref{dataset}, S\ref{dataset},
        T\ref{dataset}, P\ref{dataset}, U\ref{dataset})
        One may specify multiple data sets, and thus plot different curves
        on the same figure with different scalings and with a specified
        alignment. One should not specify this option without having
        already provided all of the data and options that reference a
        prior data set.
        \begin{itemize}
        \item[B\ref{dataset}]  This gives the index
        of the border used for annotations and labels for the new data set.
        See option \ref{border} for definitions of borders 1--4.  Values
        of 5 and 6 refer to the current data set for X and Y, and can be
        used to set values for P\ref{dataset} and U\ref{dataset} on the
        initial data set.  If B\ref{dataset} indexes a border already in
        use by another data set, that border with associated tick marks
        and labels (if any) is output immediately, and the working area of
        the plot is reduced by moving the new location of this border
        towards the center of the plot.
        \item[S\ref{dataset}] As for S\ref{border} for the new border,
        $1 \leq \text{B\ref{dataset}} \leq 4$.
        \item[T\ref{dataset}] As for T\ref{border} for the new border,
        $1 \leq \text{B\ref{dataset}} \leq 4$.
        \item[P\ref{dataset}] A distance in the same physical units as
        XSIZE and YSIZE from the left or bottom end of the border.  See
        U\ref{dataset} below.  If this is less than 0, there will be no
        effort to align the points for the different data sets.
        \item[U\ref{dataset}] A value in user coordinates associated
        with this border.  Data will be plotted in such a way that a data
        point with a coordinate with this value will be plotted at the
        distance from the left or bottom end of the border indicated by
        P\ref{dataset}.  This provides a means to align plots for the
        different data sets.
        \end{itemize}
\item\label{Xminmax} (Arguments A\ref{Xminmax} = 0, B\ref{Xminmax} = 0)
        If A\ref{Xminmax} $<$ B\ref{Xminmax}, A\ref{Xminmax}
        (B\ref{Xminmax}) gives the smallest (largest) value of X for the
        current dataset.  Values outside this range will be {\em clipped,
        i.e.}\ will not appear in the plot.  If all values are in this
        range, the plotting region is chosen as if these extreme values
        occurred.

\item\label{Yminmax} (Arguments A\ref{Yminmax} = 0, B\ref{Yminmax} = 0)
        Similar to option \ref{Xminmax}, but for Y instead of X.
\item\label{symbol}  (Argument N\ref{symbol} ...) Plot
        symbols as specified by N\ref{symbol} at the current data points.
        In addition data points are connected or not as set by digit
        $10^1$ of N\ref{opt}.

        N\ref{symbol} $<$ 0 means another N\ref{symbol} follows
        immediately, and N\ref{symbol} = 0 draws no symbols.
        If several curves are being plotted (see option \ref{ncurv}), the
        first uses the first $|$N\ref{symbol}$|$, etc.  If NY specifies
        more curves than the number of symbols specified, the
        specification for the last N\ref{symbol} $\geq$ 0 is re-used.

        $|$N\ref{symbol}$|$ is a whole number in which decimal digits
        have the following meanings:
        \begin{itemize}
        \item[$10^0\neq 1$] Plot symbols with vertices evenly spaced on a
        circle.
        \begin{itemize}
        \item[$10^0$] The number of vertices, $v$. $v=0$ describes a
          circle, in which case digits $10^{1:2}$ are ignored.
        \item[$10^1$] The number of vertices, $s$ to ``skip'' when drawing
          edges, except that $s \geq v$ means ``draw lines from the center
          to the vertices.''  To include all the vertices when $s<v$,
          $\gcd(v,s+1)$ polygons are drawn, each time starting one vertex
          counterclockwise from the previous starting vertex.
        \item[$10^2$]  Rotation, $n$.  The first vertex is at an angle of
          $n \: \frac{45^\circ}v$ counterclockwise from the $x$-axis
          direction.
        \item[$10^3$] The width of line to use, $w$, in deci-points.
         If $w=0$ then $w = 3$ is used.  If $w=9$, the symbol is filled.
        \item[$10^{4:5}$] The diameter of the circle in points.  If 0, 6
        is used.
        \item[$10^{6:}$] If this is not 0, the symbol is opaque {\em
        i.e.}\ pixels inside the symbol are cleared before drawing the
        symbol.
 \end{itemize}
        Possible values for $|$N\ref{symbol}$|$ include\vspace{5pt}
\begin{center}
\mbox{\input pl16-03b }
\end{center}

        If $s < v$ and $v$ is odd, the given data are at the averages of
        the minima and maxima of the abscissas and ordinates of the
        vertices.  Otherwise the given data are at the centers of
        circumcircles.

        \item[$10^{1:0}$=1] Plot error bars.  NY must be a multiple of 2.
        Each point on a curve is specified by 3 numbers, {\em e.g.}\
        $x_i,\ y_{i,1},$ and $y_{i,2}$ for the first curve.  At each $x_i$
        a solid vertical line is drawn from $y_{i,1} - y_{i,2}$ to
        $y_{i,1} + y_{i,2}$. The next three columns of $y$ would be used
        for the next curve, etc. Only the points $(x_i, y_{i,1})$ are
        connected for the first curve, and similarly for later curves.
        \begin{itemize}
        \item[$10^2$] Length in points of a horizontal line drawn
        centered on the vertical line at the top and bottom of each error
        bar.
        \item[$10^3$] As for $10^2$, except used for a line at $y_{i,1}$
        \item[$10^4$] Width in deci-points for the cross hatch lines.
        \item[$10^5$] The width of the vertical line, $w$, in
        deci-points. If $w=0$ then $w = 3$ is used.
        \end{itemize}
        \item[$10^{1:0}=11$] As for $10^{1:0} = 1$, except each curve is
        specified by 4 numbers (NY must be a multiple of 3) and the top of
        the vertical line is $y_{i,1} + y_{i,3}$ for the first curve.
        \item[$10^{1:0}=21$] Draw ``vector fields.'' NY (see option
        \ref{ncurv}) must be a multiple of 3.  An arrow is drawn from
        $(x_i,~y_{i,1})$ to $(x_i+y_{i,2},~y_{i,1}+y_{i,3})$ for
        the first curve.  The next three columns of $y$ would be used
        for the next curve, etc.
        \begin{itemize}
        \item[$10^2$] Length of the arrow head in points.  If 0, no
        arrow head is drawn.
        \item[$10^3$] Size of circle in points to be drawn at
        ($x_i,~y_{i,1})$.  Set to 0 if no circle is desired.
        \item[$10^4$] Width in deci-points for the line used to
        draw the circle above.  If this is 9, the circle is filled.
        \item[$10^{5}$] The width of the line use to draw the arrow, $w$,
        in deci-points. If $w=0$ then $w = 3$ is used.
        \end{itemize}
        \end{itemize}
\item\label{onesymb} (Arguments X\ref{onesymb}, Y\ref{onesymb},
        N\ref{onesymb}\ldots .)  Plot a single symbol as specified by
        N\ref{onesymb} at (X\ref{onesymb},~Y\ref{onesymb}).
        $|$N\ref{onesymb}$|$ has the same meaning as
        $|$N\ref{symbol}$|$.  If digit $10^0$ of $|$N\ref{symbol}$|$
        is 1, there are extra arguments corresponding to the
        values of $y_{i,2}$ ... required by option \ref{symbol}.
        N\ref{onesymb}~$\geq$ 0 means (X\ref{onesymb}, Y\ref{onesymb})
        (and any additional arguments) are in user coordinates,
        while N\ref{onesymb}~$<$ 0 means (X\ref{onesymb}, Y\ref{onesymb})
        and additional arguments, if any, are in physical coordinates.
\item\label{arrow} (Argument S\ref{arrow} = 0)  If S\ref{arrow}$~\neq$ 0
        draw an arrow head, with a length of S\ref{arrow} points, at
        the end of the next open curve, or at the last point given in
        the next closed curve.
\item\label{linwid} (Arguments A\ref{linwid}~= 100.0, B\ref{linwid}~=
        70.0, C\ref{linwid}~= 50.0, D\ref{linwid}~= 60.0, E\ref{linwid}~=
        30.0) \ Specify ``pens'' (as is done for option \ref{pen}) for
        various kinds of lines. Values $\leq 0$ select the default.
  \begin{itemize}
  \item[A\ref{linwid}] For borders.
  \item[B\ref{linwid}] For major ``tick'' marks.
  \item[C\ref{linwid}] For minor ``tick'' marks.
  \item[D\ref{linwid}] For lines drawn at X = 0  or Y = 0.
  \item[E\ref{linwid}] For rectangles or ellipses (see options \ref{rect},
          \ref{ellipse}, \ref{pline}, and \ref{pellipse})
  \end{itemize}
\item\label{text} (Arguments X\ref{text}, Y\ref{text}, T\ref{text})  Place
        a text annotation at (X\ref{text},~Y\ref{text}).  The text begins at
        COPT(T\ref{text}~/ 10).  If (T\ref{text}~/10) = 0, text starts
        immediately after the text for the last option of this type, or if
        this is the first, with the first position in COPT where this kind of
        option could appear.  The text pointed to must be in the same form
        as the text following an option character of 1--6 as described in
        Section \ref{Textopt}, except the ``\{...\}'' contains text to be
        printed rather than a caption and ``[...]'' describes the
        justification relative to (X\ref{text},~Y\ref{text}).

        T\ref{text}$\mod 10$ has the following meaning:
        \begin{itemize}
        \item[0 =] X\ref{text} and  Y\ref{text} are in the current user
           coordinate system.
        \item[1 =] X\ref{text} and  Y\ref{text} are in physical
          coordinates.
        \item[2--3] as for 0--1 except don't use the default prefix and
          postfix, any data of this type required is in the text string
          pointed to.
        \item[4--7] as for 0--3, and in addition an opaque borderless
        unfilled rectangle is placed ``below'' the annotation, but
        ``above'' any plotted curves or symbols.
        \end{itemize}

\item\label{number} (Arguments V\ref{number}, X\ref{number},
        Y\ref{number}, T\ref{number}).  Place a number given by
        V\ref{number} on the plot at (X\ref{number}, Y\ref{number}).
        T\ref{number} is interpreted as for T\ref{text} except that
        if the value of T\ref{number}~/ 10 is 0, the default formatting
        is used.  Otherwise the text pointed to in COPT has the same
        form as described in Section \ref{Textopt} for text following an
        option character of ``N''.

\item\label{xtick} (Arguments X\ref{xtick}, T\ref{xtick})
        Place an annotation and/or a line or tick mark at the
        value given by X\ref{xtick} on one of the borders or axes.  If
        X\ref{xtick} is not an interior point along this border or axis,
        this has no action.  Digits of T\ref{xtick} are interpreted as
        follows.
        \begin{itemize}
        \item[$10^0$] Defines the type of line to draw.
        \begin{itemize}
        \item[= 0] No line.
        \item[= 1] A major tick mark.
        \item[= 2] A minor tick mark.
        \item[= 3] A solid line from border to border.
        \item[= 4] A dashed line from border to border with 3 point dashes
                   and a 2 point space between them.
        \end{itemize}
        \item[$10^1$] Gives the index of the border or axes to which this
                     applies.
        \item[$10^{2:}$] Points to text in COPT for an annotation that is
            formatted with the same kind of rule as used for a numeric
            label, see Option \ref{number}. If 0 there is no annotation.
        \end{itemize}

\item\label{bad} (Argument A\ref{bad}, Y\ref{bad}) Values of Y equal to
        Y\ref{bad} are considered to be ``bad data.''  If A\ref{bad} is zero,
        ``bad'' values of Y are simply skipped.  If A\ref{bad}~$>$ 0 and a
        curve is being plotted, it will be terminated at the previous
        ``good'' value of Y, and re-started at the next ``good'' value of Y.
        If A\ref{bad}~$<$ 0 then testing for ``bad data'' is turned off.
        When plotting a closed curve, if this option has been selected and
        A\ref{bad}~$>$ 0, it is taken to be zero.

\item\label{nouse_1}   Not used.

\item\label{line} (Arguments XA\ref{line}, YA\ref{line}, XB\ref{line}
        YB\ref{line})  Draw a line in user coordinates from
        (XA\ref{line}, YA\ref{line}) to (XB\ref{line}, YB\ref{line})
        using the currently selected line style parameters (including the
        arrow head), and the scaling set by the most recent appearance of
        options \ref{Xminmax} or \ref{Yminmax}.

\item\label{rect} (Arguments A\ref{rect}, B\ref{rect}, C\ref{rect},
        D\ref{rect}) Draw a rectangle specified by points (A\ref{rect},
        B\ref{rect}) and (C\ref{rect}, D\ref{rect}), in user
        coordinates of any two diagonally opposite corners of the
        rectangle, after all curves are drawn, but before any annotations
        specified by options \ref{onesymb} or \ref{text} are placed.
        The rectangle is filled as specified by option \ref{fill}.

\item\label{ellipse} (Arguments X\ref{ellipse}, Y\ref{ellipse},
        A\ref{ellipse}, B\ref{ellipse}, R\ref{ellipse}) Draw an ellipse
        centered at (X\ref{ellipse}, Y\ref{ellipse}) with major axis
        A\ref{ellipse} and minor axis B\ref{ellipse}, and with the
        major axis rotated R\ref{ellipse} degrees counterclockwise from
        the $+x$-axis direction.  All but R\ref{ellipse} are in user
        coordinates.  If logarithms of user coordinates are
        ordinarily taken, logarithms of X\ref{ellipse} and Y\ref{ellipse}
        are computed before rotation.  The input values of A\ref{ellipse}
        and B\ref{ellipse} are used without taking logarithms.

\item\label{pline} Like \ref{line}, but in physical coordinates.
        Logarithms are never applied to coordinates in this case.

\item\label{prect} Like \ref{rect}, but in physical coordinates.
        Logarithms are never applied to coordinates in this case.

\item\label{pellipse} Like \ref{ellipse}, but in physical coordinates.
        Logarithms are never applied to coordinates in this case.

\item\label{fill} (Arguments F\ref{fill}=0, ...) Specify filling of closed
        curves, rectangles defined by option \ref{rect} or \ref{prect}, or
        ellipses defined by option \ref{ellipse} or \ref{pellipse}.  Some
        cases require extra arguments that are denoted by A\ref{fill},
        B\ref{fill}, and C\ref{fill}.  One must provide exactly as many
        arguments as the number required.  F\ref{fill} is a whole number
        in which the decimal digits have the following meaning:
  \begin{itemize}
  \item[$10^0$]
    \begin{itemize}
    \item[0] Do not fill~-- leave transparent.
    \item[1] Fill with solid black.
    \item[2] Erase.
    \item[3] Shade with dots of size A\ref{fill} and spacing B\ref{fill}.
    \item[4] Hatch with lines of thickness A\ref{fill}, spacing
          B\ref{fill} and angle C\ref{fill} (in degrees counter-clockwise
          from horizontal).
    \end{itemize}
  \item[$10^1$] 0 = specification applies to curves, 1 = specification
        applies to rectangles defined by option \ref{rect}, else
        specification applies to ellipses defined by option \ref{ellipse}.
  \item[$10^2$] 0 = specification applies to next curve, rectangle or
        ellipses only.  Else it applies, until changed by option
        \ref{fill}.  One can specify as many as 3 fill patterns with
        shading or hatching and all will be applied.
  \end{itemize}
        A\ref{fill} and B\ref{fill} are in points, 72.27 points/inch.

\item\label{nouse_2}   Not used.

\item\label{debug} (Argument L\ref{debug} = 0)  Print debugging output:
  \begin{itemize}
  \item[L\ref{debug}$\leq$0]  No debugging output
  \item[L\ref{debug}$>$0]  Option settings
  \item[L\ref{debug}$>$1]  Scratch file contents, including data.
  \end{itemize}
\end{enumerate}
\renewcommand{\labelenumi}{\theenumi}

\subsubsection{Textual Option Descriptions\label{Textopt}}
Data at the beginning of COPT() consists of strings headed by a single
character that identifies the kind of string that follows.  After the
last of such strings, COPT() may contain character strings pointed to by
various options.  Strings pointed to must begin with ``['', ``\{'', or
``(''.

Options specified entirely in COPT() consist of a single letter or number
code followed by text associated with the option.  All letters used in
options may be in either upper or lower case.  Option codes may be
preceded by blanks.  Where we write ``[\,]'', ``()'', or ``\{\}'' (no text
inside), the following is to be understood.

\begin{description}
\item[\hbox{[\,]}] Always optional.  Contains two or four letters between
the brackets that define how an item is centered relative to some nominal
position.  The first is either {\tt t}, {\tt c}, or {\tt b} for top,
center (vertically), and bottom.  The second letter is either {\tt l},
{\tt c}, or {\tt r}, for left, center (horizontally), and right.  One may
follow the second letter with an {\tt s} which indicates the text is
``stacked'' (vertically), and a following letter that must be {\tt l},
{\tt c}, or {\tt r}, to indicate how the text is justified horizontally
inside the stack. Inside a stack, text enclosed in balanced ``\{...\}'' or
``\$...\$'' is kept on a single line.
\item[()] Always optional.  Contains information between the parentheses
on formatting numbers and on the size of text (in points).  Items can
appear in any order, with no intervening spaces.  Letters can be in either
upper or lower case, and \# is used to denote an integer which if not
present causes the nominal value to be used.  The following can be
specified.
\begin{itemize}
\item[.] Always print a decimal point.
\item[F\#] Font size in points.  (The only case that
makes sense if ``()''
is being used in connection with text output.  Thus ``(F12)'' would be
used to indicate text or numbers that are in 12 point.)  The default is 9
point.
\item[D\#] Number of significant digits that must be printed.
\item[A\#] Number of digits that are required after the decimal point.
\item[B\#] Number of digits that are required before the decimal point,
\item[X\#] $0 < \# < 10$,  bias for selecting the exponent notation.  If
     \# is zero, it is replaced by 5.  The exponent notation is used if
     there are $9-\#$) or more zeros that are serving as place holders,
     else the usual format is used.  Note that with an input \# of 9,
     there will always be at least 0 zeros, and exponent notation must be
     used.
 \end{itemize}
\item[\{\}] Never optional, but can be replaced by a ``\#'' to get a
default value.  Between the braces there is a text string containing
at least one ``\#'' that is not preceded by a ``$\backslash$''.  When
outputting a number or text, this string is copied to the output with the
first such ``\#'' replaced by the number or text being output.  This
provides a means to change font, size, etc.  The default for \LaTeX\
is ``\{$\backslash $small \#\}'' and  for \TeX\ is ``\{\#\}''.
\end{description}\vspace{5pt}

{\bf Option}\vspace{-12pt}
\begin{description}
\item[Format] \hspace{.5in}{\bf Option Description}

\item[F\{{\em File\_name}\}] Specifies a file name for the output.
If this option is not selected, it is as if ``F\{splot.tex\}'' were used.
\item[Q]   End of the option list.  The end of the option list may also be
           indicated by \{, ( or [ appearing where the first letter of an
           option code is expected --- probably text used by option
           \ref{text}, \ref{number}, or \ref{xtick}.
\item[A\,()\{\}] Default specification for printing numbers on
borders and axes.  Should precede any B, T, L, R, X, or Y specifications.
\item[B\,()\{\}] Specification for printing numbers on
bottom border.
\item[T\,()\{\}] Specification for printing numbers on
top border.
\item[L\,()\{\}] Specification for printing numbers on
left border.
\item[R\,()\{\}] Specification for printing numbers on
right border.
\item[X\,()\{\}] Specification for printing numbers on
x-axis.
\item[Y\,()\{\}] Specification for printing numbers on
y-axis.
\item[N\,\hbox{[\,]}()\{\}] Specification for printing numbers
interior to the plot.  The default for ``[\,]'' is {\tt [cl]}.
\item[W\,\hbox{[\,]}()\{\}] Specification for printing words or other
  text somewhere on the plot.  If (...) is used, only the ``F'' for
  font size applies.  The default for ``[\,]'' is {\tt [cl]}.
\item[C\,()\{\}] Default specifications for printing border/axis
captions.
\item[I\{{\em Input\_file\_name}\}] Specifies the name of a file from
which the X and Y data are to be read.  This file is opened with {\tt
FORM~= 'UNFORMATTED'} and {\tt ACCESS~='SEQUENTIAL'}, and will have NX
records read with a sequence of statements of the form\\
{\tt \hspace*{0.125in}   read (IO) X, (Y(J), J=1,NY)}\\
where IO is the unit number, automatically selected, associated with this
file, and NY is 1 or, if set, the value of K\ref{ncurv} in option
\ref{ncurv}.  If NX is sufficently large data is read to the end of
the file.
\item[M\{{\em Raw plot output}\}]The raw plot output is sent
directly to the plot device driver, except that $\backslash$ serves
as an escape character ({\em i.e.}$\backslash$x for any ``x'' is replaced
by x).
\item[``1--6''\hbox{[\,]}()\{{\em Border/axis\_caption}\}] Specification
for a caption to be printed on a border/axis, where a single digit from 1
to 6 is used as in Option \ref{border}.  A {\tt c} for vertical positioning
on a vertical line or for horizontal positioning on a horizontal line
causes the text to be centered with respect to that direction outside any
labels that may be present.  Defaults for ``[\,]'' are {\tt [bc]}, {\tt
[cl]}, {\tt [tc]}, {\tt [cr]}, {\tt [cr]} and {\tt [tc]} for the bottom,
left, top, right, x-axis, and y-axis respectively.

The \# is not used in this case.  When using \LaTeX \ the default
``$\backslash$small'' is not generated in this case if the first
character following the \{ is a $\backslash$.

Thus ``1\{Label bottom border\}'' will place text centered below the
bottom border with a baseline at the edge of the figure.  And the default
for the $x$-axis centers text vertically just past the right end of the
axis.  If one were to use {\tt [bc]} on the top border, the text would
have a baseline at the top of the figure.  If defaults are used, a caption
for a vertical line that requires more than two character lengths
unstacked than it would require if stacked, will be stacked by adding {\tt
sc} to the positioning information.
\end{description}

\subsubsection{Incorporating the Output Into a \TeX\ or \LaTeX\
Document\label{IntoTeX}}

SPLOT generates output intended to be processed by the {\tt mfpic}
package of \TeX\ or \LaTeX\ macros.  See Section D below for more
details.  Using \LaTeX\ or PDF\LaTeX\ one can generate either ``.dvi''
files or ``.pdf'' files.  We describe both approaches, but recommend
using PDF\LaTeX\ as it gives documents that are easier to share and it
is easier to get set up.  Also for many of our plots we had to use our
own version of gftopk as the standard version does not set aside
enough memory for the character sizes that are generated for the
plots.

The general form of a \LaTeX\ document that incorporates plots generated by
SPLOT is:
\begin{verbatim}
\documentclass... % for LaTeX2e
% or \documentstyle for older LaTeX versions
  ...
\usepackage[metapost]{mfpic}
  ...
\begin{document}
  ...
\opengraphsfile{<fontname>}
  ...
\input <plotname>
  % TeX reads the SPLOT output from the file
  % <plotname>.tex, default is splot.tex
  ...
\input <anotherplotname>
  ...
\closegraphsfile
  ...
\end{document}
\end{verbatim}

The processing steps for this document illustrate usage:
\begin{enumerate}
\item Run the programs that invoke SPLOT:\\
  {\tt pl16-03} \hspace{20pt}(Creates the symbols for option \ref{symbol})\\
  {\tt drsplot} \hspace{20pt}(See Section C)
  \vspace{-5pt}
\item\label{DocP} Process the document:\\
  {\tt pdflatex ch16-03}
  \vspace{-5pt}
\item\label{mfP} Process the font produced in step \ref{DocP}:\\
  {\tt mpost pl16-03}\\
  The name {\tt pl16-03} was mentioned in the command
  ``{\tt $\backslash$opengraphsfile}'' in this document.
  \vspace{-5pt}
\item Process the document again, this time incorporating the
  generated plots from step \ref{mfP}:\\
  {\tt pdflatex ch16-03}
  \vspace{-5pt}
  \vspace{-5pt}
\end{enumerate}

Programs above may have different names and different usages on
different systems.  If you are generating ``.dvi'' files and use a
system that automatically updates a ``.dvi'' display when running
\LaTeX, you will likely need to close this window after generating a
new graphic font in order for the new graphic font to be displayed
correctly.

To get dvi output.
Replace ``$\backslash $usepackage[metapost]\{mfpic\}'' with\\
  ``$\backslash $usepackage[metafont]\{mfpic\}''.\\
Run latex instead of pdflatex.\\
Replace ``mpost pl16-03'' with ``mf pl16-03''\\
Run gftopk pl16-03.*gf\\
Run latex again.

\subsubsection{Modifications for Double Precision\label{DP}}
For double precision usage, change the REAL type statement to DOUBLE
PRECISION, and change the subroutine name to DPLOT.  The default file name
for output, used if option {\tt F} in COPT() is not selected, is
{\tt dplot.tex}.

\subsection{Examples and Remarks}

The example given is that used to generate the plot in Chapter~2.2.

If one produces a curve or set of related curves by setting the $10^0$
digit of N\ref{opt}~=~5 and calling SPLOT several times to supply data
values, the value of NY must be the same on every call that contributes to
the construction of the curve(s).

If one produces a plot by making several calls to SPLOT, using digit
$10^0$ of N\ref{opt} to indicate that the plot is or is not to be finished
on the current call, the following considerations are important:
\begin{itemize}
\item
  The specifications whether ``tick'' marks are linearly or
  logarithmically spaced are those in effect when the plot is finished.
\item
  The specifications whether to plot data as given, or to plot $\log_{10}~X$
  or $\log_{10}~Y$, are remembered from one invocation of SPLOT to the
  next, but, if changed, are separately observed.
\item
  The specification to use a polar coordinate system is the one in effect
  when the plot is finished.
\item
  The units and values of XSIZE and YSIZE are those in effect when
  the plot is finished.
\end{itemize}

All options are reset to their default values when SPLOT is first invoked,
or when invoked after finishing a plot.  When SPLOT is invoked after
having been invoked with instructions {\em not} to finish the plot, all
optional settings retain their values from the previous invocation until
changed by option selections in the current invocation, except that digit
$10^0$ of N\ref{opt} is reset to zero, and L\ref{debug} is reset to zero
before the option selections are processed.

We also give a simple example here of taking output in a file to make
a plot, and give scripts illustrating how to get the plots.
Program {\tt plotf.f} takes output from a file and calls {\tt dplot}.
Scripts {\tt plotpdf} and {\tt plotdvi} shows how to to do the
whole job where the program {\tt tedxrk8} is the program used to
generate the results and the \LaTeX\ file is {\tt stepsel.tex}.

{\bf plot.f}\vspace{-10pt}
\begin{verbatim}
c Plot data from a file; in this case file
c plot.out, assumed to have floating point
c numbers x_i, y_i, z_i on each line. y(x)
c and z(x) are plotted.  Output in this case
c goes to file stepselp.tex.
      double precision X(1), Y(1), OPT(8)
      character COPT*27
      data COPT/'I{plot.out}F{stepselp.tex}Q'/
c                      Set for 2 curves ....
      data OPT/ 0.D0, 2.D0, 100000.D0, 2.D0,
c       For x in [0, 15]
     1  8.D0, 0.D0, 15.D0, 0.D0/
      call DPLOT(4.5D0, 3.D0, X, 100000,
     1   Y, OPT, COPT)
      stop
      end
\end{verbatim}

{\bf plotpdf}\vspace{-10pt}
\begin{verbatim}
#!/bin/bash
export RUNDIR=/m/math77/ode/dxrk8/new
$RUNDIR/tedxrk8 <$RUNDIR/test.in
$RUNDIR/plotf
pdflatex stepsel
mpost stepselp
pdflatex stepsel
\end{verbatim}

{\bf plotdvi}\vspace{-10pt}
\begin{verbatim}
#!/bin/bash
export RUNDIR=/m/math77/ode/dxrk8/new
$RUNDIR/tedxrk8 <$RUNDIR/test.in
$RUNDIR/plotf
latex stepsel
mf stepselp
gftopk stepselp.*gf
latex stepsel
\end{verbatim}


\subsection{Functional Description}

\subsubsection{Sequence of Events}\label{SOE}

Each time SPLOT is invoked, the character options in COPT are processed,
then the numeric options in OPT are processed, then the data are
processed.  Most numeric options, and all data, are stored on a scratch
file, and processed when digit $10^0$ of N\ref{opt} $\leq 2$.

Numeric options are processed in the order they appear, and the setting
of one may affect the interpretation of another.  When data are processed,
the most recently previous setting of an option that affects the data or
their interpretation is effective, even if that setting occurred during a
previous invocation of SPLOT for the same plot.

The effects of options or fields of options on subsequent options, and
data, are summarized in the table at the beginning of Section
\ref{Numopt}.  The column labelled ``Scratch File'' indicates whether the
option, or some of its fields, are put onto the scratch file, along with
the data.

The plot border, axes, tick marks, labels, and captions are output first.
Then data on the scratch file is output in the following order.  First all
data for a given {\tt mfpic} group is output before that for another.  Within
each such group, curves are output, then single symbols, then rectangles
and ellipses, and finally text annotations.

\subsubsection{Context of Usage -- Interaction with \LaTeX\ and \TeX}

Plots are rendered into marks on the page by converting them to characters of
a font.  The {\tt mfpic} package instructs \TeX\ or \LaTeX\ to put commands to
produce the font into the file {\tt $<$fontname$>$.mf} specified by the {\tt
$\backslash$opengraphsfile} command (see Section \ref{IntoTeX}).

The document is thus produced by invoking the \TeX\ or \LaTeX\ program to
produce the file {\tt $<$fontname$>$.mf}, processing {\tt $<$fontname$>$.mf}
by the {\tt mf} program, and then invoking \TeX\ or \LaTeX\ again to
account for the ``font metrics'' and to incorporate the font ``characters''
that implement the plots.

After processing by the {\tt mf} program, it may also be necessary to use the
{\tt gftopk} program, depending on how the {\tt .dvi} file is ultimately
converted to a format for a specific printer.

Finally, the document is converted from ``device independent'' ({\tt .dvi})
form to a form usable by a specific printer by a {\tt dvi} program specific
to that printer.

Consult documentation specific to your system for details of usage of \TeX,
\LaTeX, {\tt mf}, {\tt gftopk} and {\tt dvi} programs.

A version of the {\tt mfpic} macros accompanies the libraries.

Some implementations of {\tt gftopk} have insufficient capacity to process
output from {\tt mf} resulting from using {\tt mfpic}.  A version of {\tt
gftopk} that has more capacity accompanies the libraries.  You will need a
C compiler to compile it for your system.

\subsection{Error Procedures and Restrictions}

Although this software has been used to produce the wide variety of plots
in the MATH77 documentation, time constraints have meant that a good
number of the options have not been checked.  It is expected that there
will be bugs in some of these unchecked options.  These will be fixed as
they arise.  Any errors due to a bug should be obvious from an examination
of the plot if the bug allows the code to get this far.  The C versions of
this code have been checked only on the simple demonstration drivers
and thus are even more likely to suffer from bugs.

An error message is produced if OPT(1) is returned non-zero.  All errors
are processed by the error message processor described in Chapter 19.3.
If the error messages provided do not clear up the problem, looking at
{\tt $<$plotname$>$.tex} may clarify the problem for you.

\subsection{Supporting Information}

The source language is ANSI Fortran~77.

\begin{tabular}{@{\bf}l@{\hspace{5pt}}l}
\bf Entry & \hspace{.35in} {\bf Required Files}\vspace{2pt} \\
SPLOT & \parbox[t]{2.7in}{\hyphenpenalty10000 \raggedright
MESS, SMESS, SPLOT, SPLOT0\rule[-5pt]{0pt}{8pt}}\\
DPLOT & \parbox[t]{2.7in}{\hyphenpenalty10000 \raggedright
DMESS, MESS, SPLOT, SPLOT0\rule[-5pt]{0pt}{8pt}}\\
\end{tabular}

Design and code by Fred T. Krogh and W. Van Snyder, JPL, December 1997.
Special thanks to Thomas Leathrum for creating {\tt mfpic} and to Geoffrey
Tobin for making his latest version of {\tt mfpic} available and for
answering questions on it's use.

\closegraphsfile
\begcode

\medskip\
\lstset{language=[77]Fortran,showstringspaces=false}
\lstset{xleftmargin=.8in}

\centerline{\bf \large DRSPLOT}\vspace{10pt}
\lstinputlisting{\codeloc{splot}}\vspace{-30pt}

\vspace{30pt}\centerline{\bf \large ODSPLOT}\vspace{10pt}
\lstset{language={}}
\lstinputlisting{\outputloc{splot}}
\end{document}

