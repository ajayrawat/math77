\documentclass[twoside]{MATH77}
\usepackage{multicol}
\usepackage[fleqn,reqno,centertags]{amsmath}
\begin{document}
\begmath 12.1 One-Dimensional Table Look Up, Interpolation, \hbox{and Differentiation}

\silentfootnote{$^\copyright$1997 Calif. Inst. of Technology, \thisyear \ Math \`a la Carte, Inc.}

\subsection{Purpose}

Given a table, $(x_i$, $y_i)$, of independent variable values and the
corresponding dependent variable values, this subroutine finds the
points in the table closest to a given value of the independent
variable and uses these points to interpolate for the corresponding
value of the dependent variable.  Error estimates, Hermite
interpolation (i.e.\ using $(x_i$, $y_i$, $y_i^\prime)$), different
look up methods, and the computation of derivative information are
available.

\subsection{Usage}

\subsubsection{Program Prototype, Single Precision}

\begin{description}
\item[INTEGER]  \ {\bf NTAB, NDEG, LUP, IOPT}$(\geq k_1)$\newline
[$k_1 \geq 2$ depends on options used.]

\item[REAL]  \ {\bf X, Y, XT}$(\geq $NTAB){\bf , YT}$(\geq $NTAB){\bf ,\newline
EOPT}$(\geq k_2$) [$k_2 \geq 1$ depends on options used .]
\end{description}
\begin{center}
\fbox{\begin{tabular}{@{\bf }c}
CALL SILUP (X, Y, NTAB, XT, YT,\\
NDEG, LUP, IOPT, EOPT)\\
\end{tabular}}
\end{center}

\subsubsection{Argument Definitions}

\begin{description}
\item[X]  [in] Independent variable where value of interpolant is desired.

\item[Y]  [out] Value of interpolant.

\item[NTAB]  [in] Number of points in the table, $\geq \text{NDEG}+k$,
  where $k=1$ if NDEG is odd and 2 if NDEG is even, and must be one
  bigger than this if an error estimate is requested.

\item[XT()]  [in] Array of independent variable values, XT($i)=x_i$. Must be
monotone increasing or monotone decreasing, but this is not checked
for. (XT(I$)=$ XT(I+1) is permitted, but has a special meaning, see
Section~C.) If the $x_i$ are equally spaced one should set LUP (below) to~3,
and provide $x_1$ in XT(1) and $x_2-x_1$ in XT(2).

\item[YT()]  [in] Array of dependent variable values. YT($i)=y_i.$

\item[NDEG]  [in] Nominal degree of the polynomial to be used in the
interpolation. Require $0\leq $ NDEG $\leq 15.$

For NDEG $=0$, Y is set to YT($k)$, where $k$ minimizes $|\text{X} -x_k|.$

For NDEG odd, Y is the result of standard polynomial interpolation of degree
NDEG using NDEG + 1 points. The interpolating function is continuous, but
the first derivative is usually discontinuous at tabular points.

For NDEG $>$ 0 and even, Y is obtained from a function that is a linear
combination of two polynomials of degree NDEG. This function is a polynomial
of degree NDEG + 1 using NDEG + 2 points that interpolates all but the two
outer points. It has a continuous first derivative but the second derivative
is usually discontinuous at tabular points.

Usually $|error|$ in the interpolant will tend to decrease in a fairly
regular way for increasing values of NDEG until either errors due to a lack
of precision in YT, or the inherent instability of high degree polynomial
interpolation causes errors to get worse. If one wants a continuous first
derivative, even values of NDEG $>$ 0 should be used.

\item[LUP]  [inout] Defines the type of look up method. (Changed only if LUP
$\leq 0$ on input.)

\begin{itemize}
\item[$\leq 0$]  \ If LUP $=0$, start with a binary search, else do a
sequential search starting with an index of $-$LUP. On exit, LUP is set to $%
-k$, where $k$ minimizes $|$X $-$ XT($k)|.$
\item[$= 1$]  Binary search. Use this when accesses are not sequential, and
XT values are not close to being equally spaced.
\item[$= 2$]  Start a sequential search with an index $=[1.5+($NTAB$-$1)
(X $-$ XT(1))/(XT(NTAB) $-$ XT(1))]. Use this when points are almost equally
spaced and accesses are not sequential.
\item[$= 3$]  YT(I) corresponds to XT(1) + (I$-$1)*XT(2), no search is
required to do the look up and XT can have dimension XT(2). This is
recommended when the XT values are equally spaced since it takes the least
space and should usually be fastest.
\item[$= 4$]  Internal information connected with the $x_k$ values used in
the last interpolation is reused. (Don't use this value if there is any
chance for an intervening call to this subroutine.) Only YT should be
changed; IOPT is not examined. This option saves time when interpolating
components after the first of a vector valued function.
\end{itemize}

\item[IOPT()]  [inout] IOPT(1) is used to return a status as follows:

\begin{itemize}
\item[$-$10 ]  \ An option index is out of range.
\item[$-$9 ]  \ NTAB is outside allowed limits.
\item[$-$8 ]  \ NDEG is outside allowed limits.
\item[$-$7 ]  \  LUP $>$ 3 when program was not ready for it.
\item[$-$6 ]  \ Option 3 (compute derivatives), has requested more than
15~derivatives.
\item[$-$5 ]  \ LUP = 3, and XT(2) = 0.
\item[$-$4 ]  \ XT(1) = XT(NTAB), and NTAB is not~1.
\item[$-$3 ]  \ Bad points (see option 6) mean only 0 or 1 points were
available for interpolation..
\item[$-$2 ]  \ There is only one table entry; the estimated error that was
requested was not computed.
\item[$-$1 ]  \ The accuracy requested was not obtained.
\item[0 ]  \ Normal return, no exceptional conditions.
\item[1 ]  \ X was outside the domain of the table, extrapolation used.
\item[2 ]  \ Available table values were so few that this restricted the
degree of the polynomial.  A valid error estimate is not returned in this
case.
\end{itemize}

Starting with IOPT(2) options are specified by integers in the range 0 to~7,
followed in some cases by integers providing argument(s) for the option.
Each option, with its arguments if any, is followed in IOPT by the next
option or by a~0.  If an option index is specified more than once,
only the last specification is used.

\begin{itemize}
\item[0]  End of the option list; this must always be the last option
specified in IOPT.

\item[1]  An error estimate is to be returned in EOPT(1).

\item[2]  (Argument: K2) K2 gives the polynomial degree to use when
extrapolating. The default for K2 is NDEG if NDEG $\leq $ 2 or if NDEG is
even, else it is NDEG~$-$~1.

\item[3]  (Arguments: K3, L3) Save $(k^{th}$ derivative of interpolating
polynomial$)/k!$ in EOPT(K3+$k-$1) for $k=1$, 2, ..., L3. These values are
the coefficients of the polynomial in the monomial basis expanded about X.
Require $0\leq $ L3 $\leq 15.$

\item[4]  (Argument K4) The absolute and relative errors expected in YT
entries are specified in EOPT(K4) and EOPT(K4+1) respectively. The values
provided here are used in estimating the error in the interpolation. An
error estimate is returned in EOPT(1).

\item[5]  (Argument K5, L5) Do the interpolation to the accuracy requested
by the absolute error tolerance specified in EOPT(K5) and the relative error
tolerance in EOPT(K5+1) respectively. An attempt is made to keep the final
error $<$ EOPT(K5) + EOPT(K5+1)\thinspace ($|$YT($k)|$ + $|$YT($k^{\prime
})|)$, where $k$ and $k^{\prime }$ are indices for table values close to X.
Standard polynomial interpolation is done, but here NDEG gives the maximal
degree polynomial to use in the interpolation. The actual degree used in
doing the interpolation is stored in the space for the argument L5. If both
EOPT's specified are $\leq $ 0, IOPT(1) is not set to $-$1, and no error
message is generated due to an unsatisfied accuracy request. An error
estimate is returned in EOPT(1).

\item[6]  (Argument K6) Do not use point (XT$(k)$, YT($k))$ in the
interpolation if YT($k$) = EOPT(K6). This option is useful if one has a table
with equally spaced points, but with some bad data points, and may also be
used by the multiple dimensional interpolation subroutine. Points are
selected for use in the interpolation as if points flagged with YT($k$) =
EOPT(K6) were not present.

\item[7] (Argument K7) YT(K7+$k$) gives the first derivative
corresponding to the function value in YT($k$).  These derivatives
are to be used in doing the interpolation.  One gets a continuous
interpolant only for NDEG = 3, 7, 11, and 15; these cases also give a
continuous first derivative in the interpolant.  The interpolating
polynomial satisfies $p$(XT($k$)) = YT($k$), $p^\prime $(XT($k$)) =
YT(K7+$k$) for values of $k$ that give values of XT close to X.
(Section~D describes how points are selected.) If NDEG is even, a
value of YT($k$) is used without using the corresponding value of
YT(K7+$k$).


\end{itemize}

\item[EOPT]  [inout] Array used to return an error estimate and used for
options.

\begin{description}
\item[EOPT$(1)$]  [out] contains an estimate of the error in the
interpolation if an error estimate has been requested by setting
option 1, 4 or~5.

\item[EOPT$(>1)$]  [in or out] for use by options~3--6.
\end{description}
\end{description}

\subsubsection{Modifications for Double Precision}

Change SILUP to DILUP, and the REAL type statement to DOUBLE PRECISION.

\subsection{Examples and Remarks}

DRSILUP below is a sample program that interpolates in a table of $\sin
(x)$ given with a spacing of~0.5 for NDEG = 2 to~10, and obtains an error
estimate. ODSILUP below gives the results of running DRSILUP on an IBM PC,
which uses IEEE 32~bit floating point arithmetic.

The user is reminded that polynomial interpolation of high degree is
hazardous, where ``high" depends strongly on the kind of data being
interpolated. The error estimates provided by the program are usually
greater than the actual error, but on any given interpolation may be much
too small. Similarly, the order selected by the program to satisfy a given
error criterion will usually do a good job, but will (infrequently) use too
low a degree due to an overly optimistic error estimate or an overly
pessimistic conclusion that the corrections are starting to diverge. When
many interpolations are going to be done in a table we recommend trying
several degrees at enough points to cover the kinds of functional behavior
in different areas of the table and examining the error estimates. Then
later interpolations can be done using the fixed degree that appears best
for the accuracy desired.

If one has a table with discontinuities, one can set XT(I) = XT(I+1) =
the point of the discontinuity. The $x_j$'s that are used in the
interpolation will be selected so that all $j$ satisfy $j \leq $ I, or
$j \geq$ I + 1.  If the discontinuity lies between two points, use
option 6 to define the value at the point of discontinuity.

\subsection{Functional Description}

The look up process identifies the point in the table, XT($k)$, nearest to
the input value X.  To obtain continuity in the interpolant, remaining
points are selected one at a time, keeping the number of points on either
side of X balanced as long as this is possible.  Thus XT($k)$ and
XT($k^{\prime})$, the second entry selected, will bracket X if
extrapolation is not required.  When NDEG $>0$ is even (and Hermite
interpolation is not being used), a linear combination of the polynomials
of degree NDEG interpolating the left NDEG + 1 and the right NDEG + 1
points is used.  (A total of NDEG + 2 points is used.) If it is not
possible to select $(\text{NDEG}+2) / 2$ points on either side of X, then
just NDEG + 1 points are used to obtain a polynomial of degree NDEG as in
the odd case.  When extrapolating, the default action is to use standard
polynomial interpolation of degree $2\times \max (1,\lfloor
$NDEG$/2\rfloor )$.  This default can be changed using an option value
of~2 in IOPT().  Interpolations are done using the Newton divided
difference form of the interpolating polynomial as described for
Algorithms I (NDEG odd) and IV (NDEG$>0$, even) by Krogh in
\cite{Krogh:1970:EAP}.  The derivatives are computed as described for
Algorithm V in \cite{Krogh:1970:EAP}.

Define $E_{Min}$ by the following:
\begin{equation*}
E_{Min}=E_{Abs}+E_{Rel\,}(|\text{YT}(i)|+|\text{YT}(i^{\prime })|),
\end{equation*}
where $E_{Abs}$ = EOPT(K4) if option~4 is used, and is otherwise~0; $E_{Rel}$
= EOPT(K4+1) if option~4 is used and is otherwise the smallest positive
floating point number that gives a number different from~1 when added to~1;
and $i$ and $i^{\prime}$ are the indices for the first two table entries
selected for use in the interpolation.

Then the error is estimated by

\begin{equation*}
\hspace{-5pt}\text{error est.}=1.5\left( |y_c-P_{k+1}|+
\frac{|P_k-P_{k-1}|}{32}\right)+E_{Min},
\end{equation*}
where $y_c$ is the value being returned for Y, $P_k$ is the result obtained
by standard polynomial interpolation of degree $k$, and $k$ is NDEG unless
the degree has been reduced because of extrapolation in which case $k$ is
the degree actually used. If there are not sufficient points available to
compute $P_{k+1}$, then the error estimate is given by~1.5 $%
|P_k-P_{k-1}|+E_{Min}.$

The choice of degree for option~5 makes use of the following quantities.
\begin{align*}
e_0 &= |P_0|\\
\overline{e}_1&=\max (.75,\ e_1/(e_1+e_0)\\
e_j &= |P_j - P_{j-1}|,\quad j = 1,\ 2,\ ...\\
\overline{e}_j&=\frac 12\overline{e}_{j-1}+e_j/(e_j+e_{j-1}),\quad j=2,
 3,\ ...\\
\widehat{e}_j&=\frac 12e_{j-1}\overline{e}_{j-1}/j,\quad j=2,\ 3,\ ...
\end{align*}
The value of $j$ is allowed to increase until either $e_j+e_{j-1}\leq $
error request, or $e_j\geq \widehat{e}_j$ for two successive steps with $j>1$%
. The quantity, $\overline{e}_j$ indicates the rate of decrease in $e_j$ for
successive values of $j$. A good rate of decrease in the past causes a
better rate to be required in the future. Also, better convergence is
required as the degree gets larger. The final error estimate is given by~1.5
$(e_j+.0625e_{j-1})+E_{Min\text{.}}$ When derivatives are being computed,
the error request is decreased by the factor $|(x-x_i)/(x_i-x_{i^{\prime }})|
$, where $i$ and $i^{\prime }$ are the indices for the first two table
entries selected.

\bibliography{math77}
\bibliographystyle{math77}

\subsection{Error Procedures and Restrictions}

Values of IOPT$(1) < 0$ ordinarily cause an error message to be printed, and
those $< -3$ do not ordinarily result in a return to the user. One can
change the action on errors by calling the message/error routine MESS
of Chapter~19.3 before calling this routine.

\subsection{Supporting Information}

The source language is ANSI Fortran~77.

\begin{tabular}{@{\bf}l@{\hspace{5pt}}l}
\bf Entry & \hspace{.35in} {\bf Required Files}\vspace{2pt} \\
DILUP & \parbox[t]{2.7in}{\hyphenpenalty10000 \raggedright
AMACH, DILUP, DMESS, MESS\rule[-5pt]{0pt}{8pt}}\\
SILUP & \parbox[t]{2.7in}{\hyphenpenalty10000 \raggedright
AMACH, MESS, SILUP, SMESS}\\
\end{tabular}

Subroutine designed and written by: Fred T. Krogh, JPL, May~1991.
Hermite interpolation added December~1994.


\begcodenp
\lstset{language=[77]Fortran,showstringspaces=false}
\lstset{xleftmargin=.8in}

\centerline{\bf \large DRSILUP}\vspace{10pt}
\lstinputlisting{\codeloc{silup}}
\newpage

\vspace{30pt}\centerline{\bf \large ODSILUP}\vspace{10pt}
\lstset{language={}}
\lstinputlisting{\outputloc{silup}}
\end{document}
