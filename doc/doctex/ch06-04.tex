\documentclass[twoside]{MATH77}
\usepackage{multicol}
\usepackage[fleqn,reqno,centertags]{amsmath}
\hyphenation{UPARAM}
\begin{document}
\begmath 6.4  One Householder Transformation

\silentfootnote{$^\copyright$1997 Calif. Inst. of Technology, \thisyear \ Math \`a la Carte, Inc.}

\subsection{Purpose}

These subroutines compute the parameters defining a Householder orthogonal
transformation which zeroes specified components in a given vector, or in
a column or row of a matrix. Also they optionally apply either a newly
defined transformation or a previously defined Householder transformation to
a vector or a set of vectors, which typically would be columns or rows of a
matrix.

Two versions are provided. For the usual case in which both the pivot vector
and the vectors to which the transformation is to be applied are column
vectors, we recommend the use of DHTCC or SHTCC. If either the pivot vector
or the vectors to which the transformations are to be applied, or both, are
row vectors, one must use DHTGEN or SHTGEN.

These subroutines are used by other library subroutines, and can be used as
modules in implementing other linear algebra algorithms.

\subsection{Usage}

\subsubsection{Usage of SHTGEN and DHTGEN}

We use the term {\em pivot vector} to mean the vector that plays a special role in
determining or defining the Householder transformation. When MODE = 1 a
transformation will be determined that is appropriate to zero components L1
through M of the pivot vector. These components are then replaced by
values that partially define the appropriate Householder transformation.
Component LPIVOT of the pivot vector, and UPARAM are also assigned values to
complete the definition of the transformation. After defining the
transformation it will optionally be applied to specified vectors.

When MODE = 2 the transformation defined in a previous call with MODE
= 1 will be applied to specified vectors.

\paragraph{Program Prototype, Single Precision}

\begin{description}

\item[INTEGER] \ {\bf MODE, LPIVOT, L1, M, LDU, LDC, NCV}

\item[LOGICAL] \ {\bf COLU, COLC}

\item[REAL] \ {\bf U}(LDU,$*$){\bf , UPARAM, C}(LDC,$*$)

\end{description}

Assign values to all integer and logical arguments and to the relevant real
arguments.

\begin{center}
\fbox{\begin{tabular}{@{\bf }c}
CALL SHTGEN(MODE, LPIVOT, L1,\\
M, U, LDU, COLU, UPARAM,\\
C, LDC, NCV, COLC)\\
\end{tabular}}
\end{center}

Computed results will be returned in U(,), UPARAM, and/or C(,),
depending on the initial settings of the integer arguments.

\paragraph{Argument Definitions}

\begin{description}

\item[MODE] \ [in] Set by the user to the value 1 or~2. If MODE = 1, the subroutine
computes parameters defining a Householder transformation as described in
Section D. If MODE = 2, it is assumed that Householder transformation
parameters have already been defined by a previous call with MODE = 1. In
either case, if NCV $> 0$, the subroutine will apply the current
transformation to the set of NCV M-vectors stored in the array C(,).

\item[LPIVOT, L1, M] \ [in] These integers specify that the vectors to be referenced
or operated upon are M-dimensional, and the components to be referenced or
operated upon in each vector are the one indexed by LPIVOT and the set
indexed from L1 through M. To define a nontrivial transformation these
integers must satisfy
\begin{equation*}
1 \leq \text{LPIVOT} <\text{L1} \leq \text{M}
\end{equation*}
If these inequalities are not all satisfied, the subroutine returns without
doing any computation. This is not regarded as an error. It has the implicit
effect of applying an identity transformation.

\item[U(,)] \ [inout] The array U(,) contains the pivot vector as either a column
or a row. Examples:

If the pivot vector is column J of an array declared as A(IDIM,JDIM), the
arguments corresponding to ``U, LDU, COLU" should be written as ``A(1, J),
IDIM, .true.".

If the pivot vector is row I of an array declared as A(IDIM,JDIM), the
arguments corresponding to ``U, LDU, COLU" should be written as ``A(I,1),
IDIM, .false.".

\item[LDU] \ [in] Leading dimensioning parameter for U(,).

\item[COLU] \ [in] Must be set to .true.\ if the pivot vector is a column of U(,)
and to .false.\ if the pivot vector is a row of U(,).

\item[UPARAM] \ [inout] Holds a value completing the definition of a Householder
transformation. See description in Section D. This value will be computed by
the subroutine when MODE = 1 and must be present on entry when MODE\ = 2.

\item[C(,)] \ [inout] If NCV $\leq$ 0, no reference will be made to C(,).
If NCV $> 0$, the array C(,) is regarded as containing a set of NCV M%
-vectors. These are regarded as column vectors if COLC = .true.\ and row
vectors if COLC\ = .false. On entry C(,) contains vectors to be
transformed, and on return C(,) contains the vectors resulting from
application of the current Householder transformation. Examples:

To apply a transformation to columns J + 1 through N of an array declared
as A(IDIM,JDIM), the arguments ``C, LDC, NCV, COLC" should be written as
``A(1, J+1), IDIM, N$-$J, .true.".

To apply a transformation to rows K + 1 through M1 of an array declared as
A(IDIM,JDIM), the arguments ``C, LDC, NCV, COLC" should be written as
``A(K+1,~1), IDIM, M1$-$K, .false.".

\item[LDC] \ [in] Leading dimensioning parameter of C(,).

\item[NCV] \ [in] Number of vectors in C(,) to be transformed. If
NCV $\leq $ 0, the array C(,) will not be referenced.

\item[COLC] \ [in] Must be set to .true.\ if the vectors to be transformed are column
vectors of C(,), and to .false.\ if the vectors to be transformed are row
vectors of C(,).
\end{description}

\subsubsection{Usage of SHTCC and DHTCC}

Subroutine DHTCC is a modification of DHTGEN specialized for the case of
COLU\ = .true.\ and COLC\ = .true.

\paragraph{Program Prototype, Single Precision}

\begin{description}

\item[INTEGER] \ {\bf MODE, LPIVOT, L1, M, LDC, NCV}

\item[REAL] \ {\bf U}$(*)${\bf , UPARAM, C}(LDC,$*)$

\end{description}

Assign values to all integer and logical arguments and to the relevant real
arguments.

\begin{center}
\fbox{\begin{tabular}{@{\bf }c}
CALL SHTCC(MODE, LPIVOT, L1, M,\\
U, UPARAM, C, LDC, NCV)\\
\end{tabular}}
\end{center}

Computed results will be returned in U(,), UPARAM, and/or C(,),
depending on the initial settings of the integer arguments.

\paragraph{Argument Definitions}

The arguments have the same meanings as the arguments of the same names in
the call to SHTGEN. Note that arguments LDU, COLU, and COLC are not used,
and U is a one-dimensional array. This subroutine functions as though COLU
= .true.\ and COLC = .true.

\subsubsection{Modifications for Double Precision}

For double precision usage change the REAL statement to DOUBLE PRECISION and
change the subroutine names SHTGEN and SHTCC to DHTGEN and DHTCC
respectively.

\subsection{Examples and Remarks}

The program DRSHTCC illustrates the use of SHTCC. The array D(,) is
initialized with a $3\times 2$ matrix, $A$, in its first 2 columns, and a $%
3\times 3$ identity matrix in columns 3 through 5. Subroutine SHTCC is used
to compute the QR factorization of $A$, $i.e.$, the program determines an
orthogonal matrix, $Q$, and a triangular matrix, $R$, such that $A = Q R$.
Results are shown in ODSHTCC.

For additional examples of the use of SHTCC and SHTGEN see the source code
for the library subroutines SHFTI or SSVDRS.

\subsection{Functional Description}

Let ${\bf u}$ be a nonzero $n$-vector and define $\beta  = -\|{\bf u}\|^2/2$. The matrix
\begin{equation*}
Q = I + \beta ^{-1}{\bf uu}^t
\end{equation*}
is symmetric and orthogonal, and is called a Householder transformation
matrix.

A product of the form ${\bf w} = Q{\bf v}$ can be computed by the steps:
\begin{align*}
\gamma &= ({\bf u}^t{\bf v})/\beta\\
{\bf w} &= {\bf v} + \gamma {\bf u}
\end{align*}
Typically one desires a ${\bf u}$ such that for some particular nonzero vector, ${\bf v}$%
, the vector ${\bf w} = Q{\bf v}$ will consist of zeros except for the first component.
Given ${\bf v}$, the appropriate ${\bf u}$ and $\beta $ to accomplish this can be
defined mathematically as follows:
\begin{align*}
\sigma &= 1 \hspace{.7in} \text{if } v_1 > 0 \text{ and } {-}{1} \text{ otherwise}\\
u_1 &= v_1 + \sigma \|{\bf v}\|\\
u_i &= v_i, \hspace{.6in} i = 2{\text, ..., }n\\
\beta  &= -\sigma u_1\|{\bf v}\| \hspace{.2in} \text{(Note that }\beta  \leq 0.)
\end{align*}
With this definition of $\sigma $, ${\bf u}$, and $\beta $, and thus of $Q$,
the vector ${\bf w} = Q{\bf v}$ will have components:
\begin{align*}
w_1 &= -\sigma \|{\bf v}\|\\
w_i &= 0, \hspace{.4in} i = 2\text{, ..., }n.
\end{align*}
In applications it is frequently convenient to regard the vectors ${\bf v}$, ${\bf u}$,
and ${\bf w}$ of the above formulas as embedded in higher dimensional vectors with
other components which play no role in the current transformation.
Specifically we regard the host vector for ${\bf v}$ (and similarly for ${\bf u}$ and ${\bf w})
$ as being of dimension M, with $v_1$ at position LPIVOT and $v_2$, ..., $%
v_n$ in positions L1 through M. Thus we are identifying $n$ with the value
M $-$ L1 + 2. We assume $1 \leq \text{LPIVOT} < \text{L1} \leq \text{M}.$

On input with MODE = 1 the array U(,) contains ${\bf v}$ embedded as specified
by the integers LPIVOT, L1, and M, and with the Fortran~77 storage mapping
as specified by the arguments LDU and COLU. The subroutine computes $w_1$
and ${\bf u}$. It stores the $u_1$ in UPARAM, $w_1$ in place of $v_1$, and $u_2$, $%
...,u_M$ in place of $v_2$, $...,v_M.$

When applying transformations the quantity $\beta $ is computed as $\beta  =
u_1w_1$, using the value of $u_1$ stored in UPARAM and $w_1$ stored in U(,).

This is essentially Algorithm H12 of \cite{Lawson:1974:SLS} with a change
in the way the user specifies the column/row options.

\bibliography{math77}
\bibliographystyle{math77}

\subsection{Error Procedures and Restrictions}

To define and/or apply a nontrivial transformation one must have
\begin{equation*}
1 \leq \text{LPIVOT} < \text{L1} \leq \text{M}
\end{equation*}
If these conditions are not all satisfied the subroutine returns
immediately, doing no computation. This is not regarded as an error
condition as it may occur intentionally as an end condition in a loop. It
will have the effect of computing and/or applying an identity
transformation.

If, on entry with MODE = 1, the components of the pivot vector indexed by
LPIVOT, and L1 through M, are all zero, the subroutine will effectively
define an identity transformation. It will set UPARAM = 0. This is not
regarded as an error condition.

\subsection{Supporting Information}

The source language is ANSI Fortran~77.

\begin{tabular}{@{\bf}l@{\hspace{5pt}}l}
\bf Entry & \hspace{.35in} {\bf Required Files}\vspace{2pt} \\
DHTCC & \parbox[t]{2.7in}{\hyphenpenalty10000 \raggedright
DHTCC, DNRM2\rule[-5pt]{0pt}{8pt}}\\
DHTGEN & \parbox[t]{2.7in}{\hyphenpenalty10000 \raggedright
DAXPY, DDOT, DHTGEN, DNRM2\rule[-5pt]{0pt}{8pt}}\\
SHTCC & \parbox[t]{2.7in}{\hyphenpenalty10000 \raggedright
SHTCC, SNRM2\rule[-5pt]{0pt}{8pt}}\\
SHTGEN & \parbox[t]{2.7in}{\hyphenpenalty10000 \raggedright
SAXPY, SDOT, SHTGEN, SNRM2}\\
\end{tabular}

Original version designed and programmed by C.  L.  Lawson and R.  J.
Hanson, JPL, 1968, and published in \cite{Lawson:1974:SLS} as subroutine
H12.  Adapted to Fortran~77 for the MATH77 library by Lawson and S.  Y.
Chiu, June~1987.


\begcodenp

\lstset{language=[77]Fortran,showstringspaces=false}
\lstset{xleftmargin=.8in}

\centerline{\bf \large DRSHTCC}\vspace{10pt}
\lstinputlisting{\codeloc{shtcc}}

\vspace{30pt}\centerline{\bf \large ODSHTCC}\vspace{10pt}
\lstset{language={}}
\lstinputlisting{\outputloc{shtcc}}
\end{document}
