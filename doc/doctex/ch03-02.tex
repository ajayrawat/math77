\documentclass[twoside]{MATH77}
\usepackage{multicol}
\usepackage[fleqn,reqno,centertags]{amsmath}
\begin{document}
\begmath 3.2 Gaussian (Normal) Random Numbers and Vectors

\silentfootnote{$^\copyright$1997 Calif. Inst. of Technology, \thisyear \ Math \`a la Carte, Inc.}

\subsection{Purpose}

Generate pseudorandom numbers or vectors from the Gaussian (normal)
distribution.

\subsection{Usage}

\subsubsection{Generating Gaussian (normal) pseudorandom numbers}

\paragraph{Program Prototype, Single Precision}

\begin{description}
\item[REAL]  \ {\bf SRANG, X}
\end{description}
$$
\fbox{{\bf X = SRANG()}}
$$
\paragraph{Argument Definitions}

\begin{description}
\item[SRANG]  \ [out] The function returns a pseudorandom number from the
Gaussian (normal) distribution with mean zero and unit standard deviation.
\end{description}

\subsubsection{Generating Gaussian (normal) pseudorandom vectors}

Given an N-vector, ${\bf \mu }$, and a symmetric positive-definite N$\times $%
N matrix, $A$, the objective is to compute pseudorandom N-vectors, ${\bf x}$%
, from the N-dimensional Gaussian (normal) distribution having mean vector, $%
{\bf \mu }$, and covariance matrix, $A$.

On the first call to SRANGV with a new covariance matrix, $A$, the user must
set HAVEC = .false. to indicate that the Cholesky factor of $A$ has not yet
been computed. SRANGV will replace $A$ in storage by its Cholesky factor, $C$%
, and set HAVEC = .true. to indicate that the array A() now contains $C$
rather than $A$.

On each call to SRANGV an N-vector, ${\bf x}$, is returned as a pseudorandom
sample from the specified distribution.

\paragraph{Program Prototype, Single Precision}

\begin{description}
\item[INTEGER]  \ {\bf NDIM, N, IERR}

\item[REAL]  \ {\bf A}(NDIM$\geq $N, $\geq $N){\bf , U}($\geq $N){\bf , X}($%
\geq $N)

\item[LOGICAL]  \ {\bf HAVEC}
\end{description}

Assign values to A(,), NDIM, N, U(), and HAVEC.

\begin{center}
\fbox{\begin{tabular}{@{\bf }c}
CALL SRANGV(A, NDIM, N, U,\\
X, HAVEC, IERR)\\
\end{tabular}}
\end{center}

Computed values will be returned in X() and IERR, and the contents of A(,)
and HAVEC may be changed.

\paragraph{Argument Definitions}

\begin{description}
\item[A(,)]  \ [inout] When HAVEC is .false., A(,) contains a user-supplied N%
$\times $N symmetric positive-definite covariance matrix, $A$. Only the
diagonal and subdiagonal elements of $A$ need be given. This subroutine will
not access the super-diagonal locations in the array, A(,). This subroutine
will replace $A$ by its lower-triangular Cholesky factor, $C$, and set HAVEC
= .true.

When HAVEC is .true., A(,) is assumed to contain the Cholesky factor, $C$.

\item[NDIM]  \ [in] First dimension of the array A(,). Require NDIM $\geq $
N.

\item[N]  \ [in] Order of the covariance matrix $A$ and dimension of the
vectors U and X.  Require N $\geq $ 1.

\item[U()]  \ [in] Contains the N-dimensional mean vector, ${\bf \mu }.$

\item[X()]  \ [out] On return will contain the N-dimensional generated
random vector.

\item[HAVEC]  \ [inout] See description above for A(,).

\item[IERR]  \ [out] IERR will only be referenced when the subroutine is
entered with HAVEC = .false. If the Cholesky factorization is successful,
the subroutine will set HAVEC = .true. and IERR = 0. Otherwise, it will
leave HAVEC = .false. and set IERR to the index of the row of $A$ in which
failure was noted. In this latter case the results returned in A(,) and X()
will not be useful.
\end{description}

\subsubsection{Modifications for Double Precision}

For double-precision usage, change the REAL statements to DOUBLE PRECISION
and change the initial ``S" of the function and subroutine names to ``D."
Note particularly that if the function name, DRANG, is used it must be typed
DOUBLE PRECISION either explicitly or via an IMPLICIT statement.

\subsection{Examples and Remarks}

The program DRSRANG demonstrates the use of SRANG to compute Gaussian random
numbers and uses SSTAT1 and SSTAT2 to compute and print statistics and a
histogram based on a sample of~10000 numbers delivered by SRANG.

To compute Gaussian random numbers with mean XMEAN and standard deviation
STDDEV, one can use the statement
$$
\text{X = XMEAN + STDDEV }*\text{ SRANG()}
$$
The program DRSRANGV demonstrates the use of SRANGV to compute pseudorandom
vectors from a 3-dimensional Gaussian distribution with mean vector and
covariance matrix specified as%
$$
{\bf \mu } =\left[
\begin{array}{c}
1 \\
2 \\
3
\end{array}
\right] \quad \text{and}\quad A=\left[
\begin{array}{rrr}
0.05 & 0.02 & 0.01 \\
0.02 & 0.07 & -0.03 \\
0.01 & -0.03 & 0.06
\end{array}
\right] .
$$
To fetch or set the seed used in the underlying pseudorandom integer
sequence use the subroutines described in Chapter~3.1.

\subsection{Functional Description}

\subparagraph{Method}

The algorithm for generation of Gaussian random numbers is based on
\cite{Bell:1968:A334}--\nocite{Box:1958:NOG}\cite{vonNeumann:1959:VTU}.
This method draws pairs of uniform random numbers $x$ from
[0,~1] and $y$ from [$-$1,~1] until a pair is obtained that
satisfies $x^2 + y^2 \leq 1$.
The probability of satisfying this constraint is $\pi /4 \approx 0.785$. It
then draws another uniform random number $u$ from [0,~1] and computes
\begin{eqnarray*}
s & = & x^2 + y^2\\
t & = & \sqrt{-2\ \log u} \,/s\\
g_1 & = & t(x^2 - y^2)\\
g_2 & = & 2xyt
\end{eqnarray*}
The numbers $g_1$ and $g_2$ are independent samples from the Gaussian
distribution with mean zero and unit standard deviation. The number $g_1$ is
returned when it is computed and $g_2$ is saved and returned the next time a
Gaussian random number is requested. The saved value will be discarded if
the underlying uniform sequence is reinitialized by a call to RAN1 or RANPUT
of Chapter~3.1.

This method is a mathematically exact transformation from the uniform
distribution to the Gaussian distribution so the statistical quality of the
delivered numbers depends entirely on the quality of the uniform
pseudorandom numbers used. The uniform numbers are obtained by calling
SRANUA or DRANUA, using the array in common block /RANCMS/ or /RANCMD/ as a
buffer as described in Chapter~3.1.

For the generation of Gaussian random vectors we are given a mean vector $%
{\bf \mu }$ and a symmetric positive-definite covariance matrix $A$. The
Cholesky method is used to factor the given covariance matrix $A$ as $A =
CC^t$ where $C$ is a lower triangular matrix. Then for each vector to be
delivered the method first constructs a vector ${\bf g}$ whose components
are independent samples from the Gaussian distribution with mean zero and
unit standard deviation. It then computes
$$
{\bf x} = {\bf \mu }+C{\bf g}
$$
which is a sample from the N-dimensional Gaussian distribution with mean $%
{\bf \mu }$ and covariance matrix $A$.

Values returned as double-precision random numbers will have random bits
throughout the word, however the quality of randomness should not be
expected to be as good in a low-order segment of the word as in a high-order
part.

\bibliography{math77}
\bibliographystyle{math77}

\subsection{Error Procedures and Restrictions}

While computing the Cholesky factorization of $A$, subroutine SRANGV or
DRANGV may determine that the matrix is not positive-definite. In that case
it will return with IERR set to the index of the row of the matrix at which
the problem was detected. In this non-positive-definite case the contents of
the array A(,) will have been altered, HAVEC will still have the value
.false., and X() will not contain useful results.

When the Cholesky factorization is successful, IERR will be set to zero.

The conditions that N $\geq $ 1, and NDIM $\geq $ N are required but
not checked.

\subsection{Supporting Information}

The source language is ANSI Fortran~77.

\begin{tabular}{@{\bf}l@{\hspace{5pt}}l}
\bf Entry & \hspace{.35in} {\bf Required Files}\vspace{2pt} \\
DRANG & \parbox[t]{2.7in}{\hyphenpenalty10000 \raggedright
DRANG, ERFIN, ERMSG, RANPK1, RANPK2\rule[-5pt]{0pt}{8pt}}\\
DRANGV & \parbox[t]{2.7in}{\hyphenpenalty10000 \raggedright
DRANG, DRANGV, ERFIN, ERMSG, RANPK1, RANPK2\rule[-5pt]{0pt}{8pt}}\\
SRANG & \parbox[t]{2.7in}{\hyphenpenalty10000 \raggedright
ERFIN, ERMSG, RANPK1, RANPK2, SRANG\rule[-5pt]{0pt}{8pt}}\\
SRANGV & \parbox[t]{2.7in}{\hyphenpenalty10000 \raggedright
ERFIN, ERMSG, RANPK1, RANPK2, SRANG, SRANGV}\\\end{tabular}

Based on subprograms written for JPL by Carl Pitts, Heliodyne Corp., April,
1969. Adapted to Fortran~77 for the JPL MATH77 library by C. L. Lawson and
S. Y. Chiu, JPL, April~1987. November~1991: Lawson reorganized and renamed
common blocks. Improved coordination between Gaussian and uniform sequences
on reinitializations.


\begcodenp

\lstset{language=[77]Fortran,showstringspaces=false}
\lstset{xleftmargin=.8in}

\centerline{\bf \large DRSRANG}\vspace{10pt}
\lstinputlisting{\codeloc{srang}}
\newpage

\vspace{30pt}\centerline{\bf \large ODSRANG}\vspace{10pt}
\lstset{language={}}
\lstinputlisting{\outputloc{srang}}
\lstset{language=[77]Fortran,showstringspaces=false}
\lstset{xleftmargin=.8in}
\newpage

\centerline{\bf \large DRSRANGV}\vspace{10pt}
\lstinputlisting{\codeloc{srangv}}
\newpage

\vspace{30pt}\centerline{\bf \large ODSRANGV}\vspace{10pt}
\lstset{language={}}
\lstinputlisting{\outputloc{srangv}}
\end{document}
