\documentclass[twoside]{MATH77}
\usepackage{multicol}
\usepackage[fleqn,reqno,centertags]{amsmath}
\begin{document}
\begmath 19.4  Converting Codes to Different Versions

\silentfootnote{$^\copyright$1997 Calif. Inst. of Technology, \thisyear \ Math \`a la Carte, Inc.}

\subsection{Purpose}

This program, {\tt m77con}, converts codes from one version to another.
It is used to convert between different precisions, and to support the
special comments used in some of the Fortran codes for input to a program
that converts programs from Fortran to C.  Support is provided for
precision greater than double for systems that support such a feature.
Support is provided for conversion between real programs and integer or
complex versions of a program, and also partial double precision and
partial quadruple precision.  Most MATH77 codes contain the comments only
for conversion between the single and double precision cases.  Although
designed for use with MATH77 codes, {\tt m77con} could be used in other
applications that want a portable means to support multiple versions.  Any
version of a code can be used as input to {\tt m77con} for the purpose of
getting another version.

\subsection{Usage}

    One uses this program by creating a control file named {\tt
m77job}, and executing the program {\tt m77con} from the command
line.  The first part of this section describes the control file which
defines what version is desired.  The second part describes the
comments in the code which indicate how different versions are to be
obtained.

\subsubsection{Transforming an Instrumented Code}

In the descriptions below, items in brackets are optional; $L_i$ is
used for letters (frequently the first letter in the name of a
program, which indicates the precision); and $T_i$ is used for
letters associated with types. Types supported include: S, D, Q, I,
C, Z, and W, which are used for Single precision, Double precision,
Quadruple precision (or at least some  precision greater than
double), Integer, Complex, double precision complex, and quadruple
precision complex respectively.

``C...'' is a comment and can appear anywhere in the control file.  A
comment of the form ``C$>>$yyyy$-$mm$-$dd...'' is treated as a special
case.  It inserts a date stamp in the file being processed, which is
used to track changes in MATH77 codes. Details on this can be found in
the code listing for {\tt m77con}.

``MAKE $L_{out}$ [with type $T_{out}$] [from $L_{in}$ [with type
$T_{in}$]]'' is used to define a letter in the names of the output
and input files and types associated with these files.  $L_{out}$ is
a letter used in output file names.  $T_{out}$ is the type for the
output file. If $T_{out}$ is not given it is assumed to be $L_{out}$
if $L_{out}$ is an allowed type, is assumed to be S or D, for
$L_{out}$ =  P or X respectively, and otherwise an error results.  (P
is assumed to be used for partial double precision, and X for partial
quadruple precision.)  $L_{in}$ and $T_{in}$ are defined as for
$L_{out}$ and $T_{out}$, except they apply to input files.  If the
``from'' clause is missing, $L_{in}$ is D, except when $L_{out}$ is
D, in which case $L_{in}$ is S. The case of letters is not
significant, except for $L_{out}$.  The case used for $L_{out}$
determines the case used when substituting $L_{out}$ or $L_{in}$ into
a file name, and thus the case is significant on systems where case
is significant in a file name.

``$T$ Defined by: $L_{exp}$, {\em keyword}, {\em type\_conv}, {\em
mach\_const}\/''\newline
``$T$ Defined by: {\em keyword}, {\em type\_conv},
{\em conj}, {\em imag\_part}\/''\newline
One of these lines is needed only if the input or output code makes
use of a nonstandard type, namely, Q, Z, or W.  Note that W makes use
of Q, and thus if it is used, Q must also be defined.  The first type
of line is used to define real (as opposed to complex) types.  $L_{exp}$
is the letter used to define the exponent part of floating point
constants, {\em keyword} is the text used to declare a variable of
type $T$, {\em type\_conv} is the name of the function used for type
conversion to type $T$, {\em mach\_const} is the name of the function
used for getting machine constants, {\em conj} and {\em imag\_part}
are the functions used to conjugate or get the imaginary part of a
complex expression, respectively.  The standard S, D, and C types are
initialized as if one had input:\newline ``S Defined by: E, real,
real, r1mach'',\newline ``D Defined by: D, double precision, dble,
d1mach'', and \newline ``C Defined by: complex, cmplx, conjg, aimag''

Note that at most one embedded blank is allowed inside {\em keyword},
and if such a blank is desired it must be included on this line.
Internally, `I' defines {\em keyword} as ``integer'', and {\em
type\_conv} as ``int''.

``VERIFY path'' if present will compare the lines of the result files
with lines in the files of the same name as the result files preceded
by ``path'', and when lines are different, both lines will be printed.
The text in ``path'' is taken exactly as given, and is considered
terminated by either a blank or an end of line.  This comparison is
not sensitive to changes in the special comment lines described in
the following subsection, or to differences in the number of
blanks.

``SET $A_1$, $A_2$, ... $A_n$'' is used to set values for the
metavariables defined by the assignments $A_i$.  These assignments, which
sometimes are necessary to get the version of a code desired, and the
metavariables set by these assignments, are described in the following
subsection.  One can have any number of ``SET...'' lines.

``STOP ...'' acts like an end of file, i.e. no more lines are read in
the control file.

``FILE {\em name}[[,] {\em ext}]'' is used to specify the name of a file
to be processed.  A ``?'' in name may be used to parameterize the name if
desired.  $L_{in}$ is substituted for the ``?'' (if any) and a file of
this name is processed as the input file.  Then $L_{out}$ is substituted
for ``?'' to obtain the name of the output file.  If the ``,~{\em ext}\/''
is present and one of the first two characters in {\em ext} is ``.'' this
name is further modified by searching backward from the end of ``{\em
name}\/'' looking for the first character in ``{\em ext}\/''.  When this
character is found, this character and the following characters in {\em
name} are replaced by ``{\em ext}\/''.  If the ``,~{\em ext} is present
and neither of the first two characters is a ``.'', {\em ext} is used as
the name of the output file.  The input and output files can have the same
name.  If the ``?'' is missing, there is no need to have had a ``MAKE...''
line.  Any characters except ``,'' can appear in ``{\em name}\/'' or
``{\em ext}\/'', as long as the results will serve as valid file names.
Examples and Remarks, below, gives samples of FILE statements.

Processing a ``FILE...'' line involves carrying out conversions on
the specified input file and placing the result in the output file.
If a ``VERIFY...'' line is present, the output file will be compared
with the file specified there.  After processing a ``FILE ...'' line,
the next line from the control file, {\tt m77job}, is read, and
actions are taken as requested there.  All metavariables defined in
previous SET instructions become undefined if there is a new SET
instruction; otherwise the old values are kept.

\subsubsection{Instrumenting a Code}

We use the notation:

\begin{tabular}{l@{ }l@{ }ll}
$L$,&$L_1$,&... $L_i$ & Letters\newline\\
$T$,&$T_1$,&... $T_i$ & Letters used to denote types\newline\\
$I$,&$I_1$,&... $I_i$ & Integer constants\newline\\
$V$,&$V_1$,&... $V_i$ & Metavariables\newline\\
$E$,&$E_1$,&... $E_i$ & Expressions\newline\\
$A$,&$A_1$,&... $A_i$ & Assignments\newline\\
\end{tabular}

The types are: S, D, Q, I, C, Z, and W, which are used for Single
precision, Double precision, Quadruple precision (or at least some
precision greater than double), Integer, Complex, double precision
complex, and quadruple precision complex respectively.  An ``$L$''
equal to one of these letters is used for a code of the corresponding
type.  Other letters are used for mixtures of types.  In particular,
P (Partial double) is used for a code that uses some D in a code that
is primarily S (perhaps to save on storage), and X (eXtended) is used
for a code that is primarily in D, but uses some Q.  No other
precisions are supported.

There are integer, logical, and string metavariables and constants.
An integer constant is defined as in Fortran.  The logical constants
are .T.\, .F.\, .U.\, .D.\, .S.\, .I.\, and .C.\ Of course .T.\ is
always true, and .F.\ is false; .U.\ is undefined (see below); .D.\,
.S.\, and .I.\ have a value of .T.\ if the upper case of the letter
$L_{out}$ in a MAKE command matches the letter between the dots, has a
value of .F. if $L_{out}$ is defined but does not match, and otherwise
has the value .U.; and .C.\ has the value .T.\ if a version of the
code used as input to the Fortran to C converter is being generated,
and is .F.  otherwise.  Initially .C.\ =.F.; its value can be set to
.T.\ with the statement ``SET .C.'' in {\tt m77job}, and can be reset
to .F.\ with the statement ``SET $\sim $.C.''.  (With T=.T., F=.F.,
and U=.U., we have, U$|$F=.U., U$|$T=.T., U\&F=.F., U\&T=.U., $\sim
$U=.U.  If U is on either side of a relational operator, the result is
.U.)

A string is defined as an arbitrary sequence of characters enclosed by
either apostrophes ({\tt '}), or quotes ({\tt "}), has the same form as a
metavariable (see below), or is .X., .Y., or .N.\ \ \ .X.  gives the
string consisting of the letter used for floating point exponents in the
output, .Y.\ gives the string consisting of the letter $T_{out}$, and
.N.\ gives the string consisting of the letter $L_{out}$.  If a
logical value of a string is required by the context, a string has the
value .U.

A metavariable is any number of nonoperators, not all of which are
digits, and not one of the letters ``TFUXN'' preceded and followed by
a ``.''.  Metavariables once defined remain defined until the end of
the file. Those metavariables defined in {\tt m77job} remain defined
past the end of file, until the next ``SET...'' line read in {\tt
m77job}. Note that metavariables appear only in ``C++...'' lines
and in lines immediately following ``C++ Substitute for ...'' lines.

Expressions may use the following operators, all of which have the same
meaning and precedence as their Fortran counterparts.

\begin{tabular}{ll}
Arithmetic:&        + \hspace{8pt} $-$ \hspace{8pt} * \hspace{8pt} /
 \hspace{8pt} **\\
Relational:&        $<$ \hspace{8pt} $>$ \hspace{8pt} == \hspace{8pt} /=
 \hspace{8pt} $<=$ \hspace{8pt} $>=$\\
Alt. Relational:&         = \hspace{8pt} $\sim $=\\
Logical:&           $|$ (or) \hspace{8pt}  \& (and) \hspace{8pt} $\sim $ (not)\\
Concatenation:&     // \\
Grouping letters:&  $\{$ \hspace{8pt}  $\}$\\
Other:&             ( \hspace{8pt}  ) \hspace{8pt} ,\\
\end{tabular}

The ``$\{$'' and ``$\}$'' are used to define a special operand.  If
the letter $L_{out}$ from the ``MAKE...'' line, see ``Transforming an
Instrumented Code'' above, appears inside the \{...\}, then this
operand has the value true, and otherwise has the value false.  The
letters inside the \{...\} may optionally be separated by commas.
Thus \{SP\} is true if $L_{out}$ is either S or P, and is false
otherwise.

We recommend using ``=='' rather than ``='' in relational
expressions.  ``='' is allowed for compatibility with comments
written for an earlier processor. The first ``='' in an assignment
statement is part of the assignment statement, not part of an
expression.

An expression can be formed as one would expect using the operators
that are available.  A diagnostic is given if one tries to mix types
in a way that make no sense.  Integer and logical are converted to
strings when they appear as an operand of ``//''.

As assignment has the form: $V = E$, or simply $V$.  The latter form is
equivalent to writing $V$=.T.

Three kinds of comments in the source code initiate special processing by
the program.  (Not counting the ``C$<<$...'' comments which are used for
date stamps.) Those headed by ``C++'' are intended primarily to support
different versions of a program.  Those headed by ``C$--$'' are intended
primarily to support different precisions, which involves changing types in
Fortran statements.  Those headed by ``C\%\%'' are used to convert the
code to the form needed for the Fortran to C conversion program.  The
actions for ``C++...'' lines precede other actions, and thus a ``C++...''
line could result in turning off a ``C$--$...'' line.  Also, if there was
no ``MAKE...'' line, type conversion is turned off, and ``C$--$...'' lines
are simply treated like any other comments.  These comments allow either
case for the ``C'' and also allow an ``*'' in column one.

``C++...'' lines have the form:

\begin{tabular}{@{}l@{ }l@{ }l}
(1) & ``C++ & Code for $E$ is active''\\
(2) & ``C++ & Code for $E$ is inactive''\\
(3) & ``C++ & END''\\
(4) & ``C++ & Of next $I$ lines, only the first $E$ are active''\\
(5) & ``C++ & Default $A_1$, $A_2$, ... $A_i$'', or\\
    & ``C++L & Default $A_1$, $A_2$, ... $A_i$'', or\\
    & ``C++(E) & Default $A_1$, $A_2$, ... $A_i$''\\
(6) & ``C++ & Current has A''\\
(7) & ``C++ & Substitute for $V_1$, $V_2$, ... $V_i$ below''\\
(8) & ``C++ & Replace $A_1$, $A_2$, ... $A_i$''\\
(9) & ``C++ & [With first index $E_1$,]\\
    &       & Save data by elements if $E_2$''
\end{tabular}

A line of form (1) or (2) initiates a scope of applicability that is
terminated by the next line of form (1), (2), or (3).  In either case
(1) or (2), if $E$=.U., no change is made in this line or in the
lines in its scope.

In the case of (1), if $E$=.T. there is no change.  If $E$=.F.,
the word ``active'' is changed to ``inactive'' and in the following
scope, columns 1--71 are shifted to columns 2--72 and a C placed in
column 1.  An error message is printed if the original line was not
blank in column 72.  (Only the first 72 characters of input lines are
read.)

In the case of (2), if $E$=.F. there is no change.  If $E$=.T. the
word ``inactive'' is replaced by ``active'' and in the following
scope, columns 2--72 are copied to columns 1--71, and a blank placed
in column 72.

No other ``C++...'' line may appear between a line of type (1) or (2)
and the next line of type (1), (2), or (3).  No code changing action
can be active when the end of a Fortran program unit is reached.

Examples from AMACH of lines of type (1), (2) and (3) are,

\begin{verbatim}
c++ Code for SYS = IEEE is ACTIVE
...
c++ Code for SYS = AMDAHL is INACTIVE
C ...
c++   END
\end{verbatim}

In case (4), one must have $I$ (an integer constant) $>$ 0, and $E$
must have an integer value $ <= I$.  The following $I$ lines in the
code must be valid Fortran statements if column~1 is set to a blank
and there must be $I$ lines preceding the last line of the Fortran
program unit. This causes the action of storing a ` ' in column~1 of
the following $E$ lines, and then storing a `C' in column~1 for the
remaining $I - E$ lines (neither shifts the line).

An example from DIVA of using this feature is (the first two lines
given here are one line in the code):

\begin{verbatim}
c++ Of next 23 lines, only the first
                        KDIM+MAXORD are active
      data B(1)  / 5.00000...0D-1 /
      ...
      data B(22) / 1.97628...5D-3 /
C     data B(23) / 1.81159...2D-3 /
\end{verbatim}

In case (5) when no L or (E) is present each assignment, $A_i$, is
executed if and only if the $V_i$ on the left of the ``='' in the $A_i$ is
not already defined.  (Recall that if $A_i$ does not contain an ``=''
sign, it is as if it had the form ``$V_i$ = .T.''.)  $V_i$ may have been
defined either from a ``SET...'' line in {\tt m77job} or from an
earlier ``C++'' line.  When L is present, L must be a sequence of
letters and the assignments will be processed as described for the
case when no L is present if and only if the current $L_{out}$ matches
one of the letters in L.  When (E) is present, the line is processed as
above if E is a logical expression with the value .T., else the line rest
of the line is ignored.

The following example from DRDIVA illustrates defining a default
value for NDIG of 4 for type of ``S'' or ``P'', and a default value for NDIG
of 10 in other cases.  If NDIG has been set in {\tt m77job}, then
the value set there will be used.

\begin{verbatim}
c++SP Default NDIG = 4
c++  Default NDIG = 10
\end{verbatim}

In case (6), the A must have an ``='' sign.  If the $V$ on the left
side of the ``=''  in A is not defined this works just like a
``C++ Default A'' line.  If $V$ is defined, then the text on the
right of the ``='' is replaced by the current value of $V$.

The following line from AMACH shows that AMACH is currently
configured for SYS = IEEE.  If SYS is not set in {\tt m77job}, this
value for SYS will continue to be used.  Otherwise the new value will
replace the {\tt IEEE} on this line.

{\tt c++ CURRENT HAS SYS = IEEE}

In case (7), the following line is examined for each of the $V_i$'s in
turn.  If such a metavariable name is found followed by an ``='' sign, and
an integer value, the integer is replaced by the integer value of $V_i$.
If the ``='' sign is followed by a logical value, this value is replaced
by the corresponding Fortran logical constant for $V_i$.  A type mismatch
in either of these cases results in a diagnostic, with no change.  A
value of .U. in this case is treated as a mismatch.

The following from DRDIVA illustrates using this feature.

\begin{verbatim}
c++ Substitute for NDIG below
      parameter (NDIG = 10)
\end{verbatim}

In case (8), the left side of each $A_i$ must be a constant character
string, i.e. either {\tt '}xx...x{\tt '} or {\tt "}xx...x{\tt "}, and
the right side, $E_i$,  must be an expression that evaluates to a
string value.  Occurrences (case of letters must match) of the
``xx...x'' on the left of $A_i$ are replaced by the value for $E_i$.
The replacement is made on ``C++ Replace...'' lines, and other lines
not starting with ``C++''.  If this line precedes a ``C$--$...'' line
of type (1), see below, then the replacement applies starting with
the first line of the file else it applies starting with the current
line.

The following from DRDRAN illustrates this feature.

\begin{verbatim}
c++S Default NDIG = 6
c++  Default NDIG = 12
c++  Replace "f15.12" = "f"//NDIG+3//"."//NDIG
...
      print'(7x,i7,'',''i5,10x,f15.12)',K...
\end{verbatim}

Case (9) is intended primarily to avoid unnecessary mess in Fortran code
that would have too many continuation lines for a data statement in
Fortran when declared with a single array.  The C converter we are using
generates much better code when such data statements are declared with a
single array instead of a separate statement for each element.  $E_1$ must
be a integer expression, and $E_2$ must be logical.  If the part
containing $E_1$ is missing the action is as if $E_1$ were present with a
value of 1.  The data statement following this must contain exactly one
data item per line.  When $E_2$ = .T., the first line and every following
line that is either a continuation line, or is a data statement for the
same name, is changed (if necessary) to be a data statement for a single
element of the array.  The index used for the array element is $E_1$ for
the first data statement and is incremented by one for the following
statements.  If $E_1$ is .F.  the first data statement line uses only the
array name, and following lines that are either continuation lines, or
data statements for the same name are changed to be continue statements.
There is no limit on the number of continued lines.

The ``C$--$...'' lines have the form:

\begin{tabular}{@{}lll}
(1) & ``C$-$$-$$L$ & Replaces ``?'': $V_1$, $V_2$, ...''\\
(2) & ``C$-$$-$$T$ & (Type)Replaces ``?'': $V_1$, $V_2$, ...''\\
(3) & ``C$-$$-$\& & $v_1$, $v_2$, ...''\\
(4) & ``C$-$$-T$ & Next line special: $S_1$, $S_2$, ...''\\
(5) & ``C$-$$-$ & Begin mask code changes''\\
(6) & ``C$-$$-$ & End mask code changes''\\
\end {tabular}

At most one of (1) or (2) can occur in a file, and all lines of type
(3) must follow immediately after a (1) (2) or (3).  (No embedded
comments.) In case (1) $L$ denotes the value of $L_{out}$ for the
current version of the code.  (See the description of the ``MAKE...''
line in Usage above.) If $L$ = $L_{out}$, then no names are changed.
When $L=L_{in}$ the $L$ in this statement is replaced by $L_{out}$.
Each of the $V_i$, or $v_i$, must be Fortran names with one of the
letters replaced by a ``?''.  These names with the ``?''  replaced by
$L_{in}$ are searched for and replaced as described below, starting
at the beginning of the file.  A line of type (2) must follow
immediately after a line of type (1) or (3).  The action for this
kind of line is similar to that for type (1),  but with $L$,
$L_{out}$, and $L_{in}$, replaced by $T$, $T_{out}$, and $T_{in}$
respectively.

The following excerpt from DWCOMP illustrates lines of type (1) and
(3).

\begin{verbatim}
c--D replaces "?": ?WCOMP,?WATAN,...
c--&   ?WSQRT,?WEXP,?WSIN,?WCOS,...
\end{verbatim}

Case (4) is used to indicate either that something different from the
standard conversion is to be used on the next line, or that
conversion to or from integer or one of the complex types is allowed.
Conversions to or from integer types occur only on lines so marked.
For a conversion to or from a code corresponding to the first letter
of one of the pairs, (C, S), (Z, D), (W, Q), the conversion is done
as if the second letter applied, except for lines marked in this
manner.  $T$ indicates the type to assume for the input on the
following line. The $T$ is replaced by the value used for the type of
the output on the next line.  The $S_i$ is either a single $L_i$, or
an expression of the form $L_i$ $=>$ $T_i$. If $L_{out}$ is the same
as any of these $L_i$, then the following line is treated as special
based on this $S_i$.  If there is no ``$=>$ $T_i$'' part then
presumably $L_i$ is I, C, Z, or W, and the following line is
modifiable to suit these types.  If the ``$=>$ $T_i$'' part is
present, then the following line is to be converted as if the output
type were $T_i$.  If none of the $L_i$ match the $L_{out}$ from the
``MAKE...'' line, the following line is treated just like any other
line.

The following from DRDIVA illustrates specifying metavariables with a
higher than usual precision when $L_{out}$ is ``P'' or ``X''.  In all
other cases the declaration here would be what one would expect
from the letter specifying the precision.

\begin{verbatim}
c--D Next line special: P=>D, X=>Q
      double precision TSPECS(4), Y(IYDIM)...
\end{verbatim}

?VECPR is one of the few codes in MATH77 which allows all the real
precisions, and also allows integer.  The first pair of lines below
get converted to the second pair when DVECPR is converted to IVECPR.

\begin{verbatim}
c--D Next line special: I
      double precision  V(N)

c--I Next line special: I
      integer           V(N)
\end{verbatim}

Lines of type (5) and (6) mark the beginning and end, respectively,
of a set of lines in which all code modifications due to ``C$--$''
commands will be inhibited.

The ``C\%\%'' lines are documented here so that those looking at the
library source will understand their purpose.  These lines only have a
special meaning when they are not being treated in a special way for some
other reason.  In particular, if a ``C++ END'' statement would be
acceptable, then such lines are not treated in a special way.  When .C.\ =
.T., the ``C'' is deleted from column one in any sequence of consecutive
lines containing ``C\%\%'', and a C is inserted at the start of the
first line (one line only) that does not start in this way.  If .C.\ =
.F., and the lines are not being treated in a special way for some other
reason, then consecutive lines starting with ``\%\%'' have a C inserted
in column 1, and the next line has the character in column 1 deleted.
These lines provide a way of indicating changes required to get the code
converted to C, without using the ``C++'' mechanism, which makes the code
difficult to read when used frequently.

\subsection{Examples and Remarks}

If one wanted to get the partial double precision version of the
integration program DIVA, Chapter~14.1, one could set {\tt m77job} to
the following for a unix machine.  (For a PC replace the {\tt .f}
with {\tt.for}.)

{\tt \begin{tabbing}
  MAKE p\\
  FILE ?iva.f\\
  FILE ?ivag.f \\
  FILE ?ivadb.f
\end{tabbing}}

To convert {\tt amach} for a VAX running UNIX, {\tt m77job} would
contain the following to obtain the output file {\tt amach.f} from
the input file {\tt amach}.

{\tt \begin{tabbing}
  SET SYS = VAX\\
  FILE amach.f
\end{tabbing}}

\subsection{Functional Description}

The program {\tt m77con} converts between different versions of Fortran~77
source code that satisfy conventions used for MATH77.  It combines
functionality from earlier programs: ``The Specializer''
\cite{Krogh:1974:ALS}, ``MARVEL'' \cite{Lawson:1980:MAR}, and ``CHGTYP''
\cite{Lawson:1988:CHG}.

The code is processed a line at a time, first doing actions on
``C++...'' lines, then for other lines doing the actions called for
by ``C++...'' lines (except for replacing strings), then doing
actions on ``C$-$$-$...'' lines, and then doing other replacements.
Processing restarts with the first line after initial processing of
``C$-$$-$'' lines of types 1--3.

Except for strings defined by ``C++ Replace...'' lines, all
comparisons done by this code are case insensitive, all replacements
are done in the same case as for the text found, and blanks are
removed before comparisons are made in the source code.

Let ``in'' denote a string, keyword, variable name, or function name
we are searching for, and ``out'' denote the corresponding
replacement.  When conversion results in substitutions, if ``out'' is
longer than ``in'', characters to the right of ``in'' are shifted
right to leave enough space to insert ``out''.  A diagnostic
results if in the process nonblank characters are shifted beyond
column 72 (and thus lost).  If ``out'' is shorter than ``in'', then
``out'' is padded with blanks so as to occupy the same space as
``in''.

When type conversion is on (as it is if there is a ``MAKE...'' line
in the control file), then all lines, except those with ``++'' or
``$--$'' in columns 2 and 3, are examined in turn for the following:

If the ``in'' {\em keyword} is found, it is replaced by the ``out''
{\em keyword}.

Then ``in'' default function names for type conversion and getting
machine constants (see ``$T$ Defined by:...'' lines in Usage above)
are searched for, and when found are replaced by the corresponding
names of the ``out'' functions.  When searching for names (here and
below) they are only found if they are preceded and followed by a
nonletter or nondigit. Function names which do not have ``mach'' as
their last four letters are found only if followed by a `('.

Then the $V_i$'s and $v_i$'s generated from the ``C$--$L...'',
``C$--$T...'', and ``C$--$\&'' lines are searched for.  When found,
$L_{out}$ (or $T_{out}$ for names defined on ``C$--$T (Type)...''
lines or ``C$--$\&'' lines following such lines) replaces the letter
in the position of the ``?'', preserving the case of the letter that
was there.

Then, those strings defined by the ``C++ Replace...'' lines are
searched for and replaced if found.

Then, floating point constants are converted.  These are assumed to
consist of (not a letter).(0 or more digits)(the letter used for
floating point constants in the current version)(an optional + or
$-$)(a digit).  If such a string is found, the letter is replaced as
specified.

Whether type conversion is on or not, lines consisting of only ``END'' are
searched for.  When found, {\tt m77con} checks if there are any ``C++...''
actions still active or if the processing is not in ``usual precision''
mode.  In either of these cases a diagnostic is given and the processing
returns to normal mode.

\bibliography{math77}
\bibliographystyle{math77}

\subsection{Error Procedures and Restrictions}
    The program prints a variety of error messages on the standard
output. An error in the control file causes the program to stop.  In
most cases of errors in the special comments of the files being
processed, processing continues.  After 10 comparison errors in
a single file, no more comparison errors are printed for that file.

\subsection{Supporting Information}

The source language is ANSI Fortran~77.

Design and code due to F. T. Krogh, JPL, October~1994.  Multiple minor
modifications through April~1996.

\end{multicols}
\end{document}
