\documentclass[twoside]{MATH77}
\usepackage[\graphtype]{mfpic}
\usepackage{multicol}
\usepackage[fleqn,reqno,centertags]{amsmath}
\begin{document}
\opengraphsfile{pl02-18}
\begmath 2.18 Digamma or $\psi $ Function

\silentfootnote{$^\copyright$1997 Calif. Inst. of Technology, \thisyear \ Math \`a la Carte, Inc.}

\subsection{Purpose}

The procedures described here compute the digamma or $\psi $ function
defined by $\psi (z) = d[\ln \Gamma (z)]/dz = \Gamma ^{\prime}(z)/\Gamma (z)$%
. Additional procedures, not necessary for simplest usage, are provided to
specify unusual options or retrieve error estimates.

\subsection{Usage}

\subsubsection{Program Prototype, Single Precision}

\begin{description}
\item[REAL]  \ {\bf U , X, SPSI}

\item[EXTERNAL]  \ {\bf SPSI}
\end{description}

Assign a value $x$ to X, and obtain U $= \psi (x)$ by using
$$
\fbox{{\bf U = SPSI(X)}}
$$

\subsubsection{Argument Definitions}

\begin{description}
\item[X]  \ [in] The argument of the function, $x$ above.
\end{description}

\subsubsection{Program Prototype, Single Precision, Specify Unusual Options}

\begin{description}
\item[REAL]  \ {\bf TOL, XERR}

\item[INTEGER]  \ {\bf MSGOFF}
\end{description}

Assign values to TOL, XERR and MSGOFF, and specify options for SPSI by using
$$
\fbox{{\bf CALL SPSIK (TOL, XERR, MSGOFF)}}
$$

\subsubsection{Argument Definitions}

\begin{description}
\item[TOL]  \ [in] Relative error tolerance for $\psi (x)$. When positive,
it indicates relative error tolerance. When negative or zero, it indicates
the default, equal to the square root of the round-off level, $\sqrt{\rho }$%
, should be used. $\rho $ is the smallest number such that the floating
point representation of $1.0+\rho \neq 1.0$. TOL only affects the threshold
for producing error messages; it does not affect the accuracy of computation.

\item[XERR]  \ [in] If non-negative, XERR provides the estimated relative
error in X. If negative, XERR indicates the default error estimate for X,
the round-off level, $\rho $, should be used.

\item[MSGOFF]  \ [in] MSGOFF is added onto the error message level before an
error message is produced by using the error message processor described in
Chapter~19.2.
\end{description}

If SPSIK is not called, the effect is as though CALL SPSIK~(0.0, $-$1.0,~0)
had been executed.
\vspace{10pt}

\hspace{5pt}\mbox{\input pl02-18 }

\subsubsection{Program Prototype, Single Precision, Determine Error Estimate}

\begin{description}
\item[REAL]  \ {\bf ERR, IERFLG}
\end{description}

Retrieve the relative error committed by the last call to SPSI, and the
internal error indicator, by using
$$
\fbox{{\bf CALL SPSIE (ERR,IERFLG)}}
$$

\subsubsection{Argument Definitions}

\begin{description}
\item[ERR]  \ [out] reports the relative error committed by the last call to
SPSI. If SPSI has not been called, ERR is returned with the value $-$1.

\item[IERFLG]  \ [out] reports the value of the internal error flag. If SPSI
has not been called, or if computation on the last call to SPSI was
completely successful, IERFLG is returned with the value zero. See Section
E below for discussion of non-zero values of IERFLG.
\end{description}

\subsubsection{Modifications for Double Precision}

Change the REAL statements to DOUBLE PRECISION and change the first letter
of the procedure names from S to D. One must declare DPSI to be DOUBLE
PRECISION, as its default type would be REAL.

\subsection{Examples and Remarks}

See DRDPSI and ODDPSI for an example of the usage of DPSIK, DPSI and
DPSIE.

For $x \geq 0$, $\psi (x)$ is a monotone increasing function of $x$. $\psi
(x)$ has singularities for $x = 0$ or $x$ a negative integer. See
\cite{ams55} for a discussion of other properties of $\psi (x).$

\subsection{Functional Description}

The computational methods used in SPSI were developed by L. Wayne Fullerton
of Los Alamos National Scientific Laboratory. The methods include rational
Chebyshev expansions, recurrences, and a reflection formula.

When X $>$ 0 SPSI estimates the relative error in $\psi ($X) is the
round-off level.  Otherwise, SPSI computes an estimate for the
relative error in $\psi ($X$) = ($relative error in X$) \times |\text{X}|
\times |d\psi /d\text{X}|\ /\ |\psi (\text{X})|$, with $d\psi /d\text{X}$
approximated by $\pi ^2 \cot ^2 \pi \text{X}$.  When X $<$ 0 and X is
approximately an integer, or $\psi ($X) is approximately zero, the
error will be large.   When X $< < $ 0 and $2\text{X}$ is not approximately an
integer, SPSI may underestimate the error it committed.

\bibliography{math77}
\bibliographystyle{math77}

\subsection{Error Procedures and Restrictions}

If SPSI estimates the specified error tolerance is not satisfied (because $x$
is too near a negative integer), $\psi (x)$ is computed, but an error
message is issued by using the error message processor described in Chapter
19.2, with LEVEL = 1 + MSGOFF, where MSGOFF is zero unless specified by a
call to SPSIK at some time before calling SPSI. The IERFLG output from SPSIE
will be $-$3.

If SPSI is called with X zero or a negative integer, $\psi (x)$ is not
defined, and an error message is issued by using the error message processor
described in Chapter~19.2, with LEVEL = 2 + MSGOFF. If error termination is
suppressed by calling SPSIK with a negative value of MSGOFF, or by calling
the ERMSET procedure described in Chapter~19.2, the function value will be
zero, and the IERFLG output from SPSIE will be~1 if X is zero, or~2 if X is
a negative integer.

\subsection{Supporting Information}

All code is written in ANSI Standard Fortran~77.

The program units SPSI, SPSIB, SPSIE and SPSIK communicate by way of a
common block /SPSIC/. The program units DPSI, DPSIB, DPSIE and DPSIK
communicate by way of a common block /DPSIC/.

\begin{tabular}{@{\bf}l@{\hspace{5pt}}l}
\bf Entry & \hspace{.35in} {\bf Required Files}\vspace{2pt} \\
DPSI & \parbox[t]{2.7in}{\hyphenpenalty10000 \raggedright
AMACH, DCSEVL, DERM1, DERV1, DINITS, DPSI, ERFIN, ERMSG, IERM1,
IERV1\rule[-5pt]{0pt}{8pt}\rule[-5pt]{0pt}{8pt}}\\
DPSIE & \parbox[t]{2.7in}{\hyphenpenalty10000 \raggedright
AMACH, DCSEVL, DERM1, DERV1, DINITS, DPSI, ERFIN, ERMSG, IERM1,
IERV1\rule[-5pt]{0pt}{8pt}\rule[-5pt]{0pt}{8pt}}\\
DPSIK & \parbox[t]{2.7in}{\hyphenpenalty10000 \raggedright
AMACH, DCSEVL, DERM1, DERV1, DINITS, DPSI, ERFIN, ERMSG, IERM1,
IERV1\rule[-5pt]{0pt}{8pt}\rule[-5pt]{0pt}{8pt}}\\
SPSI & \parbox[t]{2.7in}{\hyphenpenalty10000 \raggedright
AMACH, ERFIN, ERMSG, IERM1, IERV1, SCSEVL, SERM1, SERV1,
SINITS, SPSI\rule[-5pt]{0pt}{8pt}\rule[-5pt]{0pt}{8pt}}\\
SPSIE & \parbox[t]{2.7in}{\hyphenpenalty10000 \raggedright
AMACH, ERFIN, ERMSG, IERM1, IERV1, SCSEVL, SERM1, SERV1,
SINITS, SPSI\rule[-5pt]{0pt}{8pt}\rule[-5pt]{0pt}{8pt}}\\
SPSIK & \parbox[t]{2.7in}{\hyphenpenalty10000 \raggedright
AMACH, ERFIN, ERMSG, IERM1, IERV1, SCSEVL, SERM1, SERV1,
SINITS, SPSI}\\
\end{tabular}

Designed and programmed by L. Wayne Fullerton, Los Alamos National
Scientific Laboratory, 1977. Revised and adapted to MATH77 by W. V. Snyder,
1993, 1994.


\begcodenp
\lstset{language=[77]Fortran,showstringspaces=false}
\lstset{xleftmargin=.8in}

\centerline{\bf \large DRDPSI}\vspace{10pt}
\lstinputlisting{\codeloc{dpsi}}

\vspace{30pt}\centerline{\bf \large ODDPSI}\vspace{10pt}
\lstset{language={}}
\lstinputlisting{\outputloc{dpsi}}

\closegraphsfile
\end{document}
