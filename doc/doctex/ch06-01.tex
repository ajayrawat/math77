\documentclass[twoside]{MATH77}
\usepackage{multicol}
\usepackage[fleqn,reqno,centertags]{amsmath}
\begin{document}
\begmath 6.1  Vector and Matrix Output

\silentfootnote{$^\copyright$1997 Calif. Inst. of Technology, \thisyear \ Math \`a la Carte, Inc.}

\subsection{Purpose}

Subroutines are provided to print a vector or a matrix in a standard format
that clearly displays the component positions of the vector or matrix. An
optional descriptive header can also be printed. For similar subroutines
that provide more choices for the appearance of the output, see Chapter~6.2.

\subsection{Usage}

\subsubsection{Program Prototype, Vector Output, Single Precision}

\begin{description}

\item[INTEGER] \ N

\item[REAL] \ {\bf V}($\geq $ N)

\item[CHARACTER$*(1 \leq k \leq 132)$] {\bf TEXT}

\end{description}

Assign values to V(), N, and TEXT.
$$
\fbox{\bf CALL SVECP (V, N, TEXT)}
$$
The N-vector stored in V() will be printed.

\paragraph{Argument Definitions, Single Precision}

\begin{description}
\item[V()] \ [in] An array containing N numbers to be printed.

\item[N] \ [in] Number of components of V() to be printed. Require N $\geq 0.$

\item[TEXT] \ [in] A character string to be printed as a header preceding the vector.
The length of TEXT should be at least 1 and not more than~132. The first
character will be used as a printer line spacing control character. It is
generally most convenient to place this character string in the CALL
statement as a character literal. Thus, for example:

\centerline{{\bf CALL SVECP (V, N, $^\prime $0 V=$^\prime)$}}

Here the zero following the first apostrophe causes an extra line feed
before the header.
\end{description}

\subsubsection{Program Prototype, Matrix Output, Single Precision}

\begin{description}

\item[INTEGER] \ {\bf LDA, M, N}

\item[REAL] \ {\bf A}(LDA, $\geq $ N)\quad [LDA  $\geq $ M]

{\bf CHARACTER$*(1 \leq k \leq 132)$} {\bf TEXT}

\end{description}

Assign values to all subroutine arguments.
$$
\fbox{\bf CALL SMATP (A, LDA, M, N, TEXT)}
$$
The M$\times\text{N}$ matrix stored in A(,) will be printed.

\paragraph{Argument Definitions, Single Precision}
\begin{description}
\item[A(,)]  [in] An array containing an M$\times\text{N}$ matrix to be printed.

\item[LDA]  [in] First dimensioning parameter of the array A(,). Require LDA $\geq
\text{M}.$

\item[M]  [in] Number of rows of A(,) to be printed.

\item[N]  [in] Number of columns of A(,) to be printed.

\item[TEXT]  [in] Same specification as for the argument TEXT of subroutine SVECP
above.
\end{description}

\subsubsection{Modifications for Double Precision or Integer Data}

For double precision usage, change the REAL statements to DOUBLE PRECISION
and change the initial letter of the subroutine names from ``S" to ``D".
For integer usage, change the REAL type statements to INTEGER and change the
initial letter of the subroutine names from ``S" to ``I".

\subsection{Examples and Remarks}

See the program DRVECP and the output ODVECP for an example of the use of
SVECP, DVECP, and IVECP. See the program DRMATP and the output ODMATP for an
example of the use of SMATP, DMATP, and IMATP.

\subsection{Functional Description}

The character string, TEXT, will always be printed. The user can set TEXT to
blanks if no visible header is desired.

For the first character in TEXT the Fortran 77 standard assigns a line spacing
control interpretation to '0', '1', '+', or a blank.  In some operating
environments this interpretation is not honored, and the character simply
appear in the output.  For greatest portability, and particularly when using
the C versions of these codes, we recommend not using '1' or '+', but only
'0' or blank.  The codes are designed so that a blank produces no extra
blank line and '0' produces one extra blank line before printing TEXT(2:).

\subsubsection{Vector Output}

If N $> 0$ the subroutine will print the first N components of the array V().

The subroutines SVECP and DVECP print eight components per line with a
G15.7 format for each component if the round-off level is $> 5 \times
10^{-13}$, or six components per line with a G20.12 format otherwise.
The subroutine IVECP prints eight components
per line with an I15 format for each component.

At the left of each printed line the indices of the first and last
components in that line will be printed.  The output is written using
PRINT or WRITE($*$,...).

\subsubsection{Matrix Output}

If M $> 0$ and N $> 0$ the subroutine will print the leading M $\times $ N
matrix contained in the array A(,).

Subroutines SMATP and DMATP print the matrix
in M $\times $ 8 blocks using G15.7 format for each element if the
round-off level is $> 5 \times 10^{-13}$, and in M $\times $ 6 blocks
using G20.12 format for each element otherwise.  Subroutine IMATP prints the matrix in M $\times $ 8 blocks using I15
format for each element.

Each column will be
headed with its column number and each row will have its row number
printed along the left margin. Output is written using PRINT or WRITE($*$,...).

\subsection{Error Procedures and Restrictions}

For vector output, if N $ \leq  0$, no vector components will be
printed. For matrix output, if M $ \leq  0$ or N $ \leq  0$%
, no matrix elements will be printed.

The built-in format was designed for a line printer having 132~columns. The
results are less satisfactory, due to wraparound, on an output device having
fewer print columns. See Chapter~6.2 for similar subroutines giving more
flexibility.

\subsection{Supporting Information}

The source language is ANSI Fortran~77.

\begin{tabular}{@{\bf}l@{\hspace{5pt}}l}
\bf Entry & \hspace{.2in} {\bf Required Files}\vspace{2pt} \\
DMATP & \hspace{.35in} DMATP\\
DVECP & \hspace{.35in} DVECP\\
IMATP & \hspace{.35in} IMATP\\
IVECP & \hspace{.35in} IVECP\\
SMATP & \hspace{.35in} SMATP\\
SVECP & \hspace{.35in} SVECP\\
\end{tabular}

Based on 1969 code by C.L. Lawson, JPL. Altered by Lawson and S.
Chiu, JPL, 1983; Lawson, 1994.


\begcode

\medskip\
\lstset{language=[77]Fortran,showstringspaces=false}
\lstset{xleftmargin=.2in}

\centerline{\bf \large DRVECP\hspace{1in}}\vspace{10pt}
\lstinputlisting{\codeloc{vecp}}

\vspace{20pt}\centerline{\bf \large ODVECP\hspace{1in}}\vspace{5pt}
\lstset{language={}}
\lstinputlisting{\outputloc{vecp}}

\newpage
\lstset{language=[77]Fortran,showstringspaces=false}
\lstset{xleftmargin=.8in}

\centerline{\bf \large DRMATP}\vspace{10pt}
\lstinputlisting{\codeloc{matp}}

\vspace{20pt}\centerline{\bf \large ODMATP}\vspace{10pt}
\lstset{language={}}
\lstinputlisting{\outputloc{matp}}
\end{document}
