\documentclass[twoside]{MATH77}
\usepackage{multicol}
\usepackage[fleqn,reqno,centertags]{amsmath}
\begin{document}
\hyphenation{DCPVAL DMDVAL}
\begmath 11.2 Evaluation, Integration, and Differentiation of Polynomials

\silentfootnote{$^\copyright$1997 Calif. Inst. of Technology, \thisyear \ Math \`a la Carte, Inc.}

\subsection{Purpose}

This set of subroutines will evaluate, integrate, or differentiate
polynomials. The polynomials may be represented by coefficients relative to
either the monomial or Chebyshev basis. The data structure and
parameterization used to represent a polynomial is the same as that used by
the least-squares polynomial curve fit subroutines described in Chapter~11.1.
Special procedures for evaluation of polynomials expressed using the
Legendre and Laguerre bases are described in Chapters~2.11 and~2.12
respectively.

\subsection{Usage}

\subsubsection{Usage for Evaluation}

\paragraph{Program Prototype, Single Precision}
\begin{description}
\item[INTEGER]  \ {\bf NDEG}

\item[REAL]  \ {\bf P}$(\geq $NDEG+3){\bf , X, Y, SCPVAL, SMPVAL}
\end{description}
Assign values to NDEG, X, and P($i$), $i = 1$, NDEG$+3$. If the Chebyshev
basis is being used, use the statement:
$$
\fbox{{\bf Y = SCPVAL(P, NDEG, X)}}
$$
If the monomial basis is being used, use the statement:
$$
\fbox{{\bf Y = SMPVAL(P, NDEG, X)}}
$$
Following the appropriate one of these two statements the value of the
polynomial at the argument X will be stored in Y.

\paragraph{Argument Definitions}
\begin{description}
\item[P()]  \ [in] An array containing NDEG$+3$ parameters that define a
polynomial as described in Section D.

\item[NDEG]  \ [in] Degree of the polynomial.

\item[X]  \ [in] Argument value at which the polynomial is to be evaluated.

\item[SCPVAL]  \ [out] The value of the polynomial evaluated at X assuming
the Chebyshev basis representation.

\item[SMPVAL]  \ [out] The value of the polynomial evaluated at X assuming
the monomial basis representation.
\end{description}
\subsubsection{Usage for Integration}

\paragraph{Program Prototype, Single Precision}
\begin{description}
\item[INTEGER]  \ {\bf NDEGA, NDEGB}

\item[REAL]  \ {\bf A}($\geq $NDEGA+3){\bf , B}($\geq $NDEGA+4)
\end{description}
Assign values to NDEGA and A($i$), $i = 1$, ..., NDEGA$+3$. If the Chebyshev
basis is being used, use the statement:
$$
\fbox{{\bf CALL SCPINT (A, NDEGA, B, NDEGB)}}
$$
If the monomial basis is being used, use the statement:
$$
\fbox{{\bf CALL SMPINT (A, NDEGA, B, NDEGB)}}
$$
Following the appropriate one of these two call statements the results will
be stored in B() and NDEGB.

\paragraph{Argument Definitions}
\begin{description}
\item[A()]  \ [in] An array containing NDEGA$+3$ parameters that define the
input polynomial, say $p(x)$. See Section D for the specification of the
parameterization.

\item[NDEGA]  \ [in] Degree of the input polynomial, $p(x).$

\item[B()]  \ [out] On return B() will contain NDEGB$+3$ parameters defining
the output polynomial, say $q(x)$, which is the indefinite integral of the
input polynomial $p(x)$. Mathematically the constant term of $q(x)$ is an
arbitrary constant of integration. This subroutine will set the constant
term, B($3)$, to zero. The storage locations occupied by A() and B() must be
distinct.

\item[NDEGB]  \ [out] The subroutine sets NDEGB $= \text{NDEGA}+1$ to indicate
the degree of the output polynomial. The storage locations occupied by NDEGA
and NDEGB must be distinct.
\end{description}
\subsubsection{Usage for Differentiation}

\paragraph{Program Prototype, Single Precision}
\begin{description}
\item[INTEGER]  \ {\bf NDEGC, NDEGD}

\item[REAL]  \ {\bf C} $(\geq $NDEGC+3){\bf , D}($\geq \max(3,\text{NDEGC}+2%
))$
\end{description}
Assign values to NDEGC and C($i$), $i = 1$, ..., NDEGC$+3$. If the Chebyshev
basis is being used, use the statement:
$$
\fbox{{\bf CALL SCPDRV (C, NDEGC, D, NDEGD)}}
$$
If the monomial basis is being used, use the statement:
$$
\fbox{{\bf CALL SMPDRV (C, NDEGC, D, NDEGD)}}
$$
Following the appropriate one of these two call statements the results will
be stored in D() and NDEGD.

\paragraph{Argument Definitions}
\begin{description}
\item[C()]  \ [in] An array containing NDEGC$+3$ parameters that define the
input polynomial, say $p(x)$. See Section D for the specification of the
parameterization.

\item[NDEGC]  \ [in] Degree of the input polynomial, $p(x).$

\item[D()]  \ [out] On return D() will contain NDEGD$+3$ parameters defining
the output polynomial, say $q(x)$, which is the derivative of the input
polynomial $p(x)$. The storage locations occupied by C() and D() must be
distinct.
\newpage

\item[NDEGD]  \ [out] The subroutine sets NDEGD $=\max (0$, NDEGC$-$1) to
indicate the degree of the output polynomial. The storage locations occupied
by NDEGC and NDEGD must be distinct.
\end{description}
\subsubsection{Usage for Double Precision Evaluation, Integration or
Differentiation}

For DOUBLE PRECISION usage change the REAL type statements to DOUBLE
PRECISION and change the initial ``S'' of the function and subroutine names
to a ``D.'' Note particularly that if the function names DCPVAL or DMPVAL
are used they must be typed DOUBLE PRECISION either explicitly or via an
IMPLICIT statement.

\subsection{Examples and remarks}

Let a cubic polynomial $p(x)$ be defined relative to the Chebyshev basis as $%
p(x)=10+8T_1(u)+6T_2(u)+4T_3(u)$ where $u=(x-5)/2$. The DRSCPVAL program
computes the indefinite integral of $p(x)$ calling it $q(x)$. This
computation is checked by computing $r(x)$ as the derivative of $q(x)$. Note
that $r(x)$ agrees with $p(x)$. Finally the program evaluates the definite
integral%
\begin{equation*}
z=\int_4^6p(x)\,dx=q(6)-q(4)=10
\end{equation*}
The output from this program is shown in ODSCPVAL.

\subsection{Functional Description}

In typical expected usage the polynomial parameter vector input to any of
the subprograms of this set will have been produced by the library curve
fitting subroutine SPFIT (or DPFIT) or an integration or differentiation
subroutine of this set. The subprograms of this set are thus intended to
let the user do the operations of evaluation, integration or differentiation
of polynomials without being concerned with the details of the parametric
representation of polynomials or algorithmic details.

The following description is provided for those who wish to know more of the
internal details.

For the purposes of this set of subprograms a polynomial of degree $n$, say $%
p(x)$, is represented by a set of $n+3$ parameters, say $a_1$, ..., $a_{n+3}$%
. The first two parameters define a linear transformation of the independent
variable
\begin{equation*}
u=(x-a_1)/a_2
\end{equation*}
The remaining $n+1$ parameters are coefficients of an $n^{th}$ degree
polynomial in the transformed variable $u$. If the Chebyshev basis is used
this polynomial is%
\begin{equation*}
p(x)=q(u)=\sum_{i=0}^na_{i+3}T_i(u)
\end{equation*}
whereas if the monomial basis is used the polynomial is%
\begin{equation*}
p(x)=q(u)=\sum_{i=0}^na_{i+3}u^i
\end{equation*}
The Chebyshev polynomials $T_i(u)$ are defined by the equations%
\begin{gather*}
T_0(u)=1,\quad T_1(u)=u\\
T_i(u)=2uT_{i-1}(u)-T_{i-2}(u),\quad i=2,3,...
\end{gather*}
The formulas for differentiation and integration of polynomials expressed
using the Chebyshev basis may be derived from the following standard
identities:
\begin{align*}
0&=dT_0(u)/du\\
T_0(u)&=dT_1(u)/du\\
T_1(u)&=\frac 14 dT_2(u)/du\\
T_i(u)&=\frac 12\frac d{du}\left[ \frac{T_{i+1}(u)}{i+1}-
\frac{T_{i-1}(u)}{i-1}\right] ,\quad i=2,3,...
\end{align*}
The algorithms used by this set of subprograms are specified as follows:

\subsubsection{Monomial Basis Evaluation, SMPVAL or DMPVAL}

Given an $n^{th}$ degree polynomial $p$ represented by the parameters $p_i$,
$i=1$,\ ..., $n+3$ and an argument $x$, compute $y=p(x).$%
\begin{align*}
u&=(x-p_1)/p_2\\
z_n&=p_{n+3}\\
z_i&=u\,z_{i+1}+p_{i+3},\quad i=n-1,n-2,...,0\\
y&=z_0.
\end{align*}
\subsubsection{Monomial Basis Integration, SMPINT or DMPINT}

Given an $n^{th}$ degree polynomial $p$ represented by the parameters $a_i$,
$i=1$,\ ..., $n+3$, compute the parameters $b_i$ that represent a polynomial
$q$ that for arbitrary $u$ and $v$ satisfies%
\begin{equation*}
\int_u^vp(x)\,dx=q(v)-q(u).
\end{equation*}
The formulas used are%
\begin{gather*}
b_1=a_1,\quad b_2=a_2,\quad b_3=0\\
b_{i+3}=a_2a_{i+2}/i,\ \ \ \ \ i=1,...,n+1
\end{gather*}
\subsubsection{Monomial Basis Differentiation, SMPDRV or DMPDRV}

Given an $n^{th}$ degree polynomial $p$ represented by the parameters $c_i$,
$i=1$,\ ..., $n+3$, compute the parameters $d_i$ that represent the
polynomial $q$ satisfying%
\begin{equation*}
\frac d{dx}p(x)=q(x).
\end{equation*}
The formulas used are%
\begin{gather*}
d_1=c_1,\quad d_2=c_2\\
d_{i+3}=(i+1)c_{i+4}/c_2,\quad i=0,...,n-1
\end{gather*}
with a special case of $d_3=0$ if $n=0.$

\subsubsection{Chebyshev Basis Evaluation, SCPVAL or DCPVAL}

Given an $n^{th}$ degree polynomial $p$ represented by the parameters $p_i$,
$i=1$, ..., $n+3$ and an argument $x$ compute $y=p(x).$%
\begin{align*}
u&=(x-p_1)/p_2\\
z_n&=p_{n+3}\\
z_{n-1}&=2uz_n+p_{n+2}\\
z_i&=2uz_{i+1}-z_{i+2}+p_{i+3},\quad i=n-2,...,1\\
y&=uz_1-z_2+p_3
\end{align*}
\subsubsection{Chebyshev Basis Integration, SCPINT or DCPINT}

Given an $n^{th}$ degree polynomial $p$ represented by the parameters $a_i$,
$i = 1, $..., $n+3$, compute the parameters $b_i$ that represent a
polynomial $q$ that for arbitrary $u$ and $v$ satisfies%
\begin{equation*}
\int_u^vp(x)\,dx=q(v)-q(u).
\end{equation*}
The formulas used are
\begin{gather*}
b_1=a_1,\quad b_2=a_2,\quad b_3=0\\
b_4=a_2\left[ a_3-(1/2)a_5\right]\\
b_{i+3}=a_2(a_{i+2}-a_{i+4})/(2i),\quad i=2,...,n+1,
\end{gather*}
where $a_i$ for $i>n+3$ is taken to be zero.

\subsubsection{Chebyshev Basis Differentiation, SCPDRV or DCPDRV}

Given an $n^{th}$ degree polynomial $p$ represented by the parameters $c_i$,
$i=1$, ..., $n+3$, compute the parameters $d_i$ that represent the
polynomial $q$ satisfying%
\begin{equation*}
\frac d{dx}p(x)=q(x)
\end{equation*}
The formulas used are%
\begin{gather*}
d_1=c_1,\quad d_2=c_2\\
d_{i+3}=2(i+1)c_{i+4}/c_2,\quad i=n-1,n-2\\
d_{i+3}=d_{i+5}+2(i+1)c_{i+4}/c_2,\quad i=n-3,...,1\\
d_3=d_5/2 + c_4 / c_2
\end{gather*}
with a special case of $d_3=0$ if $n=0.$

\subsection{Error Procedures and Restrictions}

The degree of the input polynomial must be zero or positive. If it is
negative the subprograms in this set will issue an error message
using the error processor in Chapter 19.2 at level 0 and
return.

The given values of P($2)$, A(2), and C($2)$ must be nonzero.  These
conditions are not tested.

The storage locations for the input quantities in the integration and
differentiation subroutines must be distinct from the storage locations for
the output quantities.

Since DCPVAL and DMPVAL are FUNCTION type subprograms, their names must be
typed DOUBLE PRECISION in any program that uses them.

\subsection{Supporting Information}

\begin{tabular}{@{\bf}l@{\hspace{5pt}}l}
\bf Entry & \hspace{.35in} {\bf Required Files}\vspace{2pt} \\
DCPDRV & \parbox[t]{2.7in}{\hyphenpenalty10000 \raggedright
DCPDRV, ERFIN, ERMSG, IERM1, IERV1\rule[-5pt]{0pt}{8pt}}\\
DCPINT & \parbox[t]{2.7in}{\hyphenpenalty10000 \raggedright
DCPINT, ERFIN, ERMSG, IERM1, IERV1\rule[-5pt]{0pt}{8pt}}\\
DCPVAL & \parbox[t]{2.7in}{\hyphenpenalty10000 \raggedright
DCPVAL\rule[-5pt]{0pt}{8pt}}\\
DMPDRV & \parbox[t]{2.7in}{\hyphenpenalty10000 \raggedright
DMPDRV, ERFIN, ERMSG, IERM1, IERV1\rule[-5pt]{0pt}{8pt}}\\
DMPINT & \parbox[t]{2.7in}{\hyphenpenalty10000 \raggedright
DMPINT, ERFIN, ERMSG, IERM1, IERV1\rule[-5pt]{0pt}{8pt}}\\
DMPVAL & \parbox[t]{2.7in}{\hyphenpenalty10000 \raggedright
DMPVAL\rule[-5pt]{0pt}{8pt}}\\
SCPDRV & \parbox[t]{2.7in}{\hyphenpenalty10000 \raggedright
ERFIN, ERMSG, IERM1, IERV1, SCPDRV\rule[-5pt]{0pt}{8pt}}\\
SCPINT & \parbox[t]{2.7in}{\hyphenpenalty10000 \raggedright
ERFIN, ERMSG, IERM1, IERV1, SCPINT\rule[-5pt]{0pt}{8pt}}\\
SCPVAL & \parbox[t]{2.7in}{\hyphenpenalty10000 \raggedright
SCPVAL\rule[-5pt]{0pt}{8pt}}\\
SMPDRV & \parbox[t]{2.7in}{\hyphenpenalty10000 \raggedright
ERFIN, ERMSG, IERM1, IERV1, SMPDRV\rule[-5pt]{0pt}{8pt}}\\
SMPINT & \parbox[t]{2.7in}{\hyphenpenalty10000 \raggedright
ERFIN, ERMSG, IERM1, IERV1, SMPINT\rule[-5pt]{0pt}{8pt}}\\
SMPVAL & \parbox[t]{2.7in}{\hyphenpenalty10000 \raggedright
SMPVAL}\\
\end{tabular}

The source language is Fortran~77.

Designed by C. L. Lawson, JPL, 1970. Programmed by Lawson and D. Campbell,
JPL, 1970. Adapted to Fortran~77 by Lawson and S. Y. Chiu, JPL, 1984.


\begcodenp
\lstset{language=[77]Fortran,showstringspaces=false}
\lstset{xleftmargin=.8in}

\centerline{\bf \large DRSCPVAL}\vspace{10pt}
\lstinputlisting{\codeloc{scpval}}

\vspace{30pt}\centerline{\bf \large ODSCPVAL}\vspace{10pt}
\lstset{language={}}
\lstinputlisting{\outputloc{scpval}}
\end{document}
