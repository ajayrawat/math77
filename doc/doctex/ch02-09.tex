\documentclass[twoside]{MATH77}
\usepackage[\graphtype]{mfpic}
\usepackage{multicol}
\usepackage[fleqn,reqno,centertags]{amsmath}
\begin{document}
\opengraphsfile{pl02-09}
\begmath 2.9 Incomplete Elliptic Integrals

\silentfootnote{$^\copyright$1997 Calif. Inst. of Technology, \thisyear \ Math \`a la Carte, Inc.}

\subsection{Purpose}

An integral of the form
\begin{equation}
\label{O1}\int R\left( t,P(t)^{1/2}\right) dt
\end{equation}
in which P($t)$ is a polynomial of the third or fourth degree that has no
multiple roots, and R is a rational function of $t$ and P($t)^{1/2}$, is
either elementary, or is an {\em elliptic integral}. It is always possible to
express integrals of the form of Eq.~(1) linearly in terms of elementary
functions and three {\em elliptic integrals of canonical form}. These functions
are described more completely in \cite{ams55:Ellip-Int} and \cite{Byrd:1971:HEI}.
Several canonical forms have been proposed, but the most widely used are
due to Jacobi, Legendre and Carlson. In each of Eqs.~(2)--(4) we present
first Jacobi's and then Legendre's form of the canonical elliptic integrals:
\begin{gather}
\label{O2}
\begin{split}
F(\varphi ,k)&=\int_0^y\left( 1-t^2\right) ^{-1/2}\left(
1-k^2t^2\right) ^{-1/2}dt\\
&=\int_0^\varphi \left( 1-k^2\sin ^2\theta \right) ^{-1/2}d\theta
\end{split}\\
\label{O3}
\begin{split}
E(\varphi ,k)&=\int_0^y\left( 1-t^2\right) ^{-1/2}\left(
1-k^2t^2\right) ^{1/2}dt\\
&=\int_0^\varphi \left( 1-k^2\sin ^2\theta \right) ^{1/2}d\theta
\end{split}
\end{gather}
\begin{multline}
\label{O4}
\Pi (\varphi ,\alpha ,k)\\
=\int_0^y\left( 1-\alpha ^2t^2\right) ^{-1}\left( 1-t^2\right)
^{-1/2}\left( 1-k^2t^2\right) ^{-1/2}dt\\
=\int_0^\varphi \left( 1-\alpha ^2\sin ^2\theta \right)
^{-1}\left( 1-k^2\sin ^2\theta \right) ^{-1/2}d\theta
\end{multline}
in which $y=\sin \varphi $. If $\varphi $ is equal to $\pi /2$, the
integrals are said to be {\em complete}, otherwise they are {\em
incomplete}. Carlson's forms of the canonical elliptic integrals are
\begin{equation}\label{O5}
\hspace{-15pt}R_D(a,b,c)=\frac 32\int_0^\infty \left( t+a\right)
^{-1/2}\left( t+b\right) ^{-1/2}\left( t+c\right) ^{-3/2}dt
\end{equation}
in which $a$ and $b$ are nonnegative such that $a+b>0$ and $c$ is positive;
if either $a$ or $b$ is zero, the integral is {\em complete}, otherwise it is
{\em incomplete},
\begin{equation}\label{O6}
\hspace{-15pt}R_F(a,b,c)=\frac 12\int_0^\infty \left( t+a\right)
^{-1/2}\left( t+b\right) ^{-1/2}\left( t+c\right) ^{-1/2}dt
\end{equation}
in which $a$, $b$ and $c$ are nonnegative and at most one of them is zero;
if one of $a$, $b$ or $c$ is zero, the integral is {\em complete},
otherwise it is {\em incomplete}, and
\begin{multline}\label{O7}
\hspace{-10pt}R_J(a,b,c,r)\\
\hspace{-12pt}=\frac 32 \int_0^\infty \!\!\left( t+r\right)
^{-1}\left( t+a\right)^{-1/2}\left( t+b\right) ^{-1/2}\left( t+c\right)
^{-1/2}dt\hspace{-10pt}
\end{multline}
in which $a$, $b$ and $c$ are nonnegative, and at most one of them is zero,
and $r$ is nonzero; if one of $a$, $b$ or $c$ is zero, the integral is
$complete$, otherwise it is $incomplete$. Notice that $R_D(a$, $b$, $c)=R_J(a$, $%
b$, $c$, $c)$. But the necessity to compute $R_D(a$, $b$, $c)$ arises
frequently in practice, and a procedure especially tailored to compute $R_D(a
$, $b$, $c)$ is more efficient than computing $R_J(a$, $b$, $c$, $c)$. The
function $R_C(a$, $b)=R_F(a$, $b$, $b)$ is elementary, but also appears
frequently. A procedure is provided to compute $R_C(a$, $b).$

Identify a, $b$ and $c$ such that $a\leq b\leq c$, and assume $a<c$. Then
\begin{align}\label{O8}
c^{3/2}R_D(a,b,c)&=\frac 3{k^2\sin ^3\varphi }\left[ F(\varphi
,k)-E(\varphi ,k)\right]\hspace{-1in}\\
\label{O9}c^{1/2}R_F(a,b,c)&=\frac{F(\varphi ,k)}{\sin \varphi }\\
\label{O10}c^{3/2}R_J(a,b,c,r)&=\frac 3{\alpha ^2\sin ^3\varphi }\left[ \Pi
(\varphi ,\alpha ,k)-F(\varphi ,k)\right]
\end{align}
where $\cos ^2\varphi =a/c$, $k^2=(c-b)/(c-a)$ and $\alpha ^2=(c-r)/(c-a).$

The subprograms described in this chapter evaluate the canonical forms of
incomplete elliptic integrals, using either the Legendre or the Carlson
parameterization.

\subsection{Usage}

\subsubsection{Program Prototype, Single Precision, Legendre's Form, E and F}

\begin{description}
\item[REAL]  \ {\bf PHI, K, F, E}

\item[INTEGER]  \ {\bf IERR}
\end{description}

Assign values to PHI and K.
$$
\fbox{{\bf CALL SELEFI (PHI, K, F, E, IERR)}}
$$

\paragraph{Argument Definitions}

\begin{description}
\item[PHI]  \ [in] Argument, $\varphi $, of the elliptic integral. Require $|%
\text{PHI}|\leq \pi /2.$

\item[K]  \ [in] Modulus, $k$. Require $|\text{K}|\leq 1.0.$

\item[F]  \ [out] F($\varphi $, $k)$ with $\varphi $ given by PHI and $k$
given by K.

\item[E]  \ [out] E($\varphi $, $k)$ with $\varphi $ given by PHI and $k$
given by K.

\item[IERR]  \ [out] status indicator:

\begin{itemize}
\item[0 =]  no errors

\item[1 =]  Magnitude of argument too large, $|$PHI$| >\pi /2.$

\item[2 =]  Magnitude of Modulus too large, $|$K$|>1.0.$

\item[3 =]  $|$PHI$| =\pi /2$ and $|$K$|=1$, F is infinite.
\end{itemize}
\end{description}

\subsubsection{Program Prototype, Single Precision, Legendre's Form, $\Pi $}

\begin{description}
\item[REAL]  \ {\bf PHI, K2, ALPHA2, PI}

\item[INTEGER]  \ {\bf IERR}
\end{description}

Assign values to PHI, K2 and ALPHA2.
$$
\fbox{{\bf CALL SELPII (PHI, K2, ALPHA2, PI, IERR)}}
$$

\paragraph{Argument Definitions}

\begin{description}
\item[PHI]  \ [in] Argument, $\varphi $, of the elliptic integral. Require $|%
\text{PHI}|\leq \pi /2.$

\item[K2]  \ [in] Square of the modulus, $k^2$. Require $k^2\sin
^2\varphi \leq 1.0$. See Section E.

\item[ALPHA2]  \ [in] Characteristic, $\alpha ^2$. Require $\alpha
^2\sin ^2\varphi \leq 1.0$. See Section E.

\item[PI]  \ [out] $\Pi (\varphi ,\alpha ,k)$, with $\varphi $ given by PHI,
$\alpha ^2$ given by ALPHA2 and $k^2$ given by K2.

\item[IERR]  \ [out] status indicator.  If IERR = 0, there were no
errors.  Other values are produced by procedures SRFVAL and SRJVAL (see
Sections B.5 and B.6) which are
used in computing $\Pi (\phi, \alpha, k).$
\end{description}

\subsubsection{Program Prototype, Single Precision, Carlson's Form, $R_C$}

\begin{description}
\item[REAL]  \ {\bf X, Y, RC}

\item[INTEGER]  \ {\bf IERR}
\end{description}

Assign values to X and Y.
$$
\fbox{\bf CALL SRCVAL (X, Y, RC, IERR)}
$$
\paragraph{Argument Definitions}

\begin{description}
\item[X, Y]  \ [in] Arguments of the elliptic integral. Require X $\geq 0$,
Y $\neq 0$. See Section E.

\item[RC]  \ [out] The computed value of $R_C($X, Y).

\item[IERR]  \ [out] Status indicator:

\begin{itemize}
\item[0 =]  no errors

\item[1 =]  X $<0.0$ or Y = 0.0.

\item[2 =]  X + $|\text{Y}|$ too small (See Section E).

\item[3 =]  X or $|\text{Y}|\text{ or X} +|\text{Y}|$ too large (See Section E).

\item[4 =]  Y $<0$ and $|\text{Y}|$ too large and X too small (See Section
E).
\end{itemize}
\end{description}

\subsubsection{Program Prototype, Single Precision, Carlson's Form, $R_D$}

\begin{description}
\item[REAL]  \ {\bf X, Y, Z, RD}

\item[INTEGER]  \ {\bf IERR}
\end{description}

Assign values to X, Y and Z.
$$
\fbox{\bf CALL SRDVAL (X, Y, Z, RD, IERR)}
$$
\paragraph{Argument Definitions}

\begin{description}
\item[X, Y, Z]  \ [in] Arguments of the elliptic integral. Require X $\geq 0$%
, Y $\geq 0$, X + Y $>0$, Z $>0$. See Section E.

\item[RD]  \ [out] The computed value of $R_D($X, Y, Z).

\item[IERR]  \ [out] Status indicator:

\begin{itemize}
\item[0 =]  no errors

\item[1 =]  X $<0.0$ or Y $<0.0$ or Z $<0.0.$

\item[2 =]  X + Y too small or Z too small (See Section E).

\item[3 =]  X or Y or Z too large (See Section E).
\end{itemize}
\end{description}

\subsubsection{Program Prototype, Single Precision, Carlson's Form, $R_F$}

\begin{description}
\item[REAL]  \ {\bf X, Y, Z, RF}

\item[INTEGER]  \ {\bf IERR}
\end{description}

Assign values to X, Y and Z.
$$
\fbox{\bf CALL SRFVAL (X, Y, Z, RF, IERR)}
$$
\paragraph{Argument Definitions}

\begin{description}
\item[X, Y, Z]  \ [in] Arguments of the elliptic integral. Require X $\geq 0$%
, Y $\geq 0$, Z $\geq 0$, at most one of X, Y or Z equal zero. See Section E.

\item[RF]  \ [out] The computed value of $R_F($X, Y, Z).

\item[IERR]  \ [out] Status indicator:

\begin{itemize}
\item[0 =]  no errors

\item[1 =]  X $<0.0$ or Y $<0.0$ or Z $<0.0.$

\item[2 =]  X + Y or X + Z or Y + Z too small (See Section E).

\item[3 =]  X or Y or Z too large (See Section E).
\end{itemize}
\end{description}

\subsubsection{Program Prototype, Single Precision, Carlson's Form, $R_J$}

\begin{description}
\item[REAL]  \ {\bf X, Y, Z, R, RJ}

\item[INTEGER]  \ {\bf IERR}
\end{description}

Assign values to X, Y, Z and R.
$$
\fbox{\bf CALL SRJVAL (X, Y, Z, R, RJ, IERR)}
$$
\paragraph{Argument Definitions}

\begin{description}
\item[X, Y, Z, R]  \ [in] Arguments of the elliptic integral. Require X $%
\geq 0$, Y $\geq 0$, Z $\geq 0$, at most one of X, Y or Z equal zero, R $\neq
0$. See Section E.

\item[RJ]  \ [out] The computed value of $R_J($X, Y, Z, R).

\item[IERR]  \ [out] Status indicator:

\begin{itemize}
\item[0 =]  no errors

\item[1 =]  X $<0.0$ or Y $<0.0$ or Z $<0.0$ or R $=0.0.$

\item[2 =]  X + Y or X + Z or Y + Z or $|$R$|$ too small (See Section E).

\item[3 =]  X or Y or Z or $|$R$|$ too large (See Section E).
\end{itemize}
\end{description}

\subsubsection{Modifications for Double Precision}

For double precision usage, change the REAL type statements to DOUBLE
PRECISION and change the subprogram names SELEFI, SELPII, SRCVAL, SRDVAL,
SRFVAL and SRJVAL to DELEFI, DELPII, DRCVAL, DRDVAL, DRFVAL and DRJVAL,
respectively.

\subsection{Examples and Remarks}

\subsubsection{Related Functions}

Logarithms, inverse circular functions and inverse hyperbolic functions can
be expressed in terms of $R_C$, see [9,~pp.~163,~186]:
$$
\begin{array}{ll}
(\ln x)/(x-1) = R_C((\frac{1}{2}+\frac{1}{2}x)^2,x), & x>0;\\
(\sin ^{-1} x)/x = R_C(1-x^2,1), & -1\leq x\leq 1;\\
(\sinh^{-1} x)/x = R_C(1+x^2,1),& -\infty <x<\infty;\\
(\cos ^{-1} x)/(1-x^2)^{\frac{1}{2}} = R_C(x^2,1), & 0\leq x\leq 1;\\
(\cosh ^{-1} x)/(x^2-1)^{\frac{1}{2}} = R_C(x^2,1), & x\geq 1;\\
(\tan ^{-1} x)/x = R_C(1,1+x^2), & -\infty <x<\infty ;\\
(\tanh ^{-1} x)/x = R_C(1,1-x^2), & -1<x<1;\\
\cot ^{-1} x = R_C(x^2,x^2+1), & 0\leq x<\infty;\\
\coth ^{-1} x = R_C(x^2,x^2-1), & x>1.
\end{array}
$$
The first seven of these allow computing nearly indeterminate forms
with more accuracy than would be possible using the na\"\i ve
formulation.

Heuman's lambda function \cite{Heuman:1941:TCE} is a variant of Legendre's third integral:
\begin{multline}\label{O11}
\frac{\left( 1-\cos ^2\alpha \sin ^2\beta \right) ^{1/2}}{\cos
^2\alpha \sin \beta \cos \beta }\Lambda (\alpha ,\beta ,\varphi )\\
=\sin \varphi \,R_F(\cos ^2\varphi ,1-\sin ^2\alpha
\sin ^2\varphi ,1)\vspace{2pt} \\ \displaystyle\qquad +\frac{\sin ^2\alpha
\sin ^3\varphi }{3\left( 1-\cos ^2\alpha \sin ^2\beta \right) }\times
R_J \Bigl(\cos ^2\varphi ,\\
1-\sin ^2\alpha \sin^2\varphi ,1,1-\frac{\sin ^2\alpha \sin ^2\varphi }
{1-\cos ^2\alpha \sin^2\beta }\Bigr)
\end{multline}
\begin{multline}\label{O12}
\frac \pi 2\Lambda _0(\alpha ,\beta )=\Lambda (\alpha ,\beta
,\pi /2)\vspace{2pt}\\
=\sin \beta \left[ R_F(0,\cos^2\alpha ,1)-\frac 13\sin ^2\alpha
R_D(0,\cos ^2\alpha ,1)\right]\\
\times R_F\left( \cos ^2\beta ,1-\cos^2\alpha \sin ^2\beta ,1\right)
-\frac 13\cos ^2\alpha \sin ^3\beta\\
\times R_F(0,\cos ^2\alpha ,1)\times R_D(\cos ^2\beta ,1-\cos ^2\alpha
\sin ^2\beta ,1)
\end{multline}
The variants of Legendre's integrals used by Bulirsch in
\cite{Bulirsch:1965:NCE} and \cite{Bulirsch:1969:NCE} are
\begin{equation}
\label{O13}el1(x,k_c)=xR_F\left( 1,1+k_c^2x^2,1+x^2\right) ,
\end{equation}
\begin{multline}\label{O14}
el2(x,k_c,a,b)=axR_F\left( 1,1+k_c^2x^2,1+x^2\right)\\
+\frac 13(b-a)x^3R_D\left( 1,1+k_c^2x^2,1+x^2\right)
\end{multline}
\begin{multline}\label{O15}
ele3(x,k_c,p)=xR_F\left( 1,1+k_c^2x^2,1+x^2\right)\\
+\frac 13(1-p)x^3R_J\left( 1,1+k_c^2x^2,1+x^2,1+p x^2\right)
\end{multline}
\begin{equation}\label{O16}
\hspace{-15pt}cel(k_c,p,a,b)=aR_F(0,k_c^2,1)=\frac 13(b-pa)R_J\left(
0,k_c^2,1,p\right)
\end{equation}
\subsubsection{Which Procedure Should Be Used?}

Several factors influence the choice of procedure. If one
needs to write a simple program and use it once, one should probably choose
the procedure that evaluates the functions in the form most similar to the
way the problem is posed. If one needs to write a program that will have
substantial use, one should usually prefer SELEFI to SRDVAL and SRFVAL, as
the former is up to~30 times faster than the latter two. An exception to
this rule occurs if one needs to compute $R_D(a,b,c)$ with $c < \max (a,b)$,
in which case the parameters for SELEFI will be out of range. If accuracy is
an issue but speed is not, one may prefer SRDVAL and SRFVAL to SELEFI, at
least for computing F($\varphi ,k).$ (See testing in Section D below).

SRCVAL is somewhat slower, on an IBM PC/AT with a numeric data processor,
than using the equivalent Fortran intrinsic functions. This is no surprise,
as most of the intrinsic functions are implemented by hardware. But the
inverse hyperbolic functions are not. SRCVAL is roughly the same speed as
the procedures in Chapter~2.1. As mentioned above, it may be
advantageous to use SRCVAL to compute nearly indeterminate forms.

SELPII is implemented by using SRJVAL and SRFVAL (see Eqs.~(9) and
(10) above).
Thus, there is no special advantage in speed or accuracy to one or the
other. The sole criterion is how closely the forms of the functions
evaluated directly by the procedures match the forms of the functions the
user needs to evaluate.

\subsection{Functional Description}

\subsubsection{Properties of the Functions}

The first form given in Eqs.~(2)--(4) is the Jacobi or algebraic form. When
expressed in this form Eq.~(2) is finite for all real and complex $y$, including $%
\infty $, has a simple pole of order~1 for $y = \infty $, and is
logarithmically infinite for $y = 1/\alpha ^2.$

\subsubsection{Method of Computation}

The procedure SELEFI is based upon a procedure ELLPI developed by Allan V.
Hershey and modified by Alfred H.  Morris, described in
\cite{ahm:lib:ELLPI}.  The procedure uses series expansions due to
DiDonato and Hershey, described in \cite{DiDonato:1959:NFC}.  The
procedure SELPII is based upon a procedure EPI developed by Alfred H.
Morris, described in [5].  It computes $\Pi (\varphi ,k^2,\alpha ^2)$
using Eqs.~(9) and (10), as computed by SRFVAL and SRJVAL.  The procedures
SRCVAL, SRDVAL, SRFVAL and SRJVAL are based on procedures developed by B.
C.  Carlson and Elaine M.  Notis, described in \cite{Carlson:1979:CEI} and
\cite{Carlson:1981:AAI}.  All of the referenced procedures were revised to
be consistent with low level modules and naming conventions of MATH77.
\vspace{10pt}

%
% \input prepic.tex
% \input pictex.tex
% \input postpic.tex
% \beginpicture \ninepoint
% \setcoordinatesystem units <  1.797in,  2.800in>
% \setplotarea x from    .000 to   1.600, y from    .000 to   1.000
% \axis bottom label {}
%  ticks in withvalues 0 .2 .4 .6 .8 1.0 1.2 1.4 1.6 /
%  at 0 .2 .4 .6 .8 1.0 1.2 1.4 1.6 / /
% \axis left label {$k$}
%  ticks in numbered from   .0     to  1.0     by   .2     /
% \axis top ticks in quantity  9 /
% \axis right ticks in quantity  6 /
% \put {.2} at    .200   .400
% \put {.4} at    .400   .600
% \put {.6} at    .600   .400
% \put {.8} at    .830   .600
% \put {1} at   1.025   .400
% \put {1.2} at   1.220   .300
% \put {1.4} at   1.405   .200
% \setquadratic \plot
%   .2000   .00000   .2000   .00000   .2000   .00500   .2000   .01500
%   .2000   .02500   .2000   .03500   .2000   .04500   .2000   .05500
%   .2000   .0700   .2000   .0850   .2000   .1000   .2000   .1150
%   .2000   .1350   .2000   .1500   .2000   .1700   .2000   .1900
%   .2001   .2100   .2001   .2300   .2001   .2550   .2001   .2800
%   .2001   .3100   .2002   .3400   .2002   .3750
% /
% \setquadratic \plot
%   .2002   .4250   .2003   .4550   .2003   .4850   .2004   .5200
%   .2004   .5600   .2005   .6000   .2005   .6400   .2006   .6800
%   .2007   .7250   .2008   .7700   .2009   .8150   .2010   .8750
%   .2012   .9350   .2013   .9650   .2014  1.0000
% /
% \setquadratic \plot
%   .4115  1.0000   .4097   .9250   .4080   .8473   .4066   .7750
%   .4052   .6950   .4047   .6600   .4042   .6250
% /
% \setquadratic \plot
%   .4035   .5750   .4029   .5200   .4023   .4650   .4018   .4150
%   .4014   .3600   .4011   .3200   .4008   .2800   .4006   .2450
%   .4004   .2050   .4003   .1800   .4002   .1500   .4002   .1300
%   .4001   .1050   .4001   .0900   .4001   .0700   .4000   .0550
%   .4000   .04000   .4000   .03000   .4000   .02000   .4000   .01000
%   .4000   .00000
% /
% \setquadratic \plot
%   .6000   .00000   .6000   .00429   .6000   .01000   .6000   .02000
%   .6000   .03000   .6001   .04000   .6001   .05000   .6001   .06000
%   .6002   .0750   .6003   .0900   .6004   .1100   .6006   .1300
%   .6008   .1550   .6012   .1850   .6016   .2150   .6022   .2550
%   .6030   .2950   .6038   .3350   .6049   .3750
% /
% \setquadratic \plot
%   .6063   .4250   .6073   .4550   .6083   .4850   .6106   .5450
%   .6133   .6050   .6153   .6450   .6172   .6800   .6212   .7450
%   .6253   .8050   .6305   .8700   .6365   .9350   .6400   .9687
%   .6435  1.0000
% /
% \setquadratic \plot
%   .9273  1.0000   .9120   .9662   .8983   .9300   .8856   .8900
%   .8734   .8450   .8640   .8044   .8550   .7600   .8480   .7203
%   .8416   .6800   .8381   .6550   .8341   .6250
% /
% \setquadratic \plot
%   .8281   .5750   .8244   .5400   .8210   .5050   .8180   .4700
%   .8156   .4400   .8114   .3800   .8080   .3207   .8058   .2750
%   .8037   .2200   .8026   .1850   .8017   .1500   .8012   .1250
%   .8007   .09500   .8005   .08000   .8003   .06000   .8002   .04500
%   .8001   .03000   .8000   .02000   .8000   .01000   .8000   .00500
%   .8000   .00000
% /
% \setquadratic \plot
%   1.0000   .00000   1.0000   .00228   1.0000   .01000   1.0001   .02000
%   1.0001   .03000   1.0002   .04000   1.0003   .05000   1.0006   .06500
%   1.0009   .0800   1.0014   .1000   1.0020   .1200   1.0029   .1450
%   1.0042   .1750   1.0061   .2100   1.0081   .2400   1.0115   .2850
%   1.0157   .3300   1.0177   .3500   1.0206   .3750
% /
% \setquadratic \plot
%   1.0270   .4250   1.0314   .4550   1.0362   .4850   1.0416   .5150
%   1.0480   .5477   1.0551   .5800   1.0636   .6150   1.0720   .6456
%   1.0842   .6850   1.0960   .7183   1.1110   .7550   1.1252   .7850
%   1.1416   .8150   1.1574   .8400   1.1760   .8652   1.2000   .8924
%   1.2246   .9150   1.2594   .9400   1.2972   .9600   1.3360   .9746
%   1.3760   .9851   1.4320   .9940   1.4880   .9983   1.5280   .9996
%   1.5679  1.0000
% /
% \setquadratic \plot
%   1.5680   .8748   1.5360   .8598   1.5040   .8422   1.4771   .8250
%   1.4480   .8032   1.4240   .7822   1.4000   .7579   1.3763   .7300
%   1.3578   .7050   1.3415   .6800   1.3243   .6500   1.3115   .6250
%   1.2960   .5907   1.2857   .5650   1.2748   .5350   1.2651   .5050
%   1.2563   .4750   1.2497   .4500   1.2425   .4200   1.2371   .3950
%   1.2320   .3694   1.2285   .3500   1.2243   .3250
% /
% \setquadratic \plot
%   1.2171   .2750   1.2152   .2600   1.2123   .2350   1.2098   .2100
%   1.2075   .1850   1.2053   .1550   1.2031   .1200   1.2020   .0950
%   1.2011   .07000   1.2007   .05500   1.2003   .04000   1.2002   .03000
%   1.2001   .02000   1.2000   .01000   1.2000   .00000
% /
% \setquadratic \plot
%   1.4000   .00000   1.4000   .00154   1.4000   .01000   1.4001   .02000
%   1.4003   .03000   1.4005   .04000   1.4008   .05000   1.4013   .06500
%   1.4020   .0800   1.4031   .1000   1.4045   .1200   1.4070   .1500
%   1.4096   .1750
% /
% \setquadratic \plot
%   1.4161   .2250   1.4201   .2500   1.4245   .2750   1.4295   .3000
%   1.4339   .3200   1.4398   .3450   1.4451   .3650   1.4507   .3850
%   1.4584   .4100   1.4669   .4350   1.4761   .4600   1.4863   .4850
%   1.4960   .5067   1.5073   .5300   1.5206   .5550   1.5353   .5800
%   1.5516   .6050   1.5586   .6150   1.5680   .6278
% /
% \endpicture\vspace{-5pt}
%\centerline{$\varphi $}\vspace{5pt}
%\centerline{E($k,\varphi)$}\vspace{20pt}
%
% \beginpicture \ninepoint
% \setcoordinatesystem units <  1.797in,  2.800in>
% \setplotarea x from    .000 to   1.600, y from    .000 to   1.000
% \axis bottom label {}
%  ticks in withvalues 0 .2 .4 .6 .8 1.0 1.2 1.4 1.6 /
%  at 0 .2 .4 .6 .8 1.0 1.2 1.4 1.6 / /
% \axis left label {$k$}
%  ticks in numbered from   .0     to  1.0     by   .2     /
% \axis top ticks in quantity  9 /
% \axis right ticks in quantity  6 /
% \put {.2} at    .200   .300
% \put {.4} at    .400   .500
% \put {.6} at    .600   .300
% \put {.8} at    .780   .500
% \put {1} at    .990   .300
% \put {1.2} at   1.150   .500
% \put {1.4} at   1.290   .600
% \put {1.6} at   1.400   .700
% \put {1.8} at   1.450   .800
% \put {2.0} at   1.450   .900
% \setquadratic \plot
%   .2000   .00000   .2000   .00500   .2000   .01500   .2000   .02500
%   .2000   .03500   .2000   .04500   .2000   .06000   .2000   .07000
%   .2000   .0850   .2000   .1000   .2000   .1150   .2000   .1350
%   .2000   .1550   .2000   .1700   .2000   .1900   .1999   .2100
%   .1999   .2350   .1999   .2550   .1999   .2750
% /
% \setquadratic \plot
%   .1999   .3250   .1998   .3550   .1998   .3850   .1998   .4150
%   .1997   .4500   .1997   .4850   .1996   .5200   .1996   .5600
%   .1995   .6000   .1995   .6450   .1994   .6900   .1993   .7300
%   .1992   .7700   .1991   .8100   .1990   .8550   .1989   .9050
%   .1988   .9550   .1987   .9750   .1987  1.0000
% /
% \setquadratic \plot
%   .3897  1.0000   .3910   .9350   .3922   .8700   .3935   .7950
%   .3947   .7150   .3956   .6550   .3964   .5950   .3968   .5600
%   .3972   .5250
% /
% \setquadratic \plot
%   .3977   .4750   .3981   .4300   .3985   .3800   .3988   .3350
%   .3992   .2850   .3994   .2450   .3996   .2050   .3997   .1750
%   .3998   .1400   .3999   .1200   .3999   .0950   .3999   .0800
%   .4000   .06000   .4000   .05000   .4000   .03500   .4000   .02500
%   .4000   .01500   .4000   .01000   .4000   .00000
% /
% \setquadratic \plot
%   .6000   .00000   .6000   .00500   .6000   .01500   .6000   .02500
%   .6000   .03500   .5999   .04500   .5999   .06000   .5998   .07500
%   .5997   .0950   .5996   .1150   .5993   .1400   .5991   .1650
%   .5987   .1950   .5983   .2250   .5977   .2600   .5976   .2650
%   .5975   .2750
% /
% \setquadratic \plot
%   .5965   .3250   .5953   .3750   .5940   .4250   .5926   .4700
%   .5911   .5150   .5884   .5900   .5850   .6700   .5832   .7100
%   .5810   .7550   .5785   .8050   .5757   .8550   .5713   .9300
%   .5669  1.0000
% /
% \setquadratic \plot
%   .7262  1.0000   .7319   .9600   .7374   .9200   .7440   .8692
%   .7513   .8100   .7559   .7700   .7609   .7250   .7651   .6850
%   .7690   .6450   .7731   .6000   .7766   .5600   .7778   .5450
%   .7794   .5250
% /
% \setquadratic \plot
%   .7831   .4750   .7849   .4500   .7871   .4150   .7897   .3700
%   .7921   .3250   .7943   .2750   .7962   .2250   .7974   .1850
%   .7984   .1450   .7989   .1200   .7993   .0950   .7996   .0750
%   .7998   .05500   .7998   .04500   .7999   .03000   .8000   .01500
%   .8000   .00000
% /
% \setquadratic \plot
%   1.0000   .00000   1.0000   .00500   1.0000   .01500    .9999   .02500
%    .9998   .03500    .9997   .04500    .9996   .05500    .9993   .07000
%   .9990   .0850   .9985   .1050   .9979   .1250   .9969   .1500
%   .9956   .1800   .9937   .2150   .9918   .2450   .9908   .2600
%   .9897   .2750
% /
% \setquadratic \plot
%   .9856   .3250   .9838   .3450   .9814   .3700   .9782   .4000
%   .9754   .4250   .9712   .4600   .9674   .4900   .9633   .5200
%   .9589   .5500   .9543   .5800   .9495   .6100   .9440   .6427
%   .9383   .6750   .9327   .7050   .9269   .7350   .9200   .7693
%   .9125   .8050   .9048   .8400   .8960   .8784   .8880   .9121
%   .8786   .9500   .8723   .9750   .8658  1.0000
% /
% \setquadratic \plot
%    .9857  1.0000    .9982   .9700   1.0084   .9450   1.0202   .9150
%   1.0318   .8850   1.0411   .8600   1.0537   .8250   1.0640   .7952
%   1.0724   .7700   1.0821   .7400   1.0930   .7050   1.1019   .6750
%   1.1104   .6450   1.1185   .6150   1.1263   .5850   1.1336   .5550
%   1.1406   .5250
% /
% \setquadratic \plot
%   1.1514   .4750   1.1564   .4500   1.1611   .4250   1.1664   .3950
%   1.1705   .3700   1.1743   .3450   1.1772   .3250   1.1812   .2950
%   1.1849   .2650   1.1886   .2300   1.1920   .1926   1.1945   .1600
%   1.1966   .1250   1.1976   .1050   1.1986   .0800   1.1991   .0650
%   1.1996   .04500   1.1997   .03500   1.1999   .02500   1.1999   .01500
%   1.2000   .00000
% /
% \setquadratic \plot
%   1.4000   .00000   1.4000   .00500   1.3999   .01500   1.3998   .02500
%   1.3996   .03500   1.3994   .04500   1.3991   .05500   1.3985   .07000
%   1.3978   .0850   1.3966   .1050   1.3948   .1300   1.3931   .1500
%   1.3916   .1650   1.3877   .2000   1.3837   .2300   1.3791   .2600
%   1.3749   .2850   1.3703   .3100   1.3663   .3300   1.3610   .3550
%   1.3565   .3750   1.3517   .3950   1.3441   .4250   1.3387   .4450
%   1.3316   .4700   1.3240   .4950   1.3161   .5200   1.3078   .5450
%   1.2973   .5750
% /
% \setquadratic \plot
%   1.2786   .6250   1.2686   .6500   1.2561   .6800   1.2474   .7000
%   1.2340   .7300   1.2240   .7514   1.2102   .7800   1.2000   .8006
%   1.1849   .8300   1.1717   .8550   1.1581   .8800   1.1440   .9050
%   1.1296   .9300   1.1178   .9500   1.1040   .9729   1.0960   .9859
%   1.0872  1.0000
% /
% \setquadratic \plot
%   1.1723  1.0000   1.1897   .9800   1.2067   .9600   1.2234   .9400
%   1.2400   .9196   1.2560   .8995   1.2750   .8750   1.2901   .8550
%   1.3084   .8300   1.3227   .8100   1.3400   .7850   1.3534   .7650
%   1.3697   .7400   1.3760   .7300   1.3791   .7250
% /
% \setquadratic \plot
%   1.4092   .6750   1.4234   .6500   1.4343   .6300   1.4475   .6050
%   1.4600   .5800   1.4697   .5600   1.4800   .5378   1.4901   .5150
%   1.5007   .4900   1.5087   .4700   1.5183   .4450   1.5255   .4250
%   1.5341   .4000   1.5406   .3800   1.5468   .3600   1.5525   .3400
%   1.5593   .3150   1.5631   .3000   1.5680   .2795
% /
% \setquadratic \plot
%   1.5680   .6607   1.5536   .6800   1.5382   .7000   1.5222   .7200
%   1.5057   .7400   1.4887   .7600   1.4720   .7791   1.4712   .7800
%   1.4640   .7880
% /
% \setquadratic \plot
%   1.4298   .8250   1.4154   .8400   1.3957   .8600   1.3907   .8650
%   1.3806   .8750
% /
% \setquadratic \plot
%   1.3760   .8795   1.3600   .8950   1.3440   .9102   1.3280   .9251
%   1.3062   .9450   1.2894   .9600   1.2720   .9752   1.2560   .9890
%   1.2431  1.0000
% /
% \setquadratic \plot
%   1.3017  1.0000   1.3200   .9881   1.3399   .9750   1.3600   .9614
%   1.3768   .9500   1.3920   .9395   1.4126   .9250
% /
% \setquadratic \plot
%   1.4640   .8875   1.4806   .8750   1.5001   .8600   1.5191   .8450
%   1.5377   .8300   1.5520   .8182   1.5680   .8048
% /
% \endpicture\vspace{-5pt}
%\centerline{$\varphi $}\vspace{5pt}
%\centerline{F($k,\varphi)$}
%
\subsubsection{Testing}

The single precision programs for E($\varphi ,k)$, F($\varphi ,k)$, $%
R_D(a,b,c)$ and $R_F(a,b,c)$ were tested on an IBM PC/AT (using IEEE
arithmetic) by comparison to double precision results, as described below.
The relative precision of IEEE single precision arithmetic is $\rho =
2^{-23} \approx 0.119\times 10^{-6}.$

The accuracy of procedure SELEFI was assessed by comparing its results to
double precision results obtained by applying  Eqs.~(8) and (9), with $%
R_D(a,b,c)$ and $R_F(a,b,c)$ evaluated by DRDVAL and DRFVAL, respectively.
The accuracy of procedures SRDVAL and SRFVAL was assessed by comparing their
results to double precision results obtained by applying Eqs.~(8) and (9),
with E($\varphi ,k)$ and F($\varphi ,k)$ evaluated by DELEFI.

To test SELEFI, the rectangular region $0 \leq \varphi \leq \pi /2 \times 0
\leq k \leq 1$ of the $\varphi \times k$ plane was divided into 2000
regions, and a point was randomly selected in each region. To test SRDVAL
and SRFVAL, the argument $c$ was set to~1.0, the rectangular region $0 \leq
a < 1 \times 0 \leq b < 1$ of the $a\times b$ plane was divided into~2000
regions, and a point was randomly selected in each region. The maximum
relative and absolute errors are summarized in the following table.

\begin{center}
\begin{tabular}{lrr}
& \multicolumn{1}{c}{\bf Max. Rel.} & \multicolumn{1}{c}{\bf Max. Abs.}\\
{\bf Function} & \multicolumn{1}{c}{\bf Error} & \multicolumn{1}{c}{\bf Error}\\
E($\varphi ,k)$ & $.82\rho $\rule{.2in}{0pt} & $.98\rho $\rule{.2in}{0pt}\\
F($\varphi ,k)$ & $5.24\rho $\rule{.2in}{0pt} & $15.91\rho $\rule{.2in}{0pt}\\
$R_D(a,b,1)$ & $1.20\rho $\rule{.2in}{0pt} & $3.01\rho $\rule{.2in}{0pt}\\
$R_F(a,b,1)$ & $1.35\rho $\rule{.2in}{0pt} & $2.55\rho $\rule{.2in}{0pt}
\end{tabular}
\end{center}

Errors in F($\varphi ,k)$ increase as the arguments approach the infinite
singularity at $\varphi =\pi /2$ and $k=1.$

\nocite{Carlson:1977:SFA}
\bibliography{math77}
\bibliographystyle{math77}

\subsection{Error Procedures and Restrictions}

The procedure SELEFI requires $|\varphi | \leq \pi /2$, and $|k| \leq 1$.

Procedure SELPII computes $\Pi (\varphi ,k^2,\alpha ^2)$ from $R_J(a,b,c,r)$
and $R_F(a,b,c)$ using Eqs.~(9) and (10). The initial values for the arguments
are $a = \cos ^2\varphi $, $b = 1-k^2\sin ^2\varphi $, $r = 1-\alpha ^2\sin
^2\varphi $, and $c = \max (a,b,r)$. Then a, $b$ and $r$ are replaced by $%
a\times c$, $b\times c$ and $r\times c$, respectively. SELPII requires $%
|\varphi | \leq \pi /2$. Restrictions on $k^2$ and $\alpha ^2$ are enforced
indirectly by restrictions on a, $b$ and $c$ imposed by SRFVAL and SRJVAL,
described below.

The ranges for $\varphi $ and $k^2$ can be extended using formulae 113.01,
113.02, 114.01, 115.01, 115.02, 160.02, 161.02 and~162.02 from [2],
or formulae~17.4.1 through~17.4.18 from [1].

Denote the largest representable magnitude by $\Omega $, and the smallest
nonzero representable magnitude by $\omega $. General restrictions on the
arguments to procedures SRCVAL, SRDVAL, SRFVAL and SRJVAL were described
above in Section B.

SRCVAL requires $\text{X}+|\text{Y}| \geq 5\omega ,\ \text{X} \leq
\Omega /5$, $|\text{Y}| \leq \Omega /5$, and, if $\text{Y} < -2.236/\sqrt
\omega $ it requires $\text{X} \geq (\omega \Omega )^2/25$.

Denote the machine round-off level by $\rho $, that is, $\rho $ is the
smallest positive number such that
the representation of $1+\rho $\ is different from~1. Let $\varepsilon $ be
the solution of the equation $\rho = 3\varepsilon ^6(1-\varepsilon )^{-3/2}$,
$\Omega _D = 2\Omega ^{-2/3}$ and $\omega _D = \varepsilon \omega ^{-2/3}/10$%
. SRDVAL requires $\text{X}+\text{Y} \geq \omega _D$ and $\text{Z}
\geq \omega _D,\ \text{X} \leq  \Omega _D,\ \text{Y} \leq \Omega _D$
and $\text{Z} \leq \Omega _D$.

SRFVAL requires $\text{X} + \text{Y} \geq 5\omega ,\ \text{X} +
\text{Z} \geq 5\omega $, $\text{Y} + \text{Z} \geq 5\omega ,\ \text{X} \leq
\Omega /5,\ \text{Y} \leq \Omega /5$ and
$\text{Z} \leq \Omega /5$.

Let $\Omega _J = (\Omega /5)^{1/3}/5$ and $\omega _J = (5\omega )^{1/3}$.
SRJVAL requires $\text{X}+\text{Y} \geq \omega _J,\ \text{Y} +\text{Z}\geq
\omega _J,\ \text{X} +\text{Z} \geq \omega _J$, $|\text{R}| \geq
\omega _J,\ \text{X} \leq \Omega _J,\ \text{Y} \leq \Omega _J,\ \text{Z} \leq
\Omega _J$ and $|\text{R}| \leq \Omega _J$.

The accessible ranges of the
arguments may be extended beyond the ranges admissible in the procedures by
using the homogeneity of the functions:
\begin{align*}
R_F(ka, kb, kc) &= k^{-1/2} R_F(a, b, c),\quad \text{and}\\
R_J(ka,kb, kc, kr) &= k^{-3/2} R_J(a,\ b,\ c,\ r).
\end{align*}
If any of the restrictions above is violated, all procedures return an error
indicator in the argument named IERR, and invoke the error message
processor (see Chapter~19.2) with LEVEL = 0. The procedure ERMSET (see
Chapter~19.2) may be used to affect the default error processing action.

\subsection{Supporting Information}

The source language for these subroutines is ANSI Fortran 77.

The procedures SELEFI and SELPII were written by W. V. Snyder in December
1990, based on earlier procedures described by Alfred H. Morris, Naval
Surface Warfare Center, Dahlgren, VA in [5]. The procedures SRCVAL,
SRDVAL, SRFVAL and SRJVAL were written by W. V. Snyder in December 1990,
based on earlier procedures described by Carlson and Notis in \cite{Carlson:1981:AAI}.


\begin{tabular}{@{\bf}l@{\hspace{5pt}}l}
\bf Entry & \hspace{.35in} {\bf Required Files}\vspace{2pt} \\
DELEFI & \parbox[t]{2.7in}{\hyphenpenalty10000 \raggedright
AMACH, DELEFI, DERM1, DERV1, DLNREL, ERFIN, ERMSG\rule[-5pt]{0pt}{8pt}}\\DELPII & \parbox[t]{2.7in}{\hyphenpenalty10000 \raggedright
AMACH, DELPII, DERM1, DERV1, DRCVAL, DRFVAL, DRJVAL, ERFIN, ERMSG\rule[-5pt]{0pt}{8pt}}\\DRCVAL & \parbox[t]{2.7in}{\hyphenpenalty10000 \raggedright
AMACH, DERM1, DERV1, DRCVAL, ERFIN, ERMSG\rule[-5pt]{0pt}{8pt}}\\DRDVAL & \parbox[t]{2.7in}{\hyphenpenalty10000 \raggedright
AMACH, DERM1, DERV1, DRDVAL, ERFIN, ERMSG\rule[-5pt]{0pt}{8pt}}\\DRFVAL & \parbox[t]{2.7in}{\hyphenpenalty10000 \raggedright
AMACH, DERM1, DERV1, DRFVAL, ERFIN, ERMSG\rule[-5pt]{0pt}{8pt}}\\DRJVAL & \parbox[t]{2.7in}{\hyphenpenalty10000 \raggedright
AMACH, DERM1, DERV1, DRCVAL, DRFVAL, DRJVAL, ERFIN, ERMSG\rule[-5pt]{0pt}{8pt}}\\SELEFI & \parbox[t]{2.7in}{\hyphenpenalty10000 \raggedright
AMACH, ERFIN, ERMSG, SELEFI, SERM1, SERV1, SLNREL\rule[-5pt]{0pt}{8pt}}\\SELPII & \parbox[t]{2.7in}{\hyphenpenalty10000 \raggedright
AMACH, ERFIN, ERMSG, SELPII, SERM1, SERV1, SRCVAL, SRFVAL, SRJVAL\rule[-5pt]{0pt}{8pt}}\\SRCVAL & \parbox[t]{2.7in}{\hyphenpenalty10000 \raggedright
AMACH, ERFIN, ERMSG, SERM1, SERV1, SRCVAL\rule[-5pt]{0pt}{8pt}}\\SRDVAL & \parbox[t]{2.7in}{\hyphenpenalty10000 \raggedright
AMACH, ERFIN, ERMSG, SERM1, SERV1, SRDVAL\rule[-5pt]{0pt}{8pt}}\\SRFVAL & \parbox[t]{2.7in}{\hyphenpenalty10000 \raggedright
AMACH, ERFIN, ERMSG, SERM1, SERV1, SRFVAL\rule[-5pt]{0pt}{8pt}}\\SRJVAL & \parbox[t]{2.7in}{\hyphenpenalty10000 \raggedright
AMACH, ERFIN, ERMSG, SERM1, SERV1, SRCVAL, SRFVAL, SRJVAL\rule[-5pt]{0pt}{8pt}}\\\end{tabular}

\begcodenp
\lstset{language=[77]Fortran,showstringspaces=false}
\lstset{xleftmargin=.8in}

\centerline{\bf \large DRSELI}\vspace{10pt}
\lstinputlisting{\codeloc{seli}}

\vspace{30pt}\centerline{\bf \large ODSELI}\vspace{10pt}
\lstset{language={}}
\lstinputlisting{\outputloc{seli}}

\closegraphsfile
\end{document}
