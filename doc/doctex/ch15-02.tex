\documentclass[twoside]{MATH77}
\usepackage{multicol}
\usepackage[fleqn,reqno,centertags]{amsmath}
\begin{document}
\begmath 15.2 Cumulative Distribution Function and Percentage Points
\hbox{for Normal Probability Distribution}

\silentfootnote{$^\copyright$1997 Calif. Inst. of Technology, \thisyear \ Math \`a la Carte, Inc.}

\subsection{Purpose}

The procedures described in this chapter compute the Cumulative Distribution
Function (CDF) and the percentage points of the Normal or Gaussian
distribution. The CDF is sometimes called the lower tail. The lower tail,
or CDF, $g(x;\mu ,\sigma )$, and the upper
tail, $h(x;\mu ,\sigma )$ for the Normal probability distribution with mean $%
\mu $ and standard deviation $\sigma $ are defined by%
\begin{gather*}
\hspace{-15pt}g(x;\mu ,\sigma )=\frac 1{\sigma \sqrt{2\pi }}\int_{-\infty }^x\!\!\exp
\!\left( -\frac{(t-\mu )^2}{2\sigma ^2}\right) dt,\rule{50pt}{0pt}\\
\hspace{-15pt}h(x;\mu ,\sigma )=\frac 1{\sigma \sqrt{2\pi }}\int_x^\infty \!\!\exp
\!\left( -\frac{(t-\mu )^2}{2\sigma ^2}\right) dt=g(-x;\mu ,\sigma )
\end{gather*}
The percentage point of a distribution is the value of $x$ that gives the
lower tail a specified value. In this case, the problem is to compute $x$
given $u = g(x;\mu ,\sigma )$, $\mu $ and $\sigma $, that is, compute $x =
g^{-1}(u;\mu ,\sigma ).$

\subsection{Usage}

\subsubsection{Cumulative Distribution Function}
\paragraph{Program Prototype, Single Precision}

\begin{description}
\item[REAL]  \ {\bf U, X, MU, SIGMA, SCDNML}

\item[EXTERNAL]  \ {\bf SCDNML}
\end{description}

Assign values to X, MU and SIGMA and obtain U $= g(x;\mu ,\sigma )$ by using
$$
\fbox{{\bf U = SCDNML(X, MU, SIGMA)}}
$$

\paragraph{Argument Definitions}

\begin{description}
\item[X]  \ [in] Argument $x$ of the function $g(x;\mu ,\sigma ).$

\item[MU]  \ [in] Parameter $\mu $ of the function $g(x;\mu ,\sigma ).$

\item[SIGMA]  \ [in] Parameter $\sigma $ of the function $g(x;\mu ,\sigma ).$
\end{description}

\subsubsection{Percentage Points}
\paragraph{Program Prototype, Single Precision}

\begin{description}
\item[REAL]  \ {\bf U, X, MU, SIGMA, SPPNML}

\item[EXTERNAL]  \ {\bf SPPNML}
\end{description}

Assign values to U, MU and SIGMA and obtain X $= g^{-1}(u;\mu ,\sigma )$ by
using
$$
\fbox{{\bf X = SPPNML(U, MU, SIGMA)}}
$$

\paragraph{Argument Definitions}

\begin{description}
\item[U]  \ [in] Argument $u$ of the function $g^{-1}(u;\mu ,\sigma ).$
Require $0.0 \leq \text{U} \leq 1.0.$

\item[MU]  \ [in] Parameter $\mu $ of the function $g^{-1}(u;\mu ,\sigma ).$

\item[SIGMA]  \ [in] Parameter $\sigma $ of the function $g^{-1}(u;\mu
,\sigma ).$
\end{description}

\subsubsection{Modifications for Double Precision}

For double precision computation, change the REAL type statement to
DOUBLE PRECISION and change the initial letter of the function names to D.
Since these functions are not generic intrinsic functions, it is important
to declare them explicitly to be DOUBLE PRECISION, because the default
implicit type would be REAL.

\subsection{Example and Remarks}

See DRDCDNML and ODDCDNML for an example of the usage of these subprograms.

\subsection{Functional Description}

\subsubsection{Method}

To avoid cancellation error when $x-\mu << 0$, the identity $g(x;\mu
,\sigma) = \frac 12 \erfc ((\mu - x)/\sigma \sqrt 2)$ is used.  This
expression never causes more cancellation error than mathematically
equivalent alternatives, so it is used for all allowed values of $x$,
$\mu $, and $\sigma $ (see Section E for restrictions). The procedure
SERFC described in Chapter~2.2 is used to
evaluate $\erfc ((\mu - x)/\sigma \sqrt 2).$

To compute the percentage points, invert the last expression to compute $x =
g^{-1}(u;\mu ,\sigma ) = \mu - \sigma \sqrt 2$\ erfc$^{-1}(2u)$. The
procedure SERFCI described in Chapter~2.13 is used to evaluate erfc$%
^{-1}(2u).$

\subsubsection{Accuracy Tests}

See Sections 2.2.D and~2.13.D.

\subsection{Error Procedures and Restrictions}

The procedure SERFC issues a warning message by way of the error message
processor described in Chapter~19.2 if (X $-$ MU)/($\sqrt{2.0}\ \times $
SIGMA) $< -xmax$. The value of $xmax$ depends on the system and the precision.
Let $t=\sqrt {-\log(\sqrt \pi f)}$ where $f$ is the underflow limit
provided by R1MACH(1) or D1MACH(1) of Chapter~19.1.  Then $xmax = t
- ((\log \,t)/t) - 0.01.$ For example, $xmax \approx 9.18$ (26.5) for single
(double) precision IEEE arithmetic.  The procedure SERFCI issues an error
message at level 2 by way of the error message processor described in
Chapter~19.2 if U $< 0.0 $ or U $> 1.0.$
\subsection{Supporting Information}

Designed and programmed by W. V. Snyder, JPL, 1993.

\pagebreak
%\rule[-50pt]{0pt}{10pt}

\begin{tabular}{@{\bf}l@{\hspace{5pt}}l}
\bf Entry & \hspace{.35in} {\bf Required Files}\vspace{2pt} \\
DCDNML & \parbox[t]{2.7in}{\hyphenpenalty10000 \raggedright
AMACH, DCSEVL, DERF, DERM1, DERV1, DINITS, ERFIN, ERMSG, IERM1, IERV1\rule[-5pt]{0pt}{8pt}}\\
DPPNML & \parbox[t]{2.7in}{\hyphenpenalty10000 \raggedright
AMACH, DERFI, DERM1, DERV1, DPPNML, ERFIN, ERMSG\rule[-5pt]{0pt}{8pt}}\\
\end{tabular}

\begin{tabular}{@{\bf}l@{\hspace{5pt}}l}
\bf Entry & \hspace{.35in} {\bf Required Files}\vspace{2pt} \\
SCDNML & \parbox[t]{2.7in}{\hyphenpenalty10000 \raggedright
AMACH, ERFIN, ERMSG, IERM1, IERV1, SCSEVL, SERF, SERM1, SERV1, SINITS\rule[-5pt]{0pt}{8pt}}\\
SPPNML & \parbox[t]{2.7in}{\hyphenpenalty10000 \raggedright
AMACH, ERFIN, ERMSG, SERFI, SERM1, SERV1, SPPNML}\\
\end{tabular}

\begcode

\medskip\
\lstset{language=[77]Fortran,showstringspaces=false}
\lstset{xleftmargin=.8in}

\centerline{\bf \large DRDCDNML}\vspace{5pt}
\lstinputlisting{\codeloc{dcdnml}}

\vspace{15pt}\centerline{\bf \large ODDCDNML}\vspace{5pt}
\lstset{language={}}
\lstinputlisting{\outputloc{dcdnml}}
\end{document}
