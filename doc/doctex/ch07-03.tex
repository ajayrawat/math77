\documentclass[twoside]{MATH77}
\usepackage{multicol}
\usepackage[fleqn,reqno,centertags]{amsmath}
\begin{document}
\begmath 7.3 Compute Polynomial Coefficients from Roots

\silentfootnote{$^\copyright$1997 Calif. Inst. of Technology, \thisyear \ Math \`a la Carte, Inc.}

\subsection{Purpose}

Given $n$ complex numbers, $z_i$, compute (complex) coefficients, $c_i$,
such that the monic polynomial defined by%
\begin{equation*}
p(z)=c_1z^n+...+c_nz+c_{n+1}
\end{equation*}
has the numbers $z_i$ as its roots. The coefficients will be computed from
the relation%
\begin{equation*}
\hspace{-10pt}c_1z^n+.\text{ . . }+c_nz+c_{n+1}=(z-z_1)(z-z_2)^{...}(z-z_n)
\end{equation*}
Note that $c_1$ is always set to~1.

\subsection{Usage}

\subsubsection{Program Prototype, Single Precision}

\begin{description}
\item[\bf INTEGER]  \ {\bf NDEG}
\item[\bf COMPLEX]  \ {\bf ROOTS}$(\geq $NDEG),{\bf COEFS}$(\geq $NDEG+1)
\end{description}
Assign values to NDEG and ROOTS().
$$
\fbox{{\bf CALL CCOEF (NDEG, ROOTS, COEFS)}}
$$
Computed quantities are returned in COEFS().

\subsubsection{Argument Definitions}

\begin{description}
\item[NDEG]  [in] Number of roots given in ROOTS(), and thus the degree of
the polynomial whose coefficients are to be computed.
\item[ROOTS()]  [in] Roots, given as complex numbers.
\item[COEFS()]  [out] Computed coefficients, stored as complex numbers. The
arrays COEFS() and ROOTS() must be distinct. The coefficient, $c_1$, will be
the coefficient of $z^{NDEG}$ and will be set to~1.
\end{description}

\subsubsection{Modifications for Double Precision}

For double precision usage change the subroutine name from CCOEF to ZCOEF.
Recall that the Fortran~77 standard does not support a double precision
complex data type, although many Fortran compilers do, using the
declaration, COMPLEX*16. To remain within the Fortran~77 standard, use the
declarations
\begin{description}
\item[DOUBLE PRECISION]  {\bf ROOTS}(2, $\geq $ NDEG$)$,
{\bf COEFFS}(2, $\geq $NDEG$+1)$
\end{description}

and use the convention that real and imaginary parts of complex numbers are
associated with the values 1 and~2, respectively, of the first subscript.
This usage is compatible with the Fortran~90 standard.  Alternatively, if
the COMPLEX*16 declaration is available and is compatible with this storage
convention, one may use the nonstandard declaration
\begin{description}
\item[COMPLEX*16]  {\bf ROOTS}$(\geq $ NDEG$)$,
{\bf COEFFS}$(\geq $ NDEG$+1)$
\end{description}

\subsection{Examples and Remarks}

The program, DRZCOEF, with its output, ODZCOEF, illustrates the use of ZCOEF
to compute the coefficients of a quadratic and a cubic polynomial.

If this subroutine is used to assess the accuracy of a polynomial root
finder we suggest use of the double precision version, even if it is a single precision
root finder, to reduce the introduction of errors from the process of
computing the polynomial coefficients.

\subsection{Functional Description}

\subparagraph{Method}

The degree, NDEG, and roots, $z_1$, ..., $z_{NDEG}$, are given. The
coefficients, $c_i$, are computed by the following algorithm. The quantities
$z_i$ and $c_i$ are complex. In the double precision version the complex arithmetic is
coded in-line in terms of operations on the real and imaginary parts to
conform to the Fortran~77 standard.
\begin{tabbing}
\hspace{.2in}\=$c_1 = 1.0$\\
\>if( NDEG .le. 0 ) return\\
\>$c_2 = -z_1$\\
\>do $i = 2$, NDEG\\
\>\ \ \ \ \=$c_{i+1} = -c_i * z_i$\\
\>\>do $j = i$, 2, $-$1\\
\>\>\ \ \ \ $c_j = c_j - c_{j-1} * z_i$\\
\>\>enddo\\
\>enddo\\
\>return
\end{tabbing}

\subparagraph{Accuracy tests}

The logic and the accuracy of this code were checked by use with a root
finder. The accuracy was consistent with the computer system being used.

\subsection{Error Procedures and Restrictions}

If NDEG $\leq$ 0 the subroutine returns, setting COEFS$(1) = 1$. The arrays
ROOTS() and COEFS() must occupy distinct storage locations.

\subsection{Supporting Information}

The source language for these subroutines is ANSI Fortran 77.

\begin{tabular}{@{\bf}l@{\hspace{5pt}}l}
Entry & \hspace{.35in} {\bf Required Files}\vspace{2pt}\\
CCOEF & \hspace{.35in} CCOEF\rule[-5pt]{0pt}{8pt}\\
ZCOEF & \hspace{.35in} ZCOEF
\end{tabular}

Designed by C. L. Lawson, JPL, May~1986.

Programmed by C. L. Lawson and S. Y. Chiu, JPL, May~1986, Feb.~1987.


\begcodenp
\lstset{language=[77]Fortran,showstringspaces=false}
\lstset{xleftmargin=.8in}

\centerline{\bf \large DRZCOEF}\vspace{10pt}
\lstinputlisting{\codeloc{zcoef}}

\vspace{30pt}\centerline{\bf \large ODZCOEF}\vspace{10pt}
\lstset{language={}}
\lstinputlisting{\outputloc{zcoef}}
\end{document}
