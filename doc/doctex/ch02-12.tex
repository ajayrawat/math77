\documentclass[twoside]{MATH77}
\usepackage{multicol}
\usepackage[fleqn,reqno,centertags]{amsmath}
\begin{document}
\begmath 2.12 Finite Laguerre Series

\silentfootnote{$^\copyright$1997 Calif. Inst. of Technology, \thisyear \ Math \`a la Carte, Inc.}

\subsection{Purpose}

This subroutine computes the value of a finite sum of Laguerre polynomials,%
\begin{equation*}
y=\sum_{j=0}^{\text{N}}a_jL_j(x)
\end{equation*}
for a specified summation limit, N, argument, $x$, and sequence of
coefficients, $a_j$. The Laguerre polynomials are defined in
\cite{ams55:or-poly}.

\subsection{Usage}

\subsubsection{Program Prototype, Single Precision}

\begin{description}
\item[INTEGER]  \ {\bf N}

\item[REAL]  \ {\bf X, Y, A}(0:$m\geq $ N)
\end{description}

Assign values to X, N, and A(0), A(1), ..., A(N).
$$
\fbox{{\bf CALL SLASUM (X, N, A, Y)}}
$$
The sum will be stored in Y.

\subsubsection{Argument Definitions}

\begin{description}
\item[X]  \ [in] Argument of the polynomials.

\item[N]  \ [in] Highest degree of polynomials in sum.

\item[A()]  \ [in] The coefficients must be given in A(J), J = 0, ..., N.

\item[Y]  \ [out] Computed value of the sum.
\end{description}

\subsubsection{Modifications for Double Precision}

For double precision usage, change the REAL statement to DOUBLE PRECISION
and change the subroutine name from SLASUM to DLASUM.

\subsection{Examples and Remarks}

See DRSLASUM and ODSLASUM for an example of the usage of SLASUM. DRSLASUM
evaluates the following identity, the coefficients of which were obtained
from Table~22.10, page~799, of \cite{ams55:or-poly}.
\begin{equation*}
z=y-w=0,
\end{equation*}
where
\begin{multline*}
y = 7.2L_0(x)-3.2L_1(x)+108L_2(x)-144L_3(x)\\
+108L_4(x)-43.2L_5(x)+7.2L_6(x),
\end{multline*}
and
\begin{equation*}
w=0.01x^6.
\end{equation*}

\subsection{Functional Description}

The sum is evaluated by the following algorithm:%
\begin{gather*}
\hspace{-10pt}b_{N+2}=0,\quad b_{N+1}=0, \\
\hspace{-10pt}b_k=\frac{2k+1-x}{k+1}b_{k+1}-\frac{k+1}{k+2}b_{k+2}+a_k,
\quad k=\text{N},...,0,\\
\hspace{-10pt}y=b_0.
\end{gather*}
For an error analysis applying to this algorithm see \cite{Ng:1968:DSS} and
\cite{Ng:1971:RAC}. The first four Laguerre polynomials are
\begin{gather*}
L_0(x)=1,\quad L_1(x)=1-x,\\
L_2(x)=1-2x+0.5x^2,\\
L_3(x)=1-3x+1.5x^2-(1/6)x^3.
\end{gather*}
For $k \geq 2$ the Laguerre polynomials satisfy the recurrence
\begin{equation*}
kL_k(x)=(2k-1-x)L_{k-1}(x)-(k-1)L_{k-2}(x).
\end{equation*}
The Laguerre polynomials are orthogonal relative to integration with the
weight function $e^{-x}$ over the interval [0, $\infty )$, thus%
\begin{equation*}
\int_0^\infty e^{-x}L_i(x)L_j(x)\,dx=0\quad \text{if }i\neq j.
\end{equation*}
Laguerre polynomials are normally used only with an argument $x$ satisfying $%
x \geq 0.$

\bibliography{math77}
\bibliographystyle{math77}

\subsection{Error procedures and Restrictions}

The subroutine will return Y = 0 if N $< 0$. It is recommended that X
satisfy X $\geq 0.$

\subsection{Supporting Information}

The source language is ANSI Fortran~77.

\begin{tabular}{@{\bf}l@{\hspace{5pt}}l}
\bf Entry & \hspace{.2in} {\bf Required Files}\vspace{2pt} \\
DLASUM & \hspace{.35in} DLASUM\rule[-5pt]{0pt}{8pt}\\
SLASUM & \hspace{.35in} SLASUM\\
\end{tabular}

Based on a 1974 program by E.W. Ng, JPL. Present version by C.L. Lawson and
S. Y. Chiu, JPL, 1983.

\begcodenp

\lstset{language=[77]Fortran,showstringspaces=false}
\lstset{xleftmargin=.8in}

\centerline{\bf \large DRSLASUM}\vspace{10pt}
\lstinputlisting{\codeloc{slasum}}

\vspace{30pt}\centerline{\bf \large ODSLASUM}\vspace{10pt}
\lstset{language={}}
\lstinputlisting{\outputloc{slasum}}

\end{document}
