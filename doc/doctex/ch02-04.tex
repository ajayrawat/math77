\documentclass[twoside]{MATH77}
\usepackage[\graphtype]{mfpic}
\usepackage{multicol}
\usepackage[fleqn,reqno,centertags]{amsmath}
\begin{document}
\opengraphsfile{pl02-04}
\begmath 2.4 Bessel Functions $J_0$, $J_1$, $Y_0$ and $Y_1$

\silentfootnote{$^\copyright$1997 Calif. Inst. of Technology, \thisyear \ Math \`a la Carte, Inc.}

\subsection{Purpose}

These subprograms compute values of the cylindrical Bessel functions
of the first kind, $J_0$ and $J_{1}$, and of the cylindrical Bessel
functions of the second kind, $Y_0$ and $Y_{1}$.  These functions are
discussed in \cite{ams55} and \cite{Hart:1968:CA:bes}.

\subsection{Usage}

\subsubsection{Program Prototype, Single Precision}
\begin{description}
\item[REAL]  \ {\bf X,SBESJ0,SBESJ1,SBESY0,SBESY1,W}
\end{description}
Assign a value to X and use one of the following function references.

\begin{tabular}{ll}
To compute $J_0:$ & \fbox{\bf W = SBESJ0(X)}\rule[-15pt]{0pt}{8pt}\\

To compute $J_1:$ & \fbox{\bf W = SBESJ1(X)}\rule[-15pt]{0pt}{8pt}\\

To compute $Y_0$ for $x > 0:$ & \fbox{\bf W = SBESY0(X)}\rule[-15pt]{0pt}{8pt}\\

To compute $Y_1$ for $x > 0:$ & \fbox{\bf W = SBESY1(X)}
\end{tabular}

\subsubsection{Argument Definitions}
\begin{description}
\item[X]  \ [in] Argument of function. Require X $>0$ for the Y
functions.
\end{description}
\subsubsection{Modifications for Double Precision}
For double precision usage, change the REAL type statement to DOUBLE
PRECISION, and change the function names to DBESJ0, DBESJ1, DBESY0, and DBESY1,
respectively.

\subsection{Examples and Remarks}

The listing of DRSBESJ0 and ODSBESJ0 gives an example of using these
subprograms to evaluate the Wronskian identity
\begin{equation}
z(x) = (x\pi /2)[J_1(x)Y_0(x)-J_0(x)Y_1(x)] - 1 = 0\notag
\end{equation}

\subsection{Functional Description}

The functions $J_n$ and $Y_n$ are a pair of linearly
independent solutions for the differential equation
\begin{equation}
x^2\frac{d^2w}{dx^2}+x\frac{dw}{dx}+\left( x^2-n^2\right) w=0\notag
\end{equation}
The functions $J_0$ and $J_1$ are defined for all real $x$. The function $J_0
$ is even, and $J_1$ is odd. As $x\rightarrow \infty $, $J_0(x)$ and $J_1(x)$
oscillate an infinite number of times about zero with an amplitude that
diminishes asymptotically to $[2/(\pi x)]^{\frac 12}$, $i.e.$, approximately
$0.80x^{-\frac 12}$. The distance between successive zeros approaches $\pi $
as $|x|\rightarrow \infty .$

The functions $Y_0$ and $Y_1$ have real values for positive real $x$,
approach $-\infty $ as $x\rightarrow 0^{+}$ and have complex values for
negative real $x$. For large positive $x$ the Y functions have oscillatory
behavior similar to the J functions, $i.e.$, with amplitude approaching $0.80
x^{-\frac 12}$ and zero spacing approaching $\pi $. As $x\rightarrow 0^{+}$%
, $Y_0(x)\rightarrow (2/\pi )\ln (x)$ and $Y_1(x)\rightarrow -(2/\pi x).$

\vspace{10pt}

\hspace{5pt}\mbox{\input pl02-04 }

The Y subprograms treat $x\leq 0$ as an error condition. If the complex values
of $Y_0$ and $Y_1$ for negative $x$ are desired they may be computed from
the formulae
\begin{equation}
\begin{array}{ll}
\begin{array}{rcl}
Y_0(x) & = & Y_0(-x)+2iJ_0(-x)
\rule[-10pt]{0pt}{8pt} \\ Y_1(x) & = & -Y_1(-x)-2iJ_1(-x),
\end{array}
\quad x<0
\end{array}\notag
\end{equation}
where $i$ denotes the imaginary unit. See Equation 9.1.36 in \cite{ams55}.

The computer approximations for these functions were developed by L. W.
Fullerton, \cite{Fullerton:1973:FNLIB} and \cite{Fullerton:1977:PSF},
using functional forms involving sine, cosine, square root, logarithm, and
Chebyshev polynomial approximations.  These subprograms select the
polynomial degrees to adapt to machine accuracy of up to 30 decimal
places.

The single precision subprograms for $J_n(x)$ and $Y_n(x)$ were tested on a
Univac~1100 by comparison with the corresponding double precision
subprograms over various argument ranges. The relative precision of Univac
single precision arithmetic is $\rho = 2^{-27} \approx 0.745\times 10^{-8}$.

The results show that the relative error can be very large near the zeros
of any of these functions with the exception that relative accuracy can be,
and is, maintained for $J_1$ near its zero at $x = 0$. The absolute error
is large for $Y_0$ and $Y_1$ near the singularity at $x = 0.$

Test results may be summarized as follows.

\begin{tabular}{lllc}
 &\multicolumn{1}{c}{\bf Argument} & \multicolumn{1}{c}{\bf Maximum} &
\multicolumn {1}{c}{\bf (Abs. or}\\
\multicolumn{1}{c}{\bf Function}  &\multicolumn{1}{c}{\bf Interval} &
\multicolumn{1}{c}{\bf Error} & \multicolumn{1}{c}{\bf Rel.)}\\
SBESJ0 & [$-$5.6, 5.6] & \hspace{10pt} $6.5\rho $ & (Abs.)\\
 & [1.0, 1.0E6] & \hspace{10pt} $2.1\rho x^{\frac{1}{2}}$ & (Abs.)\\
SBESJ1 & [$-$7.2, 7.2] & \hspace{10pt} $3.8\rho $ & (Abs.)\\
 & [1.0, 1.0E6] & \hspace{10pt} $1.1\rho x^{\frac{1}{2}}$ & (Abs.)\\
SBESY0 & [0.00, 0.32] & \hspace{10pt} $7.6\rho $ & (Rel.)\\
 & [0.32, 1.12] & \hspace{10pt} $6.0\rho $ & (Abs.)\\
 & [1.12, 4.00] & \hspace{10pt} $1.9\rho $ & (Abs.)\\
 & [1.0, 1.0E6] & \hspace{10pt} $2.1\rho x^{\frac{1}{2}}$ & (Abs.)\\
SBESY1 & [0.0, 1.1] & \hspace{10pt} $6.6\rho $ & (Rel.)\\
 & [1.1, 5.5] & \hspace{10pt} $3.0\rho $ & (Abs.)\\
 & [1.0, 1.0E6] & \hspace{10pt} $1.3\rho x^{\frac{1}{2}}$ & (Abs.)
\end{tabular}

For the functions $J_n(x)$ and $Y_n(x)$ the absolute error is approximated
by $2|x|^{\frac{1}{2}}\rho $ for large $|x|$ while the amplitude of the
function values decreases like $0.8|x|^{-\frac{1}{2}}$. Thus no accuracy at
all can be expected when $2|x|^{\frac{1}{2}}\rho > 0.8|x|^{-\frac{1}{2}}$, $i.e.$%
, when $|x| > 0.4\rho ^{-1}$. The subprograms assume less than one decimal
digit of accuracy could be produced when $x > 0.04\rho ^{-1}$, and issue an
error message.

As a test of the double precision subprograms and an additional test of
the single precision
subprograms the test function, $z(x)$, defined above in Section C, was
evaluated in double precision and in single precision at selected points
ranging from $10^0$ to $10^7$. The machine arithmetic accuracies were $%
\rho_1 = 2^{-27} \approx 0.745\times 10^{-8}$ for single precision
and $\rho _2 = 2^{-60} \approx 1.15\times 10^{- 18}$ for double precision.
The magnitude of $z(x)$ computed in single precision was bounded
by $8\rho _1$ for $10^0 \leq x \leq 10^4$ and had the value
$22\rho _1$ for $x = 10^5$ and $x = 10^6$. The magnitude of $z(x)$ computed
in double precision was bounded by $9\rho _2$ for $10^0 \leq x \leq 10^7.$

This accuracy is much greater than the individual subprograms, SBESJ0, etc.,
deliver for large arguments. Thus the very accurate values of $z(x)$ must be
due to a functional relation existing between the algorithms implemented in
the different cylindrical Bessel function subprograms.

\bibliography{math77}
\bibliographystyle{math77}

\subsection{Error Procedures and Restrictions}

These subprograms return a zero result and issue an error message if

\begin{tabbing}
\hspace{.3in}\=(a)\quad \=$x \leq 0 \text{ for } Y_0\text{ or }Y_1,$\\
or\\
\>(b)\>$|x| > 0.04\rho ^{-1}\text{ for }J_0,\ J_1,\ Y_0\text{ or }Y_1.$
\end{tabbing}

The subprograms use R1MACH(4) or D1MACH(4) for $\rho $.  The
system-supplied sine and cosine subprograms may also have a cutoff
value close to $\rho ^{-1}$. If it is less than $0.04\rho ^{-1}$ then
there will be values of $x$ that will pass through the tests and
trigger an error message from the sine or cosine subprogram.

\subsection{Supporting Information}

The source language is ANSI Fortran~77.

\begin{tabular}{@{\bf}l@{\hspace{5pt}}l}
\bf Entry & \hspace{.35in} {\bf Required Files}\vspace{2pt} \\
DBESJ0 & \parbox[t]{2.7in}{\hyphenpenalty10000 \raggedright
AMACH, DBESJ0, DBMP0, DCSEVL, DERM1, DERV1, DINITS, ERFIN, ERMSG,
IERM1, IERV1\rule[-5pt]{0pt}{8pt}}\\
DBESJ1 & \parbox[t]{2.7in}{\hyphenpenalty10000 \raggedright
AMACH, DBESJ1, DBMP1, DCSEVL, DERM1, DERV1, DINITS, ERFIN, ERMSG,
IERM1, IERV1\rule[-5pt]{0pt}{8pt}}\\
DBESY0 & \parbox[t]{2.7in}{\hyphenpenalty10000 \raggedright
AMACH, DBESJ0, DBESY0, DBMP0, DCSEVL, DERM1, DERV1, DINITS, ERFIN, ERMSG,
IERM1, IERV1\rule[-5pt]{0pt}{8pt}}\\
DBESY1 & \parbox[t]{2.7in}{\hyphenpenalty10000 \raggedright
AMACH, DBESJ1, DBESY1, DBMP1, DCSEVL, DERM1, DERV1, DINITS, ERFIN,
ERMSG, IERM1, IERV1\rule[-5pt]{0pt}{8pt}}\\
% \end{tabular}
% \begin{tabular}{@{\bf}l@{\hspace{5pt}}l}
% \bf Entry & \hspace{.35in} {\bf Required Files}\vspace{2pt} \\
SBESJ0 & \parbox[t]{2.7in}{\hyphenpenalty10000 \raggedright
AMACH, ERFIN, ERMSG, IERM1, IERV1, SBESJ0, SBMP0, SCSEVL, SERM1,
SERV1, SINITS\rule[-5pt]{0pt}{8pt}}\\
SBESJ1 & \parbox[t]{2.7in}{\hyphenpenalty10000 \raggedright
AMACH, ERFIN, ERMSG, IERM1, IERV1, SBESJ1, SBMP1, SCSEVL, SERM1,
SERV1, SINITS\rule[-5pt]{0pt}{8pt}}\\
SBESY0 & \parbox[t]{2.7in}{\hyphenpenalty10000 \raggedright
AMACH, ERFIN, ERMSG, IERM1, IERV1, SBESJ0, SBESY0, SBMP0, SCSEVL,
SERM1, SERV1, SINITS\rule[-5pt]{0pt}{8pt}}\\
SBESY1 & \parbox[t]{2.7in}{\hyphenpenalty10000 \raggedright
AMACH, ERFIN, ERMSG, IERM1, IERV1, SBESJ1, SBESY1, SBMP1, SCSEVL,
SERM1, SERV1, SINITS\rule[-5pt]{0pt}{8pt}}\\
\end{tabular}

Subprograms SBESJ0, SBESJ1, SBESY0 and SBESY1 designed and developed by L.
W. Fullerton, Los Alamos, 1977. Adapted to Fortran~77 and the MATH77
library by C. L. Lawson and S. Chiu, JPL, 1984.


\begcodenp
\enlargethispage*{6pt}
\lstset{language=[77]Fortran,showstringspaces=false}
\lstset{xleftmargin=.8in}
\centerline{\bf \large DRSBESJ0}\vspace{-3pt}
\lstinputlisting{\codeloc{sbesj0}}
\vspace{4pt}\centerline{\bf \large ODSBESJ0}\vspace{3pt}
\lstset{language={}}
\lstinputlisting{\outputloc{sbesj0}}
\closegraphsfile
\end{document}
