\documentclass[twoside]{MATH77}
\usepackage{multicol}
\usepackage[fleqn,reqno,centertags]{amsmath}
\begin{document}
\begmath 9.1 Find a Local Minimum of a Univariate Function

\silentfootnote{$^\copyright$1997 Calif. Inst. of Technology, \thisyear \ Math \`a la Carte, Inc.}

\subsection{Purpose}

Find a local minimum of a univariate function, $f(x)$, in the closed
interval between specified abscissae, $a$ and $b$, to within a specified
tolerance, TOL.

\subsection{Usage}

This subroutine uses reverse communication, $i.e.$, it returns to the
calling program each time it needs to have $f()$ evaluated at a new value of
$x.$

\subsubsection{Program Prototype, Single Precision}
\begin{description}
\item[INTEGER]  \ {\bf MODE}
\item[REAL]  \ {\bf X, XORF, TOL}
\end{description}
\begin{tabbing}
\hspace{.2in}\= MODE = 0\\
\>X = An endpoint of the search interval\\
\>XORF = The other endpoint\\
\>TOL = Tolerance on $x$\\
10\> continue
\end{tabbing}\vspace{-3pt}
$$
\fbox{{\bf CALL SFMIN(X, XORF, MODE, TOL)}}
$$
\begin{tabbing}
\hspace{.2in}\=if( MODE .eq. 1) then\\
\>\ \ \ \ \=XORF = f() evaluated at X\\
\>\>go to 10\\
\>endif
\end{tabbing}
Computed quantities are returned in MODE, X, and XORF.

\subsubsection{Argument Definitions}
\begin{description}
\item[X, XORF, MODE]  \ [all are inout] When starting a new problem, the
user must set MODE $=0$, X = $a$, and XORF = $b$. The value of MODE should not
be changed during the solution process, except to control detailed printing as
explained in Section D.

$a$ and $b$ denote endpoints defining a closed interval in which a local minimum
is to be found. Permit $a < b$ or $a > b$ or $a = b$.

On each return after a call with MODE = 0 or 1, this subroutine will set
MODE to a value in the range [1:4] to indicate the action needed from the
calling program or the status on termination.
\begin{itemize}
\item[= 1]  \ means the calling program must evaluate $f($X), store the
value in XORF, and then call this subroutine again.
\item[= 2]  \ means normal termination. XORF contains the value of $f($X)
where X is a local minimum of the function in the interval specified.
\item[= 3]  \ same as $\text{MODE}=2$, except the requested accuracy was not
obtained.
\item[= 4]  \ means error termination due to SFMIN having been entered with
MODE $>$ 1.
\end{itemize}
\item[TOL]  \ [in] An absolute tolerance on the uncertainty in the final
estimate of the local minimum. If TOL $\leq $ 0, SFMIN attempts to get
all the accuracy it can. MODE will not be set to~3 in this case. Let $%
\epsilon $ denote the machine precision, the value of which is obtained
by reference to R1MACH(4) (D1MACH(4) for double precision), see
Chapter~19.1. The operational tolerance, $\tau $, at any trial abscissa, X,
will be
\begin{equation*}
\tau =(2/3)\times \text{TOL}+2\times |\text{X}|\times \epsilon ^{1/2}.
\end{equation*}
\end{description}
\subsubsection{Modifications for Double Precision}

For double-precision usage change the name SFMIN to DFMIN, and change the
REAL declaration to DOUBLE PRECISION.

\subsection{Examples and Remarks}

The program DRSFMIN illustrates the use of SFMIN to compute the local
minimum of the function, $f(x)=2^x+2^{-2x}$, for which the exact answer is $%
x=1/3$, with a minimum value of $(3/2)2^{1/3}\approx 1.88988$. Output is
shown in ODSFMIN.

This algorithm evaluates $f$ at an initial endpoint, $a$ or $b$, only if the
descent process leads to one of these points.

\subsection{Functional Description}

At the beginning of each iteration after the first, the algorithm has three
ordered abscissae, $a^{\prime}< x < b^{\prime}$. The point $x$ will have the
smallest function value found up to this point. All further searching will
be in the closed interval $[a^{\prime},b^{\prime}]$. The algorithm will not
necessarily have evaluated the function at $a^{\prime}$ and $b^{\prime}$.
There may also be up to two saved points, $w$ and $v$, with associated saved
function values satisfying $f(x) \leq f(w) \leq f(v)$. The points $w$ and $v$
may or may not coincide with $a^{\prime}$ and $b^{\prime}$. If $\max
(x-a^{\prime}$, $b^{\prime}-x) \leq \tau $, where $\tau $ is the operational
tolerance defined above in Section B, the algorithm terminates, returning $x$
as the approximate abscissa of the local minimum.

If the algorithm has not converged then the algorithm generates a new point
strictly between $a^{\prime}$ and $b^{\prime}$ by use of either
\begin{itemize}
\item[(1)]  \ parabolic interpolation using $x$, $w$, $v$, $f(x)$, $f(w)$,
and $f(v)$, or
\item[(2)]  \ golden section prediction using $a^{\prime}$, $x$, and $%
b^{\prime}.$
\end{itemize}
The algorithm chooses between these two methods on the basis of tests
designed to achieve a balance between rapid progress and reliability. Let $%
e_i$ denote the distance of the $i^{th}$ trial abscissa from the solution
abscissa. For a sufficiently smooth function whose first derivative has a
simple zero at the minimizing abscissa, and when the trial point is
sufficiently close to the solution, Method~1 converges according to $e_{i+1}
= e^{1.324}_i$. Method~2 is slower but more reliable, converging according
to $(b^{\prime}-a^{\prime})_{i+1} = 0.618(b^{\prime}-a^{\prime})_i$ regardless
of the smoothness of the function or the proximity to the solution. Note
that five steps of Method~2 gains somewhat more than one decimal place in
the solution since $(0.618)^5 \approx 0.090$.

This algorithm, with a Fortran~66 implementation, is presented in
\cite{Forsythe:1977:CMM}.  The algorithm is due to R.  P.  Brent,
\cite{Brent:1973:AMW}.

The algorithm as given in \cite{Forsythe:1977:CMM} never evaluates $f$ at
either of the initially given endpoints.  In cases in which the minimum is
at an endpoint, the algorithm returns a final abscissa that differs from
the endpoint by $\tau $, and may use many iterations.  We have altered
the code to permit evaluation at an endpoint when the search process
approaches an endpoint, and to keep the number of iterations used for an
endpoint minimum essentially the same as for an interior minimum.

\subparagraph{Detailed printing}

Before the initial call with MODE $= 0$, or any time during the iterative
process, the user may initiate or stop the detailed output by calling SFMIN
with a negative value of MODE. There is no problem-solving action on such a
call. A saved counter is set so detailed output will be written using
WRITE(*,...) on the next $|$MODE$|-1$ normal calls. To resume normal
computation the calling program must set MODE $= 0$ if a new problem
sequence is being started or MODE\ $= 1$ if an iterative sequence is being
continued.

The detailed output will consist of X, XORF, and IC. IC is an internal variable
that is initially~0 and is incremented by~1 whenever a golden section move
is used rather than a move determined from a parabola. If IC gets to~4, it
will either be set to $-$99, or it serves as an internal flag to check the
results at an endpoint.

\bibliography{math77}
\bibliographystyle{math77}

\subsection{Error Procedures and Restrictions}

SFMIN permits the given endpoints to satisfy $a < b,\ a > b,\text{ or }a = b$.
In the latter case the solution is $x = a$, and it will be found in one
iteration.

The user must initially set MODE $= 0$ and must not alter MODE after that
while iterating on the same problem, except to obtain the detailed
printing as described in Section D. Entering SFMIN with MODE $>$ 1
is an error condition and an error message will be issued using the
error processing routines of Chapter~19.2 with an error level of 0.

SFMIN uses the Fortran~77 SAVE statement to save values of internal
variables between the successive calls needed to solve a problem. Thus SFMIN
cannot be used to work on more than one problem at a time. In particular,
calling SFMIN with MODE $= 0$ will always initialize it for a new problem
and no data will be retained from a previous problem, except for the
internal detailed printing counter mentioned in Section D. If one needs this
type of capability, possibly because of a recursive formulation of a
multivariate minimization problem, one could make multiple copies of this
subroutine, giving a distinct name to each copy.  DMLC01,
Chapter~9.2, is recommended for most such problems, however.

\subsection{Supporting Information}

The source language is ANSI Fortran~77.

\begin{tabular}{@{\bf}l@{\hspace{5pt}}l}
\bf Entry & \hspace{.35in} {\bf Required Files}\vspace{2pt} \\
DFMIN & \parbox[t]{2.7in}{\hyphenpenalty10000 \raggedright
AMACH, DFMIN, ERFIN, ERMSG, IERM1, IERV1\rule[-5pt]{0pt}{8pt}}\\
SFMIN & \parbox[t]{2.7in}{\hyphenpenalty10000 \raggedright
AMACH, ERFIN, ERMSG, IERM1, IERV1, SFMIN}\\
\end{tabular}

Adapted to Fortran~77 for the JPL MATH77 library from the subroutine, FMIN,
in \cite{Forsythe:1977:CMM} by C. L. Lawson and F. T. Krogh, JPL, Oct.~1987.


\begcodenp
\lstset{language=[77]Fortran,showstringspaces=false}
\lstset{xleftmargin=.8in}

\centerline{\bf \large DRSFMIN}\vspace{10pt}
\lstinputlisting{\codeloc{sfmin}}

\vspace{30pt}\centerline{\bf \large ODSFMIN}\vspace{10pt}
\lstset{language={}}
\lstinputlisting{\outputloc{sfmin}}

\end{document}
