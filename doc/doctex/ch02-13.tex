\documentclass[twoside]{MATH77}
\usepackage{multicol}
\usepackage[fleqn,reqno,centertags]{amsmath}
\begin{document}
\begmath 2.13 Inverse Error Function and Inverse \hbox{Complementary Error Function}

\silentfootnote{$^\copyright$1997 Calif. Inst. of Technology, \thisyear \ Math \`a la Carte, Inc.}

\subsection{Purpose}

Given $x$, this pair of subprograms computes the value $t$ such that%
\begin{equation*}
x=\text{erf}(t)=\frac 2{\sqrt{\pi }}\int_0^te^{-s^2}ds
\end{equation*}
for the Inverse Error Function, or%
\begin{equation*}
x=\text{erfc}(t)=\frac 2{\sqrt{\pi }}\int_t^\infty e^{-s^2}ds
\end{equation*}
for the Inverse Complementary Error Function.

Procedures in Chapter~15.2 may be used to calculate the inverse of
the Gaussian or normal probability integral.

Reference \cite{ams55:erf} provides further discussion of the properties
of the error function.

\subsection{Usage}

\subsubsection{Program Prototype, Single Precision}

\begin{description}
\item[REAL]  \ {\bf X, SERFI, SERFCI, T}
\end{description}

Assign a value to X and use one of the following function references.

To compute the Inverse Error Function:
$$
\fbox{{\bf T = SERFI(X)}}
$$
To compute the Inverse Complementary Error Function:
$$
\fbox{{\bf T = SERFCI(X)}}
$$

\subsubsection{Argument Definitions}

\begin{description}
\item[X]  \ [in] Argument of function. Require $-1<\text{X}<1$ for SERFI, $0<%
\text{X}<2$ for SERFCI.
\end{description}

\subsubsection{Modification for Double Precision}

For double precision usage change the REAL type statement to DOUBLE
PRECISION and change the function names to DERFI and DERFCI respectively.

\subsection{Examples and Remarks}

See DRDERFI and ODDERFI for an example of the usage of these subprograms.

$|\frac{\text{d}}{\text{dx}}\text{erf}^{-1}(x)|$ and
$|\frac{\text{d}}{\text{dx}}\text{erfc}^{-1}(x)|$ are $\frac{\sqrt{\pi}}2
e^{x^2}$.  Therefore, the relative accuracy to be expected in evaluating
these functions decreases rapidly as $x$ increases.  For example, when using
IEEE single-precision arithmetic, one should expect no more than one digit of
the result to be correct when $x > \approx 3.6$

\subsection{Functional Description}

The computer approximations for these functions were developed by A.
Strecok, \cite{Strecok:1968:OCI}, using Chebyshev polynomial expansions.

These subprograms were tested on the IBM PC/AT, which uses IEEE
arithmetic with precision, $\rho \approx 1.19 \times 10^{-7}$ in single precision and
$\rho \approx 2.22 \times 10^{-16}$ in double precision.  We checked how well the
Error Function subprograms (SERF and SERFC or DERF and DERFC) and the
present subprograms are inverses. Two sets of relative error tests were
performed:
\begin{gather*}
\varepsilon _1 = |\erf(\erfi(x))/x - 1|\ /\ \rho,\ \ \text{and}\\
\varepsilon _2 = |\erfi(\erf(x))/x - 1|\ /\ \rho
\end{gather*}

and similarly for the complementary functions. The latter test does not
measure relative error precisely for the complementary functions, and was
not carried out in the range 1.0E$-$35 $\leq x \leq $ 1.0E$-$5 (1.0E$-$300 $%
\leq x \leq $ 1.0E$-$5 in double precision). The results are summarized below.

\begin{tabular}{llll}
 & \bf Argument & \bf Max. & \bf Max.\\
\bf Function & \bf \ \,Interval & \bf \ \ \ $\varepsilon _1$ &
\bf \ \ \ $\varepsilon _2$\\
SERFI & [0.1E$-$2, 0.8] & 0.956 & 0.993\\
 & [0.8, 0.9975] & 0.625 & 1.66\\
 & [.9975, .99999] & 0.0 & 1.50\\
SERFCI & [1.0E$-$35, 1.0E$-$5] & 113\\
 & [1.E$-$5, 2.5E$-$3] & 16.3 & 1.91E+4\\
 & [2.5E$-$3, 0.2] & 14.2 & 86.7\\
 & [0.2, 0.999] & 2.14 & 1.92\\
DERFI & [0.1E$-$2, 0.8] & 0.896 & 0.973\\
 &[0.8, 0.9975] & 0.625 & 1.59\\
 &[.9975, .99999] & 0.0 & 1.50\\
DERFCI & [1.0E$-$300, 1.0E$-$5] & 964\\
 & [1.E$-$5, 2.5E$-$3] & 13.6 & 1.90E+4\\
 &[2.5E$-$3, 0.2] & 8.29 & 64.6\\
 &[0.2, 0.999] & 1.69 & 1.60
\end{tabular}

\bibliography{math77}
\bibliographystyle{math77}

\subsection{Error Procedures and Restrictions}

These subprograms issue an error message and terminate execution if

\hspace{.4in}(a) \ \ X $\leq -1$ or X $\geq 1$ for SERFI, or

\hspace{.4in}(b) \ \ X $\leq 0$ or X $\geq 2$ for SERFCI.

The error message is issued by way of the error message processor at level 0,
and the returned function value is 0.0.  If it is desired, in the event an error
occurs, to issue the error message and then terminate program execution

\hspace{.4in}CALL ERMSET (2)

before invoking SERFI or SERFCI.

See Chapter~19.2 for further description of the error message processor.

\subsection{Supporting Information}

The source language is ANSI Fortran~77.

Subprograms designed and developed by W. V. Snyder, JPL, 1986.


\begin{tabular}{@{\bf}l@{\hspace{5pt}}l}
\bf Entry & \hspace{.35in} {\bf Required Files}\vspace{2pt} \\
DERFCI & \parbox[t]{2.7in}{\hyphenpenalty10000 \raggedright
AMACH, DERFI, DERM1, DERV1, ERFIN, ERMSG\rule[-5pt]{0pt}{8pt}}\\DERFI & \parbox[t]{2.7in}{\hyphenpenalty10000 \raggedright
AMACH, DERFI, DERM1, DERV1, ERFIN, ERMSG\rule[-5pt]{0pt}{8pt}}\\SERFCI & \parbox[t]{2.7in}{\hyphenpenalty10000 \raggedright
AMACH, ERFIN, ERMSG, SERFI, SERM1, SERV1\rule[-5pt]{0pt}{8pt}}\\SERFI & \parbox[t]{2.7in}{\hyphenpenalty10000 \raggedright
AMACH, ERFIN, ERMSG, SERFI, SERM1, SERV1}\\\end{tabular}

\begcode

\bigskip

\lstset{language=[77]Fortran,showstringspaces=false}
\lstset{xleftmargin=.8in}

\centerline{\bf \large DRDERFI}\vspace{10pt}
\lstinputlisting{\codeloc{derfi}}

\vspace{30pt}\centerline{\bf \large ODDERFI}\vspace{10pt}
\lstset{language={}}
\lstinputlisting{\outputloc{derfi}}

\end{document}
