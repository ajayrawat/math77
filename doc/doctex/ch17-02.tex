\documentclass[twoside]{MATH77}
\usepackage{multicol}
\usepackage[fleqn,reqno,centertags]{amsmath}
\begin{document}
\begmath 17.2 Computation Using Partial Derivative Arrays or
\hbox{Multivariate Taylor Series}

\silentfootnote{$^\copyright$1997 Calif. Inst. of Technology, \thisyear \ Math \`a la Carte, Inc.}

\subsection{Purpose}

This set of subroutines performs computations in which each variable is
represented by its value, and its first and second partial derivatives with
respect to N independent variables, evaluated at a specified value of these
variables. Such a set of numbers can alternatively be regarded as a scaled
representation of the low order coefficients of the multivariate Taylor
series of the dependent variable expanded at the evaluation point.

More specifically let $u({\bf x})$ denote a scalar valued function of an
N-component argument vector, ${\bf x}$. Let $u_0$, ${\bf g}_0$, and $H_0$ denote,
respectively, the value of $u$, the gradient vector, and the Hessian matrix,
evaluated at ${\bf x}_0$. Then $\{u_0$, ${\bf g}_0$, $H_0\}$ is the set of data these
programs carry to represent $u$ evaluated at ${\bf x}_0$. The Taylor series for $u$
through second order, expanded at ${\bf x}_0$, is
\begin{equation*}
u({\bf x}) = u_0 + {\bf g}^t_0({\bf x}-{\bf x}_0) + \frac{1}{2}({\bf
x}-{\bf x}_0)^t H_0({\bf x}-{\bf x}_0)
\end{equation*}
This package provides a way of computing values of the first and second
partial derivatives of a multivariate function that is defined by a sequence
of computational steps involving arithmetic and elementary functions,
without the need to derive and code expressions for the partial derivatives.

\subsection{Usage}

\subparagraph{Definition of a U-variable}

We shall use ${\bf x}$ as the generic name of the N-dimensional independent
variable vector with respect to which all partial derivatives are defined.
The term U-variable is used to denote a sequence of values, consisting of a
function value, values of the function's gradient vector, and optionally the
(packed) Hessian matrix, evaluated at some point. We assume the Hessian
matrix is symmetric, so only one copy of each symmetric pair of elements
needs to be stored. We store the lower triangle of the Hessian matrix by
rows. Equivalently, one could regard this as representing the upper triangle
by columns. Thus if U() is an array containing a U-variable, $\{u$, ${\bf
g}$, $H\}$, its contents are defined as follows:
\begin{alignat*}{2}
\text{U}(1) &=u\\
\text{U}(1+j) &= g_j &= &\:\partial u/\partial x_j,\ j = 1,\ ...,N\\
\hspace{-15pt}\text{U}(1+\text{N}+j+i(i-1)/2)&=\: h_{i,j}\: &= &\:
\partial ^2u/\partial x_i\partial x_j,\\
&&&\hspace{-.5in}i = 1,\ ..., N;\ j = 1,\ ...,\ i.
\end{alignat*}

The required dimension size for a U-variable is $dimu$ = N + 1 if only first
partial derivatives will be requested, and $dimu$ = (N +2)(N + 1) / 2 if both
first and second partial derivatives are to be computed. For convenience we
list the value of $dimu$ {\bf for the second derivative case} for some values
of N:

\begin{tabular}{@{}r*{15}{@{\ }r}}
N = & 1 & 2 & 3 & 4 & 5 & 6 & 7 & 8 & 9 & 10 & 11 & 12 & 13 & 14 & 15\\
$dimu$ = & 3 & 6 & 10 & 15 & 21 & 28 & 36 & 45 & 55 & 66 & 78 & 91 & 105 &
120 & 136
\end{tabular}

\subparagraph{Distinguishing property of an independent variable}

For $i$ in [1, N] the scalar component, $x_i$, of the independent variable
vector, ${\bf x}$, is distinguished by the unique property that all of its first
and second partial derivatives are zero except its first partial derivative
with respect to $x_i$, which has the value~1.

\subparagraph{The variables N, M1, and M2.}

N specifies the number of independent
variables, and implicitly specifies the storage layout of components within
a U-variable.  M1 and M2 specify the lowest and highest order of partial
derivatives which the called subroutines are to produce.  A program that
uses these subroutines must initially call SUSETN (or DUSETN)
to set N, M1, and M2.

\subparagraph{The subroutines}

The subroutines of this package are described in the following sections:

\begin{description}
\item[B.1]  SUSETN, Assigning values to N, M1, and M2.

\item[B.2]  SUGETN, Fetching values of N, M1, M2, L1, and L2.

\item[B.3]  SUSET, Assigning a value to a U-variable.

\item[B.4]  The computational subroutines, except SUREV.

\item[B.5]  SUREV, series reversion, or function inversion.

\item[B.6]  Modifications for double-precision.
\end{description}

\subsubsection{SUSETN, Assigning values to N, M1, and M2}

Subroutine SUSETN must be called prior to calling any of the other
subroutines of this package.

\paragraph{Program Prototype, Single Precision}
\begin{description}
\item[INTEGER]  \ {\bf N, M1, M2}
\end{description}

Assign values to N, M1, and M2.
$$
\fbox{{\bf CALL SUSETN(N, M1, M2)}}
$$

\paragraph{Argument Definitions}

\begin{description}

\item[N]  \ [in] N specifies the number of independent
variables, and implicitly specifies the storage layout of components within
a U-variable.

\item[M1, M2]  \ [in] M1 and M2 specify the lowest and highest order of partial derivatives
which the called subroutines are to produce.  M1 and M2 must satisfy:
 0 $\leq $ M1 $\leq $ M2 $\leq 2.$

Most commonly one will probably
choose to set (M1, M2) = (0,0), (0,1), or (0,2), and leave the setting
unchanged throughout a computation. See Section C for discussion of other
strategies for setting (M1,M2).
\end{description}

\subsubsection{SUGETN, Fetching values of N, M1, M2, L1, and L2}

\paragraph{Program Prototype, Single Precision}
\begin{description}
\item[INTEGER]  \ {\bf N, M1, M2, L1, L2}
$$
\fbox{{\bf CALL SUGETN(N, M1, M2, L1, L2)}}
$$
Values are returned in N, M1, M2, L1, and L2.
\end{description}

\paragraph{Argument Definitions}

\begin{description}
\item[N, M1, M2]  \ [out] Returns the values that were set on the
previous call to SUSETN.

\item[L1, L2]  \ [out] Returns values computed from N, M1, and M2.
L1 and L2 are the indices of the first and last locations in a
U-variable array that will be subject to change due to the settings
of N, M1, and M2.\vspace{5pt}
\begin{gather*}
L1 = \begin{cases}
1 & \text{if} ~ \text{M1} = 0 \\
2 & \text{if} ~ \text{M1} = 1 \\
N+2 & \text{if} ~ \text{M1} = 2
\end{cases}\\
L2 = \begin{cases}
1 & \text{if} ~ \text{M1} = 0 \\
N+1 & \text{if} ~ \text{M1} = 1 \\
1+N+((N*(N+1))/2) & \text{if} ~ \text{M1} = 2
\end{cases}
\end{gather*}
\end{description}

\subsubsection{SUSET, Assigning a value to a U-variable}

The integer values N, M1, and M2 must be set by a call to SUSETN
before calling SUSET.  SUSET will only set partial derivatives of orders
M1 through M2.

\paragraph{Program Prototype, Single Precision}
\begin{description}
\item[INTEGER]  \ {\bf KEY}

\item[REAL]  \ {\bf VAL, U}($\geq dimu)$\newline
[$dimu = \text{N}+1 \text{ or (N}+2)(\text{N}+1)/2$, see above.]

\end{description}

Assign values to N, M1, M2, VAL, and KEY.
$$
\fbox{{\bf CALL SUSET(VAL, KEY, U)}}
$$
Computed quantities are returned in U().

\paragraph{Argument Definitions}

\begin{description}
\item[VAL]  \ [in] Value to be assigned to the U-variable, U().

\item[KEY]  \ [in] Integer in the range, [0, N]. If KEY = 0, U() is set to
represent a variable that is constant relative to the N independent
variables, $i.e.$, all of its first and second partial derivatives are set
to zero.

If 1 $\leq $ KEY $\leq $ N, U() is set to represent the $\text{KEY}^{th}$
independent variable, $i.e.$, its $\text{KEY}^{th}$ first partial derivative is set
to~1.0. All of its other first partial derivatives and all of its second
partial derivatives are set to zero.

\item[U()]  \ [out] Array in which this subroutine will define a U-variable
having the value, VAL, and having partial derivative values as specified by
KEY. Only derivative values of orders M1 through M2 are set.
\end{description}

\subsubsection{The computational subroutines, except SUREV}

The integer values N, M1, and M2 must be set by a call to SUSETN
before calling any of these subroutines.  These subroutines will only set
partial derivatives of orders M1 through M2.  To use these subroutines
with M1 $>$ 0 read the remarks about M1 $>$ 0 in Section C.

In describing the following subroutines, U() and V() denote input
U-variables and Z() denotes an output U-variable. The variables A and I are
input variables that are constant relative to ${\bf x}.$

In most of these subroutines the output array Z() must occupy storage
locations distinct from any of the input data. Exceptions to this rule are
SUSUM, SUDIF, SUPRO, SUSUM1, SUDIF1, and SUPRO1. The subroutine SUQUO, which
computes $u/v \rightarrow z$, permits $u$ and $z$ to occupy the same
storage, but $v$ and $z$ must occupy distinct storage.

\paragraph{Program Prototype, Single Precision}

\begin{description}
\item[INTEGER]  \ {\bf I}

\item[REAL]  \ {\bf A, U}($\geq dimu)${\bf ,V}($\geq dimu)${\bf , Z}($\geq
dimu)$\\
{[$dimu = \text{N}+1 \text{ or (N}+2)(\text{N}+1)/2$, see page~1.]}
\end{description}

Assign values to I, A, U(), and V(), as appropriate.

\begin{center}
{\bf Two-argument operations with both arguments depending on ${\bf x}.$}

\fbox{%
\begin{tabular}{@{\bf \ }lr}
CALL\ SUSUM(U,\ V,\ Z) & $u+v \rightarrow z$\\
CALL\ SUDIF(U,\ V,\ Z) & $u-v \rightarrow z$\\
CALL\ SUPRO(U,\ V,\ Z) & $u\times v \rightarrow z$\\
CALL\ SUQUO(U,\ V,\ Z) & $u/v \rightarrow z$\\
CALL\ SUATN2(U,\ V,\ Z) & atan2($u,v) \rightarrow z$\\
\multicolumn{2}{c}{where for atan2: $-\pi <z \leq \pi \text{ and } \tan (z)
= u/v$}\\
\end{tabular}}
\end{center}

\pagebreak
\begin{center}
{\bf Two-argument operations with only one argument depending on $u.$}

\fbox{%
\begin{tabular}{@{\bf \ }lr}
CALL SUSUM1(A, V, Z) & $a+v \rightarrow z$\\
CALL SUDIF1(A, V, Z) & $a-v \rightarrow z$\\
CALL SUPRO1(A, V, Z) & $a\times v \rightarrow z$\\
CALL SUQUO1(A, V, Z) & $a/v \rightarrow z$\\
CALL SUPWRI(I, V, Z) & $v^i \rightarrow z$\\
\multicolumn{2}{c}{(See following note.)}\\
\end{tabular}}
\end{center}

Note: I may be positive, negative, or zero. If I = 0, SUPWRI sets Z(1) = 1.0
and all derivative values of $z$ to~0.0, regardless of the given value of $v$%
. It is an error to have $v = 0.0$ when I $<$ 0.

\begin{center}
{\bf One-argument operations with the argument depending on ${\bf x}.$}

\fbox{%
\begin{tabular}{@{\bf \ }lr}
CALL SUSQRT(U, Z) & $\sqrt u \rightarrow z$\\
CALL SUEXP(U, Z) & $\exp (u) \rightarrow z$\\
CALL SULOG(U, Z) & $\log (u) \rightarrow z$\\
CALL SUSIN(U, Z) & $\sin (u) \rightarrow z$\\
CALL SUCOS(U, Z) & $\cos (u) \rightarrow z$\\
CALL SUTAN(U, Z) & $\tan (u) \rightarrow z$\\
CALL SUASIN(U, Z) & asin$(u) \rightarrow z$\\
CALL SUACOS(U, Z) & acos$(u) \rightarrow z$\\
CALL SUATAN(U, Z) & atan$(u) \rightarrow z$\\
CALL SUSINH(U, Z) & sinh$(u) \rightarrow z$\\
CALL SUCOSH(U, Z) & cosh$(u) \rightarrow z$\\
CALL SUTANH(U, Z) & tanh$(u) \rightarrow z$\\
\end{tabular}}
\end{center}

\paragraph{Argument Definitions}

\begin{description}
\item[A]  \ [in] Floating point value that is independent of ${\bf x}.$

\item[I]  \ [in] Integer value that is independent of ${\bf x}.$

\item[U()]  \ [in] An input U-variable. Must contain defined values for
derivatives of orders~0 through M2.

\item[V()]  \ [in] An input U-variable. Must contain defined values for
derivatives of orders~0 through M2.

\item[Z()]  \ [inout] If M1 $>$ 0, this array must contain defined values of
derivatives of orders~0 through M1 $-$ 1 on entry. On return it will also contain
computed values of derivatives of orders M1 through M2. If M1 = 0, no input
values are required in Z().
\end{description}

\subsubsection{SUREV, Series Reversion or Function Inversion}

Let $u_i$, $i = 1$, ..., $n$, be $n$ functions of an $n$-vector, ${\bf t}$.
Suppose values of the $u_i$'s and their first and second partial
derivatives with respect to the components of ${\bf t}$ are known for a particular
value of ${\bf t}$. Then, if the Jacobian matrix of the transformation is
nonsingular at this point, one can regard the $t_j$'s as functions
of the $u_i$'s in some neighborhood of this point in $u$-space. In
this situation this subroutine can compute values of the first and second
partial derivatives of the $t_j$'s with respect to the $%
u_i$'s at this point.

Given:
\begin{gather}
\hspace{-15pt}t_j \text{ in TU}(1,\ j),\ j = 1,\ ...,\ n,\\
\hspace{-15pt}u_i \text{ in UT}(1,\ i),\ i = 1,\ ...,\ n,\\
\hspace{-15pt}\partial u_i/\partial t_j\text{ in UT}(1+j,\ i),\ j = 1,\
...,\ n;\ i = 1,\ ...\ n,\hspace{-1in}\\
\hspace{-15pt}\partial ^2u_i/\partial t_j\partial t_k \text{ in UT}
(1+n+k+j(j-1)/2),\ i),\\
\hspace{.5in}j= 1,\ ...,\ n;\ k = 1,\ ...,\ j;\ i = 1,\ ...,\ n.
\notag
\end{gather}
If the Jacobian matrix with elements $\partial u_i/\partial t_j$ is
nonsingular, this subroutine will compute the first and second partial
derivatives of the $t_j$'s with respect to the $u_i$'s and
store them as
\begin{gather}
\hspace{-15pt}\text{TU}(1+i, j) = \partial t_j/\partial u_i,\ i =
1,\ ...,\ n;\ j = 1,\ ...,\ n,\hspace{-1in}\\
\hspace{-15pt}\text{TU}(1+n+k+i(i-1)/2,\ j) = \partial ^2t_j/\partial
u_i\partial u_k,\\
\hspace{.5in}i =1,\ ...,\ n;\ k = 1,\ ...,\ i;\ j = 1,\ ...,\ n,\notag
\end{gather}
The integer values N, M1, and M2 must be set by a call to SUSETN
before calling SUREV.  N gives
the value of $n$ for Eqs.(1--6) above.
Require 0 $\leq $ M1 $\leq $ M2 $\leq $ 2.

If M2 = 0, SUREV returns with no action.

If M2 = 1, SUREV only computes Eq.(5), and it is not necessary to provide
the data of Eq.(4).

If M2 = 2 and M1 = 0 or~1, SUREV computes Eqs.(5--6).

If M2 = 2 and M1 = 2, it is assumed that the assignment of Eq.(5) has
already been done, and thus SUREV only computes Eq.(6).

\paragraph{Program Prototype, Single Precision}

\begin{description}
\item[INTEGER]  \ {\bf IWORK}(N){\bf , LDIM}

\item[REAL]  \ {\bf RCOND, TU}(LDIM, $\geq $N){\bf , UT}(LDIM, $\geq $N){\bf %
, WORK}($\geq 3\times \text{N}^2$) \ [LDIM $\geq dimu = \text{N}+1 \text{ or}\\
(\text{N}+2)(\{text{N}+1)/2$, see page 1.]
\end{description}

Assign values to LDIM, UT() and part of TU().

\begin{center}
\fbox{%
\begin{tabular}{@{\bf }c}
CALL SUREV( UT, TU, LDIM,\\
RCOND, IWORK, WORK)\\
\end{tabular}}
\end{center}

Computed quantities are returned in RCOND and TU().

\paragraph{Argument Definitions}

\begin{description}
\item[UT(,)]  \ [in] Must contain the data of Eqs.(2--4) on entry, except as
noted above for certain values of M1 and M2.

\item[TU(,)]  \ [inout] Must contain the data of Eq.(1) on entry, and will
have the data of Eqs.(5--6) assigned by SUREV, except as noted above for
particular values of M1 and M2.

\item[LDIM]  \ [in] Leading dimension for the arrays UT(,) and TU(,).
Require LDIM $\geq dimu$, where $dimu  = \text{N}+1 \text{ or }
(\text{N}+2)(\text{N}+1)/2$, see
page~1.

\item[RCOND]  \ [out] Estimate of the reciprocal condition number of the
Jacobian matrix with elements $\partial u_i/\partial t_j$, given in UT().
Only computed when M1 $\leq $ 1 and M2 $\geq 1$. Is in the range [0.0,~1.0].

A well conditioned problem has values near~1.0. Near zero means badly
conditioned. Equal to zero means singular, in which case the subroutine
cannot complete its computation.

\item[IWORK()]  \ [scratch] Integer work space of size at least N.

\item[WORK()]  \ [scratch] Floating-point work space of size at least $%
3\times \text{N}^2.$
\end{description}

\subsubsection{Modifications for Double Precision}

For double-precision usage, replace the initial S in the name of each
subroutine by D, and replace the REAL declarations by DOUBLE PRECISION.

\subsection{Examples and Remarks}

\subsubsection{Example}

As a demonstration problem, consider the following formulae for transforming
a set of 3-dimensional rectangular coordinates $(x,y,z)$ to spherical
coordinates $(r,\varphi ,\theta )$ and the inverse transformation formulae:
\begin{align*}
s &= \sqrt{x^2 + y^2}\\
r &= \sqrt{s^2 + z^2}\\
\varphi &= \atan2(y,\ x)\\
\theta &= \tan^{-1}(z/s)\\
x &= r\ \cos \ \varphi \ \cos \ \theta \\
y &= r\ \sin \ \varphi \ \cos \ \theta \\
z &= r\ \sin \ \theta
\end{align*}
The demonstration program, DRSUCOMP, performs this mapping from $(x,y,z)$ to
$(r,\varphi ,\theta )$ with computation of all of the first and second partial
derivatives of $r$, $\varphi $, and $\theta $ with respect to $x$, $y$, and $z$%
. As a check on the computation the program then transforms back to $(x,y,z)$%
. Results are shown in ODSUCOMP. Note that the final $(x,y,z)$ agrees with
the initial $(x,y,z)$ in the values and the first and second partial
derivatives.

To demonstrate series reversion (or function inversion) the program assigns
the values of $(r,\varphi ,\theta )$ and its first and second partial
derivatives with respect to $(x,y,z)$ to the array UT(,), and then assigns the
values of $(x,y,z)$ to TU(1,~1:3). It then uses SUREV to compute the first
and second partial derivatives of $(x,y,z)$ with respect to $(r,\varphi ,\theta
)$, storing these in TU(2:10,~1:3). This computation is checked by computing
the same quantities in a different way.

\subsubsection{Omitting derivative computation}

To save time when developing new code using this package, one may run with
(M1, M2) = (0,~0) until one is satisfied that the function evaluation is as
desired, and then increase M2 to activate the derivative computation.

\subsubsection{Setting M1 $>$ 0}

There are some algorithms, such as for optimization, in which one does not
need to compute partial derivatives at every point at which the function is
evaluated. Depending on how significant efficiency is in a particular
application, one may wish to consider methods of separating the function and
derivative computation using this package. One can compute only the function
value by setting (M1, M2) = (0,~0), or one can compute the function value
and first partial derivatives by setting (M1, M2) = (0,~1).

If one has computed the function value at an argument value, ${\bf x}$, with (M1,
M2) = (0,~0), and has kept all intermediate quantities in distinct storage
locations, then one can repeat the sequence of subroutine calls with (M1,
M2) = (1,~1) to compute the first partial derivatives at the same point,
${\bf x}$%
, without the function value being recomputed. In general, computation with
M1 $>$ 0 is only valid if all $in$ and $inout$ arrays in each subroutine call
contain values of all derivatives of orders~0 through M1 $-$ 1 computed
previously with the same ${\bf x}.$

\subsection{Functional Description}

This U-computation package is based on the ideas presented in
\cite{Wengert:1964:ASA}.  See the description of W-computation in
Chapter~17.1 for a summary of these ideas.  Whereas the W-computation
package generalizes the order of differentiation, this U-computation
package generalizes the number of independent variables.

Let $f()$ be a scalar-valued function of the scalar variable $u$, and use
primes to denote differentiation with respect to $u$. Let $u$ depend on the
N-vector ${\bf x}$ and denote partial derivatives of $u$ with respect to $x_i$ by
appropriate subscripts. This package computes first and second partial
derivatives of $f()$ using the formulae:
\begin{align*}
\partial f(u)/\partial x_i &= f^{\prime}(u) u_i\\
\partial ^2f(u)/\partial x_i\partial x_j &= f^{\prime}(u) u_{i,j} +
f^{\prime\prime}(u) u_i u_j
\end{align*}

\nocite{Griewank:1991:ADA}
\nocite{Lawson:1971:CDU}
\bibliography{math77}
\bibliographystyle{math77}

\subsection{Error Procedures and Restrictions}

N must be positive. M1 and M2 must satisfy 0\ $\leq $ M1
$\leq $ M2 $\leq $ 2. See the discussion in Section~C for cautions regarding
usage with M1 $>$ 0. These conditions on N, M1, and M2 are not tested by
the subroutines and their violation may cause unpredictable actions.

\subsubsection{Invalid arguments for derivative computation}

The user will likely be accustomed to avoiding sending invalid arguments to
the elementary functions, such as a negative argument to the square root. In
computing derivatives there are some additional singularities to avoid. Note
that the derivative is infinite at zero for the square root, and at $\pm $1
for arcsin and arccosine.

\subsubsection{Error handling}

Following is a list of error conditions the package detects and for which error
messages are issued. These errors are fatal in the sense that the requested
operation cannot be done; however, the default action is to return after
issuing an error message. The user can use the MATH77 library subroutine,
ERMSET of Chapter~19.2, to alter this action to cause a STOP if desired. Error conditions
not on this list, $e.g.$, negative argument in log, will be handled by the
usual host system error handler.

\begin{tabular}{@{}l@{\ \ }l}
\bf Error No.\\
\bf \& Program & \multicolumn{1}{c}{\bf Explanation}\\
\bf 1 SUASIN & Infinite derivative when arg = $-$1\ or +1\\
\bf 1 SUACOS & Infinite derivative when arg = $-1$\ or +1\\
\bf 2 SUSQRT & Infinite derivative when arg = 0\\
\bf 3 SUQUO1 & Zero divisor\\
\bf 4 SUPWRI & $\text{U}^{\text{M}}$ is infinite when U = 0 and M $<$ 0\\
\bf 5 SUQUO & Zero divisor\\
\bf 6 SUREV & Singular Jacobian matrix
\end{tabular}

\subsection{Supporting Information}

The source language is ANSI Fortran~77. All program units reference
COMMON blocks /UCOM1/ and /UCOM2/.

All of the double precision entry points except DUREV require files:

DUCOMP, ERFIN, ERMSG, IERM1, and IERV1.

All of the single precision entry points except SUREV require files:

SUCOMP, ERFIN, ERMSG, IERM1, and IERV1.

DUREV requires: DASUM, DAXPY, DDOT, DGECO, DGEFA, DGEI, DSCAL, DSWAP,
DUREV, ERFIN, ERMSG, and IDAMAX.

SUREV requires: SASUM, SAXPY, SDOT, SGECO, SGEFA, SGEI, SSCAL, SSWAP,
SUREV, ERFIN, ERMSG, and ISAMAX.\vspace{-5pt}
\begin{center}
\begin{tabular}{llll}
\multicolumn{4}{c}{\bf Entries}\\
DUACOS & DUASIN & DUATAN & DUATN2\\
DUCOS & DUCOSH & DUDIF & DUDIF1\\
DUEXP & DUGETN & DULOG & DUPRO\\
DUPRO1 & DUPWRI & DUQUO & DUQUO1\\
DUREV & DUSET & DUSETN & DUSIN\\
DUSINH & DUSQRT & DUSUM & DUSUM1\\
DUTAN & DUTANH & SUACOS & SUASIN\\
SUATAN & SUATN2 & SUCOS & SUCOSH\\
SUDIF & SUDIF1 & SUEXP & SUGETN\\
SULOG & SUPRO & SUPRO1 & SUPWRI\\
SUQUO & SUQUO1 & SUREV & SUSET\\
SUSETN & SUSIN & SUSINH & SUSQRT\\
SUSUM & SUSUM1 & SUTAN & SUTANH\\
\end{tabular}
\end{center}\vspace{-5pt}
Designed by C. L. Lawson, JPL, 1971. Adapted to Fortran~77 for the JPL
MATH77 library, Aug.~1987. Added SUREV/DUREV, February~1992.  Added
SUSETN, DUSETN, SUGETN, and DUGETN August~1994.


%\rule{0pt}{80pt}

\begcodenp
\lstset{language=[77]Fortran,showstringspaces=false}
\lstset{xleftmargin=.8in}

\centerline{\bf \large DRSUCOMP}\vspace{10pt}
\lstinputlisting{\codeloc{sucomp}}
\newpage

\vspace{30pt}\centerline{\bf \large ODSUCOMP}\vspace{10pt}
\lstset{language={}}
\lstinputlisting{\outputloc{sucomp}}
\end{document}
