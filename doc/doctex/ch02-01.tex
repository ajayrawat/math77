\documentclass[twoside]{MATH77}
\usepackage{multicol}
\usepackage[fleqn,reqno,centertags]{amsmath}
\begin{document}
\begmath 2.1 Inverse Hyperbolic Functions

\silentfootnote{$^\copyright$1997 Calif. Inst. of Technology, \thisyear \ Math \`a la Carte, Inc.}

\subsection{Purpose}

These subprograms compute the inverse hyperbolic functions.

\subsection{Usage}

\subsubsection{Program Prototype, Single Precision}

\begin{description}
\item[REAL]  \ {\bf X, U, SASINH, SACOSH, SATANH, SACTNH, SASECH, SACSCH}
\end{description}

Assign a value to X and obtain the desired value of an inverse hyperbolic
function using one of the following:
\begin{center}
\begin{tabular}{l@{\hspace{.3in}}l}
\fbox{\bf U = SASINH(X)} & \fbox{\bf U = SACOSH(X)}\\\\
\fbox{\bf U = SATANH(X)} & \fbox{\bf U = SACSCH(X)}\\\\
\fbox{\bf U = SASECH(X)} & \fbox{\bf U = SACTNH(X)}
\end{tabular}
\end{center}
where the functions SASINH, ..., SACTNH compute respectively, the inverse
hyperbolic: sine, cosine, tangent, cosecant, secant, and cotangent.

\subsubsection{Program Prototype, Double Precision}

For double precision computation use the function names DASINH, DACOSH,
DATANH, DACSCH, DASECH, and DACTNH and type the argument, function name and
result double precision.

\subsection{Example and Remarks}

See DRSASINH and ODSASINH for an example of the usage of these subprograms.

\subsection{Functional Description}

\subsubsection{Method}

The basic formulas and valid argument domains are
\begin{equation*}
\hspace{-15pt}\begin{array}{ll}
\asinh(x) = \sgn (x) \ln (|x| + \sqrt {x^2+1})& \text{all }x\\
\acosh(x) = \ln (x + \sqrt {x^2-1})& x \geq 1\\
\atanh(x) = \frac{1}{2} \sgn (x) \ln ((1+|x|)/(1-|x|))& |x| < 1\\
\acsch(x) = \asinh(1/x)& x \neq 0\\
\asech(x) = \acosh(1/x)& 0 < x \leq 1\\
\actnh(x) = \atanh(1/x)& |x| > 1
\end{array}
\end{equation*}
To avoid unnecessary loss of relative accuracy as function values approach
zero, a different method is used whenever the argument of the logarithm
function would be in the range [1.0,~2.718]. In this range the argument for
acosh or atanh is converted to an argument for asinh, and asinh is computed
by argument reduction to the interval [0.0,~0.125326] followed by evaluation
of its Taylor series. The number of terms needed in this series is
determined by use of the System Parameters subprogram, Chapter~19.1. The
prestored coefficients will support accuracy to about 25 significant decimal
digits.

For large arguments the formulas involving $x^2$ are reformulated to avoid
unnecessary overflow. Specifically, when $x > 10^{16}$ it is presumed that $%
x^2 \pm 1$ will not be distinguishable from $x^2$, and thus the formulas
given above for asinh and acosh are replaced by
\begin{align*}
\asinh(x) &= \sgn (x) [\ln (2) + \ln (|x|)], \quad \text{and}\\
\acosh(x) &= \ln (2) + \ln (x)
\end{align*}
Let $\Omega $ denote the machine overflow limit and $\rho $ denote
the difference between $1.0$ and the next smaller machine number. Define $a
= \frac{1}{2} \ln (2/\rho )$ and $b = \ln (2\Omega )$. Then the
ranges of the computed function values are
\begin{equation*}
\begin{array}{l@{\qquad }l}
|\asinh(x)\,| < b & 0 \leq \acosh(x) < b\\
|\atanh(x)\,| \leq a & 0 < |\acsch(x)\,| < b\\
0 \leq \asech(x) < b & 0 < |\actnh(x)\,| \leq a
\end{array}
\end{equation*}
\subsubsection{Accuracy Tests}

The single precision subprograms were tested on an IBM compatible PC
by comparison with the double precision subprograms at 300,000 to
800,000 points in the domain of each function.
Using $\rho = 2^{-23} \approx 0.119 \times 10^{-6}$, which is the relative
precision of IEEE single precision arithmetic, these tests may be
summarized as follows:
\begin{center}
\begin{tabular}{lcr}
& \multicolumn{1}{c}{\bf Argument} & \multicolumn{1}{c}{\bf Max. Rel.}\\
\multicolumn{1}{c}{\bf Function} & \multicolumn{1}{c}{\bf Range} &
\multicolumn{1}{c}{\bf Error}\\
SASINH & All $x$ & 0.9 $\rho $\rule[-7pt]{0pt}{8pt}\phantom{~el.}\\
SACOSH & [1.00, 1.21] & 1.6 $\rho $\phantom{~el.}\\
 & $x \geq 1.21$ & 0.5 $\rho $\rule[-7pt]{0pt}{8pt}\phantom{~el.}\\
SATANH & [$-$0.44, 0.44] & 1.3 $\rho $\phantom{~el.}\\
 & [0.44, 0.92] & 1.3 $\rho $\phantom{~el.}\\
 & [0.92, 1.0] & 0.5 $\rho $\rule[-7pt]{0pt}{8pt}\phantom{~el.}\\
SACSCH & $x \neq 0$  & 0.9 $\rho $\rule[-7pt]{0pt}{8pt}\phantom{~el.}\\
SASECH & [0.0, 0.24]  & 0.8 $\rho $\phantom{~el.}\\
 & [0.24, 0.68] & 1.2 $\rho $\phantom{~el.}\\
 & [0.68, 0.88] & 3.2 $\rho $\phantom{~el.}\\
 & [0.88, 1.] & 989.1 $\rho $\rule[-7pt]{0pt}{8pt}\phantom{~el.}\\
SACTNH & [1.0, 1.16] & 153.5 $\rho $\phantom{~el.}\\
 & [1.16, 2.2] & 1.7 $\rho $\phantom{~el.}\\
 & $x \geq 2.2$ & 1.6 $\rho $\phantom{~el.}
\end{tabular}
\end{center}
The instances of very large relative errors in this table are in regions
where the slope of the graph of the function is becoming vertical.
Note that relative errors may be significantly larger on machines
that do not have proplerly rounded arithmetic.

The functions were tested using identities of the form $x - \sinh(\asinh(x)) =
0$.  When scaled by the precision of the arithmetic used, the double precision
and single precision functions had similar perfomance.

\subsection{Error Procedures and Restrictions}

If an argument is outside the valid domain, an error message will be
issued, and the value zero will be returned.  Error messages are
processed using the subroutines of Chapter~19.2 with an error level of zero.

\subsection{Supporting Information}

The source language for these subroutines is ANSI Fortran 77.

All double precision entries are in the file DASINH, which also needs
files: AMACH, DERM1, DERV1, ERFIN, and ERMSG.

All single precision entries are in the file SASINH, which also needs
files: AMACH, ERFIN, ERMSG, SERM1, and SERV1.

Designed and programmed by C. Lawson and S. Chiu, JPL, 1983. Modified Nov.,
1988 to use R1MACH and D1MACH.


\begin{tabular}{ll@{\hspace{.3in}}ll}
\multicolumn{4}{c}{\bf Entries\rule[-6pt]{0pt}{8pt}}\\
DACOSH & DACSCH & SACOSH & SACSCH\\
DACTNH & DASECH & SACTNH & SASECH\\
DASINH & DATANH & SASINH & SATANH\\
\end{tabular}

\begcodenp

\medskip\
\lstset{language=[77]Fortran,showstringspaces=false}
\lstset{xleftmargin=.8in}

\centerline{\bf \large DRSASINH}\vspace{10pt}
\lstinputlisting{\codeloc{sasinh}}

\vspace{30pt}\centerline{\bf \large ODSASINH}\vspace{10pt}
\lstset{language={}}
\lstinputlisting{\outputloc{sasinh}}
\end{document}
