\documentclass[twoside]{MATH77}
\usepackage{multicol}
\usepackage[fleqn,reqno,centertags]{amsmath}
\begin{document}
\begmath 7.2 Roots of a Quadratic Polynomial

\silentfootnote{$^\copyright$1997 Calif. Inst. of Technology, \thisyear \ Math \`a la Carte, Inc.}

\subsection{Purpose}

Compute the two roots of a quadratic polynomial having real coefficients.

\subsection{Usage}

\subsubsection{Program Prototype, Single Precision}

\begin{description}
\item[\bf REAL]  \ {\bf A}$(\geq 3)$

\item[\bf COMPLEX]  \ {\bf Z}$(\geq 2)$
\end{description}

Assign values to A().
$$
\fbox{{\bf CALL SPOLZ2(A, Z)}}
$$
The roots are returned in Z($).$

\subsubsection{Argument Definitions}

\begin{description}
\item[A()]  [in] Contains the coefficients of the polynomial%
\begin{equation*}
a_1x^2+a_2x+a_3
\end{equation*}
Require A($1)\neq 0$. The contents of the array A() are not modified by the
subroutine.

\item[Z()]  [out] Array in which the two roots are returned. The roots will
be stored as complex numbers even in cases in which they are real.
\end{description}

\subsubsection{Modifications for Double Precision}

For double precision usage change the subroutine name to {\bf DPOLZ2} and
change the type statements to:

{\bf DOUBLE PRECISION} \ {\bf A}($\geq 3)$

{\bf DOUBLE PRECISION} \ {\bf Z}($2$, $\geq 2)$, or

{~~~~\bf COMPLEX*16} \ {\bf Z}($\geq 2)$)

On return the real and imaginary parts of the $j^{th}$ root will be stored
in Z($1,j)$ and Z($2,j)$ respectively.

\subsection{Examples and Remarks}

The program, DRSPOLZ2, uses SPOLZ2 to compute the roots of the polynomials, $%
x^2+x-1$, $2x^2-12x+26$, and $x^2-4x+4$. The output is shown in ODSPOLZ2.

See the remark in Chapter~15.1, Section~C, concerning the use of COMPLEX*16.

\subsection{Functional Description}

\subparagraph{Method}

We are given the real coefficients of a quadratic polynomial:%
\begin{equation*}
a_1x^2+a_2x+a_3
\end{equation*}
The case of $a_1=0$ is regarded as an error. The subroutine issues an error
message and returns, setting both roots to zero.

Compute $p = a_2/a_1$ and $q = a_3/a_1$. The cases of $p = 0$ or $q = 0$ are
given special treatment. Otherwise we compute the roots of $x^2 + px + q$,
knowing that $p$ and $q$ are both nonzero.

Compute $u=-p/2$. Direct use of the quadratic formula would give the roots
as $r_1=u+z$ and $r_2=u-z$, where $z= \sqrt{u^2-q}$. To extend the range of
data for which results can be obtained without overflow or underflow, the
expression for $z$ is evaluated differently depending on the magnitude of $u$%
. We use thresholds, $c_1$ and $c_2$, such that $c_1^2/16$ is the underflow
limit and $16c_2^2$ is the overflow limit. The values of $c_1$ and $c_2$ are
set on the first call to this subroutine by use of the subprograms R1MACH or
D1MACH from Chapter~19.1.

Rather than computing $z$, we compute $f=|z|$ and a quantity, $d$, having
the same sign as $u^2-q:$
\begin{tabbing}
\hspace{.2in}\=If $|u| > c_2$\\
\>\ \ \ \ \=$d=1-(q/u)/u$\\
\>\>$f = |u| \sqrt{|d|}$\\
\>Elseif $|u| < c_1$\\
\>\>$d=u(u/|q|)- \sign q$\\
\>\>$f = \sqrt{|q|} \ \sqrt{|d|}$\\
\>Else\\
\>\>$d=u^2-q$\\
\>\>$f = \sqrt{|d|}$\\
\>Endif
\end{tabbing}
If $d=0$, the polynomial has the real root, $u$, with multiplicity two. If $%
d<0$, the roots are the complex numbers, $(u,f)$ and $(u,-f).$

If $d>0$, the roots are the two real numbers, $u+f$ and $u-f$, however use
of these expressions would cause unnecessary loss of accuracy in the root of
smaller magnitude when the magnitudes of $u$ and $f$ are nearly the same.
Instead we use the expressions, $r_1=u+f \sign u$, and $r_2=q/r_1.$

\subparagraph{Accuracy tests}

A test program ran cases exercising all of the branches of the subroutine.
Accuracy was consistent with the computer being used.

\subsection{Error Procedures and Restrictions}

If the high-order coefficient, A(1), is zero, the subroutine issues an error
message via ERMSG of Chapter~19.2, with an error level of 0,
and returns with both roots set to zero.

\subsection{Supporting Information}

The Source Language is ANSI Fortran~77.

Designed by C. L. Lawson, JPL, May~1986.

Programmed by C. L. Lawson and S. Y. Chiu, JPL, May~1986, Feb.~1987.


\begin{tabular}{@{\bf}l@{\hspace{5pt}}l}
\bf Entry & \hspace{.35in} {\bf Required Files}\vspace{2pt}\\
DPOLZ2 & \parbox[t]{2.7in}{\hyphenpenalty10000 \raggedright
 AMACH, DPOLZ2, ERFIN, ERMSG\rule[-5pt]{0pt}{8pt}}\\
SPOLZ2 & \parbox[t]{2.7in}{\hyphenpenalty10000 \raggedright
 AMACH, ERFIN, ERMSG, SPOLZ2}\\
\end{tabular}

\begcode

\medskip\
\lstset{language=[77]Fortran,showstringspaces=false}
\lstset{xleftmargin=.8in}

\centerline{\bf \large DRSPOLZ2}\vspace{10pt}
\lstinputlisting{\codeloc{spolz2}}
\newpage
\vspace{30pt}\centerline{\bf \large ODSPOLZ2}\vspace{10pt}
\lstset{language={}}
\lstinputlisting{\outputloc{spolz2}}
\end{document}
