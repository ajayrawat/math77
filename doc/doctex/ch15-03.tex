\documentclass[twoside]{MATH77}
\usepackage{multicol}
\usepackage[fleqn,reqno,centertags]{amsmath}
\begin{document}
\begmath 15.3 Cumulative Distribution Function for Chi-Square
\hbox{Probability Distribution}

\silentfootnote{$^\copyright$1997 Calif. Inst. of Technology, \thisyear \ Math \`a la Carte, Inc.}

\subsection{Purpose}

The procedure described in this Section computes the Cumulative Distribution
Function (CDF) of the Chi-Square probability distribution. The CDF is
sometimes called the lower tail. The
lower tail, or CDF, $P(\chi ^2|\nu )$, and the upper tail, $Q(\chi ^2|\nu )$
for the Chi-Square probability distribution with argument $\chi $ and $\nu $
degrees of freedom are defined by%
\begin{gather*}
\hspace{-5pt}P(\chi ^2|\nu )=\frac{2^{-\nu /2}}{\Gamma (\nu /2)}\int_0^{
\chi^2}t^{(\nu / 2)-1}e^{-t/2}dt,\phantom{=1-P(\chi ^2|\nu )}\\
\hspace{-5pt}Q(\chi ^2|\nu )=\frac{2^{-\nu /2}}{\Gamma (\nu /2)}
\int_{\chi ^2}^\infty t^{(\nu / 2)-1}e^{-t/2}dt\phantom{,}=1-P(\chi ^2|\nu )
\end{gather*}
\subsection{Usage}

\subsubsection{Program Prototype, Single Precision}

\begin{description}
\item[REAL]  \ {\bf CHISQ, NU, P, Q}

\item[INTEGER]  \ {\bf IERR}
\end{description}

Assign values to CHISQ and NU, and obtain P $=P(\chi ^2|\nu )$ and Q $%
=Q(\chi ^2|\nu )$ by using
$$
\fbox{{\bf CALL SCDCHI (CHISQ, NU, P, Q, IERR)}}
$$

\subsubsection{Argument Definitions}

\begin{description}
\item[CHISQ]  \ [in] Argument $\chi ^2$ of the functions $P(\chi ^2|\nu )$
and $Q(\chi ^2|\nu ).$  Must be nonnegative.  Must be positive if NU = 0.

\item[NU]  \ [in] Degrees of freedom $\nu $ of the functions $P(\chi ^2|\nu )
$ and $Q(\chi ^2|\nu ).$  Must be nonnegative.  Must be positive if
CHISQ = 0.

\item[P]  \ [out] The value of the function $P(\chi ^2|\nu ).$

\item[Q]  \ [out] The value of the function $Q(\chi ^2|\nu ).$

\item[IERR]  \ [out] A flag that normally is zero to indicate successful
computation. See Section E below for discussion of non-zero values.
\end{description}

\subsubsection{Modifications for Double Precision}

For double precision computation, change the REAL type statement to DOUBLE
PRECISION and change the initial letter of the procedure name to D.

\subsection{Example and Remarks}

See DRDCDCHI and ODDCDCHI for an example of the usage of this subprogram.

The procedures SGAMIK and SGAMIE, described in Chapter~2.19, are used to
control the procedure SGAMI and determine the error estimate it returns.
SGAMI is used as described in Section D below.

\subsection{Functional Description}

\subsubsection{Method}

The identities $P(\chi ^2|\nu ) = P(a,x)$ and $Q(\chi ^2|\nu ) = Q(a,x)$,
with $a = \nu /2$ and $x = \chi ^2/2$, where $P(a,x)$ and $Q(a,x)$ are
incomplete gamma function ratios, are used. The procedure SGAMI described in
Chapter~2.19 is used to evaluate $P(a,x)$ and $Q(a,x).$

\subsubsection{Accuracy Tests}

See Section~2.19.D.

\subsection{Error Procedures and Restrictions}

The procedure SGAMI issues error messages, under several conditions, at
level 2 + MSGOFF, where MSGOFF is zero unless specified by a call to SGAMIK
(see Chapter~2.19) at some time before calling SCDCHI. If error termination
is suppressed by setting MSGOFF $< 0$, or by calling ERMSET (see Chapter
19.2), IERR will be set to a non-zero value.

If the desired tolerance could not be achieved, IERR is set to~2.

If SCDCHI is called with both CHISQ and NU zero, IERR is set to~3 and P is
set to~3.0.

If SCDCHI is called with one or both of CHISQ and NU negative, IERR is set
to~4 and P is set to~4.0.

\subsection{Supporting Information}

\begin{tabular}{@{\bf}l@{\hspace{5pt}}l}
\bf Entry & \hspace{.35in} {\bf Required Files}\vspace{2pt} \\
DCDCHI & \parbox[t]{2.7in}{\hyphenpenalty10000 \raggedright
AMACH, DCDCHI, DCSEVL, DERF, DERM1, DERV1, DGAM1, DGAMMA, DINITS, DRCOMP,
DREXP, DRLOG, DXPARG, ERFIN, ERMSG, IERM1, IERV1\rule[-5pt]{0pt}{8pt}}\\
SCDCHI & \parbox[t]{2.7in}{\hyphenpenalty10000 \raggedright
AMACH, ERFIN, ERMSG, IERM1, IERV1, SCDCHI, SCSEVL, SERF, SERM1, SERV1,
SGAM1, SGAMMA, SINITS, SRCOMP, SREXP, SRLOG, SXPARG}\\
\end{tabular}

Designed and programmed by W. V. Snyder, JPL, 1993.


\begcodenp
\lstset{language=[77]Fortran,showstringspaces=false}
\lstset{xleftmargin=.8in}

\centerline{\bf \large DRDCDCHI}\vspace{10pt}
\lstinputlisting{\codeloc{dcdchi}}

\vspace{30pt}\centerline{\bf \large ODDCDCHI}\vspace{10pt}
\lstset{language={}}
\lstinputlisting{\outputloc{dcdchi}}
\end{document}
