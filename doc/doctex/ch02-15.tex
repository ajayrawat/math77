\documentclass[twoside]{MATH77}
\usepackage{multicol}
\usepackage[fleqn,reqno,centertags]{amsmath}
\begin{document}
\begmath 2.15  Procedures to Avoid Loss of Precision: $\ln (1+x)$, etc.

\silentfootnote{$^\copyright$1997 Calif. Inst. of Technology, \thisyear \ Math \`a la Carte, Inc.}

\subsection{Purpose}

These procedures compute expressions involving logarithm, exponential, sine,
cosine, hyperbolic sine, hyperbolic cosine, and the gamma function, some
having arguments that are expressions, more accurately than is possible
using the Fortran intrinsic functions, or the gamma function described in
Chapter~2.3. See also discussion of the function SRCVAL in Section C
of Chapter~2.9.

\subsection{Usage}

\subsubsection{Program Prototype, Single Precision,\newline  $\ln (1+x)$}

\begin{description}

\item[REAL] \ {\bf  X, U, SLNREL}

\item[EXTERNAL] \ {\bf  SLNREL}

\end{description}

Assign a value to X and obtain U $= \ln (1+\text{X})$ by using
$$
\fbox{\bf U = SLNREL(X)}
$$
\subsubsection{Program Prototype, Single Precision,\newline $x - \ln (1+x)$}

\begin{description}

\item[REAL] \ {\bf  X, U, SRLOG1}

\item[EXTERNAL] \ {\bf  SRLOG1}

\end{description}

Assign a value to X and obtain U $= \text{X} - \ln (1+\text{X})$ by using
$$
\fbox{\bf U = SRLOG1(X)}
$$
\subsubsection{Program Prototype, Single Precision,\newline $x - 1 - \ln (x)$}

{\bf REAL \ X, U, SRLOG}

{\bf EXTERNAL \ SRLOG}

Assign a value to X and obtain U $= \text{X} - 1 - \ln (\text{X})$ by using
$$
\fbox{\bf U = SRLOG(X)}
$$
\subsubsection{Program Prototype, Single Precision,\newline  $\exp (x) - 1$}

\begin{description}

\item[REAL] \ {\bf X, U, SREXP}

\item[EXTERNAL] \ {\bf SREXP}

\end{description}

Assign a value to X and obtain U $= \exp (\text{X}) - 1$ by using
$$
\fbox{\bf U = SREXP(X)}
$$
\subsubsection{Program Prototype, Single Precision,\newline  $\sin (\pi x)$}

\begin{description}

\item[REAL] \ {\bf  X, U, SSINPX}

\item[EXTERNAL] \ {\bf  SSINPX}

\end{description}

Assign a value to X and obtain U $= \sin (\pi \text{X})$ by using
$$
\fbox{\bf U = SSINPX(X)}
$$
\subsubsection{Program Prototype, Single Precision,\newline  $\cos (\pi x)$}

\begin{description}

\item[REAL] \ {\bf  X, U, SCOSPX}

\item[EXTERNAL] \ {\bf  SCOSPX}

\end{description}

Assign a value to X and obtain U $= \cos (\pi \text{X})$ by using
$$
\fbox{\bf U = SCOSPX(X)}
$$
\subsubsection{Program Prototype, Single Precision,\newline  $(x-\sin (x))/x^3$}

\begin{description}

\item[REAL] \ {\bf  X, U, SSIN1}

\item[EXTERNAL] \ {\bf  SSIN1}

\end{description}

Assign a value to X and obtain U $= (\text{X}-\sin (\text{X}))/\text{X}^3$ by using
$$
\fbox{\bf U = SSIN1(X)}
$$
\subsubsection{Program Prototype, Single Precision,\newline  $(1-\cos (x))/x^2$}

\begin{description}

\item[REAL] \ {\bf  X, U, SCOS1}

\item[EXTERNAL] \ {\bf  SCOS1}

\end{description}

Assign a value to X and obtain U $= (1-\cos (\text{X}))/\text{X}^2$ by using
$$
\fbox{\bf U = SCOS1(X)}
$$
\subsubsection{Program Prototype, Single Precision,\newline  $\sinh(x) - x$}

\begin{description}

\item[REAL] \ {\bf  X, U, SSINHM}

\item[EXTERNAL] \ {\bf  SSINHM}

\end{description}

Assign a value to X and obtain U $= \sinh(\text{X}) - \text{X}$ by using
$$
\fbox{\bf U = SSINHM(X)}
$$
\subsubsection{Program Prototype, Single Precision,\newline  $\cosh(x) - 1$}

\begin{description}

\item[REAL] \ {\bf  X, U, SCOSHM}

\item[EXTERNAL] \ {\bf  SCOSHM}

\end{description}

Assign a value to X and obtain U $= \cosh(\text{X}) - 1$ by using
$$
\fbox{\bf U = SCOSHM(X)}
$$
\subsubsection{Program Prototype, Single Precision,\newline  $\cosh(x) -
 1 - \frac 12 x^2$}

{\bf REAL X, U, SCSHMM}

{\bf EXTERNAL SCSHMM}

Assign a value to X and obtain U $= \cosh(\text{X}) - 1 - \frac{1}{2} \text{X}^2$ by using
$$
\fbox{\bf U = SCSHMM(X)}
$$
\subsubsection{Program Prototype, Single Precision,\newline  $1/\Gamma (x+1) - 1$}

\begin{description}

\item[REAL] \ {\bf  X, U, SGAM1}

\item[EXTERNAL] \ {\bf  SGAM1}

\end{description}

Assign a value to X and obtain U $= 1/\Gamma (1+\text{X}) - 1$ by
using
$$
\fbox{\bf U = SGAM1(X)}
$$
\subsubsection{Argument Definitions}

\begin{description}

\item[X] \ [in] Argument of functions. Argument values may be constrained as
described in Section E below.
\end{description}

\subsubsection{Modifications for Double Precision}

For double precision computation, change the REAL type statement to DOUBLE
PRECISION and change the initial letter of the function name to D. Since
these functions are not generic intrinsic functions, it is important to
declare them explicitly to be DOUBLE PRECISION, because the default implicit
type would be REAL. The approximations used in SSINHM, SCOSHM and SCSHMM are
accurate to~14 decimal digits, so the REAL versions of these procedures
could be used to advantage on some platforms by changing the function type
and all internal variable types, constants, and arithmetic to double
precision.

\subsection{Example and Remarks}

See DRDLNREL and ODDLNREL for an example of the usage of these subprograms.

MATH77 does not include a special procedure to compute $\sin (x)/x$. On all
machines tested, $\sin (x)/x$ could be computed with good relative accuracy,
even for very small $x$, by using the Fortran intrinsic SIN function. The
only difficulty is that the user must test whether $x$ is exactly zero, and
use 1.0 for the result. The procedure SSIN1, described above, can be used to
compute $1-\sin (x)/x$ without loss of precision.

\subsection{Functional Description}

\subsubsection{Method}
\begin{description}
\item[For xLNREL,] when $1/\sqrt 2 < 1+x < \sqrt 2$, use approximation~2707 from
Hart, et. al., \cite{Hart:1968:CA}. Otherwise, use the Fortran intrinsic logarithm
function.

\item[For xRLOG1,] when $-0.39 \leq  x \leq  0.57$, use
rational approximations due to Alfred H. Morris, Jr., \cite{ahm:lib}. Otherwise,
use the Fortran intrinsic logarithm function.

\item[For xRLOG,] when $0.61 \leq  x \leq  1.57$, use rational
approximations due to Alfred H. Morris, Jr., \cite{ahm:lib}. Otherwise, use the
Fortran intrinsic logarithm function.

\item[For xREXP,] when $|x| \leq  0.15$, use rational approximations due
to Alfred H. Morris, Jr., \cite{ahm:lib}. Otherwise, use the Fortran intrinsic
exponential function.

\item[For xSINPX and xCOSPX,] for all values of $x$, use rational approximations
due to Alfred H. Morris, Jr., \cite{ahm:lib}.

\item[For xSIN1,] use Taylor's series for $|x| < 0.25$. Otherwise, use the Fortran
intrinsic sine function.

\item[For xCOS1,] when $\cos (x) \geq 0$ use $1-\cos (x) = \sin ^2(x) / (1+\cos (x))$%
. Otherwise, use $1-\cos (x)$. In both cases, use the Fortran intrinsic
functions.

\item[For SSINHM,] when $|x| < 1.65$, use rational approximations due to A. K.
Cline and R. J. Renka, \cite{ahm:lib}. Otherwise, use the Fortran intrinsic SINH
function.

\item[For SCOSHM,] when $|x| < 1.2$, use rational approximations due to A. K. Cline
and R. J. Renka, \cite{ahm:lib}. Otherwise, use the Fortran intrinsic COSH function.

\item[For SCSHMM,] when $|x| < 2.7$, use rational approximations due to A. K. Cline
and R. J. Renka, \cite{ahm:lib}. Otherwise, use the Fortran intrinsic COSH function.

\item[For xGAM1,] when $-0.5 \leq  x \leq  1.5$,
use rational approximations due to Alfred H. Morris, Jr., \cite{ahm:lib}.
Otherwise, use the xGAMMA routine from Chapter~2.3.
\end{description}
\subsubsection{Accuracy Tests}

The single precision subprograms were tested on an IBM PC/AT using IEEE
arithmetic, by comparison with the double precision Fortran intrinsic
functions (except the reference value for SGAM1 used DGAMMA from Chapter
2.3) at 2000~points. For each function except SSINPX and SCOSPX, the points
were chosen to cover the entire domain over which a special approximation is
used. For SSINPX and SCOSPX, 2000~points were chosen randomly in the region $%
0<x<2$, and SSINPX($10^x$), $\sin (10^x\pi )$, SCOSPX($10^x$) and $\cos
(10^x\pi )$ were then computed. For SSIN1 and SCOS1, 2000~points were chosen
randomly in the region $-8<x<-1$, then $y=10^x$,
SSIN1($y)$, $(y-\sin (y))/y^3$, SCOS1($y)$ and $(1-\cos (y))/y^2$ were
computed. The results of these tests may be summarized as follows:
\begin{table*}
\begin{center}
\begin{tabular}{l*{11}{r}}
 & \multicolumn{11}{c}{\bf Percentage of samples with specified number of
bits wrong\rule[-8pt]{0pt}{8pt}}\\
{\bf Function} & {\bf 0} & {\bf 1} & {\bf 2} & {\bf 3} & {\bf 4} &
{\bf 5} & {\bf 6} & {\bf 7} & {\bf 8} & {\bf 9} & {\bf $\geq 10$}\\
SLNREL & 100.0\\
$\ln (x+1)$ & 61.7 & 24.6 & 13.8\\
SRLOG1 & 57.3 & 23.7 & 10.9 & 6.7 & 1.5\\
$x-\ln (x+1)$ & 3.9 & 4.6 & 9.0 & 11.9 & 17.9 & 16.9 & 11.0 & 7.0 & 5.3 &
4.1 & 8.5\\
SRLOG & 56.7 & 23.9 & 10.9 & 6.9 & 1.6\\
$x-1-\ln (x)$ & 4.2 & 3.9 & 7.3 & 12.1 & 19.4 & 15.9 & 11.4 & 8.6 & 5.6 &
3.4 & 8.5\\
SREXP & 71.7 & 26.5 & 1.9\\
$\exp (x)-1$ & 8.2 & 8.8 & 16.9 & 26.0 & 18.8 & 10.8 & 5.2 & 2.8 & 1.6 &
0.4 & 0.6\\
SSINPX & 10.6 & 7.6 & 9.9 & 11.9 & 13.0 & 12.1 & 12.1 & 8.7 & 5.8 & 3.3 & 5.2\\
$\sin (\pi x)$ & 4.9 & 4.8 & 8.8 & 11.8 & 12.9 & 13.2 & 13.6 & 11.4 &
8.2 & 5.2 & 5.4\\
SCOSPX & 11.4 & 7.1 & 11.4 & 11.8 & 11.4 & 12.2 & 11.1 & 10.1 & 6.0 &
3.9 & 3.9\\
$\cos (\pi x)$ & 5.1 & 4.3 & 9.4 & 10.9 & 12.7 & 13.7 & 13.4 & 11.9 &
7.8 & 5.3 & 5.6\\
SSIN1 & 91.1 & 8.9\\
$(x-\sin (x))/x^3$ & 54.2 & 20.1 & 2.0 & 2.3 & 2.0 & 2.2 & 2.1 & 1.5 &
1.6 & 1.8 & 9.8\\
SCOS1 & 76.5 & 15.8 & 7.7 & 0.1\\
$(1-\cos (x))/x^2$ & 57.5 & 18.6 & 4.3 & 2.0 & 1.7 & 2.4 & 1.6 & 2.2 &
2.1 & 1.8 & 6.1\\
SSINHM & 66.3 & 29.4 & 4.3\\
$\sinh(x)-x$ & 13.4 & 12.3 & 23.6 & 17.0 & 8.5 & 7.7 & 5.0 & 3.7 & 2.5 &
2.0 & 4.3\\
SCOSHM & 68.7 & 28.3 & 3.0\\
$\cosh(x)-1$ & 18.8 & 19.1 & 17.5 & 13.7 & 8.5 & 6.0 & 5.0 & 3.5 & 2.8 &
1.6 & 3.7\\
SCSHMM & 46.5 & 32.9 & 19.8 & 0.8\\
$\cosh(x)-1-x^2/2$ & 12.0 & 11.4 & 16.9 & 13.7 & 10.4 & 5.7 & 5.0 & 4.0 &
3.0 & 3.3 & 14.7\\
SGAM1 & 63.3 & 27.5 & 8.9\\
$1/\Gamma (1+x)-1$ & 8.2 & 8.7 & 13.9 & 20.7 & 22.1 & 14.7 & 5.4 & 3.7 &
1.5 & 0.6 & 0.7\\
\end{tabular}
\end{center}\vspace{-10pt}
\end{table*}

\begin{tabular}{@{}l@{}rrr}
{\bf Function} & {\bf avg $|\text{E}|$} &
{\bf $\max |\text{E}|$} & {\bf std. dev.}\\
SLNREL & 0.25 & 0.50 & 0.14\\
$\ln (x+1)$ & 0.50 & 1.49 & 1.49\\
SRLOG1 & 0.46 & 3.28 & 0.43\\
$x-\ln (x+1)$ & 8075.58 & $\approx 7.9\times 10^6$ & $\approx 1.9\times 10^5$\\
SRLOG & 0.47 & 3.10 & 0.43\\
$x-1-\ln (x)$ & 3341.24 & $\approx 2.9\times 10^6$ & $\approx 7.3\times 10^4$\\
SREXP & 0.37 & 1.39 & 0.26\\
$\exp (x)-1$ & 16.51 & 7402.74 & 190.13\\
SSINPX & 58.32 & 5795.79 & 253.07\\
$\sin (\pi x)$ & 119.20 & 48061.18 & 1193.09\\
SCOSPX & 84.64 & 35894.27 & 965.42\\
$\cos (\pi x)$ & 94.27 & 19712.38 & 588.76\\
SSIN1 & 0.32 & 0.97 & 0.13\\
$(x-\sin (x))/x^3$ & 407.81 & 28944.33 & 2091.66\\
SCOS1 & 0.25 & 2.02 & 0.38\\
$(x-\cos (x))/x^2$ & 102.36 & 7931.0 & 525.75\\
SSINHM & 0.35 & 1.36 & 0.35\\
$\sinh(x)-x$ & 2584.93 & $\approx 2.6\times 10^6$ & $\approx 6.7\times 10^4$\\
SCOSHM & 0.39 & 1.35 & 0.27\\
$\cosh(x)-1$ & 2302.36 & $\approx 3.7\times 10^6$ & $\approx 8.3\times 10^4$\\
SCSHMM & 0.65 & 23.00 & 0.68\\
\multicolumn{2}{@{}l}{$\cosh(x)-1-x^2/2$}\\
 & $\approx 1.9\times 10^5$ & $>10^7$ & $\approx 1.6\times
10^6$\\
SGAM1 & 0.46 & 2.44 & 0.37\\
$1/\Gamma (1+x)-1$ & 32.94 & 28016.70 & 686.03\\
\end{tabular}

For the functions SLNREL, SRLOG1, SRLOG, SREXP, SSINHM, SCOSHM and SGAM1 the
error in units of the least significant digit was approximately constant over
the entire region. For SSINPX and SCOSPX the error in units of the least
significant digit and the absolute error both increased as the argument
increased. For SCSHMM the error in units of the least significant digit was
nearly constant over the entire region, except that one sample had 5 wrong
bits near the zero of the function.

When $\ln (1+x)$, $x-\ln (1+x)$, $x-1-\ln (x)$, $(x-\sin (x))/x^3$, $%
(1-\cos (x))/x^2$, $\sinh (x)-x$, $\cosh(x)-1$, $\cosh (x)-1-x^2/2$
and $1/\Gamma (1+x)-1$ were computed
using single precision intrinsic functions and single precision
arithmetic, there was substantial increase in the error in units of the least
significant digit near the zeros of the functions, even though the absolute
error approached zero. For $\sin (\pi x)$ and $\cos (\pi x)$ the error
in units of
the least significant digit and the absolute error increased as the argument
increased. The average, maximum and standard deviation of the errors in
units of the
least significant digit are summarized in the table at left, in units of $%
\rho = 2^{-23} \approx 1.192 \times 10^{-7}$, the relative precision of IEEE
single precision arithmetic.

For arguments in the range $0<x<1$, SSINPX and SCOSPX commit approximately
the same error as equivalent expressions constructed from the intrinsic SIN
and COS functions.

\bibliography{math77}
\bibliographystyle{math77}

\subsection{Error Procedures and Restrictions}

If the argument of xLNREL or xRLOG1 $\leq  -1.0$ or the argument of
xRLOG $\leq $ 0.0, the Fortran intrinsic logarithm function will
produce an error message. If the argument of xGAM1 $\leq -1.0$ the
xGAMMA function will produce an error message. If the magnitude of
the argument of SSINPX, DSINPX, SCOSPX or DCOSPX $> 1 / \rho $, where
$\rho $ is the smallest number that can be added to one and give a
result different from one, these routines will issue an error message
using the error message processor, described in Chapter~19.2, at
level~2. If error termination is suppressed by using ERMSET, SSINPX
and DSINPX return 0.0, while SCOSPX and DCOSPX return~1.0.

\subsection{Supporting Information}

\begin{tabular}{@{\bf}l@{\hspace{5pt}}l}
\bf Entry & \hspace{.35in}\bf Required Files\vspace{2pt}\\
DCOS1 & \parbox[t]{2.7in}{\hyphenpenalty10000 \raggedright
DCOS1\rule[-5pt]{0pt}{8pt}}\\
DCOSHM & \parbox[t]{2.7in}{\hyphenpenalty10000 \raggedright
AMACH, DCOSHM\rule[-5pt]{0pt}{8pt}}\\
DCOSPX & \parbox[t]{2.7in}{\hyphenpenalty10000 \raggedright
AMACH, ERFIN, ERMSG, DCOSPX, DERM1, DERV1\rule[-5pt]{0pt}{8pt}}\\
\end{tabular}

\begin{tabular}{@{\bf}l@{\hspace{5pt}}l}
\bf Entry & \hspace{.35in}\bf Required Files\vspace{2pt}\\
DCSHMM & \parbox[t]{2.7in}{\hyphenpenalty10000 \raggedright
AMACH, DCSHMM\rule[-5pt]{0pt}{8pt}}\\
DGAM1 & \parbox[t]{2.7in}{\hyphenpenalty10000 \raggedright
AMACH, DERM1, DERV1, DGAM1, DGAMMA, ERFIN, ERMSG\rule[-5pt]{0pt}{8pt}}\\
DLNREL & \parbox[t]{2.7in}{\hyphenpenalty10000 \raggedright
DLNREL\rule[-5pt]{0pt}{8pt}}\\
DREXP & \parbox[t]{2.7in}{\hyphenpenalty10000 \raggedright
DREXP\rule[-5pt]{0pt}{8pt}}\\
DRLOG & \parbox[t]{2.7in}{\hyphenpenalty10000 \raggedright
DRLOG\rule[-5pt]{0pt}{8pt}}\\
DRLOG1 & \parbox[t]{2.7in}{\hyphenpenalty10000 \raggedright
DRLOG\rule[-5pt]{0pt}{8pt}}\\
DSIN1 & \parbox[t]{2.7in}{\hyphenpenalty10000 \raggedright
AMACH, DSIN1\rule[-5pt]{0pt}{8pt}}\\
DSINHM & \parbox[t]{2.7in}{\hyphenpenalty10000 \raggedright
AMACH, DSINHM\rule[-5pt]{0pt}{8pt}}\\
DSINPX & \parbox[t]{2.7in}{\hyphenpenalty10000 \raggedright
AMACH, ERFIN, ERMSG, DERM1, DERV1, DSINPX\rule[-5pt]{0pt}{8pt}}\\
SCOS1 & \parbox[t]{2.7in}{\hyphenpenalty10000 \raggedright
SCOS1\rule[-5pt]{0pt}{8pt}}\\
SCOSHM & \parbox[t]{2.7in}{\hyphenpenalty10000 \raggedright
SCOSHM\rule[-5pt]{0pt}{8pt}}\\
SCOSPX & \parbox[t]{2.7in}{\hyphenpenalty10000 \raggedright
AMACH, ERFIN, ERMSG, SCOSPX, SERM1, SERV1\rule[-5pt]{0pt}{8pt}}\\
SCSHMM & \parbox[t]{2.7in}{\hyphenpenalty10000 \raggedright
SCSHMM\rule[-5pt]{0pt}{8pt}}\\
SGAM1 & \parbox[t]{2.7in}{\hyphenpenalty10000 \raggedright
AMACH, ERFIN, ERMSG, SERM1, SERV1, SGAM1, SGAMMA\rule[-5pt]{0pt}{8pt}}\\
SLNREL & \parbox[t]{2.7in}{\hyphenpenalty10000 \raggedright
SLNREL\rule[-5pt]{0pt}{8pt}}\\
SREXP & \parbox[t]{2.7in}{\hyphenpenalty10000 \raggedright
SREXP\rule[-5pt]{0pt}{8pt}}\\
SRLOG & \parbox[t]{2.7in}{\hyphenpenalty10000 \raggedright
SRLOG\rule[-5pt]{0pt}{8pt}}\\
SRLOG1 & \parbox[t]{2.7in}{\hyphenpenalty10000 \raggedright
SRLOG\rule[-5pt]{0pt}{8pt}}\\
SSIN1 & \parbox[t]{2.7in}{\hyphenpenalty10000 \raggedright
AMACH, SSIN1\rule[-5pt]{0pt}{8pt}}\\
SSINHM & \parbox[t]{2.7in}{\hyphenpenalty10000 \raggedright
SSINHM\rule[-5pt]{0pt}{8pt}}\\
SSINPX & \parbox[t]{2.7in}{\hyphenpenalty10000 \raggedright
AMACH, ERFIN, ERMSG, SERM1, SERV1, SSINPX\rule[-5pt]{0pt}{8pt}}\\
\end{tabular}

Designed and programmed by W. V. Snyder, JPL 1991 and 1993.


\begcode

\medskip\

\lstset{language=[77]Fortran,showstringspaces=false}
\lstset{xleftmargin=.8in}

\centerline{\bf \large DRDLNREL}\vspace{10pt}
\lstinputlisting{\codeloc{dlnrel}}
\newpage

\vspace{30pt}\centerline{\bf \large ODDLNREL}\vspace{10pt}
\lstset{language={}}
\lstinputlisting{\outputloc{dlnrel}}
\end{document}

