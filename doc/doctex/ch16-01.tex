\documentclass[twoside]{MATH77}
\usepackage{multicol}
\usepackage[fleqn,reqno,centertags]{amsmath}
\usepackage{moreverb}
\begin{document}
\begmath 16.1  Character-based Graphics --- One or More XY Graphs

\silentfootnote{$^\copyright$1997 Calif. Inst. of Technology, \thisyear \ Math \`a la Carte, Inc.}

\subsection{Purpose}

These subroutines produce an image in a character array that the user can
print to produce a printer plot of one or more {\em xy} data sets with
titles and numeric grid-line labels. The size of the image is user-specified.
SPRPL1/DPRPL1 will process a single {\em xy} data set, whereas SPRPL2/DPRPL2
will process multiple {\em xy} data sets.

\subsection{Usage}

\subsubsection{Usage to plot a single {\em xy} data set}

\paragraph{Program Prototype, Single Precision}

\begin{description}

\item[INTEGER] \ {\bf NP, NLINES, NCHARS, IERR}

\item[REAL] \ {\bf X}($\geq $NP){\bf , Y}($\geq $NP)

\item[CHARACTER*$n_1$] \ {\bf TITLE} \ [1 $\leq  n_1 \leq $ NCHARS]

\item[CHARACTER*$n_2$] \ {\bf XNAME} \ [1 $\leq n_2 \leq $ NCHARS]

\item[CHARACTER*$n_3$] \ {\bf YNAME} \ [1 $\leq n_3 \leq $ NLINES]

\item[CHARACTER*$n_4$] \ {\bf IMAGE}($\geq $NLINES)\\ $[n_4 \geq $ NCHARS]

\end{description}

Assign values to all arguments except IMAGE() and IERR.

\begin{center}
\fbox{\begin{tabular}{@{\bf }c@{}}
CALL SPRPL1 (X, Y, NP, TITLE, XNAME,\rule{6pt}{0pt}\\
YNAME, NLINES, NCHARS, IMAGE, IERR)\\
\end{tabular}}
\end{center}

The printer plot image is returned in IMAGE() and the termination status in
IERR.

\paragraph{Argument Definitions}

\begin{description}

\item[X(), Y()] \ [in] Arrays of $(x$, $y)$ coordinate pairs defining the curve
to be plotted.

\item[NP] \ [in] Number of $(x$, $y)$ points to be plotted.

\item[TITLE] \ [in] Character string to be placed above the plot frame as a
title for the graph.

\item[XNAME] \ [in] Character string to be placed below the plot frame to
identify the abscissa variable.

\item[YNAME] \ [in] Character string to be placed in a vertical column to the
left of the plot frame to identify the ordinate variable.

\item[NLINES] \ [in] Number of lines of a plot image to be built in IMAGE().
The array IMAGE() must be dimensioned at least NLINES.

\item[NCHARS] \ [in] Number of character positions to be used in each line in
building the plot image in IMAGE(). The declared character length of each
element of IMAGE() must be at least NCHARS.

\item[IMAGE()] \ [out] Character array in which the printer plot image is
built. The image will use NLINES positions in the vertical direction and
NCHARS positions in the horizontal direction.

\item[IERR] \ [out] Termination status indicator.
\begin{itemize}
\item[0] \ No errors.
\item[1] \ NCHARS is too small.
\item[2] \ NLINES is too small.
\end{itemize}
\end{description}

\subsubsection{Usage to plot multiple {\em xy} data sets}

\paragraph{Program Prototype, Single Precision}

\begin{description}

\item[INTEGER] \ {\bf IDIM, KC, NR, JX}($\geq $KC){\bf , JY}($\geq $KC){\bf,
NP}($\geq $KC)

\item[INTEGER] \ {\bf NLINES, NCHARS, IERR}

\item[REAL] \ {\bf XY}(IDIM, $m)$\ ${\displaystyle [m \geq \max_{1 \leq k \leq
\text{KC}} \{\text{JX}(k), \text{JY}(k)\}}$]

\item[CHARACTER*1] \ {\bf SYMBOL}($\geq $KC)

\item[CHARACTER*$n_1$] \ {\bf TITLE} \ [1 $\leq n_1 \leq $ NCHARS]

\item[CHARACTER*$n_2$] \ {\bf XNAME} \ [1 $\leq n_2 \leq $ NCHARS]

\item[CHARACTER*$n_3$] \ {\bf YNAME} \ [1 $\leq n_3 \leq $ NLINES]

\item[CHARACTER*$n_4$] \ {\bf IMAGE}($\geq $NLINES)\\
$[n_4 \geq $ NCHARS]

\end{description}

Assign values to all arguments except IMAGE() and IERR.

\begin{center}
\fbox{\begin{tabular}{@{\bf }c@{}}
CALL SPRPL2 (XY, IDIM, KC, JX, JY, NP,\\
SYMBOL, TITLE, XNAME, YNAME,\\
NLINES, NCHARS, IMAGE, IERR)\\
\end{tabular}}
\end{center}

The printer plot image is returned in IMAGE() and the termination status in
IERR.

\paragraph{Argument Definitions}

\begin{description}

\item[XY(,)] \ [in] Array of values from which $(x,\ y)$ coordinates of points
to be placed will be obtained under control of the parameters JX(), JY(),
and NP().

\item[IDIM] \ [in] Dimension of the first subscript in the XY array. Require
IDIM ${\displaystyle \geq \max_{1 \leq k \leq \text{KC}}\{\text{NP}(k)\}}$.

\item[KC] \ [in] Number of {\em xy} sets to be plotted. If KC $\leq$ 0
the subroutine will return taking no action.

\item[JX()] \ [in] JX($k)$ specifies the column (second subscript) of XY(,)
to be used as the $x$ coordinates for the $k^{th}$ {\em xy} set.

\item[JY()] \ [in] JY($k)$ specifies the column (second subscript) of XY(,)
to be used as the $y$ coordinates for the $k^{th}$ {\em xy} set.

\item[NP()] \ [in] NP($k)$ specifies the number of {\em xy} pairs from XY(,)
in the $k^{th}$ set to be plotted.

\item[SYMBOL()] \ [in] SYMBOL$(k)$ is the single character to be used for
point-plotting the $k^{th}$ data set.

\item[TITLE,]{\bf XNAME, YNAME, NLINES, NCHARS,\newline
IMAGE, IERR} \ Same as in Section B.1 above.

\end{description}

\subsubsection{Modifications for Double Precision}

Change the names SPRPL1 and SPRPL2 to DPRPL1 and DPRPL2, respectively, and
change the REAL declarations to DOUBLE PRECISION.

\subsection{Examples and Remarks}

It is permissible, and generally most convenient, to give the arguments
TITLE, XNAME, and YNAME as character literals directly in the CALL
statement. We suggest that no leading or trailing blanks be included in
these strings, since each of these strings will be centered in the image
based on its length.

The program DRSPRPL1 and its output ODSPRPL1 illustrate the use of SPRPL1 to
obtain printer-plots in two different resolutions. The first plot uses
NLINES = 45 and NCHARS = 110, producing a plot that can be displayed on
8.5 inch wide paper using compressed printing, $i.e.$, 16.67 characters per
inch. The second plot uses NLINES = 22 and NCHARS = 79, producing a plot
that can be displayed on the $25 \times 80$ character display frequently
used with personal computers. Similarly the use of SPRPL2 is illustrated by
the program DRSPRPL2 and its output ODSPRPL2.

\subsection{Functional Description}

The subroutine first scans the given {\em xy} data to determine maximum and
minimum values. It then determines end-values for the $x$-axis and a
subdivision of the $x$-axis into~3 to~10 equal-length subintervals, such
that the values at the ends and subdivision points will be representable in
decimal with a small number of nonzero digits, and the span of the $x$-axis
will encompass the given $x$ data. This same process is applied to the $y$
data.

The subroutine establishes integer values, {\em top}, {\em bottom},
{\em left}, and {\em right}, such that the plot grid will extend vertically
from index {\em top} to index {\em bottom} in the IMAGE() array, inclusive
of the top and bottom grid lines, and horizontally from character position
{\em left} to {\em right}, inclusive of the left and right grid lines. These
values will satisfy 1 $\leq$ {\em top} $<$ {\em bottom} $\leq  \text{NLINES}
- 1,$ and $1 <$ {\em left} $<$ {\em right} = NCHARS $-$ 1.  If TITLE is
nonblank, it will be copied into IMAGE(1) and {\em top} is set to~2,
otherwise {\em top} = 1. If XNAME is nonblank, it will be copied into
IMAGE(NLINES) and {\em bottom} is set to NLINES $-$ 2, otherwise {\em bottom}
= NLINES $-$ 1. {\em left} is set large enough to allow for YNAME, if
nonblank, to be placed vertically at the left edge of the plot image, for
the numeric $y$-axis labels to be positioned to the left of the plot grid,
and for the leftmost numeric $x$-axis label to extend partially to the left
of the plot grid. The width of the numeric $x$-axis labels, and $y$-axis
labels are data-dependent.

For minimal useful resolution one should probably arrange to have
({\em bottom} $-$ {\em top}) $\geq $ 20 and ({\em right} $-$ {\em left})
$\geq 20$, however the subroutine imposes the less stringent requirement
that these extents each be at least~10. The subroutine abandons the effort
and returns with IERR $\neq 0$ if either of these extents is less than~10.

It is possible that there will not be enough space to place a numeric label
below each subdivision point of the $x$-axis. The subroutine first places
one centered under the left end of the $x$-axis. This is assured to be
possible by the way {\em left} is defined. It next attempts to place one for
the right end of the $x$-axis, right justified in the available space. It then
sequentially tries to place labels centered under the second, third,
etc., $x$-axis subdivision points. It is acceptable if not all $x$-axis labels
can be placed.

\subsection{Error Procedures and Restrictions}

If the character variable TITLE or XNAME has length greater than NCHARS,
only the first NCHARS characters are placed in the image. If YNAME has
length greater than NLINES, only the first NLINES characters are placed in
the image.

This subroutine requires at least enough space for the plot grid boundary
indices, defined above in Section D, to satisfy ({\em bottom} $-$
{\em top}) $\geq 10$ and ({\em right} $-$ {\em left}) $\geq 10$. If this is not
satisfied the subroutine abandons the image-building effort, issues an error
message using the error message package of Chapter~19.2 at error
level 0, and returns with IERR = 1 or~2.

\subsection{Supporting Information}

The source language is ANSI Fortran~77.  At Release~4.0 of MATH77 the
subroutines of this Chapter were introduced and previous similar but
less general subroutines PRPL1 and PRPL2 were deleted.

Based on 1967 code by C. L. Lawson, and J.\ Hatfield, JPL.

Current version by C. L. Lawson, 1992.

\pagebreak

\begin{tabular}{@{\bf}l@{\hspace{5pt}}l}
\bf Entry & \hspace{.35in} {\bf Required Files}\vspace{2pt} \\
DPRPL1 & \parbox[t]{2.7in}{\hyphenpenalty10000 \raggedright
DPRPL1, DPRPL3, ERFIN, ERMSG, IERM1, IERV1\rule[-5pt]{0pt}{8pt}}\\
DPRPL2 & \parbox[t]{2.7in}{\hyphenpenalty10000 \raggedright
DPRPL2, DPRPL3, ERFIN, ERMSG, IERM1, IERV1\rule[-5pt]{0pt}{8pt}}\\
\end{tabular}

\begin{tabular}{@{\bf}l@{\hspace{5pt}}l}
\bf Entry & \hspace{.35in} {\bf Required Files}\vspace{2pt} \\
SPRPL1 & \parbox[t]{2.7in}{\hyphenpenalty10000 \raggedright
ERFIN, ERMSG, IERM1, IERV1, SPRPL1, SPRPL3\rule[-5pt]{0pt}{8pt}}\\
SPRPL2 & \parbox[t]{2.7in}{\hyphenpenalty10000 \raggedright
ERFIN, ERMSG, IERM1, IERV1, SPRPL2, SPRPL3}\\
\end{tabular}

\begcode

\medskip\

\lstset{language=[77]Fortran,showstringspaces=false}
\lstset{xleftmargin=.8in}

\centerline{\bf \large DRSPRPL1}\vspace{10pt}
\lstinputlisting{\codeloc{sprpl1}}

\newpage
{\scriptsize
\centerline{\bf \large ODSPRPL1}\vspace{10pt}
\lstset{language={}}
\hspace{-1.6in}\vbox{
{\ttfamily
\lstinputlisting{\outputloc{sprpl1}}
}
}
\newpage
\lstset{language=[77]Fortran,showstringspaces=false}
\lstset{xleftmargin=.8in}
\enlargethispage*{10pt}
\centerline{\bf \large DRSPRPL2}\vspace{5pt}
\lstinputlisting{\codeloc{sprpl2}}
\newpage
{\scriptsize
\centerline{\bf \large ODSPRPL2}\vspace{10pt}
\hspace{-1.6in}\vbox{
{\ttfamily
\lstinputlisting{\outputloc{sprpl2}}
}
}
\end{document}
