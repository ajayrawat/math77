\documentclass[twoside]{MATH77}
\usepackage{multicol}
\usepackage[fleqn,reqno,centertags]{amsmath}
\DeclareRobustCommand{\us}{\rule{.2pt}{0pt}\rule[-.8pt]{.4em}{.5pt}\rule{.7pt}{0pt}}
\begin{document}
\begmath 19.6  Running Problems in Matrix Market Format With Codes in Chapter 4.7

\silentfootnote{$^\copyright$ \thisyear \ Math \`a la Carte, Inc.}

\subsection{Purpose}

These programs written in C will output either a Fortran code or a C
code which will solve sparse matrix problems entered in Matrix Market
Format, see \url{http://math.nist.gov/MatrixMarket/}.

\subsection{Usage}
Edit the two define's at the start of the code for your desired use,
and compile the code.  Make up a file named mmjob which contains a
list of names for the matrix market files that you may have an
interest in running and that are stored on your machine.
\begin{center}
\fbox{\begin{tabular}{@{\bf }c}
mmgen [--Options] Path [name\us 1] [name\us 2] ...\\
\end{tabular}}
\end{center}

Options are a string of letters and numbers interpreted as follows.
\begin{description}
\item[[a] Process all lines in mmjob after processing the last
  name\us k.  If no names are given, start with the first name in mmjob.
\item[t] Save the transpose of the matrix, and solve $A^T \mathbf{x} =
  \mathbf{b}$.
\item[b0--b9] Specify the number of right hand sides in $\mathbf{b}$.
  The defauls is b1.  In the case of 0, the matrix is factored, and
  another call is made to solve the problem with a single right hand
  side.
\item[c] Compute the reciprocal of the condition number.
\item[d] Compute the determinant.
\end{description}

\subsection{Examples and Remarks}

The file mmjob listed at the end of this document gives the output
shown.

\subsection{Functional Description}

Not applicable.

\subsection{Error Procedures and Restrictions}
\label{sec:errors}
In the case of input errors, an error message is printed.

\subsection{Supporting Information}

The source language is C.

Design and programming by Fred T. Krogh, Math \`a la Carte, Inc.
March 2006.

The random number generator from Chapter 3.1 is called by the drivers
generated.
\begcode
\vspace {10pt}
~\\
\normalsize
The File mmjob\\
1138\us bus\\
CRY10000\\
CURTIS54\\
add32\\
arc130\\
cry2500\\
e20r5000\\
small1\\

\begin{verbatim}
./mmgen -acd ./mmarket

 Problem      N  Seconds   RESERR     XERR      Unused     Used    RCOND     DET   x 10^?
1138_bus   1138    0.016  1.87E-16  4.44E-16     36816     16895  3.98E-08  6.78450  2151
CRY10000  10000    1.620  1.17E-15  6.31E-03   1675872   1554563  2.52E-28  6.30195 15523
CURTIS54     54    0.000  1.46E-15  3.20E-14      2513      1930  1.15E-03 -1.78000    -8
   add32   4960    0.292  6.21E-16  1.34E-14    738755    211828  2.23E-03  1.13981 -9892
  arc130    130    0.004  4.27E-16  1.43E-09      7736     12327  8.83E-08  1.10261     3
 cry2500   2500    0.100  7.27E-16  1.19E-04     98883    240714  3.09E-24  8.65650  2445
e20r5000   4241    1.828  8.14E-14  6.79E-10   2677775   2085209  2.95E-10 -1.24088  -671
  small1     54    0.000  1.46E-15  3.20E-14      2513      1930  1.15E-03 -1.78000    -8
\end{verbatim}

\end{document}


