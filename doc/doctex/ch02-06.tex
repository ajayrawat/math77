\documentclass[twoside]{MATH77}
\usepackage[\graphtype]{mfpic}
\usepackage{multicol}
\usepackage[fleqn,reqno,centertags]{amsmath}
\begin{document}
\opengraphsfile{pl02-06}
\begmath 2.6 Bessel Functions $I_0$, $I_1$, $K_0$ and $K_1$

\silentfootnote{$^\copyright$1997 Calif. Inst. of Technology, \thisyear \ Math \`a la Carte, Inc.}

\subsection{Purpose}

These subprograms compute values of the modified (hyperbolic) Bessel
functions of the first kind, $I_0$ and $I_{1}$ and the modified
(hyperbolic) Bessel functions of the second kind, $K_0$ and $K_1$. These
functions are discussed in \cite{ams55} and \cite{Hart:1968:CA:bes}.

\subsection{Usage}

\subsubsection{Program Prototype, Single Precision}

\begin{description}
\item[REAL]  \ {\bf X, BI0, BI1, BK0, BK1}

\item[INTEGER]  \ {\bf INFO, IWANT}
\end{description}

To compute $I_0\ and/or\ K_0$:\newline
Assign values to X and IWANT, and
$$
\fbox{{\bf CALL SBI0K0(X, BI0, BK0, IWANT, INFO)}}
$$
To compute $I_1\ and/or\ K_1$:\newline
Assign values to X and IWANT, and
$$
\fbox{{\bf CALL SBI1K1(X, BI1, BK1, IWANT, INFO)}}
$$

\subsubsection{Argument Definitions}

\begin{description}
\item[X]  \ [in] Argument of function. Require X $>0$ for the K functions.

\item[BI0, BI1, BK0, BK1]  \ [out] Function values, depending on IWANT.

\item[IWANT]  \ [in] Specification of the desired functions:

\begin{tabular}{@{= }r@{: }l@{ = }l}
 1 & BI$n$ & $I_n(x)$\\
 2 & BK$n$ & $K_n(x)$\\
 3 & BI$n$ & $I_n(x)$ and BK$n = K_n(x)$\\
$-1$ & BI$n$ & $e^{-|x|} I_n(x)$\\
$-2$ & BK$n$ & $e^x K_n(x)$\\
$-3$ & BI$n$ & $e^{-|x|} I_n(x)$ and BK$n = e^x K_n(x).$
\end{tabular}

\item[INFO]  \ [out] indicates the status on termination.  INFO = 0
means the computation was successful.  See Section~E for the meaning
of nonzero values of INFO.
\end{description}

\subsubsection{Modifications for Double Precision}

For double precision usage, change the REAL type statement to DOUBLE
PRECISION, and change the subroutine names to DBI0K0 and DBI1K1, respectively.

\subsection{Examples and Remarks}

The listing of DRSBESI0 and ODSBESI0 gives an example of using these
subprograms to evaluate the Wronskian identity
\begin{equation*}
\zeta (x) = x[I_0(x)K_1(x)+I_1(x)K_0(x)] - 1 = 0
\end{equation*}

\subsection{Functional Description}

The functions $I_n$ and $K_n$ are a pair of linearly
independent solutions for the differential equation
\begin{equation*}
x^2\frac{d^2w}{dx^2}+x\frac{dw}{dx}-\left(x^2+n^2\right)w=0
\end{equation*}
The functions $I_0$ and $I_1$ are defined for all real $x$. The function $%
I_0 $ is even, and $I_1$ is odd. They are monotone increasing functions of $%
|x|$, approaching $e^{|x|} (2\pi |x|)^{-\frac{1}{2}}$ as $|x| \rightarrow
\infty $. The functions $e^{-|x|} I_0(x)$ and $e^{-|x|} I_1(x)$ are monotone
decreasing functions of $|x|$, approaching $(2\pi |x|)^{-\frac{1}{2}}$ as $%
|x| \rightarrow \infty .$

The functions $K_0$ and $K_1$ have real values for positive real $x$,
approach $+\infty $ as $x \rightarrow 0^+$ and have complex values for
negative real $x$. As $x \rightarrow 0^+$, $K_0(x) \rightarrow -\ln (x)$ and
$K_1(x) \rightarrow 1/x$. $K_n(x)$ is a monotone decreasing function of $x$,
approaching $e^{-x} (\pi /2x)^{\frac{1}{2}}$ as $x \rightarrow +\infty .$

The K subprograms treat $x\leq 0$ as an error condition. If the complex values
of $K_0$ and $K_1$ for negative $x$ are desired they may be computed from
the formulae
\begin{equation*}
\begin{array}{ll}
\begin{array}{r@{\:}c@{\:}l}
K_0(x) & = & \phantom{-}K_0(-x)-i\pi I_0(-x)
\rule[-10pt]{0pt}{8pt} \\ K_1(x) & = & -K_1(-x)-i\pi I_1(-x),
\end{array}
& \quad x<0
\end{array}
\end{equation*}
where $i$ denotes the imaginary unit. See Equation 9.6.31 in \cite{ams55}.

\vspace{10pt}

\hspace{5pt}\mbox{\input pl02-06 }

The computer approximations for these functions, except for $K_0(x)$ when
$x < 2$, were developed by L.  W.  Fullerton, \cite{Fullerton:1973:FNLIB}
and \cite{Fullerton:1977:PSF}, using functional forms involving sine,
cosine, square root, logarithm, and Chebyshev polynomial approximations.
The computer approximations for $K_0(x)$ when $x < 2$ were developed by W.
J.  Cody of Argonne National Laboratory using rational polynomial
approximations.  Where Chebyshev polynomial approximations are used, these
subprograms select the polynomial degrees to adapt to machine accuracy of
up to 30 decimal places.  The coefficients given by Cody have only 21
digits.  If 30 decimal place precision is needed, approximations due to
Fullerton are available, but for precisions up to 16 decimal places, they
are somewhat less accurate than the approximations due to Cody, because
Fullerton computes $K_0(x)$ using $I_0(x)$, $\ln (x)$ and a Chebyshev
polynomial (see formula~9.6.13 in \cite{ams55}); the errors in $I_0(x)$
and $\ln (x)$, together with the error in the Chebyshev polynomial
approximation, are greater than the errors in Cody's approximation.

The single precision subprograms for $I_n(x)$ and $K_n(x)$ were tested on an
IBM PC/AT (using IEEE arithmetic) by comparison with the corresponding
double precision subprograms over various argument ranges. Each interval was
divided into 10,000 subintervals, and a point was randomly selected in each
subinterval. The relative precision of IEEE single precision arithmetic is $%
\rho = 2^{-23} \approx 1.19\times 10^{-7}$. The test results may be
summarized as follows.

\begin{tabular}{clccc}
& \multicolumn{4}{r}{\bf Relative Error, in units of $\rho $}\\
 \multicolumn{1}{@{}c}{\bf Function} &  \multicolumn{1}{c}{\bf Interval} &
  \multicolumn{1}{c}{\bf Max.} & \multicolumn{1}{c}{\bf Mean} &
  \multicolumn{1}{c}{\bf Std. Dev.}\\
$I_0(x)$ & [0, 2.5] & 1.21 & 0.23 & 0.17\\
& [2.5, 5] & 1.15 & 0.35 & 0.25\\
& [5, 15] & 1.31 & 0.40 & 0.26\\
& [15, 80] & 1.21 & 0.39 & 0.26\\
$I_1(x)$ & [0, 2.5] & 1.01 & 0.23 & 0.17\\
& [2.5, 5] & 1.49 & 0.38 & 0.28\\
& [5, 15] & 1.39 & 0.42 & 0.28\\
& [15, 80] & 1.60 & 0.41 & 0.29\\
$K_0(x)$ & [0, 1.8] & 2.15 & 0.41 & 0.37\\
& [1.8, 2.1] & 2.28 & 0.43 & 0.34\\
& [2.1, 5] & 1.33 & 0.35 & 0.26\\
& [5, 15] & 1.36 & 0.34 & 0.24\\
& [15, 80] & 1.33 & 0.36 & 0.26\\
$K_1(x)$ & [0, 1.8] & 2.01 & 0.28 & 0.23\\
& [1.8, 2.1] & 2.30 & 0.41 & 0.31\\
& [2.1, 5] & 1.11 & 0.31 & 0.22\\
& [5, 15] & 1.09 & 0.31 & 0.22\\
& [15, 80] & 1.12 & 0.31 & 0.22
\end{tabular}

The double precision subprograms for $I_n(x)$ were tested on an IBM PC/AT
(using IEEE arithmetic) by comparison with extended precision subprograms
over various argument ranges. The double precision routines for $K_n(x)$
were tested on an IBM PC/AT (using IEEE arithmetic) by comparison with an
independent double precision procedure, due to Cody, that creates a purified
argument, that is, one for which the function can be evaluated without
significant error. Each interval was divided into 2000 subintervals, and a
point was randomly selected in each subinterval. The relative precision of
IEEE double precision arithmetic is $\rho = 2^{-52} \approx 2.22\times
10^{-16}$. The test results may be summarized as follows.

\begin{tabular}{clrrr}
& \multicolumn{4}{r}{\bf Relative Error, in units of $\rho $}\\
 \multicolumn{1}{@{}c}{\bf Function} &  \multicolumn{1}{c}{\bf Interval} &
  \multicolumn{1}{c}{\bf Max.} & \multicolumn{1}{c}{\bf Mean} &
  \multicolumn{1}{c}{\bf Std. Dev.}\\
$I_0(x)$ & [0, 2.5] & 1.38 & 0.24\rule{5pt}{0pt} & 0.31\rule{20pt}{0pt} \\
& [2.5, 5] & 1.36 & 0.34\rule{5pt}{0pt} & 0.25\rule{20pt}{0pt} \\
& [5, 9] & 5.14 & 0.63\rule{5pt}{0pt} & 0.79\rule{20pt}{0pt} \\
$I_1(x)$ & [0, 2.5] & 1.04 & 0.24\rule{5pt}{0pt} & 0.18\rule{20pt}{0pt} \\
& [2.5, 5] & 1.41 & 0.34\rule{5pt}{0pt} & 0.26\rule{20pt}{0pt} \\
& [5, 9] & 1.97 & 0.45\rule{5pt}{0pt} & 0.33\rule{20pt}{0pt} \\
$K_0(x)$ & [0, 1.8] & 3.37 & 0.07\rule{5pt}{0pt} & 0.77\rule{20pt}{0pt} \\
& [1.8, 2.1] & 2.75 & 0.03\rule{5pt}{0pt} & 0.81\rule{20pt}{0pt} \\
& [2.1, 5] & 2.94 & 0.54\rule{5pt}{0pt} & 0.52\rule{20pt}{0pt} \\
& [5, 15] & 2.24 & 0.58\rule{5pt}{0pt} & 0.50\rule{20pt}{0pt} \\
& [15, 80] & 2.78 & 0.64\rule{5pt}{0pt} & 0.55\rule{20pt}{0pt} \\
& [80, 225] & 3.21 & 0.66\rule{5pt}{0pt} & 0.59\rule{20pt}{0pt} \\
$K_1(x)$ & [0, 1.8] & 3.41 & 0.74\rule{5pt}{0pt} & 0.63\rule{20pt}{0pt} \\
& [1.8, 2.1] & 4.75 & 1.07\rule{5pt}{0pt} & 0.85\rule{20pt}{0pt} \\
& [2.1, 5] & 5.31 & 1.49\rule{5pt}{0pt} & 1.10\rule{20pt}{0pt} \\
& [5, 15] & 8.84 & 3.43\rule{5pt}{0pt} & 2.22\rule{20pt}{0pt} \\
& [15, 80] & 57.67 & 16.51\rule{5pt}{0pt} & 13.62\rule{20pt}{0pt} \\
& [80, 225] & 115.95 & 49.63\rule{5pt}{0pt} & 33.35\rule{20pt}{0pt}
\end{tabular}

The poor accuracy reported for $I_0(x)$ in [5,~9] occurs largely in [8.5,
9], and may be due to a questionable reference value.  Other reports of
poor accuracy are probably justified.

Having no adequate reference function for $I_0(x)$ and $I_1(x)$ for $x > 9.0$%
, subprograms for these functions were tested by computing the Wronskian
relation $I_0(x)K_1(x) + I_1(x)K_0(x)$ and comparing the result to $1/x$.
The test results may be summarized as follows.

\begin{tabular}{lccc}
 \multicolumn{4}{r}{\bf Relative Error, in units of $\rho $}\\
 \multicolumn{1}{c}{\bf Interval} &
  \multicolumn{1}{c}{\bf Max.} & \multicolumn{1}{c}{\bf Mean} &
  \multicolumn{1}{c}{\bf Std. Dev.}\\
 $[9, 80]$ & 1.52 & 0.43 & 0.30\\
 $[80, 700]$ & 1.66 & 0.43 & 0.30
\end{tabular}

The errors here are much smaller than the errors reported above for $K_0(x)$
and $K_1(x)$. The very accurate values of the Wronskian relation suggest a
functional dependence between the algorithms implemented in the different
hyperbolic Bessel function subprograms.

\begin{table*}
\begin{center}
\begin{tabular}{rlccll}
\multicolumn{6}{c}{\bf Argument range testing in SBI0K0 or DBI0K0:\rule[-10pt]{0pt}{8pt}}\\
\multicolumn{1}{c}{\bf INFO} & \multicolumn{1}{c}{\bf Argument Range} &
{\bf BI0} & {\bf BK0} &
\multicolumn{1}{c}{\bf IWANT} & \multicolumn{1}{c}{\bf Reason}\\
$-$3 & $x \leq 0$ & correct & $\Omega $ & $|\text{IWANT}| \geq 2$ &
$K_0(x)$ and $e^xK_0(x)$ are imaginary\\
$-$2 & $x > \text{XI}_0\text{MAX} > \text{XK}_0\text{MAX}$ &
$\Omega $ & zero & IWANT $> 0$ & $I_0(x)$ overflows, $K_0(x)$underflows\\
1 & $\text{XI}_0\text{MAX} \geq x > \text{XK}_0$MAX & correct & zero &
IWANT $\geq 2$ & $K_0(x)$ underflows\rule[-20pt]{0pt}{8pt}
\end{tabular}

\begin{tabular}{rlccll}
\multicolumn{6}{c}{\bf Argument range testing in SBI1K1 or DBI1K1:\rule[-10pt]{0pt}{8pt}}\\
\multicolumn{1}{c}{\bf INFO} & \multicolumn{1}{c}{\bf Argument Range} &
{\bf BI1} & {\bf BK1} &
\multicolumn{1}{c}{\bf IWANT} & \multicolumn{1}{c}{\bf Reason}\\
$-$4 & $0 < x < \text{XK}_1$MIN & correct & $\Omega $ & $|\text{IWANT}|
\geq 2$ & $K_1(x)$ and $e^xK_1(x)$ overflow\\
$-$3 & $x \leq 0$ & correct & $\Omega $ & $|\text{IWANT}| \geq 2$ &
$K_1(x)$ and $e^xK_1(x)$ are imaginary\\
$-$2 & $x > \text{XI}_1\text{MAX} > \text{KX}_1\text{MAX}$ & $\Omega $ &
 zero & IWANT $> 0$ & $I_1(x)$ overflows, $K_1(x)$ underflows\\
1 & $\text{XI}_1\text{MAX} \geq x > \text{XK}_1\text{MAX}$ & correct &
 zero & IWANT $\geq 2$ & $K_1(x)$ underflows
\end{tabular}
\end{center}
\end{table*}

\bibliography{math77}
\bibliographystyle{math77}

\subsection{Error Procedures and Restrictions}

These subroutines set INFO to indicate the termination status.  INFO
= 0 means the computation was successful.  INFO = $-$1 means IWANT
had an improper value.  Any other value means the argument was one for
which one of the requested functions does not have a defined real
value, or the function value would be outside the underflow or
overflow limits of the host system.

The subroutines obtain host exponent limits using R1MACH() or
D1MACH() of Chapter 19.1.  These are used to compute argument limits,
XI$_0$MAX, etc., using asymptotic formulas --- Equations 9.7.1
and~9.2.2 of \cite{ams55}.  For example on a machine with IEEE
arithmetic, these argument limits are as listed in the following
table:

\begin{tabular}{lll}
\multicolumn{1}{c}{\bf Bound} & \multicolumn{1}{c}{\bf Single Precision} &
\multicolumn{1}{c}{\bf Double Precision}\\
$\text{XI}_0\text{MAX}$ & \quad \quad 91.900 & \quad \quad 713.9869\\
$\text{XK}_0\text{MAX}$ & \quad \quad 85.337 & \quad \quad 705.3427\\
$\text{XI}_1\text{MIN}$ & \quad \quad $2.531\times 10^{-38}$ &
\quad \quad $4.45\times 10^{-308}$\\
$\text{XI}_1\text{MAX}$ & \quad \quad 91.906 & \quad \quad 713.9876\\
$\text{XK}_1\text{MIN}$ & \quad \quad $1.187\times 10^{-38}$ &
\quad \quad $2.247\times 10^{-308}$\\
$\text{XK}_1\text{MAX}$ & \quad \quad 85.34 & \quad \quad 705.3434
\end{tabular}

The different argument range tests and resulting settings of BI0,
BK0, BI1, BK1, and INFO are given in the tables above, where
$\Omega $ denotes the largest representable number.

When INFO $\ne $ 0 the error message processor of Chapter 19.2 is
called --- with LEVEL = 0 if INFO $<$ 0 and with LEVEL = $-2$ if
INFO $>$ 0.  The user can alter the default action of the error
processor by calling ERMSET of Chapter~19.2.

\subsection{Supporting Information}

The source language is ANSI Fortran~77.

Subprograms SBI0K0 and SBI1K1 designed and developed by W. V. Snyder,
JPL, 1990, based on earlier subprograms by L. W. Fullerton and W. J. Cody.


% \newpage

\begin{tabular}{@{\bf}l@{\hspace{5pt}}l}
\bf Entry & \hspace{.35in} {\bf Required Files}\vspace{2pt} \\ 
DBI0K0 & \parbox[t]{2.7in}{\hyphenpenalty10000 \raggedright
AMACH, DBI0K0, DCSEVL, DERM1, DERV1, DINITS, ERFIN, ERMSG, IERM1,
IERV1\rule[-5pt]{0pt}{8pt}}\\
DBI1K1 & \parbox[t]{2.7in}{\hyphenpenalty10000 \raggedright
AMACH, DBI1K1, DCSEVL, DERM1, DERV1, DINITS, ERFIN, ERMSG, IERM1,
IERV1\rule[-5pt]{0pt}{8pt}}\\
%\end{tabular}
%\begin{tabular}{@{\bf}l@{\hspace{5pt}}l}
%\bf Entry & \hspace{.35in} {\bf Required Files}\vspace{2pt} \\
SBI0K0 & \parbox[t]{2.7in}{\hyphenpenalty10000 \raggedright
AMACH, ERFIN, ERMSG, IERM1, IERV1, SBI0K0, SCSEVL, SERM1, SERVI,
SINITS\rule[-5pt]{0pt}{8pt}}\\
SBI1K1 & \parbox[t]{2.7in}{\hyphenpenalty10000 \raggedright
AMACH, ERFIN, ERMSG, IERM1, IERV1, SBI1K1, SCSEVL, SERM1, SERVI,
SINITS\rule[-5pt]{0pt}{8pt}}\\
\end{tabular}


\begcodenp

\lstset{language=[77]Fortran,showstringspaces=false}
\lstset{xleftmargin=.8in}

\centerline{\bf \large DRSBI0K0}\vspace{5pt}
\lstinputlisting{\codeloc{sbi0k0}}

\vspace{10pt}\centerline{\bf \large ODSBI0K0}\vspace{5pt}
\lstset{language={}}
\lstinputlisting{\outputloc{sbi0k0}}

\closegraphsfile
\end{document}
