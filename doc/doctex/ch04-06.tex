\documentclass[twoside]{MATH77}
\usepackage{multicol}
\usepackage[fleqn,reqno,centertags]{amsmath}
\begin{document}
\begmath 4.6 Solution of a Positive-Definite System \hbox{with Cholesky Factorization}

\silentfootnote{$^\copyright$1997 Calif. Inst. of Technology, \thisyear \ Math \`a la Carte, Inc.}

\subsection{Purpose}

This subroutine computes the solution vector ${\bf x}$ for a system of
equations of the form
\begin{equation}
\label{O1}P{\bf x}={\bf d},
\end{equation}
where $P$ is an N$\times $N positive-definite symmetric matrix, and ${\bf d}$ is an
N-vector. This subroutine also returns the Cholesky factor of $P$ and thus is
applicable where computing the Cholesky factor is the objective.

\subsection{Usage}

\subsubsection{Program Prototype, Double Precision}

\begin{description}
\item[DOUBLE PRECISION]  {\bf P}(LDP, $\geq $N) [LDP $\geq $ N]{\bf , D}($%
\geq $N$)${\bf , U, TOL}

\item[INTEGER]  \ {\bf LDP, N, IERR}
\end{description}

Assign values to P(,), LDP, N, D(), U, and TOL.
$$
\fbox{{\bf CALL DCHOL (P, LDP, N, D, U, TOL, IERR)}}
$$
The solution vector ${\bf x}$ will be stored in D(). Additional computed
quantities that may be of interest to the user in some situations will be
stored in P(,) and U.

\subsubsection{Argument Definitions}

\begin{description}
\item[P(,)]  \ [inout] On entry this array must contain the N$\times $N
symmetric positive-definite matrix $P$ of Eq.\,(1). It suffices to provide only
the elements on and above the diagonal. On return this array will contain the
N$\times $N upper triangular matrix F defined by Eq.\,(2) on and above the diagonal
positions of the array P(,). Locations of the array P(,) below the diagonal
will not be referenced or modified by this subroutine.

\item[LDP]  \ [in] Dimension of the first subscript of the storage array
P(,). Require LDP $\geq $ N.

\item[N]  \ [in] Order of the matrix $P$. Require N $\geq 1.$

\item[D()]  \ [inout] On entry D() must contain the vector ${\bf d}$ of Eq.
(1). On return D() contains the solution vector ${\bf x}$ for Eq.\,(1).

\item[U]  \ [inout] If U contains the number ${\bf u}$ of Eq.\,(10) or (15)
respectively on entry, then on return U will contain the number $\rho $ of
Eq.\,(11) or (16) respectively. If the user is not interested in having the
number $\rho $ computed, U should be zero on entry and will be unchanged on
return.

\item[TOL]  \ [in] A user-provided relative tolerance parameter to be used
in the conditioning test of Eq.\,(17). We suggest setting TOL to a value of $%
10^{-(k+1)}$ where $k=\min (k_A$, $k_b)$. Here $k_A$ is the user's estimate
of the number of significant decimal digits in the elements of the matrix $A$
and $k_b$ is the corresponding estimate for ${\bf b}$. See Eq.\,(7) or (12)
for the definitions of $A$ and ${\bf b}$. If the TOL input is $<\varepsilon $,
where $\varepsilon $ is the relative machine precision $(i.e$. the smallest
positive number such that $1.0+\varepsilon \neq 1.0$ in the machine's
floating point arithmetic), then $\varepsilon $ is used for TOL internally.

\item[IERR]  \ [out] On return this is set to~0 if $t_{\min }$ defined in
Eq.\,(17) is greater than~0. Otherwise results are of
questionable validity and $|$IERR$|$ will equal the index of the equation
that resulted in the value for $t_{\min }$. See Section E for more details.
\end{description}

\subsubsection{Modifications for Single Precision}

We recommend the use of double precision for this computation except on
machines such as the Cray that have $10^{-14}$ precision in single
precision. To use single precision change DCHOL to SCHOL, and the DOUBLE
PRECISION type statement to REAL.

\subsection{Examples and Remarks}

Consider the least-squares problem $A{\bf x}\simeq {\bf b}$ where $A$ is the
3~by~2 matrix and ${\bf b}$ is the 3-vector defined by the DATA statements
in the program DRDCHOL below. This program forms normal equations by
computing $P=A^TA$ and ${\bf d}=A^T{\bf b}$. It also computes u $={\bf b}^T%
{\bf b}$. It uses the subroutine DCHOL to solve the normal equations $P{\bf x%
}={\bf d}$, and to compute the quantity RNORM $=\rho =\Vert {\bf b}-A{\bf x}%
\Vert $. Output from this program is given in the file ODDCHOL.

For programming convenience one may prefer to store the matrix $A$ and the
vector ${\bf b}$ together in the same array. The code in DRDCHOL2 shows how
this example can be programmed storing $A$ and ${\bf b}$ together in the
array AB() and using the array PDU() to hold $P$, ${\bf d}$, and $u.$

\subsection{Functional Description}

Given the problem $P{\bf x}={\bf d}$, where $P$ is an N$\times $N
positive-definite symmetric matrix, there exists an upper triangular
N$\times $N matrix $F$ satisfying
\begin{equation}
\label{O2}F^TF=P
\end{equation}
Eq.\,(2) defines the Cholesky decomposition of $P$. The upper triangular
elements of $F$ will be computed from those of $P$ by the following
equations, where $i=1$, ..., N.
\begin{align}
\label{O3}g_i&=p_{i,i}-\sum_{k=1}^{i-1}f_{k,i}^2\\
\label{O4}f_{i,i}&=g_{i}^{1/2}\\
\label{O5}f_{i,j}&=\frac{\displaystyle p_{i,j}-\sum_{k=1}^{i-1}f_{k,i}f_{k,j}%
}{f_{i,i}},\ \ j=i+1,...,\text{N}\hspace{-2in}
\end{align}
In these formulas the summation is to be skipped when $i=1.$

After computing $F$ the subroutine solves the lower triangular system of
equations,%
\begin{equation*}
F^T{\bf y}={\bf d}
\end{equation*}
and then computes the vector ${\bf x}$ which satisfies $P{\bf x}={\bf d}$ by
solving the upper triangular system%
\begin{equation*}
F{\bf x}={\bf y}
\end{equation*}
Besides computing the solution vector ${\bf x}$ this subroutine uses the
input number $u$ given in the Fortran variable U to compute
\begin{equation}
\label{O6}\rho =\left[ \max (0,u-{\bf y}^T{\bf y})\right] ^{1/2}
\end{equation}
This number $\rho $ is stored in U on return. If the problem $P{\bf x}={\bf d%
}$ arose as the system of normal equations for a least-squares problem and
if $u$ was computed appropriately by the user then $\rho $ represents the
norm of the residual vector for the least-squares problem.

Specifically if the user wishes to solve the least-squares problem of
minimizing
\begin{equation}
\label{O7}\left\| {\bf b}-A{\bf x}\right\| =\left[ \left( {\bf b}-A{\bf x}%
\right) ^T\left( {\bf b}-A{\bf x}\right) \right] ^{1/2}
\end{equation}
then $P$, ${\bf d}$, and $u$ should be initialized as
\begin{align}
\label{O8}P&=A^TA\\
\label{O9}{\bf d}&=A^T{\bf b}\\
\label{O10}u&={\bf b}^T{\bf b}
\end{align}
Then theoretically, the quantity, $u-{\bf y}^T{\bf y}$ of Eq.\,(6) will be
nonnegative and the number $\rho $ of Eq.\,(6) will have the interpretation
\begin{equation}
\label{O11}\rho =\left\| {\bf b}-A{\bf x}\right\|
\end{equation}
More generally, if the user is solving the weighted least-squares problem of
minimizing
\begin{equation}
\label{O12}\left[ \left( {\bf b}-A{\bf x}\right) ^TW\left( {\bf b}-A{\bf x}%
\right) \right] ^{1/2}
\end{equation}
where $W$ is a positive definite symmetric matrix, then $P$, ${\bf d}$, and $%
u$ should be initialized as
\begin{align}
\label{O13}P&=A^TWA\\
\label{O14}{\bf d}&=A^TW{\bf b}\\
\label{O15}u&={\bf b}^TW{\bf b}
\end{align}
Then, theoretically, the quantity $u-{\bf y}^T{\bf y}$ of Eq.\,(6) will be
nonnegative and the number $\rho $ of Eq.\,(6) will have the interpretation
\begin{equation}
\label{O16}\rho =\left[ \left( {\bf b}-A{\bf x}\right) ^TW\left( {\bf b}-A%
{\bf x}\right) \right] ^{1/2}
\end{equation}
The Cholesky factor matrix $F$ will appear in the upper triangular portion
of the array P(,) on return. If IERR $\geq 0$, the user can input this
matrix $F$ to the library subroutine DCOV2 of Chapter~4.2 to compute the
unscaled covariance matrix for the associated least-squares problem. This
requires building the IP() array: IP(I) = I, for I $=1$, ..., N.

Theoretically the numbers $g_i$ of Eq.\,(3) will be strictly positive for all
$i$ if and only if the symmetric matrix $P$ is positive-definite. If all $%
g_i $ are positive but the ratio $g_i/p_{i,i}$ is very small for some $i$
this is an indication that the problem is ill-conditioned. The square of the
relative tolerance parameter TOL is used to test this ratio. Let
\begin{equation}
\label{o17}t_{\min }=\min _{1\leq i\leq \text{N}}\left\{ g_i-(TOL)^2\times
|p_{i,i}|\right\} .
\end{equation}
If $t_{\min }\geq 0$, then IERR is set to~0. Otherwise let $m$ be a value of
$i$ that gives the minimum value in Eq.\,(17). Then IERR is set to $m$ if $%
g_m>0$, and is set to $-m$ otherwise. See Section E below for more details.

If one knows or suspects that the least-squares problem is ill-conditioned
it is suggested that the Singular Value Analysis subroutine, Chapter~4.3, be
used to obtain a more complete analysis and a more reliable solution for the
problem.

A nonnegative definite symmetric matrix has a Cholesky factor even if it is
singular. In computing a Cholesky factor for such matrices this subroutine
does the following: If $g_i$ of Eq.\,(3) is nonpositive, Eqs.~(4--5) are
replaced by
\begin{equation}
\label{O18}f_{i,j}=0,\quad j=i,i+1,...,\text{N}
\end{equation}
When solving the triangular systems below Eq.\,(5), if $f_{i,i}=0$ the
solution components $y_i$ and $x_i$ are set to zero. If $P$ is a singular
nonnegative definite matrix, the matrix $F$ produced in this way is its
(nonunique) Cholesky factor, $i.e.$, it satisfies Eq.\,(2). In such a case
Eq.\,(1) may or may not have a solution and the vector ${\bf x}$ produced in
this way is the solution only if a solution exists.

\subsection{Error Procedures and Restrictions}

If $t_{\min }<0$ in Eq.\,(17), the subroutine sets IERR nonzero as
indicated above. When IERR $< 0$, at least one row of the augmented
matrix [upper triangle of $P$, $D$] will have been set to zero.

If IERR $\neq 0$ we suggest that the user apply the Singular Value Analysis
subroutine, Chapter~4.3, to the associated least-squares problem.\vfill

\subsection{Supporting Information}

The source language is ANSI Fortran~77.

\begin{tabular}{@{\bf}l@{\hspace{5pt}}l}
\bf Entry & \hspace{.35in} {\bf Required Files}\vspace{2pt} \\
DCHOL & \hspace{.3in} AMACH, DCHOL\rule[-5pt]{0pt}{8pt}\\
SCHOL & \hspace{.3in} AMACH, SCHOL\\
\end{tabular}

Programmed by: C. L. Lawson, JPL, May~1969.

Program Revised by: F. T. Krogh, JPL, September~1991.


\begcode

\medskip\
\lstset{language=[77]Fortran,showstringspaces=false}
\lstset{xleftmargin=.8in}

\centerline{\bf \large DRDCHOL}\vspace{10pt}
\lstinputlisting{\codeloc{dchol}}

\vspace{30pt}\centerline{\bf \large ODDCHOL}\vspace{10pt}
\lstset{language={}}
\lstinputlisting{\outputloc{dchol}}

\newpage\
\lstset{language=[77]Fortran,showstringspaces=false}
\lstset{xleftmargin=.8in}

\centerline{\bf \large DRDCHOL2}\vspace{10pt}
\lstinputlisting{\codeloc{dchol2}}

\vspace{30pt}\centerline{\bf \large ODDCHOL2}\vspace{10pt}
\lstset{language={}}
\lstinputlisting{\outputloc{dchol2}}
\end{document}
