\documentclass[twoside]{MATH77}
\usepackage{multicol}
\usepackage[fleqn,reqno,centertags]{amsmath}
\begin{document}
\begmath 11.3 Conversion between Chebyshev and Monomial
\hbox{Representations of a Polynomial}

\silentfootnote{$^\copyright$1997 Calif. Inst. of Technology, \thisyear \ Math \`a la Carte, Inc.}

\subsection{Purpose}

These subroutines convert a polynomial represented in the monomial basis to
a representation in the Chebyshev basis, and vice versa.

\subsection{Usage}

{\bf B.1 Program Prototype, Single Precision}
\begin{description}
\item[INTEGER]  \ {\bf N}

\item[REAL]  {\bf COEFF}(0:$\geq $N)
\end{description}
Assign values to N, and to coefficients in COEFF(). If COEFF(i) contains
coefficients of $T_i(x)$, $i = 0$, 1, ..., N, which are to be converted to
coefficients of $x^i$,
$$
\fbox{{\bf CALL SCONCM(N, COEFF)}}
$$
For the inverse operation,
$$
\fbox{{\bf CALL SCONMC(N, COEFF)}}
$$
\subsubsection{Argument Definitions}
\begin{description}
\item[N]  \ [in] The degree of the polynomial.

\item[COEFF]  \ [inout] When calling SCONCM, COEFF(i) contains the
coefficient of $T_i$, $i=0$, 1, ... N, on input, and contains the
coefficient of $x^i$ on output. When calling SCONMC, COEFF(i) contains the
coefficient of $x^i$, $i=0$, 1, ..., N on input, and the coefficient of $T_i$
on output.
\end{description}
\subsubsection{Modifications for Double Precision}

Change the names SCONCM and SCONMC to DCONCM and DCONMC respectively, and
change the REAL declaration to DOUBLE PRECISION.

\subsection{Examples and Remarks}

The program DRSCON prints out the coefficients of the Chebyshev polynomials
corresponding to $x^k$, $k = 0$, 1, ..., 6, and then prints the coefficients
in the monomial basis corresponding to the Chebyshev polynomials $T_k$, $k =
0$, 1, ..., 6. Results are in the file ODSCON.

If these subroutines are applied to a coefficient array, say P(), obtained
from SPFIT, Chapter~11.1, the zero$^{th}$ order coefficient is in P($3)$ so
the call would be of the form SCONxx(NDEG, P($3))$, where xx is either CM or
MC.

\subsection{Functional Description}

Consider the polynomial $p_n(x)$ of degree $n$,
\begin{equation}
\label{O1}p_n(x)=\sum_{k=0}^na_kx^k\equiv \sum_{k=0}^nc_kT_k(x)
\end{equation}
where $T_k(x)$ is the $k^{th}$ Chebyshev polynomial. This software
converts between the $a_k$'s and the $c_k$'s.

Using the well-known identities,
\begin{equation}
\label{O2}
\begin{split}
xT_k(x)&=\frac 12\left[ T_{k+1}(x)+T_{k-1}(x)\right],\ \ k>1\\
xT_0(x)&=T_1(x)=x,
\end{split}
\end{equation}
we can write $p_n$ in forms intermediate between the extremes represented
in Eq.\,(1). It is these intermediate forms that are used in obtaining the
recurrences. Thus
\begin{align}
\label{O3}
p_n(x) &=\sum_{k=0}^{j-1}a_kx^k+x^j\sum_{k=0}^{n-j}b_{k,j}T_k(x)\\
\label{04}
 &\equiv \sum_{k=0}^ja_kx^k+x^{j+1}\sum_{k=0}^{n-j-1}b_{k,j+1}T_k(x)
\hspace{-1in}
\end{align}
Note that $b_{k,0}\equiv c_k$. Using Eq.\,(2), Eq.\,(4) gives
\begin{multline}
\label{O5}
p_n(x)=\sum_{k=0}^{j}a_kx^k+\frac{x^j}2 \sum_{k=1}^{n-j-1}
b_{k,j+1}\bigl [ T_{k+1}(x)\\
+\ T_{k-1}(x)\bigr ]+x^jb_{0,j+1}T_1(x).
\end{multline}
Collecting like terms in Eqs.\,(3) and (5), we obtain,
\begin{equation}\label{O6}
\hspace{-15pt}\begin{split}
a_j+\frac 12b_{1,j+1} &= b_{0,j}\\
b_{0,j+1}+\frac 12 b_{2,j+1} &=b_{1,j}\\
\frac 12\left[ b_{k-1,j+1}+b_{k+1,j+1}\right] &=
b_{k,j},\ k=2,3,...,n-j-2\\
\frac 12b_{k-1,j+1} &= b_{k,j},\ k\geq n-j-1.
\end{split}
\end{equation}
A more efficient recursion is obtained with $b_{k,j}$ replaced by
$2^jB_{k,j}$. Thus,
\begin{equation}\label{O7}
\hspace{-15pt}\begin{split}
2^{-j}a_j+B_{1,j+1} &= B_{0,j}\\
2 B_{0,j+1}+B_{2,j+1} &= B_{1,j}\\
B_{k-1,j+1}+B_{k+1,j+1} &= B_{k,j},\ k=2,3,...,n-j-2\\
B_{k-1,j+1} &= B_{k,j}, \ k\geq n-j-1.
\end{split}
\end{equation}
In the code, the $B_{k-j,k}$ share space with the original $a_k$ or the
original $c_k$. If one starts with the $a_k$ then one runs $j$ from $n$ down
to~0, and otherwise $j$ runs in the opposite direction. Observe that the
innermost loop requires only a single addition.

\subsection{Error Procedures and Restrictions}

If $n < 0$, a return is made without taking any action.

\subsection{Supporting Information}

The source language is ANSI Fortran~77.
Algorithm and code by F.\ T.\ Krogh, JPL, January 1992.

\begin{tabular}{@{\bf}l@{\hspace{5pt}}l}
\bf Entry & \hspace{.35in} {\bf Required Files}\vspace{2pt} \\
DCONCM & \parbox[t]{2.7in}{\hyphenpenalty10000 \raggedright
\hspace{.35in} DCONCM\rule[-5pt]{0pt}{8pt}}\\
DCONMC & \parbox[t]{2.7in}{\hyphenpenalty10000 \raggedright
\hspace{.35in} DCONMC\rule[-5pt]{0pt}{8pt}}\\
SCONCM & \parbox[t]{2.7in}{\hyphenpenalty10000 \raggedright
\hspace{.35in} SCONCM\rule[-5pt]{0pt}{8pt}}\\
SCONMC & \parbox[t]{2.7in}{\hyphenpenalty10000 \raggedright
\hspace{.35in} SCONMC}\\
\end{tabular}

\begcode

\medskip\
\lstset{language=[77]Fortran,showstringspaces=false} \lstset{xleftmargin=.8in}

\centerline{\bf \large DRSCON}\vspace{10pt}
\lstinputlisting{\codeloc{scon}}
\newpage

\vspace{30pt}\centerline{\bf \large ODSCON}\vspace{10pt}
\lstset{language={}}
\lstinputlisting{\outputloc{scon}}
\end{document}
