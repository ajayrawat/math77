\documentclass[twoside]{MATH77}
\usepackage{multicol}
\usepackage[fleqn,reqno,centertags]{amsmath}
\begin{document}
\hyphenation{YPKNOT}
\begmath 11.4 Least-Squares Cubic Spline Fit

\silentfootnote{$^\copyright$1997 Calif. Inst. of Technology, \thisyear \ Math \`a la Carte, Inc.}

\subsection{Purpose}

A cubic spline function with NB $-$ 1 segments is a function consisting of
NB $-$ 1 pieces, each of which is a cubic polynomial. At the abscissae,
called knots, at which adjacent segments meet, the function has $C^2$
continuity, $i.e$. continuity in value, first derivative, and second
derivative.

Subroutine SC2FIT or DC2FIT will determine the (NB $-$ 1)-segment cubic spline
function, with user specified knots, that best fits a set of discrete data
in the sense of weighted least-squares, and return the values of the fitted
spline curve and its first derivative at the knots. The user can then
evaluate the curve, or its first or second derivative, at any argument using
the Hermite interpolation subroutine, SHINT or DHINT, of Chapter~12.3.

This software can be used for interpolation by setting the number of knots,
NB, to be two less than the number of data points, NXY. Setting NB $<$
NXY $-$ 2 gives least-squares approximation.

For spline fitting with more generality, see Chapter~11.5.

\subsection{Usage}

\subsubsection{Program Prototype, Single Precision}
\begin{description}
\item[INTEGER]  \ {\bf NXY, NB, LDW, IERR1}

\item[REAL]  \ {\bf X}($\geq $NXY){\bf , Y}($\geq $NXY){\bf , SD}($\geq $NXY)%
{\bf , \nolinebreak B}($\geq $NB){\bf , W}(LDW,~5){\bf , YKNOT}($\geq $NB)%
{\bf , YPKNOT}($\geq $NB){\bf , SIGFAC}
\end{description}
Assign values to X(), Y(), SD(), NXY, B(), NB, and LDW.\vspace{-15pt}
\begin{center}
\fbox{\begin{tabular}{@{\bf }c}
CALL SC2FIT(X, Y, SD, NXY, B, NB, W,\\
LDW, YKNOT, YPKNOT, SIGFAC, IERR1)\\
\end{tabular}}
\end{center}
Computed quantities are returned in YKNOT(), YPKNOT(), SIGFAC, and IERR1.

Following the use of SC2FIT the user may use SHINT of Chapter~12.3 to compute
values of the fitted curve.

\subsubsection{Argument Definitions}
\begin{description}
\item[X(), Y()]  \ [in] Data pairs (X(I), Y(I), I = 1, ..., NXY). The
contents of X() must satisfy X($1)\leq $ X(2) $\leq $ ... $\leq $ X(NXY).

\item[SD()]  \ [in] If SD($1)>0.$, each SD(I) must be positive and must be
the user's a priori estimate of the standard deviation of the uncertainty $%
(e.g.$, observational error) in the corresponding data value Y(I).

If SD($1)<0.$, $|$SD(1)$|$ will be used as the a priori standard deviation
of each data value Y(I). In this case the array SD() may be dimensioned as
SD(1).

\item[NXY]  \ [in] Number of data points. Require NXY $\geq 4.$

\item[B()]  \ [in] Set by the user to specify the knot abscissae and
endpoints for the spline curve. Must satisfy B(1) $<$ B(2)
$<$ ... $<$ B(NB).  Also require B(1) $\leq $
X(1), B(NB) $\geq $ X(NXY).

\item[NB]  \ [in] Number of knots, including endpoints. The number of
segments in the spline curve will be NB $-$ 1. The number of degrees of freedom
for the fit will be NB + 2. Require $2\leq \text{NB}\leq \text{NXY}-2.$

\item[W(\ ,\ )]  \ [scratch] Working space, dimensioned W(LDW,~5).

\item[LDW]  \ [in] Leading dimension for the work array W(,). LDW must be at
least NB + 4, but the execution time is less for larger values of LDW, up to
NB$+3+k$, where $k$ is the largest number of data points lying between any
adjacent pair of knots.

\item[YKNOT(), YPKNOT()]  \ [out] Arrays, each of length at least NB, in
which the subroutine will store a definition of the fitted spline curve as a
sequence of values of the curve and its first derivative at the knot
abscissae B(i). Letting $f$ denote the fitted curve, the elements of these
arrays will be set to

YKNOT$(i)=f(\text{B}(i))$, $i=1$, ..., NB

YPKNOT$(i)=f^{\prime }(\text{B}(i))$, $i=1$, ..., NB

\item[SIGFAC]  \ [out] Set by the subroutine as a measure of the residual
error of the fit. See Section D.

\item[IERR1]  \ [out] Error status indicator. Set on the basis of tests done
in SC2FIT as well as error indicators IERR2 set by SBACC and IERR3 set by
SBSOL. Zero indicates no errors detected. See Section E for the meaning of
nonzero values.
\end{description}
\subsubsection{Modifications for Double Precision}

For double precision usage change the REAL statement to DOUBLE PRECISION and
change the subroutine name SC2FIT to DC2FIT.

\subsection{Examples and Remarks}

{\bf Example:} Given a set of 12~data pairs $(x_i,y_i)$ compute the
uniformly weighted least-squares cubic spline fit to these data using six
uniformly spaced breakpoints, including endpoints. After determining the
spline function $f(x)$, compute and tabulate the quantities $x_i$, $y_i$, $%
f(x_i)$, and $r_i = y_i - f(x_i).$

This computation is illustrated by the program DRDC2FIT and the output
ODDC2FIT. The fitted curve is determined using DC2FIT, and is evaluated
using DHINT of Chapter~12.3.

{\bf Interpolation:} If all of the data abscissae are distinct, and one wants
interpolation, rather than the smoothing effect of least-squares
approximation, one can choose interpolation by setting NB = NXY $-$ 2. The
NB knot abscissae can be assigned in various ways, but one reasonable way is
to set B(1) = X(1), B(i) = X$(i+1)$, for $i = 2$, ..., NB $-$ 1, and B(NB) =
X(NXY).

\subsection{Functional Description}

Let knot abscissae $b_1 < b_2 < ..$.  $< b_{NB}$ be given.  A cubic spline
function defined over the interval $[b_1$, $b_{NB}]$ is a cubic polynomial
in each subinterval $[b_i$, $b_{i+1}]$, $i = 1$, ..., NB $-$ 1, with
continuity of the value, first derivative, and second derivative at each
internal knot, $b_2$, ..., $b_{NB-1}$.  The set of all cubic spline
functions defined relative to this knot set is a linear space of dimension
$d =$ NB $+$ 2.  If the knot spacing does not depart severely from
uniformity a well conditioned set of basis functions for this space is
provided by a particular set of cubic spline functions called B-splines,
$p_i(x)$, $i = 1$, ..., $d$, each of which is nonzero over at most four
adjacent subintervals, \cite{deBoor:1972:OCB}.

The problem data are $\{(x_i$, $y_i$, $s_i)$, $i=1$, ..., NXY$\}$, where $s_i$
is the a priori standard deviation of the error in the value $y_i$. The
weighted least-squares curve fitting problem then becomes one of determining
coefficients $c_j$ to minimize%
\begin{equation*}
\rho ^2({\bf c})=\sum_{i=1}^{\text{NXY}}\left[ \frac{
y_i-\sum_{j=1}^dc_jp_j(x_i)}{s_i}\right] ^2
\end{equation*}
The matrix formulation of this least-squares problem involves a matrix
having a banded form in which at most four elements are nonzero in each row.
This least-squares problem is solved using the subroutines of Chapter~4.5.

This problem will have a unique set of solution coefficients, $c_j$, if NB
$\leq \text{NXY}-2$ and the positioning of the knots is such that there
exists an indexing of some set of NB + 2 of the distinct data abscissae,
$x_i$ (not necessarily the indexing used in the subprogram) such that
$b_{i-3} < x_i < b_{i+1}$ for $i = 1$, ..., NB$+2$.  Here $b_{-2}$,
$b_{-1}$, and $b_0$ denote fictitious knots to the left of $b_1$, and
$b_{NB+1}$, $b_{NB+2}$, and $b_{NB+3}$ denote fictitious knots to the
right of $b_{NB}$, see \cite{Rice:COC:1967}.  If the solution is not
unique, no solution will be given and an error code will be returned as
described in Section E.

After determining coefficients, $c_j$, SC2FIT uses subroutine STRC2C to
evaluate the value and first derivative of the fitted curve at the knots.
These quantities are returned to the user in the arrays YKNOT() and YPKNOT()
as the defining parameters of the fitted curve.

Subroutine SC2BAS is called by both SC2FIT and STRC2C to evaluate B-spline
basis functions.

\bibliography{math77}
\bibliographystyle{math77}

\subsection{Error Procedures and Restrictions}

SC2FIT sets IERR1 and issues error messages based on internal tests as well
as propagating error information set in IERR2 by SBACC and in IERR3 by
SBSOL. See Chapter~4.5 for the meaning of IERR2 and IERR3. In all cases in
which IERR1 is set nonzero, no solution will be computed.

\begin{tabular}{ll}
{\bf IERR1} & {\bf \quad \quad Meaning}\\

\phantom{100}0 & No errors detected.\\

\phantom{1}100 & NB $< 2$ or NXY $<$ NB + 2.\\

\phantom{1}200 & B(I$) \geq $ B(I+1) for some I.\\

\phantom{1}300 & LDW $<$ NB + 4.\\

\phantom{1}400 & X(I $- 1) > $ X(I) for some I.\\

\phantom{1}500 & B($1) > $ X(1) or B(NB) $<$ X(NXY).\\

\phantom{1}600 & Need larger dimension LDW.\\

\phantom{1}700+IERR2 & SBACC set IERR2 $\neq 0$.\\

\phantom{1}800+IERR2 & SBACC set IERR2 $\neq 0$.\\

\phantom{1}900+IERR2 & SBACC set IERR2 $\neq 0$.\\

 1000+IERR3 & SBSOL set IERR3 $\neq 0$. Indicates\\
 & singularity.\\

 1100 & SD(1) = 0.0.\\

 1200 & SD(1) $>$ 0.0, and SD($i) \leq 0.0$ for\\
 & some $i \in $ {[2, NXY]}.
\end{tabular}
\subsection{Supporting Information}

\begin{tabular}{@{\bf}l@{\hspace{5pt}}l}
\bf Entry & \hspace{.35in} {\bf Required Files}\vspace{2pt} \\
DC2FIT & \parbox[t]{2.7in}{\hyphenpenalty10000 \raggedright
 AMACH, DBACC, DBSOL, DC2BAS, DC2FIT, DERM1, DERV1, DHTCC,
 DNRM2, DTRC2C, ERFIN, ERMSG, IERM1, IERV1\rule[-5pt]{0pt}{8pt}}\\
SC2FIT & \parbox[t]{2.7in}{\hyphenpenalty10000 \raggedright
 AMACH, ERFIN, ERMSG, IERM1, IERV1, SBACC, SBSOL, SC2BAS, SC2FIT,
 SERM1, SERV1, SHTCC, SNRM2, STRC2C}\\\end{tabular}

Original code designed by C. L. Lawson and R. J. Hanson, JPL, 1968.
Programmed and modified by Lawson, Hanson, T. Lang, and D. Campbell, JPL,
1968--1974. Adapted to Fortran~77 by Lawson and S. Y. Chiu, July~1987.

\begcodenp
\lstset{language=[77]Fortran,showstringspaces=false}
\lstset{xleftmargin=.8in}

\centerline{\bf \large DRDC2FIT}\vspace{10pt}
\lstinputlisting{\codeloc{dc2fit}}
\newpage

\vspace{30pt}\centerline{\bf \large ODDC2FIT}\vspace{10pt}
\lstset{language={}}
\lstinputlisting{\outputloc{dc2fit}}
\end{document}
