\documentclass[twoside]{MATH77}
\usepackage{multicol}
\usepackage[fleqn,reqno,centertags]{amsmath}
\begin{document}
\begmath 19.7 Checking and Output of Program Unit Interfaces

\silentfootnote{$^\copyright$1997 Calif. Inst. of Technology, \thisyear \ Math \`a la Carte, Inc.}

\subsection{Purpose}

The program, {\tt gloch1} checks a list of programs for Fortran syntax
errors and writes two files, {\tt glsave} and {\tt gltemp}.  The program
{\tt gloch2} reads these files, uses the data to check global interfaces,
and prints various types of cross reference information.  In particular,
it has been used to check the MATH77 library, and to obtain the data for
Appendices A and B, and for the index.  It will give diagnostics when the
declaration for and/or the calls to an entry have arguments that are
inconsistent in type or number.  It does the same for common blocks, and
in addition will flag variables in common blocks that are not both defined
and referenced someplace in one of the program units declaring the common
block.  Finally the program {\tt libchk} can be used to check a collection
of codes for the entries in a library that are used in the collection.

\subsection{Usage}
\subsubsection{The first phase -- {\tt gloch1}}

Input for {\tt gloch1} is from the file {\tt glnames} to specify the files
to be examined and optionally to specify the kinds of diagnostics desired.
The file names are given one per line and must not contain embedded
blanks, but otherwise can be quite arbitrary.  For purposes of later
output, the name reported for a file is the characters after the last
``:'', ``/'', or ``$\backslash $'' (if none of these it starts at the
start of the line), and before either a ``.'' or a blank.  If this name
has more than 8 characters, it is truncated to 8.

If the first character of the line is blank, the remaining characters
specify actions desired as indicated below.  This line would ordinarily be
the first line, and if it is not given, the program starts with defaults
as if the first line contained `` Lfeip''.  A lower case letter turns off
an action and an upper case letter turns it on.  The default is used if a
letter is not present.

\begin{description}
\item[L] Local diagnostics (should be caught by your compiler).
\item[F] Diagnose the use of Fortran 90 comments, and the Fortran 90
``implicit none''.
\item[I] Diagnose specific Intrinsic functions that could be replaced by
a generic, intrinsics in type statements, and intrinsics with more than
two arguments (to find those that cause minor problems in automated
conversion to C).
\item[P] Diagnose unreferenced Parameters.
\item[E] Write source lines containing Entry information to the file {\tt
glents} as they are encountered.
\item[G] (No ``g'' option.) This initializes internal tables from the file
{\tt glsave} generated from an earlier run of {\tt gloch1} and then
modified by running {\tt gloch2}.  Thus for example the interfaces for
programs using a collection of library codes can be checked by first
analyzing the library codes and saving the file {\tt glsave}.  Then
copying this file to {\tt glsave} when processing the programs in
question.
\end{description}

\subsubsection{The second phase -- {\tt gloch2}}
After running {\tt gloch1} one runs {\tt gloch2} to complete the
processing of the global data, thus getting the final version of the file
{\tt glsave}.  One then has other options available by providing input
as described below.

In the description below, the punctuation ``\{\}$[\,]$()$|$'' is used to
describe the input, and is not meant to be typed by the user. The case
of letters is not significant.  To define the commands read from the
standard input, let:

\begin{description}
\item[$\{xyz\}$] Denotes any combination of the letters ``$x$'',
``$y$'', and ``$z$'', with no letter used more than once.
\item[{$[\cdots ]$}] Whatever is inside the square brackets is
optional.
\item[($x|y$)] Indicates that either what is given as $x$ or what is
given as $y$ can be used.
\item[{\em file}] The name of a file (no
embedded blanks).
\item[{\em int}]  An integer.
\item[{\em lead}] Text of the form
  [{\em int}\,L][{\em int}\,I][{\em file}]''.  No blanks are allowed in
this string.  The integer before the ``L'' is used to specify the line
width (default is 79), the integer before the ``I'' is used to specify the
amount of indentation when indentation is used in the output (default is
3), and {\em file} is used to specify the file to which the output is
written until the next time {\em file} is used. (The default initially is
to use {\tt gldiagg} for output if global diagnostics are requested, and
if no global diagnostics are requested to use {\tt glstuff}.)
\item[!] A ``!'' anywhere on the line, will cause everything from that
point to the end of the line to be treated as a comment.  If the ``!'' is
in column one, the rest of the line is written to the current output file.
\end{description}\vspace{5pt}

The commands that may be read from the standard input are:\vspace{-10pt}
\begin{description}
\item[?] Gives general help.  ``?x'' gives summary help on command x.
\item[{[{\em lead}]} D] [A]\\
This is used to generate diagnostics on the global interfaces.  If the A
is present, interface information is output for all entries and common
blocks, otherwise only those with suspected errors are output.

\item[{[{\em lead}]} E] [[$-$]F][([$-$]L$|$[$-$]R)][[$-$]N]\\
Specifies how entry names are to be sorted. The ``$-$'' is used to
indicate that the order for the key following is to be reversed from the
usual order.  If a sort is not uniquely determined by the keys specified,
the sort uses an alphabetic sort on the name of the last key. The keys,
which must not be separated by blanks, are interpreted as follows.
\begin{description}
\item[F] use the order of the File containing the entry.  Then sort
entries by their position in the file.
\item[L] sort those with smallest maximum distance to a Leaf first.
(A leaf is an entry that calls no other entry.)
\item[R] sort those with smallest maximum distance to a Root first.
(A root is an entry that is called by no other entry.)
\item[N] sort on the Name of the entry.
\end{description}

\item[{[{\em lead}]} F] [([$-$]L$|$[$-$]R)][[$-$]N]\\
Specifies how file names are to be sorted.  Works similarly to the entry
sort, except there is no F key, the N key is used to indicate the name of
the file, a leaf is defined as a file with no calls to entries that are
not in the file, and a root as a file that contains no entries called from
other files.
\item[{[{\em lead}]} L] \{CEFU\}\\
  List names in the order given by the letters present. Letters are
interpreted as follows.
\begin{description}
\item[C]  Common block names
\item[E]  Entry names
\item[F]  File names
\item[U]  Undefined entries, i.e. those entries referenced, but which are
not contained in the files named in {\tt glnames}.

\end{description}

\item[{[{\em lead}]} C] (e$|$f) (c$|$e$|$f)\,[{\em tag}]\\
Gives a cross reference,  The ``c, e, and f'' are used for common blocks,
entries, and files respectively.  For every entry (or file) as indicated
in the first argument, every common block, entry, or file (as indicated
by the second argument), directly referenced is listed.  The optional tag
consists of [$(\ .\ |+|-|\ *\ )$][\#], which alters the output as follows.
\begin{description}
\item[$.$] Minimal list. Only use for C e f.\, or C f e.\; the former
lists the file containing each entry, the latter lists all of the entries
in each file.
\item[$+$] Recur with a list.  Thus for C e f$+$, the list for each entry
includes not just the files containing the entries called directly by the
entry, but also includes any files that these entries in turn require,
etc.
\item[$-$] As for ``$+$'', but each item goes on a separate line.
\item[$*$] As for ``$-$'', but repeat the full list every time.
\item[\#] Only for ``C E E'' or ``C F F'', include numbers with names
listed in the form number.name where the numbers start with 1 for the
first item listed.
\end{description}
\item[{[{\em lead}]} W] (c$|$e$|$f) (e$|$f)\,[{\em tag}]\\
Gives a ``who calls''.  As for the C command, but for each item specified
by the first argument, list those indicated by the second argument that
use it.
\item[Q] Quits the program.
\end{description}

\subsubsection{Checking the use of entries in a library -- {\tt libchk}}

If one has a collection of programs that use a library of routines, {\em
e.g.}\ MATH77, the program {\tt libchk} can be used to create a report of
how the collection of programs uses the library.  This is done as follows.

Using the file {\tt glsave} generated from running {\tt gloch1} on the
library in the current directory:
\begin{tabbing}
\hspace{.5in}\=gloch2\\
\>liblst C e f.\\
\>q
\end{tabbing}
(Note the ``.'' at the end of the second line.) Then using the file {\tt
glsave} generated from running {\tt gloch1} on the collection of programs
of interest in the current directory:
\begin{tabbing}
\hspace{.5in}\=gloch2\\
\>uselst C f e\\
\>q
\end{tabbing}

Then run the program {\tt libchk} with both the file {\tt liblst} and {\tt
uselst} generated above in the current directory.  The output in the file
{\tt uselib} contains the name of files in lower case starting in column 1
that reference entries in {\tt liblst}.  The following columns, in upper
case, contain the entries in {\tt liblst} that are referenced.  Following
this is a list of library entries that are used, with the entry given
first, then the file the entry is in, and then a count of the number of
times this entry is referenced in {\tt uselst}.

\subsection{Examples and Remarks}
The following three lines input to {\tt gloch2}

C e f+\\
L e\\
W c f

give Appendix A, the first part of Appendix B, and the last part of
Appendix B.

The file {\tt gldiagg} generated by running the subprograms in MATH77
through {\tt gloch1} and {\tt gloch2} is given at the end of this chapter.
None of these diagnostics represent errors; most of the common block items
flagged are due to the use of equivalence of arrays with scalars of the
same type.

\subsection{Functional Description}

The table at the beginning of the output given at the end of this chapter
gives a good idea of the kind of reporting provided by this software.

An argument list or common block is represented by a string of the letters
given in this table, with each letter in the string representing the type
of the corresponding argument or common block entry.  The first letter for
an argument list indicates the type of external reference or declaration.

Any common variable that uses a letter other than those in columns 4 and
8 is flagged in the output as possibly the source of a problem.  A letter
from column 1 or 4 is flagged as a potential problem in the case of
declarations, {\em i.e.}\ the formal arguments that are part of an entry
declaration.  Arguments for references are flagged, if they are
incompatible with the declaration, or if there is no declaration, if they
are incompatible with the first reference.

The program uses an {\em ad hoc} lexer and parser (if you can call it
that) to analyze just as much of the code as necessary to check argument
types and the use of common variables.  It assumes that any variable
mentioned in an equivalence statement is both referenced and defined.  It
also assumes that the user has intended to declare all common blocks in
the same way wherever they appear.  It makes no attempt to track the size
or the number of dimensions of arrays.  Dummy (formal) arguments which are
not defined, but are used as actual arguments, are traced down the call
tree, and if a routine required in order to know what happens to that
argument is not available it is flagged as unknown, i.e.  columns 4 and 8.

As long as there is a legal way to combine argument types, an additional
argument list specification is not constructed.  Thus, for example, if one
call passes a variable, and another passes a literal (a constant or
expression) of the same type in the same position, this position will be
tagged as a reference with a literal.  In the same way an array element
actual argument may get combined with a scalar and called a scalar
argument, or combined with an array and called an array argument.

Some MATH77 routines equivalence an array to a scalar in common to
simplify the setting of options or printing of output.  The implied
equivalence between the variables in common with the array is not
recognized by {\tt gloch2}, and thus it may flag such variables as
potential problems.  If an actual argument is in common and is passed to a
routine containing the same common block a warning is given.

Files used by this software are listed below.\vspace{-5pt}
\begin{description}
\item[{\tt glnames}] Contains list of files to process, and perhaps
options.
\item[{\tt gldiagl}] Local diagnostics are written to this file.
\item[{\tt glents}] If requested, lines containing the source text
associated with entries are written to this file.
\item[{\tt glsave}] Generated by {\tt gloch1} and modified by {\tt
gloch2}, this contains all the information about the global interfaces
and is required by either of these programs if access to this previously
generated information is desired.
\item[{\tt gltemp}] Generated by {\tt gloch1}, read and then deleted
by {\tt gloch2}.  This contains information about calls that contain a
dummy or common block variable.
\item[{\tt gldiagg}] Global diagnostics are written to this file
by {\tt gloch2} unless some other file has been requested in the input.
\item[{\tt glstuff}] Other output is written to this file by {\tt gloch2}
unless some other file has been requested in the input.
\item[\tt liblst] Used by {\tt libchk} as the definition of entries and
associated file names for the library.
\item[\tt uselst] Used by {\tt libchk} as the definition of entries
associated with the files contained in a collection.
\item[\tt libuse] The file containing the data generated by {\tt libchk}
\end{description}

In addition, {\tt gloch2} may use one or two scratch files while
processing the global information.

\subsection{Error Procedures and Restrictions}

There are numerous {\tt STOP} statements in the programs, all of which
print a message.  Some of these indicate that more table space is
required, and print the name of a parameter in the program which needs a
larger value.  One can change the parameter everywhere it appears,
recompile and try again.  Other cases are probably the result of bugs, and
would require changes to correct the problem.

Except for character tokens, an end of line is assumed to end a token, and
the type of a statement must be defined by the first line of a statement.
Any name with more than 8 characters is truncated to a name with 8
characters.

The user is cautioned that this code has not been used extensively.  The
only claim made is that it has done the proper thing when checking the
MATH77 library.  In particular, no effort has been made to support
extensions to Fortran 77 that are popular in some computing environments.

\subsection{Supporting Information}

The source language is ANSI Fortran~77.

Design and code due to F. T. Krogh, April 1996.  The {\tt libchk} program
added and miscellaneous corrections to {\tt gloch1}, April 1997.

\begin{tabular}{@{\bf}l@{\hspace{5pt}}l}
\bf Program & \hspace{.35in} {\bf Required to Link}\vspace{2pt}\\
GLOCH1 & \parbox[t]{2.7in}{\hyphenpenalty10000 \hspace{.25in}
GLOCH1 \rule[-5pt]{0pt}{8pt}}\\
GLOCH2 & \parbox[t]{2.7in}{\hyphenpenalty10000 \hspace{.25in}
GLOCH2, ISORT, INSORT \rule[-5pt]{0pt}{8pt}}\\
LIBCHK & \parbox[t]{2.7in}{\hyphenpenalty10000 \hspace{.25in}
LIBCHK \rule[-5pt]{0pt}{8pt}}\\
\end{tabular}


\begcode
\vspace{20pt}

\centerline{\bf \large The Output File {\tt gldiagg} from Processing
MATH77}\vspace{10pt}
\lstset{xleftmargin=.8in}
\lstset{language={}}
\lstinputlisting{\extrasloc{gldiagg.fl}}
\end{document}
