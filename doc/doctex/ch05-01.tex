\documentclass[twoside]{MATH77}
\usepackage{multicol}
\usepackage[fleqn,reqno,centertags]{amsmath}
\begin{document}
\begmath 5.1 Eigenvalues and Eigenvectors of a Symmetric Matrix

\silentfootnote{$^\copyright$1997 Calif. Inst. of Technology, \thisyear \ Math \`a la Carte, Inc.}

\subsection{Purpose}

Compute the N eigenvalues and eigenvectors of an N $\times $ N symmetric matrix $A$.

\subsection{Usage}

\subsubsection{Program Prototype, Single Precision}

\begin{description}
\item[REAL]  \ {\bf A}(LDA,$\geq $N) [LDA $\geq $ N]{\bf , EVAL}($\geq N)$%
{\bf ,\newline
WORK}($\geq N)$

\item[INTEGER]  \ {\bf LDA, N, IERR}
\end{description}

Assign values to A(,), LDA, and N.

\begin{center}
\fbox{\begin{tabular}{@{\bf }c}
CALL SSYMQL(A, LDA, N, EVAL,\\
WORK, IERR)\\
\end{tabular}}
\end{center}

Results are returned in A(,) and EVAL().

\subsubsection{Argument Definitions}

\begin{description}
\item[A(,)]  \ [inout] On entry the locations on and below the diagonal of
this array must contain the lower-triangular elements of the N $\times $ N
symmetric matrix $A$. On return the eigenvectors of $A$ will be stored as
column vectors in the array A(,). These N eigenvectors will be mutually
orthogonal and of unit Euclidean length. The eigenvector stored in column
J will be associated with the eigenvalue stored in EVAL(J).

\item[LDA]  \ [in] Dimension of the first subscript of the array A(,). Require
LDA $\geq $ N.

\item[N]  \ [in] Order of the symmetric matrix $A$. N $\geq 1.$

\item[EVAL()]  \ [out] Array in which the N eigenvalues of $A$ will be stored
by the subroutine. The eigenvalues will be sorted with the algebraically
smallest eigenvalue first.

\item[WORK()]  \ [scratch] An array of at least N locations used as
temporary space.

\item[IERR]  \ [out] On exit this is set to~0 if the QL algorithm converges,
otherwise see Section E.
\end{description}

\subsubsection{Modifications for Double Precision}

Change SSYMQL to DSYMQL, and the REAL type statement to DOUBLE PRECISION.

\subsection{Examples and Remarks}

The following symmetric matrix $A$ is
given on page~55 of~\cite{Gregory:1969:ACM}.
\begin{equation*}
A=\left[
\begin{array}{rrrr}
5 & 4 & 1 & 1 \\
4 & 5 & 1 & 1 \\
1 & 1 & 4 & 2 \\
1 & 1 & 2 & 4
\end{array}
\right]
\end{equation*}
The eigenvalues of this matrix are 1, 2, 5, and 10. Unnormalized
eigenvectors associated with these eigenvalues are (1, -1, 0, 0),
(0, 0, -1, 1), (-1, -1, 2, 2), and (2, 2, 1, 1), respectively.

The code in DRSSYMQL, given below, computes the eigenvalues and eigenvectors
of this matrix. Output from this program is given in the file ODSSYMQL.

Before the call to SSYMQL, the matrix is saved in an array ASAV() in order
to compute the relative residual matrix $D$ defined as%
\begin{equation*}
D=\left( AW-W\Lambda \right) /\gamma
\end{equation*}
where $W$ is the matrix whose columns are the computed eigenvectors
of $A$, $\Lambda $ is the diagonal matrix of eigenvalues,
and $\gamma $ is the maximum-row-sum norm of $A$.

Recall that if ${\bf v}$ is an eigenvector, then so is $\alpha {\bf
v}$ for any nonzero scalar $\alpha$.  More generally, if an
eigenvalue, $\lambda$, of a symmetric matrix occurs with multiplicity
$k$, there will be an associated  $k$-dimensional subspace in which
every vector is an eigenvector for $\lambda$.  This subroutine will
return eigenvectors constituting an orthogonal basis for such an
eigenspace.

\subsection{Functional Description}

The implicit-shift QL algorithm implemented in this subroutine is based on
the Algol procedure given in \cite{Dubrelle:1968:IQR}, pp.~337--383.  The
code combines slightly modified EISPACK routines TRED2, and IMTQL2, see
\cite{Smith:1974:MER}.  Modifications made are minor changes to convert
the code to take advantage of Fortran~77; they should not affect results.
TRED2 uses Householder orthogonal similarity transformations to transform
the matrix $A$ to tridiagonal form.  IMTQL2 uses the QL algorithm with
implicit shifts to reduce the off-diagonal elements of the tridiagonal
matrix to a magnitude of approximately the last bit of the largest element
of $A$.

The resulting diagonal elements are the eigenvalues of $A$. The matrix of
eigenvectors is computed as the product of the orthogonal transformation
matrices used in transforming $A$ first to tridiagonal form and then to
(almost) diagonal form.

The eigenvalues are sorted in nondecreasing algebraic order and the
eigenvectors are permuted as necessary to correspond to the ordered
eigenvalues.

\bibliography{math77}
\bibliographystyle{math77}

\subsection{Error Procedures and Restrictions}

If the QL algorithm fails to converge in 30~iterations on the $J^{th}$
eigenvalue the subroutine sets IERR $= J.$ In this case $J-1$ eigenvalues
and eigenvectors are computed correctly but the eigenvalues are not ordered.
If N $\leq $ 0 on entry, IERR is set to $-$1.  In either case an error
message is printed using IERM1 of Chapter 19.2 with an error level of 0,
before the return.

\subsection{Supporting Information}

The source language is ANSI Fortran~77.

\begin{tabular}{@{\bf}l@{\hspace{5pt}}l}
\bf Entry & \hspace{.35in} {\bf Required Files}\vspace{2pt} \\
DSYMQL & \parbox[t]{2.7in}{\hyphenpenalty10000 \raggedright
DIMQL, DSYMQL, ERFIN, ERMSG, IERM1, IERV1\rule[-5pt]{0pt}{8pt}}\\
SSYMQL & \parbox[t]{2.7in}{\hyphenpenalty10000 \raggedright
ERFIN, ERMSG, IERM1, IERV1, SIMQL, SSYMQL}\\
\end{tabular}

Converted by: F. T. Krogh, JPL, October~1991.


\begcode

\medskip\
\lstset{language=[77]Fortran,showstringspaces=false}
\lstset{xleftmargin=.8in}

\centerline{\bf \large DRSSYMQL}\vspace{10pt}
\lstinputlisting{\codeloc{ssymql}}

\vspace{30pt}\centerline{\bf \large ODSSYMQL}\vspace{10pt}
\lstset{language={}}
\lstinputlisting{\outputloc{ssymql}}

\end{document}
