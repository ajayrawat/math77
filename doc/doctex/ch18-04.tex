\documentclass[twoside]{MATH77}
\usepackage{multicol}
\usepackage[fleqn,reqno,centertags]{amsmath}
\begin{document}
\begmath 18.4  Sorting Data Sets too Large to Fit in Memory

\silentfootnote{$^\copyright$1997 Calif. Inst. of Technology, \thisyear \ Math \`a la Carte, Inc.}

\subsection{Purpose}

Sort a set of data that is too large to be contained in main memory. The
structure of the records, and the criteria for ordering them, are determined
by the calling program.

\subsection{Usage}

\subsubsection{Program Prototype}

\begin{description}

\item[INTEGER] \ {\bf N, L}($\geq $N){\bf , OPTION, OUTFIL}

\item[EXTERNAL] \ {\bf DATAOP}

\end{description}

Assign values to N and OPTION. Require N $\geq 4.$
\begin{center}
\fbox{\begin{tabular}{@{\bf }c}
CALL EXSORT (DATAOP, N, L,\\
OPTION, OUTFIL)
\end{tabular}}
\end{center}
\subsubsection{Argument Definitions}

\begin{description}

\item[DATAOP] \ [in] A SUBROUTINE subprogram that is used to perform operations on
the data. DATAOP is referenced by CALL DATAOP (IOP, I1, I2, IFLAG), where
all arguments are integer scalars. IOP defines the action to perform, I1 and
I2 provide parameters necessary to those actions, and IFLAG returns a
status. When I1 or I2 index a record area, we have 1\ $\leq $ I1,
I2 $\leq $ N.

The subroutine DATAOP must have capacity to store at least $\text{N} \geq 4$
records, and access to four external
sequential files, such as tape or disk files. Using the argument IOP, EXSORT
requests DATAOP to execute operations such as comparing records in the
internal region, and moving records between the internal region and
specified external files. These actions are defined below.
\begin{itemize}
\item[\bf IOP] {\hspace{.5in} \bf ACTION}

\item[1\ ] Place a datum from the set to be sorted into the record area indexed by
I2. I1 is irrelevant. Set the value of IFLAG to zero if a datum is
available, or to any nonzero value if there are no more data in the set. If
there are no more data in the set, do not modify any record areas
since EXSORT assumes they are intact, and may ask DATAOP to output from them
using IOP $= 6.$

\item[2\ ] Write the datum in the record area indexed by I2 onto the intermediate
file indexed by I1, 1 $\leq $ I1 $\leq $ 4. The value of
IFLAG is irrelevant.

\item[3\ ] Place an end-of-sequence mark to be recognized during performance of
operation~4 below onto the intermediate file indexed by I1. The values of I2
and IFLAG are irrelevant.

\item[4\ ] Read a datum from the intermediate file indexed by I1 into the record area
indexed by I2. Set the value of IFLAG to any nonzero value if the
end-of-sequence mark written by operation~3 is detected, else set the value
of IFLAG to zero.  If the end-of-sequence mark is detected, do not
modify any record areas.

\item[5\ ] Rewind the intermediate file indexed by I1. The values of I2 and IFLAG are
irrelevant. This operation is the first operation performed on the
intermediate files; thus, they should be opened here if they are not already
opened.

\item[6\ ] If I1 is zero the next datum from the sorted set is in the
record area indexed by I2.  At this point you can print the item, save it,
or process it in some other way.  If I1 is nonzero, all of the records of
the sorted set have previously been passed to DATAOP with IOP $= 6$ and I1
= zero.  The last action performed by EXSORT is to invoke DATAOP with IOP
= 6 and I1 $\neq $ zero.

\item[7\ ] Move the datum in the record area indexed by I1 to the record area indexed
by I2. The value of IFLAG is irrelevant.

\item[8\ ] Compare the datum in the record area indexed by I1 to the datum in the
record area indexed by I2. If the datum in the record area indexed by I1 is
to appear in the sorted sequence before the datum in the record area indexed
by I2, set the value of IFLAG to $-$1 or any negative integer; if the order of the
records in the areas indexed by I1 and I2 is immaterial, set the value of
IFLAG to zero; if the record in the area indexed by I1 is to appear in the
sorted sequence after the datum in the record area indexed by I2, set the
value of IFLAG to +1 or any positive integer.
\end{itemize}
EXSORT does not have access to the data.  Since DATAOP is a dummy
procedure, it may have any name.  Its name must appear in an EXTERNAL
statement in the calling program unit.
\item[N] \ [in] The maximum number of records DATAOP can hold in its internal
space. When the I1 or I2 argument of DATAOP indexes a record, its value will
be in the range from~1 to~N. We must have N $\geq 4.$  Larger values
for N will tend to give faster sorts.

\item[L()] \ [scratch] An array of length at least N.

\item[OPTION] \ [in] The first stage of sorting a large volume of data consists of
sorting parts of it that are as large as can be contained in memory. These
parts are then written onto intermediate files. If the first record of a
part that has been sorted in main memory could appear in the sorted order
after the last record that has been written onto an intermediate file, the
newly sorted part will be concatenated onto that sequence. Thus it may
happen that all of the data are contained in a single sorted sequence on one
intermediate file. This event is called fortuitous completion.

If fortuitous completion occurs and OPTION = 0, EXSORT will set OUTFIL to
the index of the intermediate file on which the completely sorted sequence
was discovered (1 $\leq $ OUTFIL $\leq $ 4). If fortuitous
completion does not occur and OPTION = 0, EXSORT will set OUTFIL = 0 and
return the sorted records by calling DATAOP with IOP = 6. If OPTION is not
zero EXSORT will not alter OUTFIL, and the sorted records are always
returned by calling DATAOP with IOP = 6, even if fortuitous completion
occurs.

To take advantage of fortuitous completion to prevent needless copying of
the sorted data, set OPTION = 0, arrange DATAOP to write data on the
intermediate files in the same format as they are written on the final
output file (when IOP = 6), and allow the consumer of the sorted data to
read them from a file specified by a variable that is computed depending on
the value of OUTFIL.

\item[OUTFIL] \ [out] See OPTION.

\end{description}

\subsection{Examples and Remarks}

The program DREXSORT illustrates the use of EXSORT to sort 10000 randomly
generated real numbers. The output should consist of the single line

``EXSORT succeeded using $n$ compares", where $n$ is about~126000.

\subparagraph{Stability}

A sorting method is said to be $stable$ if the original relative order of
equal elements is preserved.  This subroutine uses a merge sort algorithm,
which is not inherently
stable. To impose stability, add a sequence number to each record, and
include the sequence number as the least significant ordering criterion when
DATAOP is called with IOP = 8.

\subsection{Functional Description}

The basic strategy of external sorting is to sort as much of the data as
possible in main memory. If the data set is larger than can be contained in
main memory, write a sorted subset onto a file. Continue writing sorted
subsets until the entire set has been sorted into blocks that are ordered,
but elements in different blocks might not be ordered. Then merge the sorted
blocks until one sorted sequence is produced. There are several strategies
for each of these steps.

The number of passes necessary to merge the sorted blocks is
proportional to the logarithm of the number of blocks.  Since data
transfer using files is slower than sorting in memory, one should
minimize the number of blocks by making them as large as possible.

Data are sorted using calls to INSRTX which has the same functionality and
calling sequence as INSORT (Chapter~18.3), but the COMPAR subprogram has a
different specification.  It is a SUBROUTINE with four arguments, IOP, I,
J and IFLAG as described for DATAOP above.  IOP is 8, and I, J and IFLAG
are used as described for DATAOP above.

Sorted data are distributed onto two intermediate files, with the file
that receives a datum changed whenever the datum to be put onto a file
would be less than the preceding datum.  If the first datum of a sorted
subset is less than any of the last elements already recorded on
intermediate files, but the last is greater, the sorted subset is split,
with as much as possible concatenated onto one of the sequences under
construction on an intermediate file.  As a result the number of ordered
sequences on the two intermediate files is different by at most one, but
there is no bound (other than the size of the data set) on the difference
between the numbers of records on the intermediate files.  It is also
possible that only one sorted sequence will be produced, although the data
could not all be contained in memory.  See the explanation of the argument
OPTION for a discussion of this possibility.

After the data have been sorted into blocks, and distributed onto the two
intermediate files, the sequences on the intermediate files are merged
repeatedly until only one sequence remains. When more than four sequences
remain, the result of merging sequences is written to another intermediate
file. When four or fewer sequences remain, the result of merging sequences
is passed to DATAOP with IOP = 6. During the merge process, there will
generally be two sequences being merged. But when an odd sequence remains at
the end of a pass it will be merged with the first two sequences that were
produced in that pass, using a three-way merge, before the remaining
sequences produced in that pass are merged using two-way merges.

\subsection{Error Procedures and Restrictions}

EXSORT neither detects nor reports error conditions. The only limitations
are those imposed by the capacity of external storage, and those imposed by
INSRTX (see Section~18.3.E).

\subsection{Supporting Information}

The source language is ANSI Fortran 77.

Designed and coded by W. V. Snyder, JPL 1974. Adapted for MATH77,
February~1990.

\begin{tabular}{@{\bf}l@{\hspace{5pt}}l}
\bf Entry & \hspace{.35in} {\bf Required Files}\vspace{2pt} \\
EXSORT & \quad EXSORT, INSTRX, PVEC\\\end{tabular}



\begcode
\bigskip
\lstset{language=[77]Fortran,showstringspaces=false}
\lstset{xleftmargin=.8in}

\centerline{\bf \large DREXSORT}\vspace{10pt}
\lstinputlisting{\codeloc{exsort}}
\end{document}

