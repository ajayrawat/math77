\documentclass[twoside]{MATH77}
\usepackage{multicol}
\usepackage[fleqn,reqno,centertags]{amsmath}
\begin{document}
\begmath 4.2 Linear Least-Squares and Covariance Matrix

\silentfootnote{$^\copyright$1997 Calif. Inst. of Technology, \thisyear \ Math \`a la Carte, Inc.}

\subsection{Purpose}

The principal expected application of this set of subroutines is the
solution of the linear least-squares problem, $A{\bf x} \simeq {\bf b}$,
where $A$ is an $m \times n$ matrix with $m > n$, and ${\bf b}$ is an $m$%
-vector. The solution vector, ${\bf x}$, is an $n$-vector. Computation of
the covariance matrix of the solution vector, ${\bf x}$, is also supported.

This software is not limited to the usual case of $m > n$ and $A$ being of
full rank. A pseudoinverse solution is provided for all of the cases, $m > n$%
, $m = n$, and $m < n$, including the case of $A$ being rank-deficient.
Furthermore, the right-side of the problem may be an $m \times k$ matrix, $B$,
in which case the solution will be an $n \times k$ matrix, $X$.

\subsection{Usage}

\subsubsection{Solving a Least-Squares Problem}

\paragraph{Program Prototype, Single Precision}

\begin{description}
\item[INTEGER]  \ {\bf LDA, M, N, LDB, NB, KRANK, IP}($\geq $N)

\item[REAL]  \ {\bf A}(LDA, $\geq $N){\bf , TAU, RNORM}($\geq $NB){\bf ,\\
WORK}($\geq $N){\bf , B}(LDB) or {\bf B}(LDB, $\geq $ NB)
\end{description}

Assign values to A(,), LDA, M, N, B(,), LDB, NB, and TAU.

\begin{center}
\fbox{\begin{tabular}{@{\bf }c}
CALL SHFTI(A, LDA, M, N, B, LDB, NB,\\
TAU, KRANK, RNORM, WORK, IP)\\
\end{tabular}}
\end{center}

Computed quantities are returned in A(,), B(,), KRANK, RNORM(), and IP().

\paragraph{Argument Definitions}

\begin{description}
\item[A(,)]  \ [inout] On entry contains the M$\times $N matrix, $A$, of the
least-squares problem. Permit M $>$ N, M = N, or M $<$ N. On return contains
an upper triangular matrix that can be used by subroutine, SCOV2, to compute
the covariance matrix.

\item[LDA]  \ [in] First dimensioning parameter for the array, A(,). Require
LDA $\geq $ M.

\item[M]  \ [in] Number of rows of data in A(,) and B(,) on entry. Require
M $\geq 1.$

\item[N]  \ [in] Number of columns of data in A(,) on entry. Require N $\geq
0.$

\item[B(,)]  \ [inout] May be a singly or doubly subscripted array. On entry
contains the right-side M-vector, ${\bf b}$, or (M$\times $NB)-matrix, $B$,
for the least-squares problem. On return contains the solution N-vector, $%
{\bf x}$, or (N$\times $NB)-matrix, $X$.

\item[LDB]  \ [in] First dimensioning parameter for the array B(,). Require
LDB $\geq \max (\text{M},\text{N})$ when NB $\geq 1$, and LDB $\geq 1$
when NB $=0.$

\item[NB]  \ [in] Number of right-side vectors for the least-squares
problem. Require NB $\geq 0$. If NB = 1, the array, B(,), may be either
singly or doubly subscripted. If NB $>1$, the array B(,) must be doubly
subscripted. If NB = 0, this subroutine will not access B(,).

\item[TAU]  \ [in] Absolute tolerance parameter provided by the user.
Ideally should indicate the noise level of the data in the given matrix, $A$.
The value, zero, is acceptable. Will be used in estimating the rank of $A$.
Larger values of TAU will lead to a smaller estimated rank for $A$.

\item[KRANK]  \ [out] Rank of $A$ estimated by the subroutine. Will be in the
range, $0\leq \text{KRANK}\leq \min (\text{M},\text{N}).$

\item[RNORM()]  \ [out] On return, RNORM(i) will contain the square root of
sum of squares of residuals for the $i^{th}$ right-side vector.

\item[WORK()]  \ [scratch] This array, of length at least N, is used
internally by the subroutine as working space.

\item[IP()]  \ [out] On return contains a record of column interchanges.
\end{description}

\subsubsection{Computing the Covariance Matrix}

Subroutine SCOV2 is designed to be used following SHFTI; but only in cases
in which KRANK determined by SHFTI is N.

\paragraph{Program Prototype, Single Precision}

\begin{description}
\item[INTEGER]  \ {\bf LDA, N, IP}($\geq $N){\bf , IERR}

\item[REAL]  \ {\bf A}(LDA, $\geq $N){\bf , VAR}
\end{description}

On entry the values in A(,), LDA, N, and IP() should be the same as on
return from a previous call to SHFTI. Also, assign a value to VAR.
$$
\fbox{{\bf CALL SCOV2(A, LDA, N, IP, VAR, IERR)}}
$$
Computed quantities are returned in A(,) and IERR.

\paragraph{Argument Definitions}

\begin{description}
\item[A(,)]  \ [inout] On entry contains an N$\times $N upper-triangular
matrix produced by the subroutine, SHFTI. On return contains the
upper-triangular part of the symmetric covariance matrix of the original
least-squares problem. Elements in A(,) below the diagonal will not be
referenced.

\item[LDA, N]  \ [in] Must be the same as in a previous call to SHFTI.

\item[IP()]  \ [in] Contains a record of column interchanges performed by a
previous call to SHFTI.

\item[VAR]  \ [in] Estimate of the variance of the data errors in the
original right-side vector, ${\bf b}$. To compute the covariance matrix for
the $i^{th}$ right-side vector of the original problem, the user's code can
compute VAR in terms of output quantities from SHFTI as
{\tt \begin{tabbing}
\hspace{.2in}\=DOF = M - N\\
\>STDDEV = RNORM($i$)/sqrt(DOF)\\
\>VAR = STDDEV**2
\end{tabbing}}
\item[IERR]  \ [out] Error flag. Zero indicates no error was detected. A
positive value indicates that A(IERR, IERR) was zero on entry. In this
latter case no result will be produced.
\end{description}

\subsubsection{Modifications for Double Precision}

Change the REAL type statements to DOUBLE PRECISION, and change the
subroutine names from SHFTI and SCOV2 to DHFTI and DCOV2, respectively.

\subsection{Examples and Remarks}

\subsubsection{A least-squares example}

Data for a sample linear least-squares problem were generated by computing
values of%
\begin{equation*}
y=0.5+0.25\ \sin (2\pi x)+0.125\ \exp (-x)
\end{equation*}
at eleven points, $x=0.0$, 0.1, ..., 1.0, and rounding the resulting $y$%
-values to 4~decimal places, thus introducing errors bounded in magnitude by
0.00005. The program, DRSHFTI, uses SHFTI to compute a least-squares fit to
this data using the model%
\begin{equation*}
c_1+c_2\sin (2\pi x)+c_3\exp (-x)
\end{equation*}
and uses SCOV2 to compute the covariance matrix for the computed
coefficients. Results are shown in the output file, ODSHFTI.

\subsubsection{Large problems}

If M $>>$ N and storage limitations make it awkward or impossible to
allocate M$\times $N locations for the array, A(,), one can use sequential
accumulation of the rows of data to produce a smaller matrix to which SHFTI
and SCOV2 can then be applied. See Chapter~4.4 for sequential accumulation.

\subsubsection{Underdetermined problems}

If M $<$ N, or, more generally, whenever Rank($A$) $<$ N, there will be
infinitely many vectors, ${\bf x}$, that achieve the same minimal residual
norm for the least-squares problem. For such cases SHFTI computes the
pseudoinverse solution, $i.e$. the solution vector, ${\bf x}$, of least
Euclidean norm.

\subsubsection{Computing the pseudoinverse matrix}

If one sets B(,) to be the M$\times$M identity matrix, the resulting
N$\times $M solution matrix, $X$, will be the pseudoinverse matrix of $A$.

\subsubsection{Reliability issues regarding KRANK}

KRANK must be understood as an estimate of the rank of $A$ that depends not
only on $A$, but on the user's setting of TAU and the particular algorithm
used in SHFTI. A small change in the value of TAU could result in a
different value being assigned to KRANK, which in turn could result in a
large change in the solution vector.

Although this subroutine will produce a solution whatever the value of
KRANK, the occurrence of KRANK $< \min (\text{M}$, N$)$ should be regarded
as exceptional. The user should investigate such instances as it could be
due to a programming error or a very ill-conditioned model. Singular Value
Analysis (see Chapter~4.3) can be useful in analyzing an ill-conditioned model.

\subsection{Functional Description}

This software is an adaptation to Fortran~77 of the subroutine, HFTI,
given and described in \cite{Lawson:1974:SLS}.  The name, HFTI, denotes
Householder Forward Transformation with column Interchanges.

To avoid nonessential complications, we shall describe the algorithm only
for the case of M $\geq $ N and NB = 1. A sequence of up to N Householder
orthogonal transformations is applied from the left, with column
interchanges, the total effect of which may be summarized by the equation%
\begin{equation*}
Q\left[ A:{\bf b}\right] \left[
\begin{array}{cc}
P & 0 \\
0 & 1
\end{array}
\right] =\left[
\begin{array}{cc}
R & {\bf g} \\ 0 & {\bf h}
\end{array}
\right]
\end{equation*}
where Q is the M$\times $M product of the Householder matrices, $P$ is an N$%
\times $N permutation matrix accounting for the column interchanges, $R$ is an
N$\times $N upper-triangular matrix with diagonal elements in order of
decreasing magnitudes, ${\bf g}$ is an N-vector, and ${\bf h}$ is an
(M $-$ N)-vector.

If all diagonal elements of $R$ exceed TAU in magnitude, the solution, ${\bf x}
$, is computed by solving $R{\bf x}={\bf g}$. The subroutine sets KRANK $=$
N and RNORM(1)\ $=\Vert {\bf h}\Vert .$

Alternatively, if the diagonal elements of $R$ beyond position $i$ are less
than TAU, the subroutine sets KRANK $=i$. Let $[R_1:{\bf g}_1]$ denote the
first KRANK rows of $[R:{\bf g}]$, and let ${\bf g}_2$ denote the last
N $-$ KRANK components of ${\bf g}$. The subroutine applies up to KRANK
Householder transformations to $R_1$ from the right, effecting the
transformation
\begin{equation*}
R_1K=[W:0]
\end{equation*}
where $K$ is the N$\times $N product of Householder transformations and $W$ is a
KRANK$\times $KRANK non-singular upper-triangular matrix. A KRANK-vector, $%
{\bf y}_1$, is computed by solving $W{\bf y}_1={\bf g}_1$. An N-vector, $%
{\bf y}$, is formed by appending zeros to the end of ${\bf y}_1$, and the
minimal length solution vector, ${\bf x}$, is computed as ${\bf x}=\text{PK}%
{\bf y}$. RNORM(1) is computed as the Euclidean norm of the (M $-$ KRANK)-vector
formed by concatenating ${\bf g}_2$ and ${\bf h}.$

Subroutine, SCOV2, is intended for use only when Rank($A$) = N.
Algorithm, COV, from \cite{Lawson:1974:SLS} is used.  The covariance
matrix is traditionally defined as $ C=\sigma ^2(A^tA)^{-1}$, where
$\sigma ^2$ is the variance of the data error.  Using the triangular
matrix, $R$, and the (orthogonal) permutation matrix, $P$, defined above,
one can also write $C=\sigma ^2(PR^tRP^t)^{-1}=\sigma ^2PR^{-1}R^{-t}P^t$,
where $R^{-t}$ denotes the transpose of $R^{-1}$.  Thus, SCOV2 computes%
\begin{align*}
E&=R^{-1} \\ F&=EE^t\\ C&=\sigma ^2PFP^t
\end{align*}


\bibliography{math77}
\bibliographystyle{math77}

\subsection{Error Procedures and Restrictions}

In SHFTI (or DHFTI) an error message will be issued and an immediate return
will be made, setting KRANK = 0, if any of the following conditions are
noted:%
\begin{equation*}
\text{M}<1,\text{ N}<0,\text{ LDA}<\text{M, or LDB}<\max (\text{M, N})
\end{equation*}
Most commonly, on return KRANK will have the value $\min (\text{M}$, N$)$.
A smaller value of KRANK may be valid, but unless the user has reason to
expect this possibility it is likely to be due to a usage error.

In SCOV2 (or DCOV2) the N diagonal elements of A(,) must be nonzero on
entry. If so, IERR is set to zero. If not, IERR is set to the index of the
first zero element, an error message will be issued, and a return will be
made with the computation being incomplete.

\subsection{Supporting Information}

The source language is ANSI Fortran~77.

\begin{tabular}{@{\bf}l@{\hspace{5pt}}l}
\bf Entry & \hspace{.35in} {\bf Required Files}\vspace{2pt} \\
DCOV2 & \parbox[t]{2.7in}{\hyphenpenalty10000 \raggedright
DCOV2, DDOT, DSWAP, ERFIN, ERMSG, IERM1, IERV1\rule[-5pt]{0pt}{8pt}}\\
DHFTI & \parbox[t]{2.7in}{\hyphenpenalty10000 \raggedright
AMACH, DAXPY, DDOT, DHFTI, DHTCC, DHTGEN, DNRM2, ERFIN, ERMOR, ERMSG, IERM1, IERV1\rule[-5pt]{0pt}{8pt}}\\
SCOV2 & \parbox[t]{2.7in}{\hyphenpenalty10000 \raggedright
ERFIN, ERMSG, IERM1, IERV1, SCOV2, SDOT, SSWAP\rule[-5pt]{0pt}{8pt}}\\
SHFTI & \parbox[t]{2.7in}{\hyphenpenalty10000 \raggedright
AMACH, ERFIN, ERMOR, ERMSG, IERM1, IERV1, SAXPY, SDOT, SHFTI, SHTCC, SHTGEN, SNRM2}\\\end{tabular}

Adapted to Fortran~77 from \cite{Lawson:1974:SLS} by C.  L.  Lawson and S.
Y.  Chiu, JPL, May~1986, June~1987.


\begcode
\bigskip
\lstset{language=[77]Fortran,showstringspaces=false}
\lstset{xleftmargin=.8in}

\centerline{\bf \large DRSHFTI}\vspace{10pt}
\lstinputlisting{\codeloc{shfti}}

\vspace{30pt}\centerline{\bf \large ODSHFTI}\vspace{10pt}
\lstset{language={}}
\lstinputlisting{\outputloc{shfti}}
\end{document}
