\documentclass[twoside]{MATH77}
\usepackage{multicol}
\usepackage[fleqn,reqno,centertags]{amsmath}
\begin{document}
\begmath 19.2  Error Message Processor

\silentfootnote{$^\copyright$1997 Calif. Inst. of Technology, \thisyear \ Math \`a la Carte, Inc.}

\subsection{Purpose}

This package of subroutines processes diagnostic messages, providing options
to write or not write the message, and to return or stop after message
processing. These subroutines are intended primarily for use by other
library subprograms although they are not restricted to that usage.

By passing error messages through these subroutines it is possible to (a)
avoid having Fortran I/O statements in other library subprograms that do
not otherwise execute any I/O functions, and (b) provide a means, via
the subroutine ERMSET, for a user program to alter the nominal action of
error message processing.

\subsection{Usage}

To process an error message the program detecting the error must make
a sequence of one or more calls to subroutines of this package. Such
a sequence of calls is distinguished by the condition that the
argument, FLAG, has the value~$^{\prime },^{\prime }$ in all but the
last call, and has the value~$^{\prime }.^{\prime }$ in the last call
of the sequence.

The first, and possibly only, call of a sequence must be to ERMSG, SERM1,
DERM1, or IERM1.  These subroutines all have the argument, LEVEL, that
specifies the nominal action level for the entire sequence of calls.

If there are additional calls in the sequence, the subsequent calls must
be to SERV1, DERV1, IERV1, or ERMOR, each of which may be called any
number of times.  These calls provide additional data values or character
strings to be included in the printed error message.

The package contains two additional subroutines: ERMSET, which can be called
by a user to alter the nominal action of the package, and ERFIN which is
called by other subroutines of the package when FLAG = $^{\prime}.^{\prime}$
to handle the common final steps of processing an error message.

\subsubsection{Type Statements for Arguments}
\begin{description}
\item[INTEGER] \ {\bf IERR, LEVEL, IDATA, IDELTA, NVAL}

\item[CHARACTER] \ {\bf SUBNAM*$n_1$, MESS*$n_2$,\\
FLAG, LABEL*$n_3$}

\item[CHARACTER*$n_4$] \ {\bf LABELS}($\geq $NVAL)

\item[REAL] \ {\bf SDATA, SVALS}($\geq $NVAL)

\item[DOUBLE PRECISION] \ {\bf DDATA, DVALS}($\geq $NVAL)
\end{description}

\subsubsection{Call Statements}

In each of these calls, all arguments must be assigned values before the
call and none of the argument values will be changed by the subroutine. It
will usually be most convenient to supply most of the arguments,
particularly those of type character, as literals in the call statement.

To initiate an error message:

\begin{center}
\fbox{\begin{tabular}{@{\bf }c}
CALL ERMSG (SUBNAM, IERR, LEVEL,\\
MESS, FLAG)\\
\end{tabular}}
\end{center}

To initiate a message, including one REAL datum:

\begin{center}
\fbox{\begin{tabular}{@{\bf }c}
CALL SERM1 (SUBNAM, IERR, LEVEL,\\
MESS, LABEL, SDATA, FLAG)\\
\end{tabular}}
\end{center}

To initiate a message, including one DOUBLE PRECISION datum:

\begin{center}
\fbox{\begin{tabular}{@{\bf }c}
CALL DERM1 (SUBNAM, IERR, LEVEL,\\
MESS, LABEL, DDATA, FLAG)\\
\end{tabular}}
\end{center}

To initiate a message, including one INTEGER datum:

\begin{center}
\fbox{\begin{tabular}{@{\bf }c}
CALL IERM1 (SUBNAM, IERR, LEVEL,\\
MESS, LABEL, IDATA, FLAG)\\
\end{tabular}}
\end{center}

To add a REAL datum to the current message:
$$
\fbox{\bf CALL SERV1 (LABEL, SDATA, FLAG)}
$$
To add a DOUBLE PRECISION datum to the current message:
$$
\fbox{\bf CALL DERV1 (LABEL, DDATA, FLAG)}
$$
To add an INTEGER datum to the current message:
$$
\fbox{\bf CALL IERV1 (LABEL, IDATA, FLAG)}
$$
\pagebreak

To add an additional character string, MESS, to the current message:
$$
\fbox{\bf CALL ERMOR (MESS, FLAG)}
$$
To alter the nominal message processing action:
$$
\fbox{\bf CALL ERMSET (IDELTA)}
$$
Subroutine called by other subroutines of the package when
FLAG = $^{\prime}.^{\prime}$ to print the closing line of dollar signs
and take the final action of returning or stopping:
$$
\fbox{\bf CALL ERFIN}
$$
\subsubsection{Argument Definitions}

\begin{description}

\item[SUBNAM] \ [in] Name of subprogram in which error has been detected. Suggest
length of name $\leq $ 12.

\item[IERR] \ [in] Identification number for the error.

\item[LEVEL] \ [in] Should be set to~2, 0, or~$-$2 to specify the nominal
action desired.
\begin{itemize}
\item[2] = print and stop
\item[0] = print and return
\item[$-$2] = return with no printing
\end{itemize}
This specification applies over a complete sequence of calls to the package
as described above. In particular if a stop is to occur, it will not happen
until the last call of a sequence, $i.e$. a call with
FLAG = $^{\prime}.^{\prime}$ .

The actual action taken is governed by:

\hspace{.2in}NALPHA = NDELTA + LEVEL,

where NDELTA is a saved local variable in the package. The initial value of
NDELTA is zero but it can be changed by use of subroutine ERMSET.

\item[MESS] \ [in] Error message of length $\leq $ 72.

\item[FLAG] \ [in] A single character, either a comma or a period. A comma means
further data or character strings to be included in the error message will
be provided by subsequent calls. A period means this is the last call
relating to the current error message.

\item[LABEL] \ [in] An identifying label to be printed with the data value given in
SDATA, DDATA, or IDATA. Suggest length of label $\leq $ 35.

\item[SDATA] \ [in] Data item of type REAL to be printed with the error message.

\item[DDATA] \ [in] Data item of type DOUBLE PRECISION to be printed with the error
message.

\item[IDATA] \ [in] Data item of type INTEGER to be printed with the error message.

\item[NVAL] \ [in] Number of array elements to be printed from LABELS() and from
SVALS() or DVALS().

\item[LABELS()] \ [in] An array of NVAL labels to be printed with the data values
given in SVALS() or DVALS(). The array elements should have CHARACTER
length $\leq $ 35.

\item[SVALS()] \ [in] An array of NVAL REAL values to be printed.

\item[DVALS()] \ [in] An array of NVAL DOUBLE PRECISION values to be printed.

\item[IDELTA] \ [in] New value to be assigned to the saved local variable NDELTA.
Alters the action of the error processing.

\end{description}

\subsection{Examples and Remarks}

Usage is illustrated by the program DRERMSG and the output ODERMSG.

\subsection{Functional Description}

\subsubsection{Levels of action}

The actual action level, NALPHA, computed as LEVEL + NDELTA, determines
the action as follows:

\begin{tabular}{ll}
NALPHA $\geq $ 2 & print message and STOP.\\
NALPHA $= -1$, 0 or 1 & print message and RETURN.\\
NALPHA $\leq$ $-$2 & RETURN, doing no printing.\\
\end{tabular}

\subsubsection{Effects of resetting NDELTA}

The saved local variable, NDELTA, initially has the value zero. If it is
changed by a call to ERMSET the effect can be interpreted as follows:

{\bf NDELTA \hspace{.75in} Effect}\vspace{-3pt}
\begin{itemize}
\item[2\ ]  Print and stop on all diagnostics that would otherwise be printed.

\item[0\ ]  Take the standard action.

\item[$-$1\ ]  Do not stop on any diagnostics. Print as usual.

\item[$-$2\ ]  Never stop. Print only those diagnostics that nominally result in a stop.

\item[$-$4\ ]  Do not print or stop on any diagnostic.
\end{itemize}

\subsubsection{Form of the error message}

The message will begin with a line of~72 dollar signs. The next two lines
will be:

Subprogram SUBNAM reports Error No. IERR

The initial message, MESS.

Following may be lines of the following forms:

\begin{tabbing}
(1)\hspace{.2in}\=LABEL = $value$\\
\\
where $value$ is SDATA, DDATA, or IDATA,\\
\\
(2)\>$label1 = val1$, $label2 = val2$, ...\\
\\
where the $labels$ are from LABELS() and the $values$ are\\
from SVALS() or DVALS(), or\\
\\
(3)\>MESS\\
\\
transmitted by subroutine ERMOR.
\end{tabbing}

Finally the message will be terminated with another line of~72 dollar signs.

\subsection{Error Procedures and Restrictions}

This package shares data via a COMMON block named M77ERR. The user must not
use this name for any other COMMON block.

If any of the character string arguments are longer than the suggested
maximum lengths, the corresponding printed line of the error message will
exceed a length of 72~characters.

The subroutine ERMSET should not be called between the beginning and end of
a sequence of calls to the error processing subroutines.

\subsection{Supporting Information}

The source language is ANSI Fortran~77. Uses common block M77ERR.

\begin{tabular}{@{\bf}l@{\hspace{5pt}}l}
\bf Entry & \hspace{.35in} {\bf Required Files}\vspace{2pt} \\
DERM1 & \parbox[t]{2.7in}{\hyphenpenalty10000 \raggedright
DERM1, DERV1, ERFIN, ERMSG\rule[-5pt]{0pt}{8pt}}\\
DERV1 & \parbox[t]{2.7in}{\hyphenpenalty10000 \raggedright
DERV1, ERFIN\rule[-5pt]{0pt}{8pt}}\\
ERFIN & \parbox[t]{2.7in}{\hyphenpenalty10000 \raggedright
ERFIN\rule[-5pt]{0pt}{8pt}}\\
ERMOR & \parbox[t]{2.7in}{\hyphenpenalty10000 \raggedright
ERFIN, ERMOR\rule[-5pt]{0pt}{8pt}}\\
ERMSET & \parbox[t]{2.7in}{\hyphenpenalty10000 \raggedright
ERFIN, ERMSG\rule[-5pt]{0pt}{8pt}}\\
ERMSG & \parbox[t]{2.7in}{\hyphenpenalty10000 \raggedright
ERFIN, ERMSG\rule[-5pt]{0pt}{8pt}}\\
IERM1 & \parbox[t]{2.7in}{\hyphenpenalty10000 \raggedright
ERFIN, ERMSG, IERM1, IERV1\rule[-5pt]{0pt}{8pt}}\\
IERV1 & \parbox[t]{2.7in}{\hyphenpenalty10000 \raggedright
ERFIN, IERV1\rule[-5pt]{0pt}{8pt}}\\
SERM1 & \parbox[t]{2.7in}{\hyphenpenalty10000 \raggedright
ERFIN, ERMSG, SERM1, SERV1\rule[-5pt]{0pt}{8pt}}\\
SERV1 & \parbox[t]{2.7in}{\hyphenpenalty10000 \raggedright
ERFIN, SERV1\rule[-5pt]{0pt}{8pt}}\\
\end{tabular}

This package is based on the subroutine EMESS, designed by F. T. Krogh, JPL,
1974, and programmed by S. A. Singletary, JPL, 1974. Present version
designed and programmed by C. L. Lawson and S. Chiu, JPL, April~1983.


\begcode

\medskip\
\lstset{language=[77]Fortran,showstringspaces=false}
\lstset{xleftmargin=.8in}

\centerline{\bf \large DRERMSG}\vspace{10pt}
\lstinputlisting{\codeloc{ermsg}}
\newpage

\vspace{30pt}\centerline{\bf \large ODERMSG}\vspace{10pt}
\lstset{language={}}
\lstinputlisting{\outputloc{ermsg}}
\end{document}
